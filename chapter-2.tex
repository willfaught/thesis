\chapter{Model of Computation}

Embedding server languages within client languages as expressions succinctly expresses the client-server interaction of interoperation.  Embedded server expressions evaluate to values and convert to equivalent client values.  Value conversion represents servers sending data and clients receiving them.  Since Haskell, ML, and Scheme are functional languages, extended lambda calculi can represent them.

Lambda calculus is a minimal model of computation that embodies the notions of abstracting patterns into functions and applying function abstractions to concrete instances of patterns.  Computation is done by substituting function arguments for free occurrences of corresponding variables within function bodies.  Interoperation between lambda calculi is represented by nesting languages within languages expressions embedded within special syntactic forms that indicate a change of language and expected and actual types of values for both sides of boundaries.  Evaluation contexts determine the evaluation strategy.

\section{Core Calculi}

\subsection{Haskell}

Lambda calculus is comprised of functions, variables, and function applications.  Haskell extends lambda calculus in several ways.  First, it adds natural numbers, addition and subtraction operations for natural numbers, and a condition that uses the natural number zero as true and all other natural numbers as false.

Second, it adds list constructions containing pairs of head and tail expressions, empty lists, operations to access head and tail expressions within list constructions, a condition that uses empty lists as true and list constructions as false, and error reports for head and tail operations that operate on empty lists.

Third, it adds a static type system called System F.  System F adds types for natural numbers, lists, and functions and the expressions that contain them.  If the type of an expression is valid, it is well-typed; otherwise, it is ill-typed.  Type errors cannot occur in well-typed expressions.  Haskell does not compute ill-typed expressions.  It adds type annotations to some expressions to enable the calculation and validation of expression types.  It adds parametric polymorphism whereby the types of expressions can be parameterized by type variables and type abstractions and then instantiated with type applications.  It adds types for type variables and type abstractions.

System F renders Haskell Turing-incomplete because Haskell cannot express recursive functions.  Haskell adds a fixed-point operator to restore Turing-completeness.  The operand of the fixed-point operator is a function of a recursive function over the body of the recursive function.

Fourth, the values and evaluation contexts of Haskell follow the spirit of lazy evaluation in that expressions are reduced only where necessary.

An expression is an instance of any syntactic form.  A reduction or evaluation is a step of computation that applies a rewrite rule of an operational semantics to an expression.  A value is an expression that is irreducible.  Letter subscripts of grammar non-terminals and relations denote the language to which they belong.

Types, denoted $T$, are comprised of natural numbers, denoted $N$; type variables, denoted $X$; functions, denoted $T\rightarrow T$; type abstractions, denoted $\forall X.T$; and lists, denoted $[T]$.

Haskell values, denoted $v_{H}$, are comprised of natural numbers, denoted $\overline{n}$, which syntactically represents the natural number $n$; list constructions, denoted $\mathtt{cons}$ $e_{H}$ $e_{H}$; empty lists, denoted $\mathtt{nil}^{T}$; functions, denoted $\lambda x:T.e_{H}$; and type abstractions, denoted $\Lambda X.e_{H}$.

Haskell expressions, denoted $e_{H}$, are comprised of variables, denoted $x$; function applications, denoted $e_{H}$ $e_{H}$; type applications, denoted $e_{H}$ $\lbrace T\rbrace$; addition and subtraction operations, denoted $+$ $e_{H}$ $e_{H}$ and $-$ $e_{H}$ $e_{H}$, respectively; natural number conditions, denoted $\mathtt{if0}$ $e_{H}$ $e_{H}$ $e_{H}$; head and tail operations, denoted $\mathtt{hd}$ $e_{H}$ and $\mathtt{tl}$ $e_{H}$, respectively; list conditions, denoted $\mathtt{ifnil}$ $e_{H}$ $e_{H}$ $e_{H}$; fixed-point operations, denoted $\mathtt{fix}$ $e_{H}$; and error reports, denoted $\mathtt{wrong}^{T}$ string.

Haskell evaluation contexts, denoted $E_{H}$, determine the Haskell evaluation strategy by constraining which expression matches the Haskell hole.  Haskell evaluation contexts are comprised of the Haskell hole, denoted $[\,]_{H}$; left expressions of function applications, denoted $E_{H}$ $e_{H}$; left expressions of type applications, denoted $E_{H}$ $\lbrace T\rbrace$; left operands of arithmetic operations, denoted $o$ $E_{H}$ $e_{H}$; right operands of arithmetic operations if left operands are values, denoted $o$ $v_{H}$ $E_{H}$; guards of natural number conditions, denoted $\mathtt{if0}$ $E_{H}$ $e_{H}$ $e_{H}$; operands of head and tail operations, denoted $f$ $E_{H}$; guards of list conditions, denoted $\mathtt{ifnil}$ $E_{H}$ $e_{H}$ $e_{H}$; and operands of fixed-point operations, denoted $\mathtt{fix}$ $E_{H}$.

Variables and type variables are unique across all languages.  Alpha conversion implicitly resolves name conflicts.

\begin{figure}[tp]
%\onehalfspacing
\centering
\begin{tabular}{lcl}
$e_{H}$ & $=$ & $v_{H}$ $\vert$ $x$ $\vert$ $e_{H}$ $e_{H}$ $\vert$ $e_{H}$ $\lbrace T\rbrace$ $\vert$ $o$ $e_{H}$ $e_{H}$ $\vert$ $\mathtt{if0}$ $e_{H}$ $e_{H}$ $e_{H}$ $\vert$ $f$ $e_{H}$ \\
&& $\vert$ $\mathtt{ifnil}$ $e_{H}$ $e_{H}$ $e_{H}$ $\vert$ $\mathtt{fix}$ $e_{H}$ $\vert$ $\mathtt{wrong}^{T}$ string \\
$v_{H}$ & $=$ & $\overline{n}$ $\vert$ $\mathtt{cons}$ $e_{H}$ $e_{H}$ $\vert$ $\mathtt{nil}^{T}$ $\vert$ $\lambda x:T.e_{H}$ $\vert$ $\Lambda X.e_{H}$ \\
$T$ & $=$ & $N$ $\vert$ $X$ $\vert$ $T\rightarrow T$ $\vert$ $\forall X.T$ $\vert$ $[T]$ \\
$o$ & $=$ & $\mathtt{+}$ $\vert$ $\mathtt{-}$ \\
$f$ & $=$ & $\mathtt{hd}$ $\vert$ $\mathtt{tl}$ \\
$E_{H}$ & $=$ & $[\,]_{H}$ $\vert$ $E_{H}$ $e_{H}$ $\vert$ $E_{H}$ $\lbrace T\rbrace$ $\vert$ $o$ $E_{H}$ $e_{H}$ $\vert$ $o$ $v_{H}$ $E_{H}$ $\vert$ $\mathtt{if0}$ $E_{H}$ $e_{H}$ $e_{H}$ $\vert$ $f$ $E_{H}$ \\
&& $\vert$ $\mathtt{ifnil}$ $E_{H}$ $e_{H}$ $e_{H}$ $\vert$ $\mathtt{fix}$ $E_{H}$
\end{tabular}
\caption{Haskell core expressions}
\label{hce}
\end{figure}

The Haskell typing relation, denoted $\Gamma\vdash_{H}e_{H}:T$, is defined by a set of typing rules that assign types to expressions \cite{pierce02}.  $\Gamma\vdash_{H}e_{H}:T$ is a typing statement that asserts the type of an expression and typing rules are implications between typing statements.  Where the type of an expression depends on the types of its subexpressions, its typing rule uses typing statements about those subexpressions as premises for its conclusion.  An expression $e_{H}$ is well-typed if there is some $T$ with the context $\Gamma$ such that $\Gamma\vdash_{H}e_{H}:T$.  A context is a set of assumptions about the types of variables and the abstraction of type variables.  Since all programs are assumed to be closed, programs containing free variables or free type variables are ill-typed.  Adding to the context $\Gamma$ assumptions about the type variable $X$ and the type $T$ of variable $x$ is denoted $\Gamma,X$ and $\Gamma,x:T$, respectively.  The typing statement $\Gamma\vdash_{H}T$ asserts the type $T$ is well-formed with the context $\Gamma$.  Where the context is empty, it is omitted from typing statements.

Natural number values, addition and subtraction operations and their operands, and guards of natural number conditions have type $N$.  Functions, left expressions of function applications, and operands of fixed-point operations have type $T_{1}\rightarrow T_{2}$.  Type abstractions and expressions of type applications have type $\forall X.T$.  List constructions, tails of list constructions, empty lists, tail operations, operands of head and tail operations, and guards of list conditions have type $[T]$.

Variables have the types they are assumed to have by the context.  If the context does not assume the type of a variable, the variable occurs free and the expression is ill-typed.  If the context does not assume the abstraction of a type variable, the type variable occurs free and the expression is ill-typed.  Function values, empty lists, and error reports have type annotations because their types cannot be calculated otherwise.  Type substitution substitutes one type, $T_{1}$, for free occurrences of a type variable, $X$, within a second type, $T_{2}$, denoted $T_{2}[T_{1}/X]$.  Number subscripts and superscripts of grammar symbols in a typing rule denote distinct instances of them.  Numbers are absent where there is a single instance of a grammar symbol in a typing rule.

\begin{figure}[p]
\label{hctr}
\caption{Haskell core typing rules}
\[
\frac{}{\vdash_{H}N}
\quad
\frac{X\in\Gamma}{\Gamma\vdash_{H}X}
\quad
\frac{\Gamma\vdash_{H}T_{1}\quad\Gamma\vdash_{H}T_{2}}{\Gamma\vdash_{H}T_{1}\rightarrow T_{2}}
\quad
\frac{\Gamma ,X\vdash_{H}T}{\Gamma\vdash_{H}\forall X.T}
\quad
\frac{\Gamma\vdash_{H}T}{\Gamma\vdash_{H}[T]}
\]
\bigskip
\[
\frac{}{\vdash_{H}\overline{n}:N}
\quad
\frac{\Gamma\vdash_{H}e_{H}^{1}:T\quad\Gamma\vdash_{H}e_{H}^{2}:[T]}{\Gamma\vdash_{H}\mathtt{cons}\;e_{H}^{1}\;e_{H}^{2}:[T]}
\quad
\frac{\Gamma\vdash_{H}T}{\Gamma\vdash_{H}\mathtt{nil}^{T}:[T]}
\]
\[
\frac{\Gamma\vdash_{H}T_{1}\quad\Gamma,x:T_{1}\vdash_{H}e_{H}:T_{2}}{\Gamma\vdash_{H}\lambda x:T_{1}.e_{H}:T_{1}\rightarrow T_{2}}
\quad
\frac{\Gamma,X\vdash_{H}e_{H}:T}{\Gamma\vdash_{H}\Lambda X.e_{H}:\forall X.T}
\quad
\frac{x:T\in\Gamma}{\Gamma\vdash_{H}x:T}
\]
\[
\frac{\Gamma\vdash_{H}e_{H}^{1}:T_{1}\rightarrow T_{2}\quad\Gamma\vdash_{H}e_{H}^{2}:T_{1}}{\Gamma\vdash_{H}e_{H}^{1}\;e_{H}^{2}:T_{2}}
\quad
\frac{\Gamma\vdash_{H}T_{1}\quad\Gamma\vdash_{H}e_{H}:\forall X.T_{2}}{\Gamma\vdash_{H}e_{H}\;\lbrace T_{1}\rbrace:T_{2}[T_{1}/X]}
\]
\[
\frac{\Gamma\vdash_{H}e_{H}^{1}:N\quad\Gamma\vdash_{H}e_{H}^{2}:N}{\Gamma\vdash_{H}o\;e_{H}^{1}\;e_{H}^{2}:N}
\quad
\frac{\Gamma\vdash_{H}e_{H}^{1}:N\quad\Gamma\vdash_{H}e_{H}^{2}:T\quad\Gamma\vdash_{H}e_{H}^{3}:T}{\Gamma\vdash_{H}\mathtt{if0}\;e_{H}^{1}\;e_{H}^{2}\;e_{H}^{3}:T}
\]
\[
\frac{\Gamma\vdash_{H}e_{H}:[T]}{\Gamma\vdash_{H}\mathtt{hd}\;e_{H}:T}
\quad
\frac{\Gamma\vdash_{H}e_{H}:[T]}{\Gamma\vdash_{H}\mathtt{tl}\;e_{H}:[T]}
\]
\[
\frac{\Gamma\vdash_{H}e_{H}^{1}:[T_{1}]\quad\Gamma\vdash_{H}e_{H}^{2}:T_{2}\quad\Gamma\vdash_{H}e_{H}^{3}:T_{2}}{\Gamma\vdash_{H}\mathtt{ifnil}\;e_{H}^{1}\;e_{H}^{2}\;e_{H}^{3}:T_{2}}
\quad
\frac{\Gamma\vdash_{H}e_{H}:T\rightarrow T}{\Gamma\vdash_{H}\mathtt{fix}\;e_{H}:T}
\]
\[
\frac{\Gamma\vdash_{H}T}{\Gamma\vdash_{H}\mathtt{wrong}^{T}\;\mathrm{string}:T}
\]
\end{figure}

Haskell computations are defined by a set of rewrite rules called an operational semantics.  A rewrite rule rewrites a reducible expression in an evaluation context that matches its left side according to its right side.  Function applications substitute the argument for free occurrences of the function parameter within the function body.  Type applications substitute the argument for free occurrences of the type abstraction parameter within the type abstraction body.  Addition and subtraction operations reduce to the addition and subtraction of their operands, respectively.  Natural number conditions reduce to their true or false expressions if their guards are zero or not zero, respectively.  Head and tail operations on list constructions reduce to the heads and tails of their operands, respectively.  Head and tail operations on empty lists reduce to error reports.  List conditions reduce to their true or false expressions if their guards are empty lists or list constructions, respectively.  Fixed-point operations reduce to their operands substituted for the function parameters of their operands within the function bodies of their operands.  Error reports reduce to errors and terminate the computation.

Expression substitution substitutes one expression, $e_{H}^{1}$, for a variable, $x$, within a second expression, $e_{H}^{2}$, denoted $e_{H}^{2}[e_{H}^{1}/x]$.  Variable instances that occur only on the right side of a rewrite rule must be unique.  All rewrite rules are defined with an unspecified evaluation context $\mathscr{E}$.  The evaluation of a single language would instantiate $\mathscr{E}$ to $E_{H}$ for Haskell, to $E_{M}$ for ML, and to $E_{S}$ for Scheme.  The interoperation of languages would instantiate $\mathscr{E}$ according to the language in which programs begin and end.

\begin{figure}[p]
%\onehalfspacing
\centering
\begin{tabular}{rcl}
$\mathscr{E}[(\lambda x:T.e_{H}^{1})$ $e_{H}^{2}]_{H}$ & $\rightarrow$ & $\mathscr{E}[e_{H}^{1}[e_{H}^{2}/x]]$ \\
$\mathscr{E}[(\Lambda X.e_{H})$ $\lbrace T\rbrace]_{H}$ & $\rightarrow$ & $\mathscr{E}[e_{H}[T/X]]$ \\
$\mathscr{E}[+$ $\overline{n_{1}}$ $\overline{n_{2}}]_{H}$ & $\rightarrow$ & $\mathscr{E}[\overline{n_{1}+n_{2}}]$ \\
$\mathscr{E}[-$ $\overline{n_{1}}$ $\overline{n_{2}}]_{H}$ & $\rightarrow$ & $\mathscr{E}[\overline{max(n_{1}-n_{2},0)}]$ \\
$\mathscr{E}[\mathtt{if0}$ $\overline{0}$ $e_{H}^{1}$ $e_{H}^{2}]_{H}$ & $\rightarrow$ & $\mathscr{E}[e_{H}^{1}]$ \\
$\mathscr{E}[\mathtt{if0}$ $\overline{n}$ $e_{H}^{1}$ $e_{H}^{2}]_{H}$ & $\rightarrow$ & $\mathscr{E}[e_{H}^{2}]$ $(n\neq0)$ \\
$\mathscr{E}[\mathtt{hd}$ $(\mathtt{cons}$ $e_{H}^{1}$ $e_{H}^{2})]_{H}$ & $\rightarrow$ & $\mathscr{E}[e_{H}^{1}]$ \\
$\mathscr{E}[\mathtt{tl}$ $(\mathtt{cons}$ $e_{H}^{1}$ $e_{H}^{2})]_{H}$ & $\rightarrow$ & $\mathscr{E}[e_{H}^{2}]$ \\
$\mathscr{E}[\mathtt{hd}$ $\mathtt{nil}^{T}]_{H}$ & $\rightarrow$ & $\mathscr{E}[\mathtt{wrong}^{T}$ ``Empty list"$]$ \\
$\mathscr{E}[\mathtt{tl}$ $\mathtt{nil}^{T}]_{H}$ & $\rightarrow$ & $\mathscr{E}[\mathtt{wrong}^{[T]}$ ``Empty list"$]$ \\
$\mathscr{E}[\mathtt{ifnil}$ $\mathtt{nil}^{T}$ $e_{H}^{1}$ $e_{H}^{2}]_{H}$ & $\rightarrow$ & $\mathscr{E}[e_{H}^{1}]$ \\
$\mathscr{E}[\mathtt{ifnil}$ $(\mathtt{cons}$ $e_{H}^{1}$ $e_{H}^{2})$ $e_{H}^{3}$ $e_{H}^{4}]_{H}$ & $\rightarrow$ & $\mathscr{E}[e_{H}^{4}]$ \\
$\mathscr{E}[\mathtt{fix}$ $(\lambda x:T.e_{H})]_{H}$ & $\rightarrow$ & $\mathscr{E}[e_{H}[(\mathtt{fix}$ $(\lambda x:T.e_{H}))/x]]$ \\
$\mathscr{E}[\mathtt{wrong}^{T}$ string$]_{H}$ & $\rightarrow$ & \textbf{Error}: string
\end{tabular}
\caption{Haskell core operational semantics}
\label{hcos}
\end{figure}

\subsection{ML}

ML is identical to Haskell except that it uses strict evaluation instead of lazy evaluation.  This difference affects list constructions and evaluation contexts.  Haskell reduces expressions only where necessary and ML reduces expressions regardless of necessity, thus their definition of values are different.  Haskell list constructions are values regardless of whether their heads and tails are values, but ML list constructions are values where their heads and tails are values.  To reflect these differences, ML adds a list construction expression where the head and tail are expressions, denoted $\mathtt{cons}$ $e_{M}$ $e_{M}$, and changes the head and tail of the list construction value to values, denoted $\mathtt{cons}$ $v_{M}$ $v_{M}$.

ML evaluation contexts extend Haskell evaluation contexts such that all expressions are reduced and the order of evaluation is deterministic.  In addition to the Haskell definition, ML evaluation contexts are right expressions of function applications where left expression are values, heads of list constructions, and tails of list construction where heads are values.

\begin{figure}[ph!]
\centering
\begin{tabular}{lcl}
$e_{M}$ & $=$ & $v_{M}$ $\vert$ $x$ $\vert$ $e_{M}$ $e_{M}$ $\vert$ $e_{M}$ $\lbrace T\rbrace$ $\vert$ $o$ $e_{M}$ $e_{M}$ $\vert$ $\mathtt{if0}$ $e_{M}$ $e_{M}$ $e_{M}$ $\vert$ $\mathtt{cons}$ $e_{M}$ $e_{M}$ \\

\vspace{5pt}

&& $\vert$ $f$ $e_{M}$ $\vert$ $\mathtt{ifnil}$ $e_{M}$ $e_{M}$ $e_{M}$ $\vert$ $\mathtt{fix}$ $e_{M}$ $\vert$ $\mathtt{wrong}^{T}$ string \\

\vspace{5pt}

$v_{M}$ & $=$ & $\overline{n}$ $\vert$ $\mathtt{cons}$ $v_{M}$ $v_{M}$ $\vert$ $\mathtt{nil}^{T}$ $\vert$ $\lambda x:T.e_{M}$ $\vert$ $\Lambda X.e_{M}$ \\

\vspace{5pt}

$T$ & $=$ & $N$ $\vert$ $X$ $\vert$ $T\rightarrow T$ $\vert$ $\forall X.T$ $\vert$ $[T]$ \\

\vspace{5pt}

$o$ & $=$ & $\mathtt{+}$ $\vert$ $\mathtt{-}$ \\

\vspace{5pt}

$f$ & $=$ & $\mathtt{hd}$ $\vert$ $\mathtt{tl}$ \\

\vspace{5pt}

$E_{M}$ & $=$ & $[\,]_{M}$ $\vert$ $E_{M}$ $e_{M}$ $\vert$ $v_{M}$ $E_{M}$ $\vert$ $E_{M}$ $\lbrace T\rbrace$ $\vert$ $o$ $E_{M}$ $e_{M}$ $\vert$ $o$ $v_{M}$ $E_{M}$ \\

\vspace{5pt}

&& $\vert$ $\mathtt{if0}$ $E_{M}$ $e_{M}$ $e_{M}$ $\vert$ $\mathtt{cons}$ $E_{M}$ $e_{M}$ $\vert$ $\mathtt{cons}$ $v_{M}$ $E_{M}$ $\vert$ $f$ $E_{M}$ \\

\vspace{5pt}

&& $\vert$ $\mathtt{ifnil}$ $E_{M}$ $e_{M}$ $e_{M}$ $\vert$ $\mathtt{fix}$ $E_{M}$
\end{tabular}
\caption{ML core expressions}
\label{mce}
\end{figure}

\begin{figure}[p]
\label{mctr}
\caption{ML core typing rules}
\[
\frac{}{\vdash_{M}N}
\quad
\frac{X\in\Gamma}{\Gamma\vdash_{M}X}
\quad
\frac{\Gamma\vdash_{M}T_{1}\quad\Gamma\vdash_{M}T_{2}}{\Gamma\vdash_{M}T_{1}\rightarrow T_{2}}
\quad
\frac{\Gamma,X\vdash_{M}T}{\Gamma\vdash_{M}\forall X.T}
\quad
\frac{\Gamma\vdash_{M}T}{\Gamma\vdash_{M}[T]}
\]
\bigskip
\[
\frac{}{\vdash_{M}\overline{n}:N}
\quad
\frac{\Gamma\vdash_{M}e_{M}^{1}:T\quad\Gamma\vdash_{M}e_{M}^{2}:[T]}{\Gamma\vdash_{M}\mathtt{cons}\;e_{M}^{1}\;e_{M}^{2}:[T]}
\quad
\frac{\Gamma\vdash_{M}T}{\Gamma\vdash_{M}\mathtt{nil}^{T}:[T]}
\]
\[
\frac{\Gamma\vdash_{M}T_{1}\quad\Gamma,x:T_{1}\vdash_{M}e_{M}:T_{2}}{\Gamma\vdash_{M}\lambda x:T_{1}.e_{M}:T_{1}\rightarrow T_{2}}
\quad
\frac{\Gamma,X\vdash_{M}e_{M}:T}{\Gamma\vdash_{M}\Lambda X.e_{M}:\forall X.T}
\quad
\frac{x:T\in\Gamma}{\Gamma\vdash_{M}x:T}
\]
\[
\frac{\Gamma\vdash_{M}e_{M}^{1}:T_{1}\rightarrow T_{2}\quad\Gamma\vdash_{M}e_{M}^{2}:T_{1}}{\Gamma\vdash_{M}e_{M}^{1}\;e_{M}^{2}:T_{2}}
\quad
\frac{\Gamma\vdash_{M}T_{1}\quad\Gamma\vdash_{M}e_{M}:\forall X.T_{2}}{\Gamma\vdash_{M}e_{M}\;\lbrace T_{1}\rbrace:T_{2}[T_{1}/X]}
\]
\[
\frac{\Gamma\vdash_{M}e_{M}^{1}:N\quad\Gamma\vdash_{M}e_{M}^{2}:N}{\Gamma\vdash_{M}o\;e_{M}^{1}\;e_{M}^{2}:N}
\quad
\frac{\Gamma\vdash_{M}e_{M}^{1}:N\quad\Gamma\vdash_{M}e_{M}^{2}:T\quad\Gamma\vdash_{M}e_{M}^{3}:T}{\Gamma\vdash_{M}\mathtt{if0}\;e_{M}^{1}\;e_{M}^{2}\;e_{M}^{3}:T}
\]
\[
\frac{\Gamma\vdash_{M}e_{M}:[T]}{\Gamma\vdash_{M}\mathtt{hd}\;e_{M}:T}
\quad
\frac{\Gamma\vdash_{M}e_{M}:[T]}{\Gamma\vdash_{M}\mathtt{tl}\;e_{M}:[T]}
\]
\[
\frac{\Gamma\vdash_{M}e_{M}^{1}:[T_{1}]\quad\Gamma\vdash_{M}e_{M}^{2}:T_{2}\quad\Gamma\vdash_{M}e_{M}^{3}:T_{2}}{\Gamma\vdash_{M}\mathtt{ifnil}\;e_{M}^{1}\;e_{M}^{2}\;e_{M}^{3}:T_{2}}
\quad
\frac{\Gamma\vdash_{M}e_{M}:T\rightarrow T}{\Gamma\vdash_{M}\mathtt{fix}\;e_{M}:T}
\quad
\]
\[
\frac{\Gamma\vdash_{M}T}{\Gamma\vdash_{M}\mathtt{wrong}^{T}\;\mathrm{string}:T}
\]
\end{figure}

\begin{figure}[p]
%\onehalfspacing
\centering
\begin{tabular}{rcl}
$\mathscr{E}[(\lambda x:T.e_{M})$ $v_{M}]_{M}$ & $\rightarrow$ & $\mathscr{E}[e_{M}[v_{M}/x]]$ \\
$\mathscr{E}[(\Lambda X.e_{M})$ $\lbrace T\rbrace]]_{M}$ & $\rightarrow$ & $\mathscr{E}[e_{M}[T/X]]$ \\
$\mathscr{E}[+$ $\overline{n_{1}}$ $\overline{n_{2}}]_{M}$ & $\rightarrow$ & $\mathscr{E}[\overline{n_{1}+n_{2}}]$ \\
$\mathscr{E}[-$ $\overline{n_{1}}$ $\overline{n_{2}}]_{M}$ & $\rightarrow$ & $\mathscr{E}[\overline{max(n_{1}-n_{2},0)}]$ \\
$\mathscr{E}[\mathtt{if0}$ $\overline{0}$ $e_{M}^{1}$ $e_{M}^{2}]_{M}$ & $\rightarrow$ & $\mathscr{E}[e_{M}^{1}]$ \\
$\mathscr{E}[\mathtt{if0}$ $\overline{n}$ $e_{M}^{1}$ $e_{M}^{2}]_{M}$ & $\rightarrow$ & $\mathscr{E}[e_{M}^{2}]$ $(n\neq0)$ \\
$\mathscr{E}[\mathtt{hd}$ $(\mathtt{cons}$ $v_{M}^{1}$ $v_{M}^{2})]_{M}$ & $\rightarrow$ & $\mathscr{E}[v_{M}^{1}]$ \\
$\mathscr{E}[\mathtt{tl}$ $(\mathtt{cons}$ $v_{M}^{1}$ $v_{M}^{2})]_{M}$ & $\rightarrow$ & $\mathscr{E}[v_{M}^{2}]$ \\
$\mathscr{E}[\mathtt{hd}$ $\mathtt{nil}^{T}]_{M}$ & $\rightarrow$ & $\mathscr{E}[\mathtt{wrong}^{T}$ ``Empty list"$]$ \\
$\mathscr{E}[\mathtt{tl}$ $\mathtt{nil}^{T}]_{M}$ & $\rightarrow$ & $\mathscr{E}[\mathtt{wrong}^{[T]}$ ``Empty list"$]$ \\
$\mathscr{E}[\mathtt{ifnil}$ $\mathtt{nil}^{T}$ $e_{M}^{1}$ $e_{M}^{2}]_{M}$ & $\rightarrow$ & $\mathscr{E}[e_{M}^{1}]$ \\
$\mathscr{E}[\mathtt{ifnil}$ $(\mathtt{cons}$ $v_{M}^{1}$ $v_{M}^{2})$ $e_{M}^{1}$ $e_{M}^{2}]_{M}$ & $\rightarrow$ & $\mathscr{E}[e_{M}^{2}]$ \\
$\mathscr{E}[\mathtt{fix}$ $(\lambda x:T.e_{M})]_{M}$ & $\rightarrow$ & $\mathscr{E}[e_{M}[(\mathtt{fix}$ $(\lambda x:T.e_{M}))/x]]$ \\
$\mathscr{E}[\mathtt{wrong}^{T}$ string$]_{M}$ & $\rightarrow$ & \textbf{Error}: string 
\end{tabular}
\caption{ML core operational semantics}
\label{mcos}
\end{figure}

\subsection{Scheme}

Scheme is identical to ML except that it removes the static type system and adds a dynamic type system.  It removes all the types and type annotations for functions, empty lists, and error reports.  It removes type abstractions and type applications and the evaluation contexts, typing rules, and rewrite rules that contain them.

The dynamic type system adds type predicate expressions that determine the types of values.  $\mathtt{num?}$ $e_{S}$ determines whether $e_{S}$ is a natural number.  $\mathtt{list?}$ $e_{S}$ determines whether $e_{S}$ is a list.  $\mathtt{fun?}$ $e_{S}$ determines whether $e_{S}$ is a function.

In addition to ML evaluation contexts, Scheme evaluation contexts are the expression of type predicates, denoted $p$ $E_{S}$.

\begin{figure}[ph!]
\centering
\begin{tabular}{lcl}
$e_{S}$ & $=$ & $v_{S}$ $\vert$ $x$ $\vert$ $e_{S}$ $e_{S}$ $\vert$ $o$ $e_{S}$ $e_{S}$ $\vert$ $\mathtt{if0}$ $e_{S}$ $e_{S}$ $e_{S}$ $\vert$ $\mathtt{cons}$ $e_{S}$ $e_{S}$ $\vert$ $f$ $e_{S}$ \\

\vspace{5pt}

&& $\vert$ $\mathtt{ifnil}$ $e_{S}$ $e_{S}$ $e_{S}$ $\vert$ $p$ $e_{S}$ $\vert$ $\mathtt{wrong}$ $\mathrm{string}$ \\

\vspace{5pt}

$v_{S}$ & $=$ & $\overline{n}$ $\vert$ $\mathtt{cons}$ $v_{S}$ $v_{S}$ $\vert$ $\mathtt{nil}$ $\vert$ $\lambda x.e_{S}$ \\

\vspace{5pt}

$o$ & $=$ & $\mathtt{+}$ $\vert$ $\mathtt{-}$ \\

\vspace{5pt}

$f$ & $=$ & $\mathtt{hd}$ $\vert$ $\mathtt{tl}$ \\

\vspace{5pt}

$p$ & $=$ & $\mathtt{num?}$ $\vert$ $\mathtt{list?}$ $\vert$ $\mathtt{fun?}$ \\

\vspace{5pt}

$E_{S}$ & $=$ & $[\,]_{S}$ $\vert$ $E_{S}$ $e_{S}$ $\vert$ $v_{S}$ $E_{S}$ $\vert$ $o$ $E_{S}$ $e_{S}$ $\vert$ $o$ $v_{S}$ $E_{S}$ $\vert$ $\mathtt{if0}$ $E_{S}$ $e_{S}$ $e_{S}$ $\vert$ $\mathtt{cons}$ $E_{S}$ $e_{S}$ \\

\vspace{5pt}

&& $\vert$ $\mathtt{cons}$ $v_{S}$ $E_{S}$ $\vert$ $f$ $E_{S}$ $\vert$ $\mathtt{ifnil}$ $E_{S}$ $e_{S}$ $e_{S}$ $\vert$ $p$ $E_{S}$
\end{tabular}
\caption{Scheme core expressions}
\label{sce}
\end{figure}

It adds a single type called The Scheme Type, $TST$.  Well-typed expressions do not have free variables and have type $TST$.

\begin{figure}
\[
\frac{}{\vdash_{S}TST}
\]
\bigskip
\[
\frac{}{\vdash_{S}\overline{n}:TST}
\quad
\frac{}{\vdash_{S}\mathtt{nil}:TST}
\quad
\frac{}{\vdash_{S}\mathtt{wrong}\;\mathrm{string}:TST}
\]
\[
\frac{x:TST\in\Gamma}{\Gamma\vdash_{S}x:TST}
\quad
\frac{\Gamma\vdash_{S}e_{S}^{1}:TST\quad\Gamma\vdash_{S}e_{S}^{2}:TST}{\Gamma\vdash_{S}\mathtt{cons}\;e_{S}^{1}\;e_{S}^{2}:TST}
\]
\[
\frac{\Gamma\vdash_{S}e_{S}^{1}:TST\quad\Gamma\vdash_{S}e_{S}^{2}:TST\quad\Gamma\vdash_{S}e_{S}^{3}:TST}{\Gamma\vdash_{S}\mathtt{if0}\;e_{S}^{1}\;e_{S}^{2}\;e_{S}^{3}:TST}
\]
\[
\frac{\Gamma,x:TST\vdash_{S}e_{S}:TST}{\Gamma\vdash_{S}\lambda x.e_{S}:TST}
\quad
\frac{\Gamma\vdash_{S}e_{S}^{1}:TST\quad\Gamma\vdash_{S}e_{S}^{2}:TST}{\Gamma\vdash_{S}e_{S}^{1}\;e_{S}^{2}:TST}
\]
\[
\quad
\frac{\Gamma\vdash_{S}e_{S}^{1}:TST\quad\Gamma\vdash_{S}e_{S}^{2}:TST}{\Gamma\vdash_{S}o\;e_{S}^{1}\;e_{S}^{2}:TST}
\quad
\frac{\Gamma\vdash_{S}e_{S}:TST}{\Gamma\vdash_{S}f\;e_{S}:TST}
\quad
\frac{\Gamma\vdash_{S}e_{S}:TST}{\Gamma\vdash_{S}p\;e_{S}:TST}
\]
\caption{Scheme core typing rules}
\label{sctr}
\end{figure}

It changes the natural number condition to reduce to one if the guard is not zero, and possibly not a natural number.  It changes the list condition to reduce to one if the guard is not an empty list, and possibly not a list.  The type predicates reduce to the natural number zero if true and the natural number one if false.

\begin{figure}[p]
%\onehalfspacing
\centering
\begin{tabular}{rcl}
$\mathscr{E}[(\lambda x.e_{S})$ $v_{S}]_{S}$ & $\rightarrow$ & $\mathscr{E}[e_{S}[v_{S}/x]]$ \\
$\mathscr{E}[v_{S}^{1}$ $v_{S}^{2}]_{S}$ & $\rightarrow$ & $\mathscr{E}[\mathtt{wrong}$ ``Not a function"$]$ $(v_{S}^{1}\neq\lambda x.e_{S})$ \\
$\mathscr{E}[+$ $\overline{n_{1}}$ $\overline{n_{2}}]_{S}$ & $\rightarrow$ & $\mathscr{E}[\overline{n_{1}+n_{2}}]$ \\
$\mathscr{E}[-$ $\overline{n_{1}}$ $\overline{n_{2}}]_{S}$ & $\rightarrow$ & $\mathscr{E}[\overline{max(n_{1}-n_{2},0)}]$ \\
$\mathscr{E}[o$ $v_{S}^{1}$ $v_{S}^{2}]_{S}$ & $\rightarrow$ & $\mathscr{E}[\mathtt{wrong}$ ``Not a number"$]$ $(v_{S}^{1}\neq\overline{n}$ or $v_{S}^{2}\neq\overline{n})$ \\
$\mathscr{E}[\mathtt{if0}$ $\overline{0}$ $e_{S}^{1}$ $e_{S}^{2}]_{S}$ & $\rightarrow$ & $\mathscr{E}[e_{S}^{1}]$ \\
$\mathscr{E}[\mathtt{if0}$ $v_{S}$ $e_{S}^{1}$ $e_{S}^{2}]_{S}$ & $\rightarrow$ & $\mathscr{E}[e_{S}^{2}]$ $(v_{S}\neq\overline{0})$ \\
$\mathscr{E}[\mathtt{hd}$ $(\mathtt{cons}$ $v_{S}^{1}$ $v_{S}^{2})]_{S}$ & $\rightarrow$ & $\mathscr{E}[v_{S}^{1}]$ \\
$\mathscr{E}[\mathtt{tl}$ $(\mathtt{cons}$ $v_{S}^{1}$ $v_{S}^{2})]_{S}$ & $\rightarrow$ & $\mathscr{E}[v_{S}^{2}]$ \\
$\mathscr{E}[\mathtt{hd}$ $\mathtt{nil}]_{S}$ & $\rightarrow$ & $\mathscr{E}[\mathtt{wrong}$ ``Empty list"$]$ \\
$\mathscr{E}[\mathtt{tl}$ $\mathtt{nil}]_{S}$ & $\rightarrow$ & $\mathscr{E}[\mathtt{wrong}$ ``Empty list"$]$ \\
$\mathscr{E}[f$ $v_{S}^{1}]_{S}$ & $\rightarrow$ & $\mathscr{E}[\mathtt{wrong}$ ``Not a list"$]$ \\
&& $(v_{S}^{1}\neq\mathtt{cons}$ $v_{S}^{2}$ $v_{S}^{3}$ and $v_{S}^{1}\neq\mathtt{nil})$ \\
$\mathscr{E}[\mathtt{ifnil}$ $\mathtt{nil}$ $e_{S}^{1}$ $e_{S}^{2}]_{S}$ & $\rightarrow$ & $\mathscr{E}[e_{S}^{1}]$ \\
$\mathscr{E}[\mathtt{ifnil}$ $v_{S}$ $e_{S}^{1}$ $e_{S}^{2}]_{S}$ & $\rightarrow$ & $\mathscr{E}[e_{S}^{2}]$ $(v_{S}\neq\mathtt{nil})$ \\
$\mathscr{E}[\mathtt{num?}$ $\overline{n}]_{S}$ & $\rightarrow$ & $\mathscr{E}[\overline{0}]$ \\
$\mathscr{E}[\mathtt{num?}$ $v_{S}]_{S}$ & $\rightarrow$ & $\mathscr{E}[\overline{1}]$ ($v_{S}\neq\overline{n}$) \\
$\mathscr{E}[\mathtt{list?}$ $(\mathtt{cons}$ $v_{S}^{1}$ $v_{S}^{2})]_{S}$ & $\rightarrow$ & $\mathscr{E}[\overline{0}]$ \\
$\mathscr{E}[\mathtt{list?}$ $\mathtt{nil}]_{S}$ & $\rightarrow$ & $\mathscr{E}[\overline{0}]$ \\
$\mathscr{E}[\mathtt{list?}$ $v_{S}^{1}]_{S}$ & $\rightarrow$ & $\mathscr{E}[\overline{1}]$ $(v_{S}^{1}\neq\mathtt{cons}$ $v_{S}^{2}$ $v_{S}^{3}$ and $v_{S}^{1}\neq\mathtt{nil})$ \\
$\mathscr{E}[\mathtt{fun?}$ $(\lambda x.e_{S})]_{S}$ & $\rightarrow$ & $\mathscr{E}[\overline{0}]$ \\
$\mathscr{E}[\mathtt{fun?}$ $v_{S}]_{S}$ & $\rightarrow$ & $\mathscr{E}[\overline{1}]$ ($v_{S}\neq\lambda x.e_{S}$) \\
$\mathscr{E}[\mathtt{wrong}$ $\mathrm{string}]_{S}$ & $\rightarrow$ & \textbf{Error}: string
\end{tabular}
\caption{Scheme core operational semantics}
\label{scos}
\end{figure}

\section{Interoperation Calculi}

The interoperation calculi extend the core calculi with new expressions, evaluation contexts, typing rules, and rewrite rules to express interoperation.  They add boundary expressions, which represent values with actual (server) and expected (client) types crossing from server languages to client languages.  Boundaries are denoted by two-letter acronyms, where the first letter names servers and the second letters names clients.  Expected types are superscripts to the left of first letters, and actual types are superscripts to the right of second letters.  Expressions to be reduced to values and cross between languages are to the right of the letters and types, separated by a space.  For example, the expression $^{T_{1}}HM^{T_{2}}$ $e_{M}$ denotes ML expression $e_{M}$ with actual type $T_{2}$ and expected type $T_{1}$ crossing from ML to Haskell.  Since a set of $n$ interoperable languages requires $n\times(n-1)$ boundary expressions, this model requires six boundaries.

They add evaluation contexts for the expressions of boundaries, $^{T}HM^{T}$ $E_{M}$ for example.

They add typing rules for boundaries.  For a boundary to be well-typed, its actual and expected types must be well-formed and equivalent and the type of its expression must equal the actual type of the boundary.  The types of boundaries are their expected types.  $TST$ is omitted from boundary notation because all well-typed Scheme expressions have type $TST$.

They add rewrite rules for every combination of boundary, expected and actual types, and syntactic forms of values.  Rewrite rules for boundaries containing Scheme values verify that their syntactic forms match their expected types and report type errors.

The actual and expected types of boundaries determine their reduction.

\subsection{Natural Number Types}

Natural numbers do not change when they cross languages because the languages have the same number domain.  For example, $^{N}HM^{N}$ $\overline{n}$ reduces to $\overline{n}$.  For $^{N}HS$ $v_{S}$ and $^{N}MS$ $v_{S}$, where $v_{S}$ is not a natural number, the boundaries reduce to type error reports.

\subsection{Function Types}

Functions cannot be directly converted when they cross boundaries because each language grammar is different, Haskell and ML do not have a reasonable equivalent for every Scheme function, and functions may behave differently if they are evaluated with a different evaluation strategy.  Instead, a function from the sending language is wrapped inside a function from the receiving language with a boundary expression.  The wrapper function passes its argument across a boundary from the receiving language to the sending language.  Within the sending language, the original function is applied to the argument, and the result is passed back across a boundary to the receiving language.  The wrapper function produces the result from the sending language as its result.  To do this, a language is allowed to perform substitution within itself across boundaries.  For example, the Scheme function $\lambda x_{1}.e_{S}$ represented in Haskell with type $T_{1}\rightarrow T_{2}$ is $\lambda x_{2}:T_{1}.(^{T_{2}}HS$ $((\lambda x_{1}.e_{S})$ $(SH^{T_{1}}$ $x_{2})))$.  If the Haskell function is applied to an argument, the argument crosses the $SH^{T_{1}}$ boundary from Haskell to Scheme, the Scheme function is applied to it, and its result crosses the $^{T_{2}}HS$ boundary from Scheme back to Haskell, which becomes the result of the function application.

\subsection{Type Abstraction Types}

If the expected and actual types of a boundary expression are a universal type, the contained value is either a type abstraction or a Scheme value contained within another boundary expression.  If it is a type abstraction, $\Lambda X.e_{M}$ with type $\forall X.T$ for example, the boundary expression reduces to a type abstraction of $X$ for a boundary with expected and actual type $T$ containing $e_{M}$.  Therefore $^{\forall X.T}HM^{\forall X.T}$ $(\Lambda X.e_{M})\rightarrow\Lambda X.(^{T}HM^{T}$ $e_{M})$.  If it is a Scheme value contained within another boundary expression, $^{\forall X.T}MS$ $v_{S}$ for example, the boundary expression reduces to a new boundary expression where $v_{S}$ crosses from Scheme to Haskell with expected type $\forall X.T$ and actual type $TST$.  Therefore $^{\forall X.T}HM^{\forall X.T}$ $(^{\forall X.T}MS$ $v_{S})\rightarrow{^{\forall X.T}H}S$ $v_{S}$.

If the expected and actual types of a boundary expression are a universal type and the Scheme type, respectively, the contained value is a Scheme value.  Such a boundary expression is irreducible because Scheme does not have expressions with universal types.  Therefore $^{\forall X.T_{1}}HS$ $v_{S}$ and $^{\forall X.T_{1}}MS$ $v_{S}$ are values.  Instead, such a boundary expression can be applied to a type $T_{2}$ like a type abstraction, which could substitute the type argument $T_{2}$ for $X$ within $T_{1}$ for its expected and actual types.  However, if $T_{1}$ is a function type that contains $X$ and $v_{S}$ is a function, a safeguard must ensure that $v_{S}$ does not break the parametricity of Haskell and ML.  For example, if $T_{1}$ is $X\rightarrow X$, Haskell and ML assume $v_{S}$ is the identity function, but it may be another function that satisfies the type and thus breaks parametricity.

To ensure Scheme cannot break parametricity, the safeguard must do two things.  First, it must prevent polymorphic Scheme functions from determining their behavior by the types and values of their arguments.  Consider the following reductions, where a polymorphic Scheme function would break the parametricity of Haskell:

\begin{tabular}{ll}
& $((^{\forall X.(X\rightarrow X)}HS$ $(\lambda x_{1}.(\mathtt{if0}$ $x_{1}$ $1$ $x_{1})))$ $\lbrace N\rbrace)$ $\overline{0}$ \\
$\rightarrow$ & $(^{N\rightarrow N}HS$ $(\lambda x_{1}.(\mathtt{if0}$ $x_{1}$ $1$ $x_{1})))$ $\overline{0}$ \\
$\rightarrow$ & $(\lambda x_{2}:N.(^{N}HS$ $((\lambda x_{1}.(\mathtt{if0}$ $x_{1}$ $1$ $x_{1}))$ $(SH^{N}$ $x_{2}))))$ $\overline{0}$ \\
$\rightarrow$ & $^{N}HS$ $((\lambda x_{1}.(\mathtt{if0}$ $x_{1}$ $1$ $x_{1}))$ $(SH^{N}$ $\overline{0}))$
\end{tabular}

Next, the $SH^{N}\;\overline{0}$ boundary expression would reduce and the Haskell function argument $\overline{0}$ would cross the boundary from Haskell to Scheme and reduce to the Scheme value $\overline{0}$.  The Scheme polymorphic function can now determine the type and value of the argument and thereby break parametricity.  Therefore to preserve parametricity, $SH^{N}\;\overline{0}$ cannot be reducible, and in fact all such boundary expressions in the context of being arguments to polymorphic Scheme functions cannot be reducible as well.  Since such boundary expressions also occur for the conversion of non-polymorphic Scheme functions, such boundary expressions must be marked in such a way that they are reducible for non-polymorphic functions but irreducible for polymorphic functions.  This is done by introducing a new type called a type label, denoted $T^{a}$, and changing the type arguments to labeled types before performing the type substitution.  Even though labeled types are used, the type of the expression is still expected to be the type underlying the label, so the type of HS and MS expressions and the types of expressions within SH and SM expressions are their expected and actual types, respectively, with their labeled types replacd with their corresponding underlying types, denoted $T[T_{i}/T_{i}^{a}]$.  Furthermore, since arguments may be returned, there has to be a way to reduce the nested HS and SH boundaries to retrieve the Haskell value.  Therefore the rewrite rule $\mathscr{E}[^{T^{a}}HS$ $(SH^{T^{a}}$ $e_{H})\rightarrow\mathscr{E}[e_{H}]$ is added.

Second, it must verify that the same argument corresponds to every instantiation of $X$ within $T_{1}$.  Otherwise, it is possible for a Scheme function with type $\forall X_{1}.(\forall X_{2}.(X_{1}\rightarrow X_{2}\rightarrow X_{1}))$ and instantiated with $N$ and again with $N$ to return the second argument and not the first as required by parametricity.  Therefore every type label used for type substitutions must be unique and the type labels of nested HS and SH and MS and SM expressions must match up.  Otherwise, an error is signaled to indicate that parametricity was broken.

Even if $T_{1}$ is not a function type, the type labels are still used in case a function type is contained by it.

If the expected and actual types of a boundary expression are the Scheme type and a universal type, respectively, the contained value is either a type abstraction or a Scheme value contained within another boundary expression.

If it is a type abstraction, $\Lambda X.e_{H}$ with type $\forall X.T$ for example, it may contain parametrically polymorphic functions of types containing $X$.  Since those functions are parametrically polymorphic, they cannot examine the types or values of their arguments.  Therefore Scheme arguments for these functions cannot cross the boundary to Haskell and ML.  In addition, the function types must be instantiated with a type that would enable them to accept any Scheme argument.  The lump type $L$ is added to be used in these situations.  $^{L}HS$ $v_{S}$ and $^{L}MS$ $v_{S}$ are irreducible.  For example, $SH^{\forall X.(T\rightarrow T)}$ $\Lambda X.(\lambda x:X.x)\rightarrow SH^{(T\rightarrow T)[L/X]}$ $(\Lambda X.(\lambda x:X.x))$ $\lbrace L\rbrace\rightarrow SH^{L\rightarrow L}$ $(\lambda x:L.x)$.

If it is a Scheme value contained within another boundary expression, the boundary expression reduces to the Scheme value.  For example, $SH^{T}$ $(^{T}HS$ $v_{S})\rightarrow$ $v_{S}$.

!!! TODO: explain label removal t[ti/tia] syntax

\subsection{List Types}

If the expected and actual types of a boundary expression are a list type, the contained value is either an empty list or a list construction.  If it is an empty list, the boundary expression reduces to the empty list.  For example, $^{[T]}HM^{[T]}$ $\mathtt{nil}^{T}$ reduces to $\mathtt{nil}^{T}$.  If it is an ML list construction crossing to Haskell, the boundary expression reduces to a list construction where the head and tail are the head and tail of the ML list construction wrapped in boundary expressions, respectively.  For example, $\mathtt{cons}$ $v_{M}^{1}$ $v_{M}^{2}$ with type $[T]$ would reduce to $\mathtt{cons}$ $(^{T}HM^{T}$ $v_{M}^{1})$ $(^{[T]}HM^{[T]}$ $v_{M}^{2})$.  If it is a Haskell list construction crossing to ML, the boundary expression should be irreducible.  Since Haskell list constructions are values, they can be recursive and therefore infinite in length.  Converting a list construction from Haskell to ML like a list construction is converted from ML to Haskell would reduce the tail until it is a value.  If the list construction were infinite, the tail may reduce to the list construction, and thus never reduce to a value, and the computation would loop forever.  Therefore $^{[T]}MH^{[T]}$ $(\mathtt{cons}$ $e_{H}^{1}$ $e_{H}^{2})$ is a value.

If the expected and actual types of a boundary expression are a list type and the Scheme type, respectively, the contained value is either an empty list or a list construction.  If it is an empty list, the boundary expression reduces to an empty list of the corresponding type.  For example, $^{[T]}HS$ $\mathtt{nil}$ reduces to $\mathtt{nil}^{T}$.  If it is a list construction, the boundary expression reduces to a list construction where the new head and tail are the old head and tail wrapped in boundary expressions, respectively.  For example, $^{[T]}HS$ $\mathtt{cons}$ $v_{S}^{1}$ $v_{S}^{2}$ reduces to $\mathtt{cons}$ $(^{T}HS$ $v_{S}^{1})$ $(^{[T]}HS$ $v_{S}^{2})$.

If the expected and actual types of a boundary expression are the Scheme type and a list type, respectively, the contained value is either an empty list or a list construction.  If it is an empty list, the boundary expression reduces to an empty list.  For example, $SH^{[T]}$ $\mathtt{nil}^{T}$ reduces to $\mathtt{nil}$.  If it is an ML list construction, the boundary expression reduces to a list construction where the new head and tail are the old head and tail wrapped in boundary expressions, respectively.  For example, $SM^{[T]}$ $v_{M}^{1}$ $v_{M}^{2}$ reduces to $\mathtt{cons}$ $(SM^{T}$ $v_{M}^{1})$ $(SM^{[T]}$ $v_{M}^{2})$.  If it is a Haskell list construction, it should be irreducible for the same reason that $^{[T]}MH^{[T]}$ $(\mathtt{cons}$ $e_{H}^{1}$ $e_{H}^{2})$ is a value, as discussed above.  Therefore $SH^{[T]}$ $(\mathtt{cons}$ $e_{H}^{1}$ $e_{H}^{2})$ is a value.

\begin{figure}[p]
%\onehalfspacing
\centering
\begin{tabular}{lcl}
$e_{H}$ & $=$ & $\cdots$ $\vert$ $^{T}HM^{T}$ $e_{M}$ $\vert$ $^{T}HS$ $e_{S}$ \\
$v_{H}$ & $=$ & $\cdots$ $\vert$ $^{\forall X.T}HS$ $v_{S}$ $\vert$ $^{L}HS$ $v_{S}$ \\
$T$ & $=$ & $\cdots$ $\vert$ $L$ $\vert$ $T^{a}$ \\
$a$ & $=$ & Type labels (a, b, \ldots) \\
$E_{H}$ & $=$ & $\cdots$ $\vert$ $^{T}HM^{T}$ $E_{M}$ $\vert$ $^{T}HS$ $E_{S}$
\end{tabular}
\caption{Haskell interoperation expressions}
\label{hie}
\end{figure}
\begin{figure}
\[
\frac{}{\Gamma\vdash_{H}L}
\quad
\frac{\Gamma\vdash_{H}T}{\Gamma\vdash_{H}T^{a}}
\]
\bigskip
\[
\frac{\Gamma\vdash_{S}t_{S}:TST\quad\Gamma\vdash_{H}T}{\Gamma\vdash_{H}\;^{T}HS\;t_{S}:T[T_{i}/T^{a}_{i}]}
\quad
\frac{\Gamma\vdash_{M}t_{M}:T_{2}\quad\Gamma\vdash_{H}T_{1}\quad\Gamma\vdash_{M}T_{2}\quad T_{1}=T_{2}}{\Gamma\vdash_{H}\;^{T_{1}}HM^{T_{2}}\;t_{M}:T_{1}}
\]
\caption{Haskell interoperation typing rules}
\label{hitr}
\end{figure}
\begin{figure}
\onehalfspacing
\begin{center}
\begin{tabular}{rcl}
% hm - number
$\mathscr{E}[^{N}HM^{N}\;\overline{n}]_{H}$ & $\rightarrow$ & $\mathscr{E}[\overline{n}]$ \\
% hs - number
$\mathscr{E}[^{N}HS\;\overline{n}]_{H}$ & $\rightarrow$ & $\mathscr{E}[\overline{n}]$ \\
% hs - number error
$\mathscr{E}[^{N}HS\;v_{S}]_{H}$ & $\rightarrow$ & $\mathscr{E}[^{N}HS\;(\mathtt{wrong}\;\mathrm{``Not\;a\;number"})]$ \\
&& ($v_{S}\neq\overline{n}$) \\
% hm - function
$\mathscr{E}[^{T_{1}\rightarrow T_{2}}HM^{T_{1}\rightarrow T_{2}}\;(\lambda x_{1}:T_{1}.e_{M})]_{H}$ & $\rightarrow$ & $\mathscr{E}[\lambda x_{2}:T_{1}.(^{T_{2}}HM^{T_{2}}\;((\lambda x_{1}:T_{1}.e_{M})\;(^{T_{1}}MH^{T_{1}}\;x_{2})))]$ \\
% hs - function
$\mathscr{E}[^{T_{1}\rightarrow T_{2}}HS\;(\lambda x_{1}.e_{S})]_{H}$ & $\rightarrow$ & $\mathscr{E}[\lambda x_{2}:T_{1}[T_{i}/T^{a}_{i}].(^{T_{2}}HS\;((\lambda x_{1}.e_{S})\;(SH^{T_{1}}\;x_{2})))]$ \\
% hs - function error
$\mathscr{E}[^{T_{1}\rightarrow T_{2}}HS\;v_{S}]_{H}$ & $\rightarrow$ & $\mathscr{E}[^{T_{1}\rightarrow T_{2}}HS\;(\mathtt{wrong}\;\mathrm{``Not\;a\;procedure"})]$ \\
&& ($v_{S}\neq\lambda x.e_{S}$) \\
% hm - universal
$\mathscr{E}[^{\forall X.T}HM^{\forall X.T}\;(\Lambda X.e_{M})]_{H}$ & $\rightarrow$ & $\mathscr{E}[\Lambda X.(^{T}HM^{T}\;e_{M})]$ \\
% hm - universal ms
$\mathscr{E}[^{\forall X.T}HM^{\forall X.T}\;(^{\forall X.T}MS\;v_{S})]_{H}$ & $\rightarrow$ & $\mathscr{E}[^{\forall X.T}HS\;v_{S}]$ \\
% hs - universal
$\mathscr{E}[(^{\forall X.T_{1}}HS\;v_{S})\;\lbrace T_{2}\rbrace]_{H}$ & $\rightarrow$ & $\mathscr{E}[^{T_{1}[T^{a}_{2}/X]}HS\;v_{S}]$ \\
% hm - list - cons
$\mathscr{E}[^{[T]}HM^{[T]}\;(\mathtt{cons}\;v_{M}^{1}\;v_{M}^{2})]_{H}$ & $\rightarrow$ & $\mathscr{E}[\mathtt{cons}\;(^{T}HM^{T}\;v_{M}^{1})\;(^{[T]}HM^{[T]}\;v_{M}^{2})]$ \\
% hm - list - mh cons
$\mathscr{E}[^{[T]}HM^{[T]}\;(^{[T]}MH^{[T]}\;(\mathtt{cons}\;e_{H}^{1}\;e_{H}^{2}))]_{H}$ & $\rightarrow$ & $\mathscr{E}[\mathtt{cons}\;e_{H}^{1}\;e_{H}^{2}]$ \\
% hs - list - cons
$\mathscr{E}[^{[T]}HS\;(\mathtt{cons}\;v_{S}^{1}\;v_{S}^{2})]_{H}$ & $\rightarrow$ & $\mathscr{E}[\mathtt{cons}\;(^{T}HS\;v_{S}^{1})\;(^{[T]}HS\;v_{S}^{2})]$ \\
% hs - list - sh cons
$\mathscr{E}[^{[T]}HS\;(SH^{[T]}\;(\mathtt{cons}\;e_{H}^{1}\;e_{H}^{2}))]_{H}$ & $\rightarrow$ & $\mathscr{E}[\mathtt{cons}\;e_{H}^{1}\;e_{H}^{2}]$ \\
% hm - list - nil
$\mathscr{E}[^{[T]}HM^{[T]}\;\mathtt{nil}^{T}]_{H}$ & $\rightarrow$ & $\mathscr{E}[\mathtt{nil}^{T}]$ \\
% hs - list - nil
$\mathscr{E}[^{[T]}HS\;\mathtt{nil}]_{H}$ & $\rightarrow$ & $\mathscr{E}[\mathtt{nil}^{T}]$ \\
% hs - list error
$\mathscr{E}[^{[T]}HS\;v_{S}^{1}]_{H}$ & $\rightarrow$ & $\mathscr{E}[^{[T]}HS\;(\mathtt{wrong}\;\mathrm{``Not\;a\;list"})]$ \\
&& ($v_{S}^{1}\neq\mathtt{cons}\;v_{S}^{2}\;v_{S}^{3}$ and $v_{S}^{1}\neq\mathtt{nil}$) \\
% hm - lump
$\mathscr{E}[^{L}HM^{L}\;(^{L}MS\;v_{S})]_{H}$ & $\rightarrow$ & $\mathscr{E}[^{L}HS\;v_{S}]$ \\
% hs - label
$\mathscr{E}[^{T^{a}}HS\;(SH^{T^{a}}\;e_{H})]_{H}$ & $\rightarrow$ & $\mathscr{E}[e_{H}]$ \\
% hs - label error
$\mathscr{E}[^{T^{a}}HS\;v_{S}]_{H}$ & $\rightarrow$ & $\mathscr{E}[^{T^{a}}HS\;(\mathtt{wrong}\;\mathrm{``Parametricity\;violated"})]$ \\
&& ($v_{S}\neq SH^{T^{a}}\;e_{H}$)
\end{tabular}
\end{center}
\caption{Haskell interoperation operational semantics}
\label{fig:hios}
\end{figure}
\begin{figure}[p]
%\onehalfspacing
\centering
\begin{tabular}{lcl}
$e_{M}$ & $=$ & $\cdots$ $\vert$ $^{T}MH^{T}$ $e_{H}$ $\vert$ $^{T}MS$ $e_{S}$ \\
$v_{M}$ & $=$ & $\cdots$ $\vert$ $^{[T]}MH^{[T]}$ $(\mathtt{cons}$ $e_{H}$ $e_{H})$ $\vert$ $^{\forall X.T}MS$ $v_{S}$ $\vert$ $^{L}MS$ $v_{S}$ \\
$T$ & $=$ & $\cdots$ $\vert$ $L$ $\vert$ $T^{a}$ \\
$a$ & $=$ & Type labels (a, b, \ldots) \\
$E_{M}$ & $=$ & $\cdots$ $\vert$ $^{T}MH^{T}$ $E_{H}$ $\vert$ $^{T}MS$ $E_{S}$
\end{tabular}
\caption{ML interoperation expressions}
\label{mie}
\end{figure}
\begin{figure}[ph!]
\[
\frac{}{\vdash_{M}L}
\quad
\frac{\Gamma\vdash_{M}T}{\Gamma\vdash_{M}T^{a}}
\]
\bigskip
\[
\frac{\Gamma\vdash_{M}T\quad\Gamma\vdash_{H}T\quad\Gamma\vdash_{H}e_{H}:T}{\Gamma\vdash_{M}\;^{T}MH^{T}\;e_{H}:T}
\quad
\frac{\Gamma\vdash_{M}T\quad\Gamma\vdash_{S}e_{S}:TST}{\Gamma\vdash_{M}\;^{T}MS\;e_{S}:T[T_{i}/T^{a}_{i}]}
\]
\caption{ML interoperation typing rules}
\label{mitr}
\end{figure}
\begin{figure}
\onehalfspacing
\begin{center}
\begin{tabular}{rcl}
$\mathscr{E}[^{N}MH^{N}\;\overline{n}]_{M}$ & $\rightarrow$ & $\mathscr{E}[\overline{n}]$ \\
$\mathscr{E}[^{\forall X_{1}.T}MH^{\forall X_{1}.T}\;(\Lambda X_{1}.e_{H})]_{M}$ & $\rightarrow$ & $\mathscr{E}[\Lambda X_{2}.(^{T[X_{2}/X_{1}]}MH^{T[X_{2}/X_{1}]}\;((\Lambda X_{1}.e_{H})\;\lbrace X_{2}\rbrace))]$ \\
$\mathscr{E}[^{T_{1}\rightarrow T_{2}}MH^{T_{1}\rightarrow T_{2}}\;(\lambda x_{1}:T_{1}.e_{H})]_{M}$ & $\rightarrow$ & $\mathscr{E}[\lambda x_{2}:T_{1}.(^{T_{2}}MH^{T_{2}}\;((\lambda x_{1}:T_{1}.e_{H})\;(^{T_{1}}HM^{T_{1}}\;x_{2})))]$ \\
$\mathscr{E}[\mathtt{hd}\;(^{[T]}MH^{[T]}\;(\mathtt{cons}\;e_{H}^{1}\;e_{H}^{2}))]_{M}$ & $\rightarrow$ & $\mathscr{E}[^{T}MH^{T}\;e_{H}^{1}]$ \\
$\mathscr{E}[\mathtt{tl}\;(^{[T]}MH^{[T]}\;(\mathtt{cons}\;e_{H}^{1}\;e_{H}^{2}))]_{M}$ & $\rightarrow$ & $\mathscr{E}[^{[T]}MH^{[T]}\;e_{H}^{2}]$ \\
$\mathscr{E}[^{N}MS\;\overline{n}]_{M}$ & $\rightarrow$ & $\mathscr{E}[\overline{n}]$ \\
$\mathscr{E}[^{N}MS\;v_{S}]_{M}$ & $\rightarrow$ & $\mathscr{E}[^{N}MS\;(\mathtt{wrong}\;\mathrm{``Not\;a\;number"})]$ \\
&& $(v_{S}\neq\overline{n}$ for any $\overline{n})$ \\
$\mathscr{E}[(^{\forall X.T_{1}}MS\;v_{S})\;\lbrace T_{2}\rbrace]_{M}$ & $\rightarrow$ & $\mathscr{E}[^{T_{1}[X/T^{a}_{2}]}MS\;v_{S}]$ \\
$\mathscr{E}[^{T_{1}\rightarrow T_{2}}MS\;(\lambda x_{1}.e_{S})]_{M}$ & $\rightarrow$ & $\mathscr{E}[\lambda x_{2}:T_{1}[T_{i}/T^{a}_{i}].(^{T_{2}}MS\;((\lambda x_{1}.e_{S})\;(SM^{T_{1}}\;x_{2})))]$ \\
$\mathscr{E}[^{T_{1}\rightarrow T_{2}}MS\;v_{S}]_{M}$ & $\rightarrow$ & $\mathscr{E}[^{T_{1}\rightarrow T_{2}}MS\;(\mathtt{wrong}\;\mathrm{``Not\;a\;function"})]$ \\
&& $(v_{S}\neq\lambda x.e_{S}$ for any $x,e_{S})$ \\
$\mathscr{E}[^{T^{a}}MS\;(SM^{T^{a}}\;v_{M})]_{M}$ & $\rightarrow$ & $\mathscr{E}[v_{M}]$ \\
$\mathscr{E}[^{T^{a}}MS\;v_{S}]_{M}$ & $\rightarrow$ & $\mathscr{E}[^{T^{a}}MS\;(\mathtt{wrong}\;\mathrm{``Parametricity\;violated"})]$ \\
&& $(v_{S}\neq SM^{T^{a}}\;v_{M}$ for any $v_{M})$ \\
$\mathscr{E}[^{[T]}MS\;(\mathtt{cons}\;v_{S}^{1}\;v_{S}^{2})]_{M}$ & $\rightarrow$ & $\mathscr{E}[\mathtt{cons}\;(^{T}MS\;v_{S}^{1})\;(^{[T]}MS\;v_{S}^{2})]$ \\
$\mathscr{E}[^{[T]}MS\;v_{S}^{1}]_{M}$ & $\rightarrow$ & $\mathscr{E}[^{[T]}MS\;(\mathtt{wrong}\;\mathrm{``Not\;a\;list"})]$ \\
&& $(v_{S}^{1}\neq\mathtt{cons}\;v_{S}^{2}\;v_{S}^{3}$ for any $v_{S}^{2},v_{S}^{3})$
\end{tabular}
\end{center}
\caption{ML interoperation operational semantics}
\label{fig:mios}
\end{figure}
\begin{figure}[ph!]
\centering
\begin{tabular}{lcl}
\vspace{5pt}

$e_{S}$ & $=$ & $\cdots$ $\vert$ $SH^{T}$ $e_{H}$ $\vert$ $SM^{T}$ $e_{M}$ \\

\vspace{5pt}

$v_{S}$ & $=$ & $\cdots$ $\vert$ $SH^{[T]}$ $(\mathtt{cons}$ $e_{H}$ $e_{H})$ $\vert$ $SH^{T^{a}}$ $e_{H}$ $\vert$ $SM^{T^{a}}$ $v_{M}$ \\

\vspace{5pt}

$E_{S}$ & $=$ & $\cdots$ $\vert$ $SH^{T}$ $E_{H}$ $\vert$ $SM^{T}$ $E_{M}$
\end{tabular}
\caption{Scheme interoperation expressions}
\label{sie}
\end{figure}
\begin{figure}[p]
\label{sitr}
\caption{Scheme interoperation typing rules}
\[
\frac{\Gamma\vdash_{H}T\quad\Gamma\vdash_{H}e_{H}:T[T_{i}/T^{a}_{i}]}{\Gamma\vdash_{S}\;SH^{T}\;e_{H}:TST}
\quad
\frac{\Gamma\vdash_{M}T\quad\Gamma\vdash_{M}e_{M}:T[T_{i}/T^{a}_{i}]}{\Gamma\vdash_{S}\;SM^{T}\;e_{M}:TST}
\]
\end{figure}
\begin{figure}
\onehalfspacing
\begin{center}
\begin{tabular}{rcl}
$\mathscr{E}[SH^{N}\;\overline{n}]_{S}$ & $\rightarrow$ & $\mathscr{E}[\overline{n}]$ \\
$\mathscr{E}[SH^{\forall X.T}\;(\Lambda X.e_{H})]_{S}$ & $\rightarrow$ & $\mathscr{E}[SH^{T[L/X]}\;((\Lambda X.e_{H})\;\lbrace L\rbrace)]$ \\
$\mathscr{E}[SH^{\forall X.T}\;(^{\forall X.T}HS\;v_{S})]_{S}$ & $\rightarrow$ & $\mathscr{E}[v_{S}]$ \\
$\mathscr{E}[SH^{T_{1}\rightarrow T_{2}}\;(\lambda x_{1}:T_{1}.e_{H})]_{S}$ & $\rightarrow$ & $\mathscr{E}[\lambda x_{2}.(SH^{T_{2}}\;((\lambda x_{1}:T_{1}.e_{H})\;(^{T_{1}}HS\;x_{2})))]$ \\
$\mathscr{E}[SH^{L}\;(^{L}HS\;v_{S})]_{S}$ & $\rightarrow$ & $\mathscr{E}[v_{S}]$ \\
$\mathscr{E}[\mathtt{hd}\;(SH^{[T]}\;(\mathtt{cons}\;e_{H}^{1}\;e_{H}^{2}))]_{S}$ & $\rightarrow$ & $\mathscr{E}[SH^{T}\;e_{H}^{1}]$ \\
$\mathscr{E}[\mathtt{tl}\;(SH^{[T]}\;(\mathtt{cons}\;e_{H}^{1}\;e_{H}^{2}))]_{S}$ & $\rightarrow$ & $\mathscr{E}[SH^{[T]}\;e_{H}^{2}]$ \\
$\mathscr{E}[SM^{N}\;\overline{n}]_{S}$ & $\rightarrow$ & $\mathscr{E}[\overline{n}]$ \\
$\mathscr{E}[SM^{\forall X.T}\;(\Lambda X.e_{H})]_{S}$ & $\rightarrow$ & $\mathscr{E}[SM^{T[L/X]}\;((\Lambda X.e_{M})\;\lbrace L\rbrace)]$ \\
$\mathscr{E}[SM^{\forall X.T}\;(^{\forall X.T}MS\;v_{S})]_{S}$ & $\rightarrow$ & $\mathscr{E}[v_{S}]$ \\
$\mathscr{E}[SM^{T_{1}\rightarrow T_{2}}\;(\lambda x_{1}:T_{1}.e_{M})]_{S}$ & $\rightarrow$ & $\mathscr{E}[\lambda x_{2}.(SM^{T_{2}}\;((\lambda x_{1}:T_{1}.e_{M})\;(^{T_{1}}MS\;x_{2})))]$ \\
$\mathscr{E}[SM^{L}\;(^{L}MS\;v_{S})]_{S}$ & $\rightarrow$ & $\mathscr{E}[v_{S}]$ \\
$\mathscr{E}[(SM^{[T]}\;(\mathtt{cons}\;v_{M}^{1}\;v_{M}^{2})]_{S}$ & $\rightarrow$ & $\mathscr{E}[\mathtt{cons}\;(SM^{T}\;v_{M}^{1})\;(SM^{[T]}\;v_{M}^{2})]$
\end{tabular}
\end{center}
\caption{Scheme interoperation operational semantics}
\label{fig:sios}
\end{figure}