\chapter{Proof of Type Soundness}

% TODO: Well-typed terms are closed, no need to specify both.

Proving the progress of expressions and the preservation of types proves the type soundness of the model of computation.  Progress ensures that a well-typed, closed expression is either an unforced value, reducible to another expression, or reducible to an error.  Preservation ensures that if a well-typed expression reduces to another expression, the other expression is well-typed and has the same type.  The proof extends the proof by Kinghorn \cite{kinghorn07}, which was based on proofs by Pierce \cite{pierce02} and Matthews and Findler \cite{matthews07}.  Cases common to two or more languages are elided for brevity.

\section{Proof of Expression Progress}

Progress will be proven by structural induction on a well-typed, closed expression of each syntactic form.  In each case, the expression will be proven to be either an unforced value, reducible to another expression, or reducible to an error.  The reduction of a subexpression is the reduction of its parent expression.  If a subexpression reduces to an error, its parent expression reduces to the error.  In some cases, the syntactic form of a subexpression must be determined to reduce its parent expression.  Determining the unique type of a subexpression determines its syntactic form.

\begin{lemma}

\label{leminv}

The syntactic forms of well-typed expressions determine the types of their subexpressions.

\begin{enumerate}

% Haskell

% \x:t.e

\item If \judeh{\env}{\expfabss{\varvarh}{\first{\vartyh}}{\varexph}}{\second{\vartyh}} then $\second{\vartyh} = \tyfun{\first{\vartyh}}{\third{\vartyh}}$, \judth{\env}{\first{\vartyh}}, and \judeh{\envexte{\varvarh}{\first{\vartyh}}}{\varexph}{\third{\vartyh}}.

% \\u.e

\item If \judeh{\env}{\exptabs{\tyvarh}{\varexph}}{\first{\vartyh}} then $\first{\vartyh} = \tyfor{\tyvarh}{\second{\vartyh}}$ and \judeh{\envextt{\tyvarh}}{\varexph}{\second{\vartyh}}.

% n

\item If \judeh{}{\expnum{\symnum}}{\vartyh} then $\vartyh = \tynum$.

% nil t

\item If \judeh{\env}{\expnils{\first{\vartyh}}}{\second{\vartyh}} then $\second{\vartyh} = \tylist{\first{\vartyh}}$ and \judth{\env}{\first{\vartyh}}.

% cons e e

\item If \judeh{\env}{\expcons{\first{\varexph}}{\second{\varexph}}}{\first{\vartyh}} then $\first{\vartyh} = \tylist{\second{\vartyh}}$, \judeh{\env}{\first{\varexph}}{\second{\vartyh}}, and \judeh{\env}{\second{\varexph}}{\tylist{\second{\vartyh}}}.

% x

\item \judeh{\envexte{\varvarh}{\vartyh}}{\varvarh}{\vartyh}.

% e e

\item If \judeh{\env}{\expfapp{\first{\varexph}}{\second{\varexph}}}{\first{\vartyh}} then \judeh{\env}{\first{\varexph}}{\tyfun{\second{\vartyh}}{\first{\vartyh}}} and \judeh{\env}{\second{\varexph}}{\second{\vartyh}}.

% fix e

\item If \judeh{\env}{\expfix{\varexph}}{\vartyh} then \judeh{\env}{\varexph}{\tyfun{\vartyh}{\vartyh}}.

% e<t>

\item If \judeh{\env}{\exptapp{\varexph}{\first{\vartyh}}}{\second{\vartyh}} then $\second{\vartyh} = \tysubst{\third{\vartyh}}{\first{\vartyh}}{\tyvarh}$, \judth{\env}{\vartyh}, and \judeh{\env}{\varexph}{\tyfor{\tyvarh}{\third{\vartyh}}}.

% hd e

\item If \judeh{\env}{\exphd{\varexph}}{\vartyh} then \judeh{\env}{\varexph}{\tylist{\vartyh}}.

% tl e

\item If \judeh{\env}{\exptl{\varexph}}{\first{\vartyh}} then $\first{\vartyh} = \tylist{\second{\vartyh}}$ and \judeh{\env}{\varexph}{\tylist{\second{\vartyh}}}.

% o e e

\item If $\Gamma\vdash_{A}o$ $e_{A}^{1}$ $e_{A}^{2}:T$ then $T=N$, $\Gamma\vdash_{A}e_{A}^{1}:N$, and $\Gamma\vdash_{A}e_{A}^{2}:N$ where $A\in\lbrace H,M\rbrace$.

\item If \judeh{\env}{\expop{\first{\varexph}}{\second{\varexph}}}{ % TODO

% null? e

\item If $\Gamma\vdash_{A}\mathtt{null?}$ $e_{A}:T$ then $T=N$ and $\Gamma\vdash_{A}e_{A}:[T_{1}]$ where $A\in\lbrace H,M\rbrace$.

% if0 e e e

\item If $\Gamma\vdash_{A}\mathtt{if0}$ $e_{A}^{1}$ $e_{A}^{2}$ $e_{A}^{3}:T$ then $T=T_{1}$, $\Gamma\vdash_{A}e_{A}^{1}:N$, $\Gamma\vdash_{A}e_{A}^{2}:T_{1}$, and $\Gamma\vdash_{A}e_{A}^{3}:T_{1}$ where $A\in\lbrace H,M\rbrace$.

% wrong t string

\item If $\Gamma\vdash_{A}\mathtt{wrong}^{T_{1}}$ $\mathrm{string}:T$ then $T=T_{1}$ where $A\in\lbrace H,M\rbrace$.

\item If $\Gamma\vdash_{A}{^{T_{1}}A}B^{T_{1}}$ $e_{B}:T$ then $T=T_{1}$, $\Gamma\vdash_{A}T_{1}$, $\Gamma\vdash_{B}T_{1}$, and $\Gamma\vdash_{B}e_{B}:T_{1}$ where $(A,B)\in\lbrace(H,M),(M,H)\rbrace$.

\item If $\Gamma\vdash_{A}{^{T_{1}}A}S$ $e_{S}:T$ then $T=T_{1}[T_{i}/T_{i}^{a}]$, $\Gamma\vdash_{A}T_{1}$, and $\Gamma\vdash_{S}e_{S}:TST$ where $A\in\lbrace H,M\rbrace$.

% ML

% Scheme

\item If $\Gamma\vdash_{S}\lambda x.e_{S}:TST$ then $\Gamma,x:TST\vdash_{S}e_{S}:TST$.

\item $\vdash_{S}\overline{n}:TST$.

\item $\vdash_{S}\mathtt{nil}:TST$.

\item If $\Gamma\vdash_{S}\mathtt{cons}$ $e_{S}^{1}$ $e_{S}^{2}:TST$ then $\Gamma\vdash_{S}e_{S}^{1}:TST$ and $\Gamma\vdash_{S}e_{S}^{2}:TST$.

\item If $\Gamma\vdash_{S}x:TST$ then $x:TST\in\Gamma$.

\item If $\Gamma\vdash_{S}e_{S}^{1}$ $e_{S}^{2}:TST$ then $\Gamma\vdash_{S}e_{S}^{1}:TST$ and $\Gamma\vdash_{S}e_{S}^{2}:TST$.

\item If $\Gamma\vdash_{S}f$ $e_{S}:TST$ then $\Gamma\vdash_{S}e_{S}:TST$.

\item If $\Gamma\vdash_{S}o$ $e_{S}^{1}$ $e_{S}^{2}:TST$ then $\Gamma\vdash_{S}e_{S}^{1}:TST$ and $\Gamma\vdash_{S}e_{S}^{2}:TST$.

\item If $\Gamma\vdash_{S}p$ $e_{S}:TST$ then $\Gamma\vdash_{S}e_{S}:TST$.

\item If $\Gamma\vdash_{S}\mathtt{if0}$ $e_{S}^{1}$ $e_{S}^{2}$ $e_{S}^{3}:TST$ then $\Gamma\vdash_{S}e_{S}^{1}:TST$, $\Gamma\vdash_{S}e_{S}^{2}:TST$, and $\Gamma\vdash_{S}e_{S}^{3}:TST$.

\item $\vdash_{S}\mathtt{wrong}$ $\mathrm{string}:TST$.

\item $\Gamma\vdash_{S}SA^{T_{1}}$ $e_{A}:TST$, $\Gamma\vdash_{A}T_{1}$, and $\Gamma\vdash_{A}e_{A}:T_{1}[T_{i}/T_{i}^{a}]$ where $A\in\lbrace H,M\rbrace$.

\end{enumerate}

\begin{proof}

Immediate from the typing rules.

\end{proof}

\end{lemma}


Well-typed Haskell and ML expressions have unique types.

\begin{lemma}{Uniqueness of Types}

\label{lemuni}

If \varexph, \varexpm, and \varexps are well-typed then they have only one type.

\begin{proof}

Straightforward structural induction on \varexph, \varexpm, and \varexps using the induction hypothesis and the \proinv.

\end{proof}

\end{lemma}


The types of Haskell and ML values determine their syntactic forms.

\begin{lemma}{Canonical Forms}

\label{lemcan}

The syntactic forms of \prouvs for each type.

\begin{enumerate}

% Haskell

% L

\item If \judeh{\env}{\varvalfh}{\tylump} then $\varvalfh \in \lbrace \exphm{\tylump}{\vartym}{\varvalfm}, \exphs{\cslump}{\varvalfs} \rbrace$.

% N

\item If \judeh{\env}{\varvalfh}{\tynum} then $\varvalfh = \expnum{\symnum}$.

% {t}

\item If \judeh{\env}{\varvalfh}{\tylist{\vartyh}} then $\varvalfh \in \lbrace \expnils{\vartyh}, \expcons{\first{\varexph}}{\second{\varexph}} \rbrace$.

% t->t

\item If \judeh{\env}{\varvalfh}{\tyfun{\first{\vartyh}}{\second{\vartyh}}} then $\varvalfh = \expfabss{\varvarh}{\first{\vartyh}}{\varexph}$.

% Au.t

\item If \judeh{\env}{\varvalfh}{\tyfor{\tyvarh}{\vartyh}} then $\varvalfh = \exptabs{\tyvarh}{\varexph}$.

% ML

% L

\item If \judem{\env}{\varvalfm}{\tylump} then $\varvalfm \in \lbrace \expmh{\tylump}{\vartym}{\varvalfm}, \expms{\cslump}{\varvalfs} \rbrace$.

% N

\item If \judem{\env}{\varvalfm}{\tynum} then $\varvalfm = \expnum{\symnum}$.

% {t}

\item If \judem{\env}{\varvalfm}{\tylist{\vartym}} then $\varvalfm \in \lbrace \expnils{\vartym}, \expcons{\first{\varvalum}}{\second{\varvalum}} \rbrace$.

% t->t

\item If \judem{\env}{\varvalfm}{\tyfun{\first{\vartym}}{\second{\vartym}}} then $\varvalfm = \expfabss{\varvarh}{\first{\vartym}}{\varexph}$.

% Au.t

\item If \judem{\env}{\varvalfm}{\tyfor{\tyvarm}{\vartym}} then $\varvalfm = \exptabs{\tyvarm}{\varexph}$.

\end{enumerate}

\begin{proof}

Immediate from the definitions of unforced values and the typing relations.

\end{proof}

\end{lemma}


\begin{theorem}{Haskell Progress}

\label{thmpsh}

If \judeh{}{\first{\varexph}}{\vartyh} then \pshyp{\first{\varexph}}{\second{\varexph}}.

\begin{proof}

By structural induction on \first{\varexph} and theorems \ref{thmpsm} and \ref{thmpss}.

% x

\newcommand{\psvar}{\varvarh\xspace}

\begin{case}{\psvar}

Cannot occur because \varexph is closed.

\end{case}

% \x:t.e

\newcommand{\psfabss}{\expfabss{\varvarh}{\vartyh}{\varexph}\xspace}

\begin{case}{\psfabss}

\psfabss is an \prouv.

\end{case}

% \\u.e

\newcommand{\pstabs}{\exptabs{\tyvarh}{\varexph}\xspace}

\begin{case}{\pstabs}

\pstabs is an \prouv.

\end{case}

% n

\newcommand{\psnum}{\expnum{\symnum}\xspace}

\begin{case}{\psnum}

\psnum is an \prouv.

\end{case}

% nil t

\newcommand{\psnils}{\expnils{\vartyh}\xspace}

\begin{case}{\psnils}

\psnils is an \prouv.

\end{case}

% cons e e

\newcommand{\psconsh}{\expcons{\first{\varexph}}{\second{\varexph}}\xspace}

\begin{case}{\psconsh}

\psconsh is an \prouv.

\end{case}

% e e

\newcommand{\psfapph}{\expfapp{\first{\varexph}}{\second{\varexph}}\xspace}
\renewcommand{\x}{\expfabss{\varvarh}{\first{\vartyh}}{\third{\varexph}}\xspace}

\begin{case}{\psfapph}

\pshypby
{\first{\varexph}}
{\third{\varexph}}
\psvalifeqh
{\first{\varexph}}
{\tyfun{\first{\vartyh}}{\second{\vartyh}}}
{\x}
\psred
{\expfapp{(\x)}{\second{\varexph}}}
{\expsubst{\third{\varexph}}{\second{\varexph}}{\varvarh}}
\pssub
{\first{\varexph}}
{\third{\varexph}}
{\psfapph}
{\expfapp{\third{\varexph}}{\second{\varexph}}}
\pserr
{\first{\varexph}}
{\psfapph}

\end{case}

% e<t>

\newcommand{\pstapp}{\exptapp{\first{\varexph}}{\first{\vartyh}}\xspace}
\renewcommand{\x}{\exptabs{\tyvarh}{\second{\varexph}}\xspace}

\begin{case}{\pstapp}

\pshypby
{\first{\varexph}}
{\second{\varexph}}
\psvalifeqh
{\first{\varexph}}
{\tyfor{\tyvarh}{\second{\vartyh}}}
{\x}
\psred
{\exptapp{(\x)}{\first{\vartyh}}}
{\expsubst{\second{\varexph}}{\first{\vartyh}}{\tyvarh}}
\pssub
{\first{\varexph}}
{\second{\varexph}}
{\pstapp}
{\exptapp{\second{\varexph}}{\first{\vartyh}}}
\pserr
{\first{\varexph}}
{\pstapp}

\end{case}

% fix e

\newcommand{\psfix}{\expfix{\first{\varexph}}\xspace}
\renewcommand{\x}{\expfabss{\varvarh}{\vartyh}{\second{\varexph}}\xspace}
\renewcommand{\y}{\expfix{(\x)}}

\begin{case}{\psfix}

\pshypby
{\first{\varexph}}
{\second{\varexph}}
\psvalifeqh
{\first{\varexph}}
{\tyfun{\vartyh}{\vartyh}}
{\x}
\psred
{\y}
{\expsubst{\second{\varexph}}{\y}{\varvarh}}
\pssub
{\first{\varexph}}
{\second{\varexph}}
{\psfix}
{\expfix{\second{\varexph}}}
\pserr
{\first{\varexph}}
{\psfix}

\end{case}

% o e e

\newcommand{\psop}{\expop{\first{\varexph}}{\second{\varexph}}\xspace}
\renewcommand{\x}{\first{\expnum{\varnum}}\xspace}
\renewcommand{\y}{\second{\expnum{\varnum}}\xspace}

\begin{case}{\psop}

\pshypby
{\first{\varexph}}
{\third{\varexph}}
\psvalifeqh
{\first{\varexph}}
{\tynum}
{\x}
\pssub
{\first{\varexph}}
{\third{\varexph}}
{\psop}
{\expop{\third{\varexph}}{\second{\varexph}}}
\pserr
{\first{\varexph}}
{\psop}
\pshypby
{\second{\varexph}}
{\third{\varexph}}
\psvalifeqh
{\second{\varexph}}
{\tynum}
{\y}
\pssuband
{\second{\varexph}}
{\third{\varexph}}
{\first{\varexph}}
{\psop}
{\expop{\first{\varexph}}{\third{\varexph}}}
\pserrand
{\second{\varexph}}
{\first{\varexph}}
{\psop}
\psred
{\expadd{\x}{\y}}
{\expnum{\first{\varnum} + \second{\varnum}}}
\psred
{\expsub{\x}{\y}}
{\expnum{\formvar{max}(\first{\varnum} - \second{\varnum}, 0)}}

\end{case}

% if0 e e e

\newcommand{\psif}{\expif{\first{\varexph}}{\second{\varexph}}{\third{\varexph}}\xspace}
\renewcommand{\x}{\expnum{\varnum}\xspace}

\begin{case}{\psif}

\pshypby
{\first{\varexph}}
{\fourth{\varexph}}
\psvalifeqh
{\first{\varexph}}
{\tynum}
{\x}
\psred
{\expif{\expnum{0}}{\second{\varexph}}{\third{\varexph}}}
{\second{\varexph}}
\psrednote
{\expif{\x}{\second{\varexph}}{\third{\varexph}}}
{\third{\varexph}}
{n \neq 0}
\pssub
{\first{\varexph}}
{\fourth{\varexph}}
{\psif}
{\expif{\fourth{\varexph}}{\second{\varexph}}{\third{\varexph}}}
\pserr
{\first{\varexph}}
{\psif}

\end{case}

% f e

\newcommand{\psfield}{\expfield{\first{\varexph}}\xspace}
\renewcommand{\x}{\expnils{\vartyh}\xspace}
\renewcommand{\y}{\expcons{\second{\varexph}}{\third{\varexph}}\xspace}

\begin{case}{\psfield}

\pshypby
{\first{\varexph}}
{\second{\varexph}}
\psvalifinh
{\first{\varexph}}
{\tylist{\vartyh}}
{\x, \y}
\psred
{\exphd{(\x)}}
{\expwrongs{\vartyh}{\errempty}}
\psred
{\exptl{(\x)}}
{\expwrongs{\tylist{\vartyh}}{\errempty}}
\psred
{\exphd{(\y)}}
{\second{\varexph}}
\psred
{\exptl{(\y)}}
{\third{\varexph}}
\pssub
{\first{\varexph}}
{\second{\varexph}}
{\psfield}
{\expfield{\second{\varexph}}}
\pserr
{\first{\varexph}}
{\psfield}

\end{case}

% null? e

\newcommand{\pspnull}{\exppnull{\first{\varexph}}\xspace}
\renewcommand{\x}{\expnils{\vartyh}\xspace}
\renewcommand{\y}{\expcons{\second{\varexph}}{\third{\varexph}}}

\begin{case}{\pspnull}

\pshypby
{\first{\varexph}}
{\second{\varexph}}
\psvalifinh
{\first{\varexph}}
{\tylist{\vartyh}}
{\x, \y}
\psred
{\exppnull{(\x)}}
{\expnum{0}}
\psred
{\exppnull{(\y)}}
{\expnum{1}}
\pssub
{\first{\varexph}}
{\second{\varexph}}
{\pspnull}
{\exppnull{\second{\varexph}}}
\pserr
{\first{\varexph}}
{\pspnull}

\end{case}

% wrong t string

\newcommand{\pswrongs}{\expwrongs{\vartyh}{\varstr}\xspace}

\begin{case}{\pswrongs}

\psred
{\pswrongs}
{\emph{\experr{\varstr}}}

\end{case}

% hm t t e

\newcommand{\pshm}{\exphm{\first{\vartyh}}{\first{\vartym}}{\first{\varexpm}}}

\begin{case}{\pshm}

\pshypby
{\first{\varexpm}}
{\second{\varexpm}}
\pssub
{\first{\varexpm}}
{\second{\varexpm}}
{\pshm}
{\exphm{\first{\vartyh}}{\first{\vartym}}{\second{\varexpm}}}
\pserr
{\first{\varexpm}}
{\pshm}
\pscasestwo
{\first{\varexpm}}
{\first{\vartyh}}
{\first{\vartym}}
{\pshm}

% L, *

\begin{subcase}{\first{\vartyh} $=$ \tylump}

\exphm{\tylump}{\first{\vartym}}{\first{\varexpm}} is an \prouv.

\end{subcase}

% !L, L

\begin{subcase}{\first{\vartyh} $\neq$ \tylump and \first{\vartym} $=$ \tylump}

\psvalinh
{\first{\varexpm}}
{\tylump}
{\expmh{\tylump}{\second{\vartyh}}{\varexph}, \expms{\cslump}{\varvalfs}}
\psrednote
{\exphm{\first{\vartyh}}{\tylump}{(\expmh{\tylump}{\second{\vartyh}}{\varexph})}}
{\varexph}
{\first{\vartyh} = \second{\vartyh}}
\psrednote
{\exphm{\first{\vartyh}}{\tylump}{(\expmh{\tylump}{\second{\vartyh}}{\varexph})}}
{\varexph}
{\first{\vartyh} \neq \second{\vartyh}}
\psred
{\exphm{\first{\vartyh}}{\tylump}{(\expms{\cslump}{\varvalfs})}}
{\expwrongs{\first{\vartyh}}{\errvalue}}

\end{subcase}

% N, N

\begin{subcase}{\first{\vartyh} $=$ \tynum and \first{\vartym} $=$ \tynum}

\psvaleqm
{\first{\varexpm}}
{\tynum}
{\expnum{\varnum}}
\psred
{\exphm{\tynum}{\tynum}{\expnum{\varnum}}}
{\expnum{\varnum}}

\end{subcase}

% {t}, {t}

\renewcommand{\w}{\tylist{\second{\vartyh}}}
\renewcommand{\x}{\tylist{\second\vartym}}
\renewcommand{\y}{\expnils{\third{\vartym}}}
\renewcommand{\z}{\expcons{\first{\varvalum}}{\second{\varvalum}}}

\begin{subcase}{\first{\vartyh} $=$ \w and \first{\vartym} $=$ \x}

\psvalinm
{\first{\varexpm}}
{\tylist{\third{\vartym}}}
{\y, \z}
\psred
{\exphm{\w}{\x}{(\y)}}
{\expnils{\second{\vartyh}}}
\psred
{\exphm{\w}{\x}{(\z)}}
{\expcons{(\exphm{\second{\vartyh}}{\second{\vartym}}{\first{\varvalum}})}{(\exphm{\w}{\x}{\second{\varvalum}})}}

\end{subcase}

% t->t, t->t

\renewcommand{\x}{\tyfun{\second{\vartyh}}{\third{\vartyh}}}
\renewcommand{\y}{\tyfun{\second{\vartym}}{\third{\vartym}}}
\renewcommand{\z}{\expfabss{\varvarm}{\fourth{\vartym}}{\second{\varexpm}}}

\begin{subcase}{\first{\vartyh} $=$ \x and \first{\vartym} $=$ \y}

\psvaleqm
{\first{\varexpm}}
{\tyfun{\fourth{\vartym}}{\fifth{\vartym}}}
{\z}
\psred
{\exphm{(\x)}{(\y)}{(\z)}}
{\expfabss{\varvarh}{\second{\vartyh}}{\exphm{\third{\vartyh}}{\third{\vartym}}{(\expfapp{(\z)}{(\expmh{\second{\vartym}}{\second{\vartyh}}{\varvarh})})}}}

\end{subcase}

% Au.t, Au.t

\renewcommand{\x}{\tyfor{\first{\tyvarh}}{\second{\vartyh}}}
\renewcommand{\y}{\tyfor{\first{\tyvarm}}{\second{\vartym}}}
\renewcommand{\z}{\exptabs{\second{\tyvarm}}{\second{\varexpm}}}

\begin{subcase}{\first{\vartyh} $=$ \x and \first{\vartym} $=$ \y}

\psvaleqm
{\first{\varexpm}}
{\tyfor{\second{\tyvarm}}{\third{\vartym}}}
{\z}
\psred
{\exphm{(\x)}{(\y)}{(\z)}}
{\exptabs{\first{\tyvarh}}{\exphm{\second{\vartyh}}{\tysubst{\second{\vartym}}{\tylump}{\first{\tyvarm}}}{\expsubst{\second{\varexpm}}{\tylump}{\second{\tyvarm}}}}}

\end{subcase}

\end{case}

% hs k e

\newcommand{\pshs}{\exphs{\first{\varcsh}}{\first{\varexps}}}

\begin{case}{\pshs}

\pshypby
{\first{\varexps}}
{\second{\varexps}}
\pssub
{\first{\varexps}}
{\second{\varexps}}
{\pshs}
{\exphs{\first{\varcsh}}{\second{\varexps}}}
\pserr
{\first{\varexps}}
{\pshs}
\pscasesone
{\first{\varexps}}
{\first{\varcsh}}
{\pshs}

% L

\begin{subcase}{\cslump}

\exphs{\cslump}{\first{\varexps}} is an \prouv.

\end{subcase}

% N

\begin{subcase}{\csnum}

\psred
{\exphs{\csnum}{\expnum{\varnum}}}
{{\expnum{\varnum}}}
\psrednote
{\exphs{\csnum}{\first{\varexps}}}
{\expwrongs{\tynum}{\errnum}}
{\first{\varexps} \neq \expnum{\varnum}}

\end{subcase}

% {k}

\renewcommand{\x}{\cslist{\second{\varcsh}}}

\begin{subcase}{\x}

\psred
{\exphs{\x}{\expnild}}
{\expnils{\tyunbrand{\second{\varcsh}}}}
\psred
{\exphs{\x}{(\expcons{\first{\varvalus}}{\second{\varvalus}})}}
{\expcons{(\exphs{\varcsh}{\first{\varvalus}})}{(\exphs{\x}{\second{\varvalus}})}}
\psrednote
{\exphs{\x}{\first{\varexps}}}
{\expwrongs{\tyunbrand{\x}}{\errlist}}
{\first{\varexps} \neq \expnild$ and $\first{\varexps} \neq \expcons{\first{\varvalus}}{\second{\varvalus}}}

\end{subcase}

\renewcommand{\x}{\csbrand{\varbrand}{\vartyh}}

\begin{subcase}{\x}

% b.t

\psred
{\exphs{(\x)}{(\expsh{(\x)}{\varexph})}}
{\varexph}
\psrednote
{\exphs{(\x)}{\first{\varexps}}}
{\expwrongs{\vartyh}{\errbrand}}
{\first{\varexps} \neq \expsh{(\x)}{\varexph}}

\end{subcase}

% k->k

\renewcommand{\x}{\csfun{\second{\varcsh}}{\third{\varcsh}}}

\begin{subcase}{\x}

\psred
{\exphs{(\x)}{(\expfabsd{\varvars}{\varexps})}}
{\expfabss{\varvarh}{\tyunbrand{\second{\varcsh}}}{\exphs{\third{\varcsh}}{(\expfapp{(\expfabsd{\varvars}{\varexps})}{(\expsh{\second{\varcsh}}{\varvarh})})}}}
\psrednote
{\exphs{(\x)}{\first{\first{\varexps}}}}
{\expwrongs{\tyunbrand{\x}}{\errfun}}
{\first{\varexps} \neq \expfabsd{\varvars}{\varexps}}

\end{subcase}

% Au.k

\renewcommand{\x}{\csfor{\csvarh}{\second{\varcsh}}}

\begin{subcase}{\x}

\psred
{\exphs{(\x)}{\first{\varexps}}}
{\exptabs{\tyvarh}{\exphs{\second{\varcsh}}{\first{\varexps}}}}

\end{subcase}

\end{case}

\end{proof}

\end{theorem}


\begin{theorem}{ML Progress}

\label{thmpsm}

If \judem{}{\first{\varexpm}}{\vartym} then \pshyp{\first{\varexpm}}{\second{\varexpm}}.

\begin{proof}

By structural induction on \first{\varexpm}.  Cases similar to Haskell cases are omitted.

% mh t t e

\newcommand{\psmh}{\expmh{\first{\vartym}}{\first{\vartyh}}{\first{\varexph}}\xspace}

\begin{case}{\psmh}

\pshypby
{\first{\varexph}}
{\second{\varexph}}
\pscasestwo
{\first{\varexph}}
{\first{\vartym}}
{\first{\vartyh}}
{\psmh}

% TODO:

TODO

\begin{subcase}{}

\end{subcase}

\pssub
{\first{\varexph}}
{\second{\varexph}}
{\psmh}
{\expmh{\first{\vartym}}{\first{\vartyh}}{\second{\varexph}}}
\pserr
{\first{\varexph}}
{\psmh}

\end{case}

% e e

\newcommand{\psfappm}{\expfapp{\first{\varexpm}}{\second{\varexpm}}}
\renewcommand{\x}{\expfabss{\varvarm}{\first{\vartym}}{\third{\varexpm}}}

\begin{case}{\psfappm}

\pshypby
{\first{\varexpm}}
{\third{\varexpm}}
\psvalifeqm
{\first{\varexpm}}
{\tyfun{\first{\vartym}}{\second{\vartym}}}
{\x}
\pssub
{\first{\varexpm}}
{\third{\varexpm}}
{\psfappm}
{\expfapp{\third{\varexpm}}{\second{\varexpm}}}
\pserr
{\first{\varexpm}}
{\psfappm}
\pshypby
{\second{\varexpm}}
{\third{\varexpm}}
\pssuband
{\second{\varexpm}}
{\third{\varexpm}}
{\first{\varexpm}}
{\psfappm}
{\expfapp{\first{\varexpm}}{\third{\varexpm}}}
\pserrand
{\second{\varexpm}}
{\first{\varexpm}}
{\psfappm}
\psredif
{\first{\varexpm}}
{\second{\varexpm}}
{\expfapp{(\x)}{\second{\varexpm}}}
{\expsubst{\third{\varexpm}}{\second{\varexpm}}{\varvarm}}

\end{case}

% cons e e

\newcommand{\psconsem}{\expcons{\first{\varexpm}}{\second{\varexpm}}\xspace}

\begin{case}{\psconsem}

\pshypby
{\first{\varexpm}}
{\third{\varexpm}}
\pssub
{\first{\varexpm}}
{\third{\varexpm}}
{\psconsem}
{\expcons{\third{\varexpm}}{\second{\varexpm}}}
\pserr
{\first{\varexpm}}
{\psconsem}
\pssuband
{\second{\varexpm}}
{\third{\varexpm}}
{\first{\varexpm}}
{\psconsem}
{\expcons{\third{\varexpm}}{\second{\varexpm}}}
\pserrand
{\second{\varexpm}}
{\first{\varexpm}}
{\psconsem}
\psvaliftwo
{\first{\varexpm}}
{\second{\varexpm}}
{\psconsem is an \prouv.}

\end{case}

\end{proof}

\end{theorem}


\begin{figure}[p]
\onehalfspacing
\centering
\begin{tabular}{l}

% hs N n

\redruleh
{\exphs{\csnum}{\expnum{\varnum}}}
{{\expnum{\varnum}}} \\

% hs N v

\redruleh
{\exphs{\csnum}{\varvalfs}}
{\expwrongs{\tynum}{\errnum}}
$(\varvalfs \neq \expnum{\varnum})$ \\

% hs {k} nil

\redruleh
{\exphs{\cslist{\varcsh}}{\expnild}}
{\expnils{\tyunbrand{\varcsh}}} \\

% hs {k} (cons v v)

\redruleh
{\exphs{\cslist{\varcsh}}{(\expcons{\first{\varvalus}}{\second{\varvalus}})}}
{\expcons{(\exphs{\varcsh}{\first{\varvalus}})}{(\exphs{\cslist{\varcsh}}{\second{\varvalus}})}} \\

% hs {k} v

\redruleh
{\exphs{\cslist{\varcsh}}{\varvalfs}}
{\expwrongs{\tyunbrand{\cslist{\varcsh}}}{\errlist}} \\

\redsp $(\varvalfs \neq \expnild$ and $\varvalfs \neq \expcons{\first{\varvalus}}{\second{\varvalus}})$ \\

% hs (b.t) (sh (b.t) e)

\redruleh
{\exphs{(\csbrand{\varbrand}{\vartyh})}{(\expsh{(\csbrand{\varbrand}{\vartyh})}{\varexph})}}
{\varexph} \\

% hs (b.t) v

\redruleh
{\exphs{(\csbrand{\varbrand}{\vartyh})}{\varvalfs}}
{\expwrongs{\vartyh}{\errbrand}}
$(\varvalfs \neq \expsh{(\csbrand{\varbrand}{\vartyh})}{\varexph})$ \\

% hs (k->k) (\x.e)

\redruleh
{\exphs{(\csfun{\first{\varcsh}}{\second{\varcsh}})}{(\expfabsd{\varvars}{\varexps})}}
{\expfabss{\varvarh}{\tyunbrand{\first{\varcsh}}}{\exphs{\second{\varcsh}}{(\expfapp{(\expfabsd{\varvars}{\varexps})}{(\expsh{\first{\varcsh}}{\varvarh})})}}} \\

% hs (k->k) v

\redruleh
{\exphs{(\csfun{\first{\varcsh}}{\second{\varcsh}})}{\varvalfs}}
{\expwrongs{\tyunbrand{\csfun{\first{\varcsh}}{\second{\varcsh}}}}{\errfun}} \\

\redsp $(\varvalfs \neq \expfabsd{\varvars}{\varexps})$ \\

% hs (Au.k) w

\redruleh
{\exphs{(\csfor{\csvarh}{\varcsh})}{\varvalfs}}
{\exptabs{\tyvarh}{\exphs{\varcsh}{\varvalfs}}} \\

\end{tabular}
\caption{Haskell-Scheme operational semantics}
\label{fighsos}
\end{figure}

\section{Proof of Type Preservation}

Preservation will be proven by cases on the reduction rules.  In each case, the right side will be proven to be well-typed and have the same type as the left side.  Lemmas \ref{leminv} and \ref{lemuni} determine the types of the left side and its subexpressions and the type of the right side.  Some reduction rules use expression and type substitutions.

If \first{\varexph} is substituted for free occurrences of \first{\varvarh} within \second{\varexph} and \first{\varexph} and \first{\varvarh} have the same type then the result has the same type as \second{\varexph}.

\begin{lemma}{Expression Substitution Preservation}

\label{lemexp}

If \judeh{\envexte{\env}{\first{\varvarh}}{\first{\vartyh}}}{\first{\varexph}}{\second{\vartyh}} and \judeh{\env}{\second{\varexph}}{\first{\vartyh}} then \judeh{\env}{\expsubst{\first{\varexph}}{\second{\varexph}}{\first{\varvarh}}}{\second{\vartyh}}.  If \judem{\envexte{\env}{\first{\varvarm}}{\first{\vartym}}}{\first{\varexpm}}{\second{\vartym}} and \judem{\env}{\second{\varexpm}}{\first{\vartym}} then \judem{\env}{\expsubst{\first{\varexpm}}{\second{\varexpm}}{\first{\varvarm}}}{\second{\vartym}}.  If \judes{\envexte{\env}{\first{\varvars}}{\tytst}}{\first{\varexps}}{\tytst} and \judes{\env}{\second{\varexps}}{\tytst} then \judes{\env}{\expsubst{\first{\varexps}}{\second{\varexps}}{\first{\varvars}}}{\tytst}.

\begin{proof}

By structural induction.

\end{proof}

\end{lemma}


If \first{\vartyh} is substituted for free occurrences of \first{\tyvarh} within \first{\varexph} of type \second{\vartyh}, the type of the result is \first{\vartyh} substituted for free occurrences of \tyvarh within \second{\vartyh}.

\begin{lemma}
\label{tes}
If $\Gamma,X\vdash_{HM}e_{HM}:T_{1}$ and $\vdash_{HM}T_{2}$ then $\Gamma\vdash_{HM}e_{HM}[T_{2}/X]:T_{1}[T_{2}/X]$.
\begin{proof}
By structural induction.
\end{proof}
\end{lemma}

\begin{lemma}{Evaluation Context Preservation}

\label{lemeva}

If \judeh{}{\first{\varexph}}{\first{\vartyh}}, \judeh{}{\second{\varexph}}{\first{\vartyh}}, and \judeh{}{\redconh{\first{\varexph}}}{\second{\vartyh}} then \judeh{}{\redconh{\second{\varexph}}}{\second{\vartyh}}.
If \judem{}{\first{\varexpm}}{\first{\vartym}}, \judem{}{\second{\varexpm}}{\first{\vartym}}, and \judem{}{\redconm{\first{\varexpm}}}{\second{\vartym}} then \judem{}{\redconm{\second{\varexpm}}}{\second{\vartym}}.
If \judes{}{\first{\varexps}}{\tytst}, \judes{}{\second{\varexps}}{\tytst}, and \judes{}{\redcons{\first{\varexps}}}{\tytst} then \judes{}{\redcons{\second{\varexps}}}{\tytst}.

\begin{proof}

By structural induction.

\end{proof}

\end{lemma}


\begin{theorem}{Haskell Preservation}

\label{thmpnh}

If \judeh{\env}{\first{\varexph}}{\first{\vartyh}} and \redruleh{\first{\varexph}}{\second{\varexph}} then \judeh{\env}{\second{\varexph}}{\first{\vartyh}}.

\begin{proof}

By cases on the reduction \redruleh{\first{\varexph}}{\second{\varexph}} and lemma \ref{lemeva}.

% e e

\begin{case}{\osfapph}

\pnpremise
{\judeh{\env}{\osfapplh}{\second{\vartyh}}}
\pntypes
{
\judeh{\env}{\expfabss{\first{\varvarh}}{\first{\vartyh}}{\first{\varexph}}}{\tyfun{\first{\vartyh}}{\second{\vartyh}}};
\judeh{\envexte{\env}{\first{\varvarh}}{\first{\vartyh}}}{\first{\varexph}}{\second{\vartyh}};
\judeh{\env}{\second{\varexph}}{\first{\vartyh}}; and
\judeh{\envexte{\env}{\first{\varvarh}}{\first{\vartyh}}}{\first{\varvarh}}{\first{\vartyh}}
}
\pnexpsubst
{\judeh{\env}{\osfapprh}{\second{\vartyh}}}

\end{case}

% e<t>

\begin{case}{\ostapp}

\pnpremise
{\judeh{\env}{\ostappl}{\second{\vartyh}}}
\pntypes
{
\judeh{\env}{\exptabs{\first{\tyvarh}}{\first{\varexph}}}{\tyfor{\first{\tyvarh}}{\third{\vartyh}}};
\judeh{\envextt{\env}{\first{\tyvarh}}}{\first{\varexph}}{\third{\vartyh}}; and
\second{\vartyh} $=$ \tysubst{\third{\vartyh}}{\first{\vartyh}}{\first{\tyvarh}}
}
\pntysubst
{\judeh{\env}{\ostappr}{\tysubst{\third{\vartyh}}{\first{\vartyh}}{\first{\tyvarh}}}}

\end{case}

% fix e

\begin{case}{\osfix}

$\mathtt{fix}$ $(\lambda x:T_{1}.e_{A})\rightarrow e_{A}[(\mathtt{fix}$ $(\lambda x:T_{1}.e_{A}))/x]$ where $A\in\lbrace H,M\rbrace$

$\Gamma\vdash_{A}\mathtt{fix}$ $(\lambda x:T_{1}.e_{A}):T$ by premise and uniqueness of types (Lemma \ref{uot}).  $\Gamma\vdash_{A}\lambda x:T_{1}.e_{A}:T_{1}\rightarrow T_{2}$, $\Gamma,x:T_{1}\vdash_{A}e_{A}:T_{1}$, $\Gamma,x:T_{1}\vdash_{A}x:T_{1}$, and $T=T_{1}$ by inversion (Lemma \ref{i}) and uniqueness of types.  $\Gamma\vdash_{A}e_{A}[(\mathtt{fix}$ $(\lambda x:T_{1}.e_{A}))/x]:T_{1}$ by term substitution (Lemma \ref{tms}).  $\Gamma\vdash_{A}e_{A}[(\mathtt{fix}$ $(\lambda x:T_{1}.e_{A}))/x]:T$ because $T_{1}=T$.
\end{case}

% + n n

\begin{case}{\osadd}
$+$ $\overline{n_{1}}$ $\overline{n_{2}}\rightarrow\overline{n_{1}+n_{2}}$ where $A\in\lbrace H,M\rbrace$

$\vdash_{A}+$ $\overline{n_{1}}$ $\overline{n_{2}}:T$ by premise and uniqueness of types (Lemma \ref{uot}).  $\vdash_{A}\overline{n_{1}}:N$, $\vdash_{A}\overline{n_{2}}:N$, $T=N$, and $\vdash_{A}\overline{n_{1}+n_{2}}:N$ by inversion (Lemma \ref{i}) and uniqueness of types.  $\vdash_{A}\overline{n_{1}+n_{2}}:T$ because $N=T$.
\end{case}

% - n n

\begin{case}{\ossub}
$-$ $\overline{n_{1}}$ $\overline{n_{2}}\rightarrow\overline{max(n_{1}-n_{2},0)}$ where $A\in\lbrace H,M\rbrace$

$\vdash_{A}-$ $\overline{n_{1}}$ $\overline{n_{2}}:T$ by premise and uniqueness of types (Lemma \ref{uot}).  $\vdash_{A}\overline{n_{1}}:N$, $\vdash_{A}\overline{n_{2}}:N$, $T=N$, and $\vdash_{A}\overline{max(n_{1}-n_{2},0)}:N$ by inversion (Lemma \ref{i}) and uniqueness of types.  $\vdash_{A}\overline{max(n_{1}-n_{2},0)}:T$ because $N=T$.
\end{case}

% if0 0 e e

\begin{case}{\osiftrue}

$\mathtt{if0}$ $\overline{0}$ $e_{A}^{1}$ $e_{A}^{2}\rightarrow e_{A}^{1}$ where $A\in\lbrace H,M\rbrace$

$\Gamma\vdash_{A}\mathtt{if0}$ $\overline{0}$ $e_{A}^{1}$ $e_{A}^{2}:T$ by premise and uniqueness of types (Lemma \ref{uot}).  $\Gamma\vdash_{A}e_{A}^{1}:T_{1}$ and $T=T_{1}$ by inversion (Lemma \ref{i}) and uniqueness of types.  $\Gamma\vdash_{A}e_{A}^{1}:T$ because $T_{1}=T$.
\end{case}

% if0 n e e

\begin{case}{\osiffalse}

$\mathtt{if0}$ $\overline{n}$ $e_{A}^{1}$ $e_{A}^{2}\rightarrow e_{A}^{2}$ $(n\neq0)$ where $A\in\lbrace H,M\rbrace$

$\Gamma\vdash_{A}\mathtt{if0}$ $\overline{n}$ $e_{A}^{1}$ $e_{A}^{2}:T$ by premise and uniqueness of types (Lemma \ref{uot}).  $\Gamma\vdash_{A}e_{A}^{2}:T_{1}$ and $T=T_{1}$ by inversion (Lemma \ref{i}) and uniqueness of types.  $\Gamma\vdash_{A}e_{A}^{2}:T$ because $T_{1}=T$.
\end{case}

% hd nil

\begin{case}{\oshdnil}

$\mathtt{hd}$ $\mathtt{nil}^{T_{1}}\rightarrow\mathtt{wrong}^{T_{1}}$ \emph{``Empty list"} where $A\in\lbrace H,M\rbrace$

$\Gamma\vdash_{A}\mathtt{hd}$ $\mathtt{nil}^{T_{1}}:T$ by premise and uniqueness of types (Lemma \ref{uot}).  $\Gamma\vdash_{A}\mathtt{nil}^{T_{1}}:[T_{1}]$, $T=T_{1}$, and $\Gamma\vdash_{A}\mathtt{wrong}^{T_{1}}$ $\mathrm{``Empty}$ $\mathrm{list"}:T_{1}$ by inversion (Lemma \ref{i}) and uniqueness of types.  $\Gamma\vdash_{A}\mathtt{wrong}^{T_{1}}$ $\mathrm{``Empty}$ $\mathrm{list"}:T$ because $T_{1}=T$.
\end{case}

% tl nil

\begin{case}{\ostlnil}

$\mathtt{tl}$ $\mathtt{nil}^{T_{1}}\rightarrow\mathtt{wrong}^{[T_{1}]}$ \emph{``Empty list"} where $A\in\lbrace H,M\rbrace$

$\Gamma\vdash_{A}\mathtt{tl}$ $\mathtt{nil}^{T_{1}}:T$ by premise and uniqueness of types (Lemma \ref{uot}).  $\Gamma\vdash_{A}\mathtt{nil}^{T_{1}}:[T_{1}]$, $T=[T_{1}]$, and $\Gamma\vdash_{A}\mathtt{wrong}^{[T_{1}]}$ $\mathrm{``Empty}$ $\mathrm{list"}:[T_{1}]$ by inversion (Lemma \ref{i}) and uniqueness of types.  $\Gamma\vdash_{A}\mathtt{wrong}^{[T_{1}]}$ $\mathrm{``Empty}$ $\mathrm{list"}:T$ because $[T_{1}]=T$.
\end{case}

% hd (cons e e)

\begin{case}{\oshdconsh}

$\mathtt{hd}$ $(\mathtt{cons}$ $e_{H}^{1}$ $e_{H}^{2})\rightarrow e_{H}^{1}$

$\Gamma\vdash_{H}\mathtt{hd}$ $(\mathtt{cons}$ $e_{H}^{1}$ $e_{H}^{2}):T$ by premise and uniqueness of types (Lemma \ref{uot}).  $\Gamma\vdash_{H}e_{H}^{1}:T_{1}$, $\Gamma\vdash_{H}\mathtt{cons}$ $e_{H}^{1}$ $e_{H}^{2}:[T_{1}]$, and $T=T_{1}$ by inversion (Lemma \ref{i}) and uniqueness of types.  $\Gamma\vdash_{H}e_{H}^{1}:T$ because $T_{1}=T$.
\end{case}

% tl (cons e e)

\begin{case}{\ostlconsh}

$\mathtt{tl}$ $(\mathtt{cons}$ $e_{H}^{1}$ $e_{H}^{2})\rightarrow e_{H}^{2}$

$\Gamma\vdash_{H}\mathtt{tl}$ $(\mathtt{cons}$ $e_{H}^{1}$ $e_{H}^{2}):T$ by premise and uniqueness of types (Lemma \ref{uot}).  $\Gamma\vdash_{H}e_{H}^{2}:[T_{1}]$, $\Gamma\vdash_{H}\mathtt{cons}$ $e_{H}^{1}$ $e_{H}^{2}:[T_{1}]$, and $T=[T_{1}]$ by inversion (Lemma \ref{i}) and uniqueness of types.  $\Gamma\vdash_{H}e_{H}^{2}:T$ because $[T_{1}]=T$.
\end{case}

% null nil

\begin{case}{\osnullnil}

$\mathtt{null?}$ $\mathtt{nil}^{T_{1}}\rightarrow\overline{0}$ where $A\in\lbrace H,M\rbrace$

$\vdash_{A}\mathtt{null?}$ $\mathtt{nil}^{T_{1}}:T$ by premise and uniqueness of types (Lemma \ref{uot}).  $T=N$ and $\vdash_{A}\overline{0}:N$ by inversion (Lemma \ref{i}) and uniqueness of types.  $\vdash_{A}\overline{0}:T$ because $N=T$.
\end{case}

% null (cons e e)

\begin{case}{\osnullconsh}

$\mathtt{null?}$ $(\mathtt{cons}$ $B_{A}^{1}$ $B_{A}^{2})\rightarrow\overline{1}$ where $(A,B)\in\lbrace(H,e),(M,v)\rbrace$

$\Gamma\vdash_{A}\mathtt{null?}$ $(\mathtt{cons}$ $B_{A}^{1}$ $B_{A}^{2}):T$ by premise and uniqueness of types (Lemma \ref{uot}).  $T=N$ and $\vdash_{A}\overline{1}:N$ by inversion (Lemma \ref{i}) and uniqueness of types.  $\vdash_{A}\overline{1}:T$ because $N=T$.
\end{case}

% wrong t string

\begin{case}{\oswrong}

$\mathtt{wrong}^{T_{1}}$ \emph{string} $\rightarrow$ \emph{\textbf{Error}: string}

Irrelevant because an error terminates the computation.
\end{case}

% hm t L (mh L t e)

\begin{case}{\oshmmhneqa \oshmmhneqb}

TEST

\end{case}

% hm mh lump

\begin{case}
$^{L}AB^{L}$ $(^{L}BS$ $v_{S})\rightarrow{^{L}A}S$ $v_{S}$ where $(A,B)\in\lbrace(H,M),(M,H)\rbrace$

$\Gamma\vdash_{A}{^{L}A}B^{L}$ $(^{L}BS$ $v_{S}):T$ by premise and uniqueness of types (Lemma \ref{uot}).  $\Gamma\vdash_{S}v_{S}:TST$ by inversion (Lemma \ref{i}).  $\Gamma\vdash_{B}{^{L}B}S$ $v_{S}:L$ and $T=L$ by inversion and uniqueness of types.  $\Gamma\vdash_{A}{^{L}A}S$ $v_{S}:L$ by inversion and uniqueness of types.  $\Gamma\vdash_{A}{^{L}A}S$ $v_{S}:T$ because $L=T$.
\end{case}

% hm mh number

\begin{case}
$^{N}AB^{N}$ $\overline{n}\rightarrow\overline{n}$ where $(A,B)\in\lbrace(H,M),(M,H)\rbrace$

$\vdash_{A}{^{N}A}B^{N}$ $\overline{n}:T$ by premise and uniqueness of types (Lemma \ref{uot}).  $\vdash_{A}\overline{n}:N$ and $T=N$ by inversion (Lemma \ref{i}) and uniqueness of types.
\end{case}

% hs ms number

\begin{case}
$^{N}AS$ $\overline{n}\rightarrow\overline{n}$ where $A\in\lbrace H,M\rbrace$

$\vdash_{A}{^{N}A}S$ $\overline{n}:T$ by premise and uniqueness of types (Lemma \ref{uot}).  $\vdash_{A}\overline{n}:N$ and $T=N$ by inversion (Lemma \ref{i}) and uniqueness of types.
\end{case}

% hs ms number error

\begin{case}
$^{N}AS$ $v_{S}\rightarrow{^{N}A}S$ $(\mathtt{wrong}$ $\mathrm{``Not}$ $\mathrm{a}$ $\mathrm{number"})$ $(v_{S}\neq\overline{n})$ where $A\in\lbrace H,M\rbrace$

$\Gamma\vdash_{A}{^{N}AS}$ $v_{S}:T$ by premise and uniqueness of types (Lemma \ref{uot}).  $T=N$ by inversion (Lemma \ref{i}) and uniqueness of types.  $\vdash_{S}\mathtt{wrong}$ $\mathrm{``Not}$ $\mathrm{a}$ $\mathrm{number"}:TST$ by inversion.  $\vdash_{A}{^{N}A}S$ $(\mathtt{wrong}$ $\mathrm{``Not}$ $\mathrm{a}$ $\mathrm{number"}):N$ by inversion and uniqueness of types.  $\vdash_{A}{^{N}A}S$ $(\mathtt{wrong}$ $\mathrm{``Not}$ $\mathrm{a}$ $\mathrm{number"}):T$ because $N=T$.
\end{case}

% hm mh list nil

\begin{case}
$^{[T_{1}]}AB^{[T_{1}]}$ $\mathtt{nil}^{T_{1}}\rightarrow\mathtt{nil}^{T_{1}}$ where $(A,B)\in\lbrace(H,M),(M,H)\rbrace$

$\Gamma\vdash_{A}{^{[T_{1}]}A}B^{[T_{1}]}$ $\mathtt{nil}^{T_{1}}:T$ by premise and uniqueness of types (Lemma \ref{uot}).  $\Gamma\vdash_{A}\mathtt{nil}^{T_{1}}:[T_{1}]$ and $T=[T_{1}]$ by inversion (Lemma \ref{i}) and uniqueness of types.  $\Gamma\vdash_{A}\mathtt{nil}^{T_{1}}:T$ because $[T_{1}]=T$.
\end{case}

% hs ms list nil

\begin{case}
$^{[T_{1}]}AS$ $\mathtt{nil}\rightarrow\mathtt{nil}^{T_{1}}$ where $A\in\lbrace H,M\rbrace$

$\Gamma\vdash_{A}{^{[T_{1}]}A}S$ $\mathtt{nil}:T$ by premise and uniqueness of types (Lemma \ref{uot}).  $T=[T_{1}]$ and $\Gamma\vdash_{A}\mathtt{nil}^{T_{1}}:[T_{1}]$ by inversion (Lemma \ref{i}) and uniqueness of types.  $\Gamma\vdash_{A}\mathtt{nil}^{T_{1}}:T$ because $[T_{1}]=T$.
\end{case}

% hm list cons

\begin{case}
$^{[T_{1}]}HM^{[T_{1}]}$ $(\mathtt{cons}$ $v_{M}^{1}$ $v_{M}^{2})\rightarrow\mathtt{cons}$ $(^{T_{1}}HM^{T_{1}}$ $v_{M}^{1})$ $(^{[T_{1}]}HM^{[T_{1}]}$ $v_{M}^{2})$

$^{[T_{1}]}HM^{[T_{1}]}$ $(\mathtt{cons}$ $v_{M}^{1}$ $v_{M}^{2}):T$ by premise and uniqueness of types (Lemma \ref{uot}).  $\Gamma\vdash_{M}v_{M}^{1}:T_{1}$, $\Gamma\vdash_{M}v_{M}^{2}:[T_{1}]$, $\Gamma\vdash_{M}\mathtt{cons}$ $v_{M}^{1}$ $v_{M}^{2}:[T_{1}]$, $T=[T_{1}]$, $\Gamma\vdash_{H}{^{T_{1}}H}M^{T_{1}}$ $v_{M}^{1}:T_{1}$, $\Gamma\vdash_{H}{^{[T_{1}]}H}M^{[T_{1}]}$ $v_{M}^{2}:[T_{1}]$, and $\Gamma\vdash_{H}\mathtt{cons}$ $(^{T_{1}}HM^{T_{1}}$ $v_{M}^{1})$ $(^{[T_{1}]}HM^{[T_{1}]}$ $v_{M}^{2}):[T_{1}]$ by inversion (Lemma \ref{i}) and uniqueness of types.  $\Gamma\vdash_{H}\mathtt{cons}$ $(^{T_{1}}HM^{T_{1}}$ $v_{M}^{1})$ $(^{[T_{1}]}HM^{[T_{1}]}$ $v_{M}^{2}):T$ because $[T_{1}]=T$.
\end{case}

% hs ms list cons

\begin{case}
$^{[T_{1}]}AS$ $(\mathtt{cons}$ $v_{S}^{1}$ $v_{S}^{2})\rightarrow\mathtt{cons}$ $(^{T_{1}}AS$ $v_{S}^{1})$ $(^{[T_{1}]}AS$ $v_{S}^{2})$ where $A\in\lbrace H,M\rbrace$

$\Gamma\vdash_{A}{^{[T_{1}]}A}S$ $(\mathtt{cons}$ $v_{S}^{1}$ $v_{S}^{2}):T$ by premise and uniqueness of types (Lemma \ref{uot}).  $\Gamma\vdash_{S}v_{S}^{1}:TST$, $\Gamma\vdash_{S}v_{S}^{2}:TST$, and $\Gamma\vdash_{S}\mathtt{cons}$ $v_{S}^{1}$ $v_{S}^{2}:TST$ by inversion (Lemma \ref{i}).  $T=[T_{1}]$, $\Gamma\vdash_{A}{^{T_{1}}A}S$ $v_{S}^{1}:T_{1}$, $\Gamma\vdash_{A}{^{[T_{1}]}A}S$ $v_{S}^{2}:[T_{1}]$, and $\Gamma\vdash_{A}\mathtt{cons}$ $(^{T_{1}}AS$ $v_{S}^{1})$ $(^{[T_{1}]}AS$ $v_{S}^{2}):[T_{1}]$ by inversion and uniqueness of types.  $\Gamma\vdash_{A}\mathtt{cons}$ $(^{T_{1}}AS$ $v_{S}^{1})$ $(^{[T_{1}]}AS$ $v_{S}^{2}):T$ because $[T_{1}]=T$.
\end{case}

% list mh cons

\begin{case}
$^{[T_{1}]}HM^{[T_{1}]}$ $(^{[T_{1}]}MH^{[T_{1}]}$ $(\mathtt{cons}$ $e_{H}^{1}$ $e_{H}^{2}))\rightarrow\mathtt{cons}$ $e_{H}^{1}$ $e_{H}^{2}$

$^{[T_{1}]}HM^{[T_{1}]}$ $(^{[T_{1}]}MH^{[T_{1}]}$ $(\mathtt{cons}$ $e_{H}^{1}$ $e_{H}^{2})):T$ by premise and uniqueness of types (Lemma \ref{uot}).  $\Gamma\vdash_{H}\mathtt{cons}$ $e_{H}^{1}$ $e_{H}^{2}:[T_{1}]$, $\Gamma\vdash_{H}{^{[T_{1}]}M}H^{[T_{1}]}$ $(\mathtt{cons}$ $e_{H}^{1}$ $e_{H}^{2}):[T_{1}]$, and $T=[T_{1}]$ by inversion (Lemma \ref{i}) and uniqueness of types.  $\Gamma\vdash_{H}\mathtt{cons}$ $e_{H}^{1}$ $e_{H}^{2}:T$ because $[T_{1}]=T$.
\end{case}

% list sh cons

\begin{case}
$^{[T_{1}]}HS$ $(SH^{[T_{1}]}$ $(\mathtt{cons}$ $e_{H}^{1}$ $e_{H}^{2}))\rightarrow\mathtt{cons}$ $e_{H}^{1}$ $e_{H}^{2}$

$^{[T_{1}]}HS$ $(SH^{[T_{1}]}$ $(\mathtt{cons}$ $e_{H}^{1}$ $e_{H}^{2})):T$ by premise and uniqueness of types (Lemma \ref{uot}).  $\Gamma\vdash_{H}\mathtt{cons}$ $e_{H}^{1}$ $e_{H}^{2}:[T_{1}]$ by inversion (Lemma \ref{i}) and uniqueness of types.  $\Gamma\vdash_{S}SH^{[T_{1}]}$ $(\mathtt{cons}$ $e_{H}^{1}$ $e_{H}^{2}):TST$ by inversion.  $T=[T_{1}]$ by inversion and uniqueness of types.  $\Gamma\vdash_{H}\mathtt{cons}$ $e_{H}^{1}$ $e_{H}^{2}:T$ because $[T_{1}]=T$.
\end{case}

% hs ms list error

\begin{case}
$^{[T_{1}]}AS$ $v_{S}^{1}\rightarrow{^{[T_{1}]}A}S$ $(\mathtt{wrong}$ $\mathrm{``Not}$ $\mathrm{a}$ $\mathrm{list"})$ $(v_{S}^{1}\neq\mathtt{cons}$ $v_{S}^{2}$ $v_{S}^{3}$ and $v_{S}^{1}\neq\mathtt{nil})$ where $A\in\lbrace H,M\rbrace$

$\Gamma\vdash_{A}{^{[T_{1}]}AS}$ $v_{S}^{1}:T$ by premise and uniqueness of types (Lemma \ref{uot}).  $T=[T_{1}]$ by inversion (Lemma \ref{i}) and uniqueness of types.  $\vdash_{S}\mathtt{wrong}$ $\mathrm{``Not}$ $\mathrm{a}$ $\mathrm{list"}:TST$ by inversion.  $\Gamma\vdash_{A}{^{[T_{1}]}A}S$ $(\mathtt{wrong}$ $\mathrm{``Not}$ $\mathrm{a}$ $\mathrm{list"}):[T_{1}]$ by inversion and uniqueness of types.  $\Gamma\vdash_{A}{^{[T_{1}]}A}S$ $(\mathtt{wrong}$ $\mathrm{``Not}$ $\mathrm{a}$ $\mathrm{list"}):T$ because $[T_{1}]=T$.
\end{case}

% hs ms label

\begin{case}
$^{T_{1}^{a}}AS$ $(SA^{T_{1}^{a}}$ $B_{A})\rightarrow B_{A}$ where $(A,B)\in\lbrace(H,e),(M,v)\rbrace$

$^{T_{1}^{a}}AS$ $(SA^{T_{1}^{a}}$ $B_{A}):T$ by premise and uniqueness of types (Lemma \ref{uot}).  $\Gamma\vdash_{A}B_{A}:T_{1}^{a}[T_{i}/T_{i}^{a}]$ by inversion (Lemma \ref{i}) and uniqueness of types.  $\Gamma\vdash_{S}SA^{T_{1}^{a}}$ $B_{A}:TST$ by inversion.  $T=T_{1}^{a}[T_{i}/T_{i}^{a}]$ by inversion and uniqueness of types.  $\Gamma\vdash_{A}B_{A}:T$ because $T_{1}^{a}[T_{i}/T_{i}^{a}]=T$.
\end{case}

% hs ms label error

\begin{case}
$^{T_{1}^{a}}AS$ $v_{S}\rightarrow{^{T_{1}^{a}}A}S$ $(\mathtt{wrong}$ $\mathrm{``Parametricity}$ $\mathrm{violated"})$ $(v_{S}\neq SA^{T_{1}^{a}}$ $B_{A})$ where $(A,B)\in\lbrace(H,e),(M,v)\rbrace$

$\Gamma\vdash_{A}{^{T_{1}^{a}}A}S$ $v_{S}:T$ by premise and uniqueness of types (Lemma \ref{uot}).  $T=T_{1}^{a}[T_{i}/T_{i}^{a}]$ by inversion (Lemma \ref{i}) and uniqueness of types.  $\vdash_{S}\mathtt{wrong}$ \emph{``Parametricity} $\mathrm{violated"}:TST$ by inversion.  $^{T_{1}^{a}}AS$ $(\mathtt{wrong}$ $\mathrm{``Parametricity}$ $\mathrm{violated"}):T_{1}^{a}[T_{i}/T_{i}^{a}]$ by inversion and uniqueness of types.  $^{T_{1}^{a}}AS$ $(\mathtt{wrong}$ $\mathrm{``Parametricity}$ $\mathrm{violated"}):T$ because $T_{1}^{a}[T_{i}/T_{i}^{a}]=T$.
\end{case}

% hm mh function

\begin{case}{\redrule{\oshmfabsa}{\oshmfabsb}}

$^{T_{1}\rightarrow T_{2}}AB^{T_{1}\rightarrow T_{2}}$ $(\lambda x_{1}:T_{1}.e_{B})\rightarrow\lambda x_{2}:T_{1}.(^{T_{2}}AB^{T_{2}}$ $((\lambda x_{1}:T_{1}.e_{B})$ $(^{T_{1}}BA^{T_{1}}$ $x_{2})))$ where $(A,B)\in\lbrace(H,M),(M,H)\rbrace$

$\Gamma\vdash_{A}{^{T_{1}\rightarrow T_{2}}}AB^{T_{1}\rightarrow T_{2}}$ $(\lambda x_{1}:T_{1}.e_{B}):T$ by premise and uniqueness of types (Lemma \ref{uot}).  $\Gamma\vdash_{B}\lambda x_{1}:T_{1}.e_{B}:T_{1}\rightarrow T_{2}$, $T=T_{1}\rightarrow T_{2}$, $\Gamma,x_{2}:T_{1}\vdash_{A}x_{2}:T_{1}$, $\Gamma,x_{2}:T_{1}\vdash_{B}{^{T_{1}}B}A^{T_{1}}$ $x_{2}:T_{1}$, $\Gamma,x_{2}:T_{1}\vdash_{B}(\lambda x_{1}:T_{1}.e_{B})$ $(^{T_{1}}BA^{T_{1}}$ $x_{2}):T_{2}$, $\Gamma,x_{2}:T_{1}\vdash_{A}{^{T_{2}}A}B^{T_{2}}$ $((\lambda x_{1}:T_{1}.e_{B})$ $(^{T_{1}}BA^{T_{1}}$ $x_{2})):T_{2}$, and $\Gamma\vdash_{A}\lambda x_{2}:T_{1}.(^{T_{2}}AB^{T_{2}}$ $((\lambda x_{1}:T_{1}.e_{B})$ $(^{T_{1}}BA^{T_{1}}$ $x_{2}))):T_{1}\rightarrow T_{2}$ by inversion (Lemma \ref{i}) and uniqueness of types.  $\Gamma\vdash_{A}\lambda x_{2}:T_{1}.(^{T_{2}}AB^{T_{2}}$ $((\lambda x_{1}:T_{1}.e_{B})$ $(^{T_{1}}BA^{T_{1}}$ $x_{2}))):T$ because $T_{1}\rightarrow T_{2}=T$.
\end{case}

% hs ms function

\begin{case}
$^{T_{1}\rightarrow T_{2}}AS$ $(\lambda x_{1}.e_{S})\rightarrow\lambda x_{2}:T_{1}[T_{i}/T_{i}^{a}].(^{T_{2}}AS$ $((\lambda x_{1}.e_{S})$ $(SA^{T_{1}}$ $x_{2})))$ where $A\in\lbrace H,M\rbrace$

$\Gamma\vdash_{A}{^{T_{1}\rightarrow T_{2}}A}S$ $(\lambda x_{1}.e_{S}):T$ by premise and uniqueness of types (Lemma \ref{uot}).  $\Gamma\vdash_{S}\lambda x_{1}.e_{S}:TST$ by inversion (Lemma \ref{i}).  $T=(T_{1}\rightarrow T_{2})[T_{i}/T_{i}^{a}]$ by inversion and uniqueness of types.  $\Gamma,x_{2}:T_{1}[T_{i}/T_{i}^{a}]\vdash_{A}x_{2}:T_{1}[T_{i}/T_{i}^{a}]$ by inversion and uniqueness of types.  $\Gamma,x_{2}:T_{1}[T_{i}/T_{i}^{a}]\vdash_{S}SA^{T_{1}}$ $x_{2}:TST$ and $\Gamma,x_{2}:T_{1}[T_{i}/T_{i}^{a}]\vdash_{S}(\lambda x_{1}.e_{S})$ $(SA^{T_{1}}$ $x_{2}):TST$ by inversion.  $\Gamma,x_{2}:T_{1}[T_{i}/T_{i}^{a}]\vdash_{A}{^{T_{2}}A}S$ $((\lambda x_{1}.e_{S})$ $(SA^{T_{1}}$ $x_{2})):T_{2}[T_{i}/T_{i}^{a}]$ and $\Gamma\vdash_{A}\lambda x_{2}:T_{1}[T_{i}/T_{i}^{a}].(^{T_{2}}AS$ $((\lambda x_{1}.e_{S})$ $(SA^{T_{1}}$ $x_{2}))):T_{1}[T_{i}/T_{i}^{a}]\rightarrow T_{2}[T_{i}/T_{i}^{a}]$ by inversion and uniqueness of types.  $\Gamma\vdash_{A}\lambda x_{2}:T_{1}[T_{i}/T_{i}^{a}].(^{T_{2}}AS$ $((\lambda x_{1}.e_{S})$ $(SA^{T_{1}}$ $x_{2}))):T$ because $T_{1}[T_{i}/T_{i}^{a}]\rightarrow T_{2}[T_{i}/T_{i}^{a}]=(T_{1}\rightarrow T_{2})[T_{i}/T_{i}^{a}]=T$.
\end{case}

% function error

\begin{case}
$^{T_{1}\rightarrow T_{2}}AS$ $v_{S}\rightarrow{^{T_{1}\rightarrow T_{2}}A}S$ $(\mathtt{wrong}$ $\mathrm{``Not}$ $\mathrm{a}$ $\mathrm{function"})$ $(v_{S}\neq\lambda x.e_{S})$ where $A\in\lbrace H,M\rbrace$

$\Gamma\vdash_{A}{^{T_{1}\rightarrow T_{2}}A}S$ $v_{S}:T$ by premise and uniqueness of types (Lemma \ref{uot}).  $T=(T_{1}\rightarrow T_{2})[T_{i}/T_{i}^{a}]$ by inversion (Lemma \ref{i}) and uniqueness of types.  $\vdash_{S}\mathtt{wrong}$ $\mathrm{``Not}$ $\mathrm{a}$ $\mathrm{function"}:TST$ by inversion.  $\Gamma\vdash_{A}{^{T_{1}\rightarrow T_{2}}A}S$ $(\mathtt{wrong}$ $\mathrm{``Not}$ $\mathrm{a}$ $\mathrm{function"}):(T_{1}\rightarrow T_{2})[T_{i}/T_{i}^{a}]$ by inversion and uniqueness of types.  $\Gamma\vdash_{A}{^{T_{1}\rightarrow T_{2}}A}S$ $(\mathtt{wrong}$ $\mathrm{``Not}$ $\mathrm{a}$ $\mathrm{function"}):T$ because $(T_{1}\rightarrow T_{2})[T_{i}/T_{i}^{a}]=T$.
\end{case}

% hm mh universal

\begin{case}
$^{\forall X.T_{1}}AB^{\forall X_{1}.T_{1}}$ $(\Lambda X.e_{B})\rightarrow\Lambda X.(^{T_{1}}AB^{T_{1}}$ $e_{B})$ where $(A,B)\in\lbrace(H,M),$ $(M,H)\rbrace$

$\Gamma\vdash_{A}{^{\forall X.T_{1}}A}B^{\forall X.T_{1}}$ $(\Lambda X_{1}.e_{B}):T$ by premise and uniqueness of types (Lemma \ref{uot}).  $\Gamma,X\vdash_{B}e_{B}:T_{1}$, $\Gamma\vdash_{B}\Lambda X.e_{B}:\forall X.T_{1}$, $T=\forall X.T_{1}$, $\Gamma,X\vdash_{A}{^{T_{1}}A}B^{T_{1}}$ $e_{B}:T_{1}$, and $\Gamma\vdash_{A}\Lambda X.(^{T_{1}}AB^{T_{1}}$ $e_{B}):\forall X.T_{1}$ by inversion (Lemma \ref{i}) and uniqueness of types.  $\Gamma\vdash_{A}\Lambda X.(^{T_{1}}AB^{T_{1}}$ $e_{B}):T$ because $\forall X.T_{1}=T$.
\end{case}

% universal ms hs

\begin{case}
$^{\forall X.T_{1}}AB^{\forall X.T_{1}}$ $(^{\forall X.T_{1}}BS$ $v_{S})\rightarrow{^{\forall X.T_{1}}A}S$ $v_{S}$ where $(A,B)\in\lbrace(H,M),$ $(M,H)\rbrace$

$\Gamma\vdash_{A}{^{\forall X.T_{1}}A}B^{\forall X.T_{1}}$ $(^{\forall X.T_{1}}BS$ $v_{S}):T$ by premise and uniqueness of types (Lemma \ref{uot}).  $\Gamma\vdash_{S}v_{S}:TST$ by inversion (Lemma \ref{i}).  $\Gamma\vdash_{B}{^{\forall X.T_{1}}B}S$ $v_{S}:\forall X.T_{1}$, $T=\forall X.T_{1}$, and $\Gamma\vdash_{A}{^{\forall X.T_{1}}A}S$ $v_{S}:\forall X.T_{1}$ by inversion and uniqueness of types.  $\Gamma\vdash_{A}{^{\forall X.T_{1}}A}S$ $v_{S}:T$ because $\forall X.T_{1}=T$.
\end{case}

% hs ms universal

\begin{case}
$(^{\forall X.T_{1}}AS$ $v_{S})$ $\lbrace T_{2}\rbrace\rightarrow{^{T_{1}[T_{2}^{a}/X]}A}S$ $v_{S}$ where $A\in\lbrace H,M\rbrace$

$\Gamma\vdash_{A}(^{\forall X.T_{1}}AS$ $v_{S})$ $\lbrace T_{2}\rbrace:T$ by premise and uniqueness of types (Lemma \ref{uot}).  $\Gamma\vdash_{S}v_{S}:TST$ by inversion (Lemma \ref{i}).  $\Gamma\vdash_{A}{^{\forall X.T_{1}}A}S$ $v_{S}:\forall X.T_{1}$ and $T=T_{1}[T_{2}/X]$ by inversion and uniqueness of types.  $\Gamma\vdash_{A}{^{T_{1}[T_{2}^{a}/X]}A}S$ $v_{S}:T_{1}[T_{2}^{a}/X][T_{i}/T_{i}^{a}]$ by inversion and uniqueness of types.  $\Gamma\vdash_{A}{^{T_{1}[T_{2}^{a}/X]}A}S$ $v_{S}:T$ because $T_{1}[T_{2}^{a}/X][T_{i}/T_{i}^{a}]=T_{1}[T_{2}/X]=T$.
\end{case}

\end{proof}

\end{theorem}


\begin{theorem}{ML Preservation}

\label{thmpnm}

If $\Gamma\vdash_{A}e_{A}^{1}:T$ and $e_{A}^{1}\rightarrow e_{A}^{2}$ then $\Gamma\vdash_{A}e_{A}^{2}:T$ where $A\in\lbrace H,M\rbrace$.  If $\Gamma\vdash_{S}e_{S}^{1}:TST$ and $e_{S}^{1}\rightarrow e_{S}^{2}$ then $\Gamma\vdash_{S}e_{S}^{2}:TST$.

\begin{proof}

By cases on the reductions $e_{A}^{1}\rightarrow e_{A}^{2}$ and $e_{S}^{1}\rightarrow e_{S}^{2}$ and evaluation context preservation (Lemma \ref{ec}).  Straightforward cases of Scheme preservation are elided.

% m head cons

\begin{case}
$\mathtt{hd}$ $(\mathtt{cons}$ $v_{M}^{1}$ $v_{M}^{2})\rightarrow v_{M}^{1}$

$\Gamma\vdash_{M}\mathtt{hd}$ $(\mathtt{cons}$ $v_{M}^{1}$ $v_{M}^{2}):T$ by premise and uniqueness of types (Lemma \ref{uot}).  $\Gamma\vdash_{M}v_{M}^{1}:T_{1}$, $\Gamma\vdash_{M}\mathtt{cons}$ $v_{M}^{1}$ $v_{M}^{2}:[T_{1}]$, and $T=T_{1}$ by inversion (Lemma \ref{i}) and uniqueness of types.  $\Gamma\vdash_{M}v_{M}^{1}:T$ because $T_{1}=T$.
\end{case}

% m tail cons

\begin{case}
$\mathtt{tl}$ $(\mathtt{cons}$ $v_{M}^{1}$ $v_{M}^{2})\rightarrow v_{M}^{2}$

$\Gamma\vdash_{M}\mathtt{tl}$ $(\mathtt{cons}$ $v_{M}^{1}$ $v_{M}^{2}):T$ by premise and uniqueness of types (Lemma \ref{uot}).  $\Gamma\vdash_{M}v_{M}^{2}:[T_{1}]$, $\Gamma\vdash_{M}\mathtt{cons}$ $v_{M}^{1}$ $v_{M}^{2}:[T_{1}]$, and $T=[T_{1}]$ by inversion (Lemma \ref{i}) and uniqueness of types.  $\Gamma\vdash_{M}v_{M}^{2}:T$ because $[T_{1}]=T$.
\end{case}

% mh list head cons

\begin{case}
$\mathtt{hd}$ $(^{[T_{1}]}MH^{[T_{1}]}$ $(\mathtt{cons}$ $e_{H}^{1}$ $e_{H}^{2}))\rightarrow{^{T_{1}}M}H^{T_{1}}$ $e_{H}^{1}$

$\Gamma\vdash_{M}\mathtt{hd}$ $(^{[T_{1}]}MH^{[T_{1}]}$ $(\mathtt{cons}$ $e_{H}^{1}$ $e_{H}^{2})):T$ by premise and uniqueness of types (Lemma \ref{uot}).  $\Gamma\vdash_{H}e_{H}^{1}:T_{1}$, $\Gamma\vdash_{H}\mathtt{cons}$ $e_{H}^{1}$ $e_{H}^{2}:[T_{1}]$, $\Gamma\vdash_{M}{^{[T_{1}]}M}H^{[T_{1}]}$ $(\mathtt{cons}$ $e_{H}^{1}$ $e_{H}^{2}):[T_{1}]$, $T=T_{1}$, and $^{T_{1}}MH^{T_{1}}$ $e_{H}^{1}:T_{1}$ by inversion (Lemma \ref{i}) and uniqueness of types.  $^{T_{1}}MH^{T_{1}}$ $e_{H}^{1}:T$ because $T_{1}=T$.
\end{case}

% mh list tail cons

\begin{case}
$\mathtt{tl}$ $(^{[T_{1}]}MH^{[T_{1}]}$ $(\mathtt{cons}$ $e_{H}^{1}$ $e_{H}^{2}))\rightarrow{^{[T_{1}]}M}H^{[T_{1}]}$ $e_{H}^{2}$

$\Gamma\vdash_{M}\mathtt{tl}$ $(^{[T_{1}]}MH^{[T_{1}]}$ $(\mathtt{cons}$ $e_{H}^{1}$ $e_{H}^{2})):T$ by premise and uniqueness of types (Lemma \ref{uot}).  $\Gamma\vdash_{H}e_{H}^{2}:[T_{1}]$, $\Gamma\vdash_{H}\mathtt{cons}$ $e_{H}^{1}$ $e_{H}^{2}:[T_{1}]$, $\Gamma\vdash_{M}{^{[T_{1}]}M}H^{[T_{1}]}$ $(\mathtt{cons}$ $e_{H}^{1}$ $e_{H}^{2}):[T_{1}]$, $T=[T_{1}]$, and $^{[T_{1}]}MH^{[T_{1}]}$ $e_{H}^{2}:[T_{1}]$ by inversion (Lemma \ref{i}) and uniqueness of types.  $^{[T_{1}]}MH^{[T_{1}]}$ $e_{H}^{2}:T$ because $[T_{1}]=T$.
\end{case}

% m null mh cons

\begin{case}
$\mathtt{null?}$ $(^{[T]}MH^{[T]}$ $(\mathtt{cons}$ $e_{H}^{1}$ $e_{H}^{2}))\rightarrow\overline{1}$

$\Gamma\vdash_{M}\mathtt{null?}$ $(^{[T]}MH^{[T]}$ $(\mathtt{cons}$ $e_{H}^{1}$ $e_{H}^{2})):T$ by premise and uniqueness of types (Lemma \ref{uot}).  $T=N$ and $\vdash_{A}\overline{1}:N$ by inversion (Lemma \ref{i}) and uniqueness of types.  $\vdash_{A}\overline{1}:T$ because $N=T$.
\end{case}

% ms list sh cons

\begin{case}
$^{[T_{1}]}MS$ $(SH^{[T_{1}]}$ $(\mathtt{cons}$ $e_{H}^{1}$ $e_{H}^{2}))\rightarrow{^{[T_{1}]}M}H^{[T_{1}]}$ $(\mathtt{cons}$ $e_{H}^{1}$ $e_{H}^{2})$

$^{[T_{1}]}MS$ $(SH^{[T_{1}]}$ $(\mathtt{cons}$ $e_{H}^{1}$ $e_{H}^{2})):T$ by premise and uniqueness of types (Lemma \ref{i}).  $\Gamma\vdash_{H}\mathtt{cons}$ $e_{H}^{1}$ $e_{H}^{2}:[T_{1}]$ by inversion (Lemma \ref{i}) and uniqueness of types.  $\Gamma\vdash_{S}SH^{[T_{1}]}$ $(\mathtt{cons}$ $e_{H}^{1}$ $e_{H}^{2}):TST$ by inversion.  $T=[T_{1}]$ and $\Gamma\vdash_{M}{^{[T_{1}]}M}H^{[T_{1}]}$ $(\mathtt{cons}$ $e_{H}^{1}$ $e_{H}^{2}):[T_{1}]$ by inversion and uniqueness of types.  $\Gamma\vdash_{M}{^{[T_{1}]}M}H^{[T_{1}]}$ $(\mathtt{cons}$ $e_{H}^{1}$ $e_{H}^{2}):T$ because $[T_{1}]=T$.
\end{case}

\end{proof}

\end{theorem}


\begin{theorem}{Scheme Preservation}

\label{thmpns}

If \judes{\env}{\first{\varexps}}{\tytst} and \first{\varexps} \red \second{\varexps} then \judes{\env}{\second{\varexps}}{\tytst}.

\begin{proof}

By cases on the reductions $e_{A}^{1}\rightarrow e_{A}^{2}$ and $e_{S}^{1}\rightarrow e_{S}^{2}$ and evaluation context preservation (Lemma \ref{ec}).  Straightforward cases of Scheme preservation are elided.

% sh list head cons

\begin{case}
$\mathtt{hd}$ $(SH^{[T_{1}]}$ $(\mathtt{cons}$ $e_{H}^{1}$ $e_{H}^{2}))\rightarrow SH^{T_{1}}$ $e_{H}^{1}$

$\Gamma\vdash_{S}\mathtt{hd}$ $(SH^{[T_{1}]}$ $(\mathtt{cons}$ $e_{H}^{1}$ $e_{H}^{2})):TST$ by premise.  $\Gamma\vdash_{H}e_{H}^{1}:T_{1}$ and $\Gamma\vdash_{H}\mathtt{cons}$ $e_{H}^{1}$ $e_{H}^{2}:[T_{1}]$ by inversion (Lemma \ref{i}) and uniqueness of types (Lemma \ref{uot}).  $\Gamma\vdash_{S}SH^{[T_{1}]}$ $(\mathtt{cons}$ $e_{H}^{1}$ $e_{H}^{2}):TST$ and $\Gamma\vdash_{S}SH^{T_{1}}$ $e_{H}^{1}:TST$ by inversion.
\end{case}

% sh list tail cons

\begin{case}
$\mathtt{tl}$ $(SH^{[T_{1}]}$ $(\mathtt{cons}$ $e_{H}^{1}$ $e_{H}^{2}))\rightarrow SH^{[T_{1}]}$ $e_{H}^{2}$

$\Gamma\vdash_{S}\mathtt{tl}$ $(SH^{[T_{1}]}$ $(\mathtt{cons}$ $e_{H}^{1}$ $e_{H}^{2})):TST$ by premise.  $\Gamma\vdash_{H}e_{H}^{2}:[T_{1}]$ and $\Gamma\vdash_{H}\mathtt{cons}$ $e_{H}^{1}$ $e_{H}^{2}:[T_{1}]$ by inversion (Lemma \ref{i}) and uniqueness of types (Lemma \ref{uot}).  $\Gamma\vdash_{S}SH^{[T_{1}]}$ $(\mathtt{cons}$ $e_{H}^{1}$ $e_{H}^{2}):TST$ and $\Gamma\vdash_{S}SH^{[T_{1}]}$ $e_{H}^{2}:TST$ by inversion.
\end{case}

% sm list cons

\begin{case}
$SM^{[T_{1}]}$ $(\mathtt{cons}$ $v_{M}^{1}$ $v_{M}^{2})\rightarrow\mathtt{cons}$ $(SM^{T_{1}}$ $v_{M}^{1})$ $(SM^{[T_{1}]}$ $v_{M}^{2})$

$\Gamma\vdash_{S}SM^{[T_{1}]}$ $(\mathtt{cons}$ $v_{M}^{1}$ $v_{M}^{2}):TST$ by premise.  $\Gamma\vdash_{M}v_{M}^{1}:T_{1}$, $\Gamma\vdash_{M}v_{M}^{2}:[T_{1}]$, and $\Gamma\vdash_{M}\mathtt{cons}$ $v_{M}^{1}$ $v_{M}^{2}:[T_{1}]$ by inversion (Lemma \ref{i}) and uniqueness of types (Lemma \ref{uot}).  $\Gamma\vdash_{S}SM^{T_{1}}$ $v_{M}^{1}:TST$, $\Gamma\vdash_{S}SM^{[T_{1}]}$ $v_{M}^{2}:TST$, and $\Gamma\vdash_{S}\mathtt{cons}$ $(SM^{T_{1}}$ $v_{M}^{1})$ $(SM^{[T_{1}]}$ $v_{M}^{2}):TST$ by inversion.
\end{case}

% sm list mh cons

\begin{case}
$SM^{[T_{1}]}$ $(^{[T_{1}]}MH^{[T_{1}]}$ $(\mathtt{cons}$ $e_{H}^{1}$ $e_{H}^{2}))\rightarrow SH^{[T_{1}]}$ $(\mathtt{cons}$ $e_{H}^{1}$ $e_{H}^{2})$

$SM^{[T_{1}]}$ $(^{[T_{1}]}MH^{[T_{1}]}$ $(\mathtt{cons}$ $e_{H}^{1}$ $e_{H}^{2})):TST$ by premise.  $\Gamma\vdash_{H}\mathtt{cons}$ $e_{H}^{1}$ $e_{H}^{2}:[T_{1}]$ and $\Gamma\vdash_{M}{^{[T_{1}]}M}H^{[T_{1}]}$ $(\mathtt{cons}$ $e_{H}^{1}$ $e_{H}^{2}):[T_{1}]$ by inversion (Lemma \ref{i}) and uniqueness of types (Lemma \ref{uot}).  $\Gamma\vdash_{S}SH^{[T_{1}]}$ $(\mathtt{cons}$ $e_{H}^{1}$ $e_{H}^{2}):TST$ by inversion.
\end{case}

% sh sm function

\begin{case}
$SA^{T_{1}\rightarrow T_{2}}$ $(\lambda x_{1}:T_{1}[T_{i}/T_{i}^{a}].e_{A})\rightarrow\lambda x_{2}.(SA^{T_{2}}$ $((\lambda x_{1}:T_{1}[T_{i}/T_{i}^{a}].e_{A})$ $(^{T_{1}}AS$ $x_{2})))$ where $A\in\lbrace H,M\rbrace$

$\Gamma\vdash_{S}SA^{T_{1}\rightarrow T_{2}}$ $(\lambda x_{1}:T_{1}[T_{i}/T_{i}^{a}].e_{A}):TST$ by premise.  $\Gamma\vdash_{A}\lambda x_{1}:T_{1}[T_{i}/T_{i}^{a}].e_{A}$ $:T_{1}[T_{i}/T_{i}^{a}]\rightarrow T_{2}[T_{i}/T_{i}^{a}]$ by inversion (Lemma \ref{i}) and uniqueness of types (Lemma \ref{uot}).  $\Gamma,x_{2}:TST\vdash_{S}x_{2}:TST$ by inversion.  $\Gamma,x_{2}:TST\vdash_{A}{^{T_{1}}A}S$ $x_{2}:T_{1}[T_{i}/T_{i}^{a}]$ and $\Gamma,x_{2}:TST\vdash_{A}(\lambda x_{1}:T_{1}[T_{i}/T_{i}^{a}].e_{A})$ $(^{T_{1}}AS$ $x_{2}):T_{2}[T_{i}/T_{i}^{a}]$ by inversion and uniqueness of types.  $\Gamma,x_{2}:TST\vdash_{S}SA^{T_{2}}$ $((\lambda x_{1}:T_{1}[T_{i}/T_{i}^{a}].e_{A})$ $(^{T_{1}}AS$ $x_{2})):TST$ and $\Gamma\vdash_{S}\lambda x_{2}.(SA^{T_{2}}$ $((\lambda x_{1}:T_{1}[T_{i}/T_{i}^{a}].e_{A})$ $(^{T_{1}}AS$ $x_{2}))):TST$ by inversion.
\end{case}

% sh sm universal

\begin{case}
$SA^{\forall X.T_{1}}$ $(\Lambda X.e_{A})\rightarrow SA^{T_{1}[L/X]}$ $((\Lambda X.e_{A})$ $\lbrace L\rbrace)$ where $A\in\lbrace H,M\rbrace$

$\Gamma\vdash_{S}SA^{\forall X.T_{1}}$ $(\Lambda X.e_{A}):TST$ by premise.  $\Gamma\vdash_{A}\Lambda X.e_{A}:\forall X.T_{1}$ and $\Gamma\vdash_{A}(\Lambda X.e_{A})$ $\lbrace L\rbrace:T_{1}[L/X]$ by inversion (Lemma \ref{i}) and uniqueness of types (Lemma \ref{uot}).  $\Gamma\vdash_{S}SA^{T_{1}[L/X]}$ $((\Lambda X.e_{A})$ $\lbrace L\rbrace):TST$ by inversion.
\end{case}

\end{proof}

\end{theorem}

