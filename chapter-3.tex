\chapter{Proof of Type Soundness}

The type soundness of the model of computation must be proven for it to be useful.  Type soundness is ensured if progress of expressions and preservation of types are ensured.  Progress ensures that a well-typed, closed expression is either a value, reducible to another expression, or reducible to an error.  Preservation ensures that if a well-typed expression reduces to another expression, the other expression is well-typed and has the same type.  Proving progress and preservation proves type soundness.  The proof extends the proof by Kinghorn \cite{kinghorn07}, which was based on proofs by Pierce \cite{pierce02} and Matthews and Findler \cite{matthews07}.

\section{Progress}

Progress will be proven by structural induction on a well-typed, closed expression of each syntactic form.  In each case, the expression will be proven to be either a value, reducible to another expression, or reducible to an error.  Reductions of subexpressions are reductions of the top expression.  If a subexpression reduces to an error, the top expression reduces to the error.  In some cases, the syntactic forms of subexpressions must be determined to reduce the top expression.  If unique types for those subexpressions can be determined, they can be used to determine their syntactic forms.

Inverting the typing relations enables the syntactic forms of well-typed expressions to determine the types of their subexpressions:

\begin{lemma}

\label{leminv}

The syntactic forms of well-typed expressions determine the types of their subexpressions.

\begin{enumerate}

% Haskell

% \x:t.e

\item If \judeh{\env}{\expfabss{\varvarh}{\first{\vartyh}}{\varexph}}{\second{\vartyh}} then $\second{\vartyh} = \tyfun{\first{\vartyh}}{\third{\vartyh}}$, \judth{\env}{\first{\vartyh}}, and \judeh{\envexte{\varvarh}{\first{\vartyh}}}{\varexph}{\third{\vartyh}}.

% \\u.e

\item If \judeh{\env}{\exptabs{\tyvarh}{\varexph}}{\first{\vartyh}} then $\first{\vartyh} = \tyfor{\tyvarh}{\second{\vartyh}}$ and \judeh{\envextt{\tyvarh}}{\varexph}{\second{\vartyh}}.

% n

\item If \judeh{}{\expnum{\symnum}}{\vartyh} then $\vartyh = \tynum$.

% nil t

\item If \judeh{\env}{\expnils{\first{\vartyh}}}{\second{\vartyh}} then $\second{\vartyh} = \tylist{\first{\vartyh}}$ and \judth{\env}{\first{\vartyh}}.

% cons e e

\item If \judeh{\env}{\expcons{\first{\varexph}}{\second{\varexph}}}{\first{\vartyh}} then $\first{\vartyh} = \tylist{\second{\vartyh}}$, \judeh{\env}{\first{\varexph}}{\second{\vartyh}}, and \judeh{\env}{\second{\varexph}}{\tylist{\second{\vartyh}}}.

% x

\item \judeh{\envexte{\varvarh}{\vartyh}}{\varvarh}{\vartyh}.

% e e

\item If \judeh{\env}{\expfapp{\first{\varexph}}{\second{\varexph}}}{\first{\vartyh}} then \judeh{\env}{\first{\varexph}}{\tyfun{\second{\vartyh}}{\first{\vartyh}}} and \judeh{\env}{\second{\varexph}}{\second{\vartyh}}.

% fix e

\item If \judeh{\env}{\expfix{\varexph}}{\vartyh} then \judeh{\env}{\varexph}{\tyfun{\vartyh}{\vartyh}}.

% e<t>

\item If \judeh{\env}{\exptapp{\varexph}{\first{\vartyh}}}{\second{\vartyh}} then $\second{\vartyh} = \tysubst{\third{\vartyh}}{\first{\vartyh}}{\tyvarh}$, \judth{\env}{\vartyh}, and \judeh{\env}{\varexph}{\tyfor{\tyvarh}{\third{\vartyh}}}.

% hd e

\item If \judeh{\env}{\exphd{\varexph}}{\vartyh} then \judeh{\env}{\varexph}{\tylist{\vartyh}}.

% tl e

\item If \judeh{\env}{\exptl{\varexph}}{\first{\vartyh}} then $\first{\vartyh} = \tylist{\second{\vartyh}}$ and \judeh{\env}{\varexph}{\tylist{\second{\vartyh}}}.

% o e e

\item If $\Gamma\vdash_{A}o$ $e_{A}^{1}$ $e_{A}^{2}:T$ then $T=N$, $\Gamma\vdash_{A}e_{A}^{1}:N$, and $\Gamma\vdash_{A}e_{A}^{2}:N$ where $A\in\lbrace H,M\rbrace$.

\item If \judeh{\env}{\expop{\first{\varexph}}{\second{\varexph}}}{ % TODO

% null? e

\item If $\Gamma\vdash_{A}\mathtt{null?}$ $e_{A}:T$ then $T=N$ and $\Gamma\vdash_{A}e_{A}:[T_{1}]$ where $A\in\lbrace H,M\rbrace$.

% if0 e e e

\item If $\Gamma\vdash_{A}\mathtt{if0}$ $e_{A}^{1}$ $e_{A}^{2}$ $e_{A}^{3}:T$ then $T=T_{1}$, $\Gamma\vdash_{A}e_{A}^{1}:N$, $\Gamma\vdash_{A}e_{A}^{2}:T_{1}$, and $\Gamma\vdash_{A}e_{A}^{3}:T_{1}$ where $A\in\lbrace H,M\rbrace$.

% wrong t string

\item If $\Gamma\vdash_{A}\mathtt{wrong}^{T_{1}}$ $\mathrm{string}:T$ then $T=T_{1}$ where $A\in\lbrace H,M\rbrace$.

\item If $\Gamma\vdash_{A}{^{T_{1}}A}B^{T_{1}}$ $e_{B}:T$ then $T=T_{1}$, $\Gamma\vdash_{A}T_{1}$, $\Gamma\vdash_{B}T_{1}$, and $\Gamma\vdash_{B}e_{B}:T_{1}$ where $(A,B)\in\lbrace(H,M),(M,H)\rbrace$.

\item If $\Gamma\vdash_{A}{^{T_{1}}A}S$ $e_{S}:T$ then $T=T_{1}[T_{i}/T_{i}^{a}]$, $\Gamma\vdash_{A}T_{1}$, and $\Gamma\vdash_{S}e_{S}:TST$ where $A\in\lbrace H,M\rbrace$.

% ML

% Scheme

\item If $\Gamma\vdash_{S}\lambda x.e_{S}:TST$ then $\Gamma,x:TST\vdash_{S}e_{S}:TST$.

\item $\vdash_{S}\overline{n}:TST$.

\item $\vdash_{S}\mathtt{nil}:TST$.

\item If $\Gamma\vdash_{S}\mathtt{cons}$ $e_{S}^{1}$ $e_{S}^{2}:TST$ then $\Gamma\vdash_{S}e_{S}^{1}:TST$ and $\Gamma\vdash_{S}e_{S}^{2}:TST$.

\item If $\Gamma\vdash_{S}x:TST$ then $x:TST\in\Gamma$.

\item If $\Gamma\vdash_{S}e_{S}^{1}$ $e_{S}^{2}:TST$ then $\Gamma\vdash_{S}e_{S}^{1}:TST$ and $\Gamma\vdash_{S}e_{S}^{2}:TST$.

\item If $\Gamma\vdash_{S}f$ $e_{S}:TST$ then $\Gamma\vdash_{S}e_{S}:TST$.

\item If $\Gamma\vdash_{S}o$ $e_{S}^{1}$ $e_{S}^{2}:TST$ then $\Gamma\vdash_{S}e_{S}^{1}:TST$ and $\Gamma\vdash_{S}e_{S}^{2}:TST$.

\item If $\Gamma\vdash_{S}p$ $e_{S}:TST$ then $\Gamma\vdash_{S}e_{S}:TST$.

\item If $\Gamma\vdash_{S}\mathtt{if0}$ $e_{S}^{1}$ $e_{S}^{2}$ $e_{S}^{3}:TST$ then $\Gamma\vdash_{S}e_{S}^{1}:TST$, $\Gamma\vdash_{S}e_{S}^{2}:TST$, and $\Gamma\vdash_{S}e_{S}^{3}:TST$.

\item $\vdash_{S}\mathtt{wrong}$ $\mathrm{string}:TST$.

\item $\Gamma\vdash_{S}SA^{T_{1}}$ $e_{A}:TST$, $\Gamma\vdash_{A}T_{1}$, and $\Gamma\vdash_{A}e_{A}:T_{1}[T_{i}/T_{i}^{a}]$ where $A\in\lbrace H,M\rbrace$.

\end{enumerate}

\begin{proof}

Immediate from the typing rules.

\end{proof}

\end{lemma}


Well-typed Haskell and ML expressions have unique types:

\begin{lemma}
\label{uot}
\onehalfspacing
$e_{A}$ has at most one type $T$ for a given context $\Gamma$ where $A\in\lbrace H,M\rbrace$.
\begin{proof}
By structural induction on $e_{A}$ using inversion (Lemma \ref{i}).
\end{proof}
\end{lemma}

The types of Haskell and ML values determine their syntactic forms:

\begin{lemma}
\label{cf}
%\onehalfspacing
The possible syntactic forms of values of various types.
\begin{enumerate}
\item If $v_{A}:N$ then $v_{A}=\overline{n}$ where $A\in\lbrace H,M\rbrace$.
\item If $v_{A}:T_{1}\rightarrow T_{2}$ then $v_{A}=\lambda x:T_{1}.e_{A}$ where $A\in\lbrace H,M\rbrace$.
\item If $v_{A}:\forall X.T$ then $v_{A}\in\lbrace\Lambda X.e_{A},{^{\forall X.T}A}S$ $v_{S}\rbrace$ where $A\in\lbrace H,M\rbrace$.
\item If $v_{H}:[T]$ then $v_{H}\in\lbrace\mathtt{cons}$ $e_{H}^{1}$ $e_{H}^{2},\mathtt{nil}^{T}\rbrace$.
\item If $v_{M}:[T]$ then $v_{M}\in\lbrace\mathtt{cons}$ $v_{M}^{1}$ $v_{M}^{2},\mathtt{nil}^{T},{^{[T]}M}H^{[T]}$ $(\mathtt{cons}$ $e_{H}^{1}$ $e_{H}^{2})\rbrace$.
\item If $v_{A}:L$ then $v_{A}={^{L}A}S$ $v_{S}$ where $A\in\lbrace H,M\rbrace$.
\end{enumerate}
\begin{proof}
Immediate from the definitions of values and the typing relations.
\end{proof}
\end{lemma}

\input{proof/theorems/haskell-ml-progress.tex}

\input{proof/theorems/scheme-progress.tex}

\section{Preservation}

Preservation will be proven by cases on the rewrite rules.  In each case, the right side is be proven to be well-typed and have the same type as the left side.  Inversion (Lemma \ref{i}) and uniqueness of types (Lemma \ref{uot}) are used to determine the types of the left side and its subexpressions and the type of the right side.  Some rewrite rules use expression and type substitutions.

If $e_{A}^{1}$ is substituted for free occurrences of $x$ within $e_{A}^{2}$, $e_{A}^{1}$ and $x$ have the same type, and the result has the same type as $e_{A}^{2}$, where $A\in\lbrace H,M,S\rbrace$:

\begin{lemma}{Expression Substitution Preservation}

\label{lemexp}

If \judeh{\envexte{\first{\varvarh}}{\first{\vartyh}}}{\first{\varexph}}{\second{\vartyh}} and \judeh{\env}{\second{\varexph}}{\first{\vartyh}} then \judeh{\env}{\expsubst{\first{\varexph}}{\second{\varexph}}{\first{\varvarh}}}{\second{\vartyh}}.  If \judem{\envexte{\first{\varvarm}}{\first{\vartym}}}{\first{\varexpm}}{\second{\vartym}} and \judem{\env}{\second{\varexpm}}{\first{\vartym}} then \judem{\env}{\expsubst{\first{\varexpm}}{\second{\varexpm}}{\first{\varvarm}}}{\second{\vartym}}.  If \judes{\envexte{\first{\varvars}}{\tytst}}{\first{\varexps}}{\tytst} and \judes{\env}{\second{\varexps}}{\tytst} then \judes{\env}{\expsubst{\first{\varexps}}{\second{\varexps}}{\first{\varvars}}}{\tytst}.

\begin{proof}

By structural induction.

\end{proof}

\end{lemma}


If $T_{1}$ is substituted for free occurrences of $X$ within $e_{A}$ of type $T_{2}$, the type of the result is $T_{1}$ substituted for free occurrences of $X$ within $T_{2}$, where $A\in\lbrace H,M\rbrace$:

\begin{lemma}
\label{tes}
If $\Gamma,X\vdash_{HM}e_{HM}:T_{1}$ and $\vdash_{HM}T_{2}$ then $\Gamma\vdash_{HM}e_{HM}[T_{2}/X]:T_{1}[T_{2}/X]$.
\begin{proof}
By structural induction.
\end{proof}
\end{lemma}

\begin{lemma}
\label{ecp}
\onehalfspacing
If $\Gamma\vdash_{H}e_{H}^{1}:T_{1}$, $\Gamma\vdash_{H}e_{H}^{2}:T_{2}$, and $\mathscr{E}[e_{H}^{1}]:T_{1}$ then $\mathscr{E}[e_{H}^{2}]:T_{2}$.
\begin{proof}
By structural induction because typing rules use types and not forms of sub-terms.
\end{proof}
\end{lemma}

\begin{theorem}
\label{pn}
\onehalfspacing
If $\Gamma\vdash_{HMS}e_{HMS}^{1}:T$ and $e_{HMS}^{1}\rightarrow e_{HMS}^{2}$ then $\Gamma\vdash_{HMS}e_{HMS}^{2}:T$.
\begin{proof}
By cases on the reduction $e_{HMS}^{1}\rightarrow e_{HMS}^{2}$.  Scheme reductions that do not contain Haskell or ML terms are omitted because demonstrating the preservation of $TST$ is straightforward.
\begin{case}
$(\lambda x:T_{1}.e_{HM}^{1})\;e_{HM}^{2}\rightarrow e_{HM}^{1}[e_{HM}^{2}/x]$

$\Gamma\vdash_{HM}(\lambda x:T_{1}.e_{HM}^{1})\;e_{HM}^{2}:T$ by the premise and uniqueness of types (Lemma \ref{uot}).  $\Gamma\vdash_{HM}\lambda x:T_{1}.e_{HM}^{1}:T_{1}\rightarrow T$, $\Gamma,x:T_{1}\vdash_{HM}e_{HM}^{1}:T$, $\Gamma\vdash_{HM}e_{HM}^{2}:T_{1}$, and $\Gamma,x:T_{1}\vdash_{HM}x:T_{1}$ by inversion (Lemma \ref{i}) and uniqueness of types.  $e_{HM}^{1}[e_{HM}^{2}/x]:T$ by term substitution (Lemma \ref{tms}).
\end{case}
\begin{case}
$(\Lambda X.e_{HM})\;\lbrace T_{1}\rbrace\rightarrow e_{HM}[T_{1}/X]$

$\Gamma\vdash_{HM}(\Lambda X.e_{H})\;\lbrace T_{1}\rbrace:T$ by premise and uniqueness of types (Lemma \ref{uot}).  $\Gamma\vdash_{HM}\Lambda X.e_{HM}:\forall X.T_{2}$, $\Gamma,X\vdash_{HM}e_{HM}:T_{2}$, and $T=T_{2}[T_{1}/X]$ by inversion (Lemma \ref{i}) and uniqueness of types.  $\Gamma\vdash_{HM}e_{HM}[T_{1}/X]:T_{2}[T_{1}/X]$ by type substitution (Lemma \ref{tes}).  $\Gamma\vdash_{HM}e_{HM}[T_{1}/X]:T$ because $T=T_{2}[T_{1}/X]$.
\end{case}
\begin{case}
$\mathtt{if0}\;\overline{0}\;e_{HM}^{1}\;e_{HM}^{2}\rightarrow e_{HM}^{1}$

$\Gamma\vdash_{HM}\mathtt{if0}\;\overline{0}\;e_{HM}^{1}\;e_{HM}^{2}:T$ by premise and uniqueness of types (Lemma \ref{uot}).  $\Gamma\vdash_{HM}e_{HM}^{1}:T$ by inversion (Lemma \ref{i}) and uniqueness of types.
\end{case}
\begin{case}
$\mathtt{if0}\;\overline{n}\;e_{HM}^{1}\;e_{HM}^{2}\rightarrow e_{HM}^{2}\;(n\neq0)$

$\Gamma\vdash_{HM}\mathtt{if0}\;\overline{n}\;e_{HM}^{1}\;e_{HM}^{2}:T$ by premise and uniqueness of types (Lemma \ref{uot}).  $\Gamma\vdash_{HM}e_{HM}^{2}:T$ by inversion (Lemma \ref{i}) and uniqueness of types.
\end{case}
\begin{case}
$+\;\overline{n_{1}}\;\overline{n_{2}}\rightarrow\overline{n_{1}+n_{2}}$

$\vdash_{HM}+\;\overline{n_{1}}\;\overline{n_{2}}:N$ by inversion (Lemma \ref{i}) and uniqueness of types (Lemma \ref{uot}).  $\vdash_{HM}\overline{n_{1}+n_{2}}:N$ by inversion and uniqueness of types.
\end{case}
\begin{case}
$-\;\overline{n_{1}}\;\overline{n_{2}}\rightarrow\overline{max(n_{1}-n_{2},0)}$ where $A\in\lbrace H,M\rbrace$

$\vdash_{A}-\;\overline{n_{1}}\;\overline{n_{2}}:N$ by inversion (Lemma \ref{i}) and uniqueness of types (Lemma \ref{uot}).  $\vdash_{A}\overline{max(n_{1}-n_{2},0)}:N$ by inversion and uniqueness of types.
\end{case}
\input{cases/preservation/head-cons.tex}
\begin{case}
$\mathtt{tl}\;(\mathtt{cons}\;e_{HM}^{1}\;e_{HM}^{2})\rightarrow e_{HM}^{2}$

$\Gamma\vdash_{HM}\mathtt{tl}\;(\mathtt{cons}\;e_{HM}^{1}\;e_{HM}^{2}):T$ by premise and uniqueness of types (Lemma \ref{uot}).  $\Gamma\vdash_{HM}e_{HM}^{2}:T$ by inversion and uniqueness of types (Lemma \ref{uot}).
\end{case}
\begin{case}
$\mathtt{hd}\;\mathtt{nil}^{T_{1}}\rightarrow\,^{T}HS\;(\mathtt{wrong}\;\mathrm{``Empty\;list"})$

$\Gamma\vdash_{HM}\mathtt{hd}\;\mathtt{nil}^{T_{1}}:T$ by premise and uniqueness of types (Lemma \ref{uot}).  $\Gamma\vdash_{HM}\,^{T}HS\;(\mathtt{wrong}\;\mathrm{``Empty\;list"}):T$ by inversion (Lemma \ref{i}) and uniqueness of types (Lemma \ref{uot}).
\end{case}
\begin{case}
$\mathtt{tl}\;\mathtt{nil}^{T_{1}}\rightarrow\mathtt{nil}^{T_{1}}$ where $A\in\lbrace H,M\rbrace$

$\Gamma\vdash_{A}\mathtt{tl}\;\mathtt{nil}^{T_{1}}:T$ by premise and uniqueness of types (Lemma \ref{uot}).  $\Gamma\vdash_{A}\mathtt{nil}^{T_{1}}:[T_{1}]$ and $T=[T_{1}]$ by inversion (Lemma \ref{i}) and uniqueness of types.  $\Gamma\vdash_{A}\mathtt{nil}^{T_{1}}:T$ because $[T_{1}]=T$.
\end{case}
\begin{case}
$\mathtt{fix}\;(\lambda x:T_{1}.e_{A})\rightarrow e_{A}[(\mathtt{fix}\;(\lambda x:T_{1}.e_{A}))/x]$ where $A\in\lbrace H,M\rbrace$

$\Gamma\vdash_{A}\mathtt{fix}\;(\lambda x:T_{1}.e_{A}):T$ by premise and uniqueness of types (Lemma \ref{uot}).  $T=T_{1}$, $\Gamma,x:T_{1}\vdash_{A}e_{A}:T_{1}$, and $\Gamma,x:T_{1}\vdash_{A}x:T_{1}$ by inversion (Lemma \ref{i}) and uniqueness of types.  $\Gamma\vdash_{A}e_{A}[(\mathtt{fix}\;(\lambda x:T_{1}.e_{A}))/x]:T_{1}$ by term substitution (Lemma \ref{tms}).  $\Gamma\vdash_{A}e_{A}[(\mathtt{fix}\;(\lambda x:T_{1}.e_{A}))/x]:T$ because $T_{1}=T$.
\end{case}
\begin{case}

$e_{A}=\overline{n}$ where $A\in\lbrace H,M\rbrace$

$\overline{n}$ is an unforced value.

\end{case}
\begin{case}
$^{\forall X_{1}.T}B^{\forall X_{1}.T}\;(\Lambda X_{1}.e_{HM})\rightarrow\Lambda X_{2}.(^{T[X_{2}/X_{1}]}B^{T[X_{2}/X_{1}]}\;((\Lambda X_{1}.e_{HM})\;\lbrace X_{2}\rbrace))$ where $B\in\lbrace HM,MH\rbrace$

$\Gamma\vdash_{HM}\,^{\forall X_{1}.T}B^{\forall X_{1}.T}\;(\Lambda X_{1}.e_{HM}):\forall X_{1}.T$ by premise and inversion (Lemma \ref{i}) and uniqueness of types (Lemma \ref{uot}).  NOT DONE.
\end{case}
\begin{case}
$e_{S}=\lambda x.e_{S}^{1}$

$\lambda x.e_{S}^{1}$ is a forced value.
\end{case}
\begin{case}
$^{[T]}HM^{[T]}\;(\mathtt{cons}\;v_{M}^{1}\;v_{M}^{2})\rightarrow\mathtt{cons}\;(^{T}HM^{T}\;v_{M}^{1})\;(^{[T]}HM^{[T]}\;v_{M}^{2})$

$\Gamma\vdash_{H}^{[T]}HM^{[T]}\;(\mathtt{cons}\;v_{M}^{1}\;v_{M}^{2}):[T]$ by premise and inversion (Lemma \ref{i}) and uniqueness of types (Lemma \ref{uot}).  $\Gamma\vdash_{M}v_{M}^{1}:T$ and $\Gamma\vdash_{M}v_{M}^{1}:[T]$ by inversion (Lemma \ref{i}) and uniqueness of types (Lemma \ref{uot}).  $\Gamma\vdash_{H}\,^{T}HM^{T}\;v_{M}^{1}:T$ and $\Gamma\vdash_{H}\,^{[T]}HM^{[T]}\;v_{M}^{2}:[T]$ by the induction hypothesis and uniqueness of types (Lemma \ref{uot}).  $\Gamma\vdash_{H}\mathtt{cons}\;(^{T}HM^{T}\;v_{M}^{1})\;(^{[T]}HM^{[T]}\;v_{M}^{2}):[T]$.
\end{case}
\input{cases/preservation/list-cons-m.tex}
\input{cases/preservation/list-nil.tex}
%\input{cases/preservation/.tex}
\end{proof}
\end{theorem}