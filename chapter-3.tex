\chapter{Proof of Type Soundness}

% TODO: Well-typed terms are closed, no need to specify both.

Proving the progress of expressions and the preservation of types proves the type soundness of the model of computation.  Progress ensures that a well-typed, closed expression is either an unforced value, reducible to another expression, or reducible to an error.  Preservation ensures that if a well-typed expression reduces to another expression, the other expression is well-typed and has the same type.  The proof extends the proof by Kinghorn \cite{kinghorn07}, which was based on proofs by Pierce \cite{pierce02} and Matthews and Findler \cite{matthews07}.  Cases common to two or more languages are elided for brevity.

\section{Proof of Expression Progress}

Progress will be proven by structural induction on a well-typed, closed expression of each syntactic form.  In each case, the expression will be proven to be either an unforced value, reducible to another expression, or reducible to an error.  The reduction of a subexpression is the reduction of its parent expression.  If a subexpression reduces to an error, its parent expression reduces to the error.  In some cases, the syntactic form of a subexpression must be determined to reduce its parent expression.  Determining the unique type of a subexpression determines its syntactic form.

\subsection{Inversion Lemma}

Inverting the typing rules enables the syntactic forms of well-typed expressions to determine the types of their subexpressions.

\begin{lemma}

\label{leminv}

The syntactic forms of well-typed expressions determine the types of their subexpressions.

\begin{enumerate}

% Haskell

% \x:t.e

\item If \judeh{\env}{\expfabss{\varvarh}{\first{\vartyh}}{\varexph}}{\second{\vartyh}} then $\second{\vartyh} = \tyfun{\first{\vartyh}}{\third{\vartyh}}$, \judth{\env}{\first{\vartyh}}, and \judeh{\envexte{\varvarh}{\first{\vartyh}}}{\varexph}{\third{\vartyh}}.

% \\u.e

\item If \judeh{\env}{\exptabs{\tyvarh}{\varexph}}{\first{\vartyh}} then $\first{\vartyh} = \tyfor{\tyvarh}{\second{\vartyh}}$ and \judeh{\envextt{\tyvarh}}{\varexph}{\second{\vartyh}}.

% n

\item If \judeh{}{\expnum{\symnum}}{\vartyh} then $\vartyh = \tynum$.

% nil t

\item If \judeh{\env}{\expnils{\first{\vartyh}}}{\second{\vartyh}} then $\second{\vartyh} = \tylist{\first{\vartyh}}$ and \judth{\env}{\first{\vartyh}}.

% cons e e

\item If \judeh{\env}{\expcons{\first{\varexph}}{\second{\varexph}}}{\first{\vartyh}} then $\first{\vartyh} = \tylist{\second{\vartyh}}$, \judeh{\env}{\first{\varexph}}{\second{\vartyh}}, and \judeh{\env}{\second{\varexph}}{\tylist{\second{\vartyh}}}.

% x

\item \judeh{\envexte{\varvarh}{\vartyh}}{\varvarh}{\vartyh}.

% e e

\item If \judeh{\env}{\expfapp{\first{\varexph}}{\second{\varexph}}}{\first{\vartyh}} then \judeh{\env}{\first{\varexph}}{\tyfun{\second{\vartyh}}{\first{\vartyh}}} and \judeh{\env}{\second{\varexph}}{\second{\vartyh}}.

% fix e

\item If \judeh{\env}{\expfix{\varexph}}{\vartyh} then \judeh{\env}{\varexph}{\tyfun{\vartyh}{\vartyh}}.

% e<t>

\item If \judeh{\env}{\exptapp{\varexph}{\first{\vartyh}}}{\second{\vartyh}} then $\second{\vartyh} = \tysubst{\third{\vartyh}}{\first{\vartyh}}{\tyvarh}$, \judth{\env}{\vartyh}, and \judeh{\env}{\varexph}{\tyfor{\tyvarh}{\third{\vartyh}}}.

% hd e

\item If \judeh{\env}{\exphd{\varexph}}{\vartyh} then \judeh{\env}{\varexph}{\tylist{\vartyh}}.

% tl e

\item If \judeh{\env}{\exptl{\varexph}}{\first{\vartyh}} then $\first{\vartyh} = \tylist{\second{\vartyh}}$ and \judeh{\env}{\varexph}{\tylist{\second{\vartyh}}}.

% o e e

\item If $\Gamma\vdash_{A}o$ $e_{A}^{1}$ $e_{A}^{2}:T$ then $T=N$, $\Gamma\vdash_{A}e_{A}^{1}:N$, and $\Gamma\vdash_{A}e_{A}^{2}:N$ where $A\in\lbrace H,M\rbrace$.

\item If \judeh{\env}{\expop{\first{\varexph}}{\second{\varexph}}}{ % TODO

% null? e

\item If $\Gamma\vdash_{A}\mathtt{null?}$ $e_{A}:T$ then $T=N$ and $\Gamma\vdash_{A}e_{A}:[T_{1}]$ where $A\in\lbrace H,M\rbrace$.

% if0 e e e

\item If $\Gamma\vdash_{A}\mathtt{if0}$ $e_{A}^{1}$ $e_{A}^{2}$ $e_{A}^{3}:T$ then $T=T_{1}$, $\Gamma\vdash_{A}e_{A}^{1}:N$, $\Gamma\vdash_{A}e_{A}^{2}:T_{1}$, and $\Gamma\vdash_{A}e_{A}^{3}:T_{1}$ where $A\in\lbrace H,M\rbrace$.

% wrong t string

\item If $\Gamma\vdash_{A}\mathtt{wrong}^{T_{1}}$ $\mathrm{string}:T$ then $T=T_{1}$ where $A\in\lbrace H,M\rbrace$.

\item If $\Gamma\vdash_{A}{^{T_{1}}A}B^{T_{1}}$ $e_{B}:T$ then $T=T_{1}$, $\Gamma\vdash_{A}T_{1}$, $\Gamma\vdash_{B}T_{1}$, and $\Gamma\vdash_{B}e_{B}:T_{1}$ where $(A,B)\in\lbrace(H,M),(M,H)\rbrace$.

\item If $\Gamma\vdash_{A}{^{T_{1}}A}S$ $e_{S}:T$ then $T=T_{1}[T_{i}/T_{i}^{a}]$, $\Gamma\vdash_{A}T_{1}$, and $\Gamma\vdash_{S}e_{S}:TST$ where $A\in\lbrace H,M\rbrace$.

% ML

% Scheme

\item If $\Gamma\vdash_{S}\lambda x.e_{S}:TST$ then $\Gamma,x:TST\vdash_{S}e_{S}:TST$.

\item $\vdash_{S}\overline{n}:TST$.

\item $\vdash_{S}\mathtt{nil}:TST$.

\item If $\Gamma\vdash_{S}\mathtt{cons}$ $e_{S}^{1}$ $e_{S}^{2}:TST$ then $\Gamma\vdash_{S}e_{S}^{1}:TST$ and $\Gamma\vdash_{S}e_{S}^{2}:TST$.

\item If $\Gamma\vdash_{S}x:TST$ then $x:TST\in\Gamma$.

\item If $\Gamma\vdash_{S}e_{S}^{1}$ $e_{S}^{2}:TST$ then $\Gamma\vdash_{S}e_{S}^{1}:TST$ and $\Gamma\vdash_{S}e_{S}^{2}:TST$.

\item If $\Gamma\vdash_{S}f$ $e_{S}:TST$ then $\Gamma\vdash_{S}e_{S}:TST$.

\item If $\Gamma\vdash_{S}o$ $e_{S}^{1}$ $e_{S}^{2}:TST$ then $\Gamma\vdash_{S}e_{S}^{1}:TST$ and $\Gamma\vdash_{S}e_{S}^{2}:TST$.

\item If $\Gamma\vdash_{S}p$ $e_{S}:TST$ then $\Gamma\vdash_{S}e_{S}:TST$.

\item If $\Gamma\vdash_{S}\mathtt{if0}$ $e_{S}^{1}$ $e_{S}^{2}$ $e_{S}^{3}:TST$ then $\Gamma\vdash_{S}e_{S}^{1}:TST$, $\Gamma\vdash_{S}e_{S}^{2}:TST$, and $\Gamma\vdash_{S}e_{S}^{3}:TST$.

\item $\vdash_{S}\mathtt{wrong}$ $\mathrm{string}:TST$.

\item $\Gamma\vdash_{S}SA^{T_{1}}$ $e_{A}:TST$, $\Gamma\vdash_{A}T_{1}$, and $\Gamma\vdash_{A}e_{A}:T_{1}[T_{i}/T_{i}^{a}]$ where $A\in\lbrace H,M\rbrace$.

\end{enumerate}

\begin{proof}

Immediate from the typing rules.

\end{proof}

\end{lemma}


\subsection{Uniqueness of Types Lemma}

Well-typed Haskell and ML expressions have unique types.

\begin{lemma}
\label{uot}
\onehalfspacing
$e_{A}$ has at most one type $T$ for a given context $\Gamma$ where $A\in\lbrace H,M\rbrace$.
\begin{proof}
By structural induction on $e_{A}$ using inversion (Lemma \ref{i}).
\end{proof}
\end{lemma}

\subsection{Canonical Forms Lemma}

The types of Haskell and ML values determine their syntactic forms.

\begin{lemma}
\label{cf}
%\onehalfspacing
The possible syntactic forms of values of various types.
\begin{enumerate}
\item If $v_{A}:N$ then $v_{A}=\overline{n}$ where $A\in\lbrace H,M\rbrace$.
\item If $v_{A}:T_{1}\rightarrow T_{2}$ then $v_{A}=\lambda x:T_{1}.e_{A}$ where $A\in\lbrace H,M\rbrace$.
\item If $v_{A}:\forall X.T$ then $v_{A}\in\lbrace\Lambda X.e_{A},{^{\forall X.T}A}S$ $v_{S}\rbrace$ where $A\in\lbrace H,M\rbrace$.
\item If $v_{H}:[T]$ then $v_{H}\in\lbrace\mathtt{cons}$ $e_{H}^{1}$ $e_{H}^{2},\mathtt{nil}^{T}\rbrace$.
\item If $v_{M}:[T]$ then $v_{M}\in\lbrace\mathtt{cons}$ $v_{M}^{1}$ $v_{M}^{2},\mathtt{nil}^{T},{^{[T]}M}H^{[T]}$ $(\mathtt{cons}$ $e_{H}^{1}$ $e_{H}^{2})\rbrace$.
\item If $v_{A}:L$ then $v_{A}={^{L}A}S$ $v_{S}$ where $A\in\lbrace H,M\rbrace$.
\end{enumerate}
\begin{proof}
Immediate from the definitions of values and the typing relations.
\end{proof}
\end{lemma}

\subsection{Haskell and ML Progress Theorem}

%\newcommand{\}{}
\newcommand{\pshyp}[2]{#1 is a \profv or #1 \red #2 or #1 \red \emph{\experr{\varstr}}\xspace}
\newcommand{\pshypref}[2]{\pshyp{#1}{#2} by the induction hypothesis.\xspace}
\newcommand{\psval}[3]{If #1 is a \profv then \judeh{}{#1}{#2} by lemmas \ref{leminv} and \ref{lemuni} and #1 $=$ #3 by lemma \ref{lemcan}.\xspace}
\newcommand{\pssub}[4]{If #1 \red #2 then #3 \red #4.\xspace}
\newcommand{\psred}[2]{\redrule{#1}{#2}.\xspace}
\newcommand{\pserr}[2]{If #1 \red \emph{\experr{\varstr}} then #2 \red \emph{\experr{\varstr}}.\xspace}

\begin{theorem}{Haskell Progress}

\label{thmhps}

If \judeh{}{\varexph}{\vartyh} then \pshyp{\first{\varexph}}{\second{\varexph}}.

\begin{proof}

By structural induction on \varexph.

% Haskell

% \x:t.e

\newcommand{\psfabss}{\expfabss{\varvarh}{\vartyh}{\varexph}\xspace}

\begin{case}

\psfabss

\psfabss is a \profv.

\end{case}

% \\u.e

\newcommand{\pstabs}{\exptabs{\tyvarh}{\varexph}\xspace}

\begin{case}

\pstabs

\pstabs is a \profv.

\end{case}

% n

\newcommand{\psnum}{\expnum{\symnum}\xspace}

\begin{case}

\psnum

\psnum is a \profv.

\end{case}

% nil t

\newcommand{\psnils}{\expnils{\vartyh}\xspace}

\begin{case}

\psnils

\psnils is a \profv.

\end{case}

% cons e e

\newcommand{\psconsh}{\expcons{\first{\varexph}}{\second{\varexph}}\xspace}

\begin{case}

\psconsh

\psconsh is a \profv.

\end{case}

% x

\newcommand{\psvar}{\varvarh\xspace}

\begin{case}

\psvar

Cannot occur because \varexph is closed.

\end{case}

% e e

\newcommand{\psfapp}{\expfapp{\first{\varexph}}{\second{\varexph}}\xspace}
\newcommand{\x}{\expfabss{\varvarh}{\first{\vartyh}}{\third{\varexph}}\xspace}

\begin{case}

\psfapp

\pshypref
{\first{\varexph}}
{\third{\varexph}}
\psval
{\first{\varexph}}
{\tyfun{\first{\vartyh}}{\second{\vartyh}}}
{\x}
\psred
{\expfapp{(\x)}{\second{\varexph}}}
{\expsubst{\third{\varexph}}{\second{\varexph}}{\varvarh}}
\pssub
{\first{\varexph}}
{\third{\varexph}}
{\psfapp}
{\expfapp{\third{\varexph}}{\second{\varexph}}}
\pserr
{\first{\varexph}}
{\psfapp}

\end{case}

% fix e

\begin{case}

$e_{A}=\mathtt{fix}$ $e_{A}^{1}$ where $A\in\lbrace H,M\rbrace$

$e_{A}^{1}$ is an unforced value or $e_{A}^{1}\rightarrow e_{A}^{2}$ or $e_{A}^{1}\rightarrow$ \emph{\textbf{Error}: string} by the induction hypothesis.  If $e_{A}^{1}$ is an unforced value then $e_{A}^{1}:T\rightarrow T$ by inversion (Lemma \ref{i}) and uniqueness of types (Lemma \ref{uot}) and $e_{A}^{1}=\lambda x:T.e_{A}^{3}$ by canonical forms (Lemma \ref{cf}).  $\mathtt{fix}$ $(\lambda x:T.e_{A}^{3})\rightarrow e_{A}^{3}[\mathtt{fix}$ $(\lambda x:T.e_{A}^{3})/x]$.  If $e_{A}^{1}\rightarrow e_{A}^{2}$ then $\mathtt{fix}$ $e_{A}^{1}\rightarrow\mathtt{fix}$ $e_{A}^{2}$.  If $e_{A}^{1}\rightarrow$ \emph{\textbf{Error}: string} then $\mathtt{fix}$ $e_{A}^{1}\rightarrow$ \emph{\textbf{Error}: string}.

\end{case}

% e<t>

\begin{case}

$e_{A}=e_{A}^{1}$ $\lbrace T_{1}\rbrace$ where $A\in\lbrace H,M\rbrace$

$e_{A}^{1}$ is an unforced value or $e_{A}^{1}\rightarrow e_{A}^{2}$ or $e_{A}^{1}\rightarrow$ \emph{\textbf{Error}: string} by the induction hypothesis.  If $e_{A}^{1}$ is an unforced value then $e_{A}^{1}:\forall X.T_{2}$ by inversion (Lemma \ref{i}) and uniqueness of types (Lemma \ref{uot}) and $e_{A}^{1}\in\lbrace\Lambda X.e_{A}^{3},{^{\forall X.T_{2}}A}S$ $v_{S}\rbrace$ by canonical forms (Lemma \ref{cf}).  $(\Lambda X.e_{A}^{3})$ $\lbrace T_{1}\rbrace\rightarrow e_{A}^{3}[T_{1}/X]$.  $(^{\forall X.T_{2}}AS$ $v_{S})$ $\lbrace T_{1}\rbrace\rightarrow{^{T_{2}[T_{1}^{a}/X]}A}S$ $v_{S}$.  If $e_{A}^{1}\rightarrow e_{A}^{2}$ then $e_{A}^{1}$ $\lbrace T_{1}\rbrace\rightarrow e_{A}^{2}$ $\lbrace T_{1}\rbrace$.  If $e_{A}^{1}\rightarrow$ \emph{\textbf{Error}: string} then $e_{A}^{1}$ $\lbrace T_{1}\rbrace\rightarrow$ \emph{\textbf{Error}: string}.

\end{case}

% ML cons e e

\begin{case}

$e_{M}=\mathtt{cons}$ $e_{M}^{1}$ $e_{M}^{2}$

$e_{M}^{1}$ is an unforced value or $e_{M}^{1}\rightarrow e_{M}^{3}$ or $e_{M}^{1}\rightarrow$ \emph{\textbf{Error}: string} by the induction hypothesis.  If $e_{M}^{1}\rightarrow e_{M}^{3}$ then $\mathtt{cons}$ $e_{M}^{1}$ $e_{M}^{2}\rightarrow\mathtt{cons}$ $e_{M}^{3}$ $e_{M}^{2}$.  If $e_{M}^{1}\rightarrow$ \emph{\textbf{Error}: string} then $\mathtt{cons}$ $e_{M}^{1}$ $e_{M}^{2}\rightarrow$ \emph{\textbf{Error}: string}.  $e_{M}^{2}$ is an unforced value or $e_{M}^{2}\rightarrow e_{M}^{4}$ or $e_{M}^{2}\rightarrow$ \emph{\textbf{Error}: string} by the induction hypothesis.  If $e_{M}^{2}\rightarrow e_{M}^{4}$ and $e_{M}^{1}$ is an unforced value then $\mathtt{cons}$ $e_{M}^{1}$ $e_{M}^{2}\rightarrow\mathtt{cons}$ $e_{M}^{1}$ $e_{M}^{4}$.  If $e_{M}^{2}\rightarrow$ \emph{\textbf{Error}: string} and $e_{M}^{1}$ is an unforced value then $\mathtt{cons}$ $e_{M}^{1}$ $e_{M}^{2}\rightarrow$ \emph{\textbf{Error}: string}.  If $e_{M}^{1}$ and $e_{M}^{2}$ are unforced values then $\mathtt{cons}$ $e_{M}^{1}$ $e_{M}^{2}$ is an unforced value.

\end{case}

% Haskell f e

\begin{case}

$e_{H}=f$ $e_{H}^{1}$

$e_{H}^{1}$ is an unforced value or $e_{H}^{1}\rightarrow e_{H}^{2}$ or $e_{H}^{1}\rightarrow$ \emph{\textbf{Error}: string} by the induction hypothesis.  If $e_{H}^{1}$ is an unforced value then $e_{H}^{1}:[T]$ by inversion (Lemma \ref{i}) and uniqueness of types (Lemma \ref{uot}) and $e_{H}^{1}\in\lbrace\mathtt{nil}^{T},\mathtt{cons}$ $e_{H}^{3}$ $e_{H}^{4}\rbrace$ by canonical forms (Lemma \ref{cf}).  $\mathtt{hd}$ $\mathtt{nil}^{T}\rightarrow\mathtt{wrong}^{T}$ \emph{``Empty list"}.  $\mathtt{tl}$ $\mathtt{nil}^{T}\rightarrow\mathtt{wrong}^{[T]}$ \emph{``Empty list"}.  $\mathtt{hd}$ $(\mathtt{cons}$ $e_{H}^{3}$ $e_{H}^{4})\rightarrow e_{H}^{3}$.  $\mathtt{tl}$ $(\mathtt{cons}$ $e_{H}^{3}$ $e_{H}^{4})\rightarrow e_{H}^{4}$.  If $e_{H}^{1}\rightarrow e_{H}^{2}$ then $f$ $e_{H}^{1}\rightarrow f$ $e_{H}^{2}$.  If $e_{H}^{1}\rightarrow$ \emph{\textbf{Error}: string} then $f$ $e_{H}^{1}\rightarrow$ \emph{\textbf{Error}: string}.

\end{case}

% ML f e

\begin{case}

$e_{M}=f$ $e_{M}^{1}$

$e_{M}^{1}$ is an unforced value or $e_{M}^{1}\rightarrow e_{M}^{2}$ or $e_{M}^{1}\rightarrow$ \emph{\textbf{Error}: string} by the induction hypothesis.  If $e_{M}^{1}$ is an unforced value then $e_{M}^{1}:[T]$ by inversion (Lemma \ref{i}) and uniqueness of types (Lemma \ref{uot}) and $e_{M}^{1}\in\lbrace\mathtt{nil}^{T},\mathtt{cons}$ $v_{M}^{1}$ $v_{M}^{2},{^{[T]}M}H^{[T]}$ $(\mathtt{cons}$ $e_{H}^{1}$ $e_{H}^{2})\rbrace$ by canonical forms (Lemma \ref{cf}).  $\mathtt{hd}$ $\mathtt{nil}^{T}\rightarrow\mathtt{wrong}^{T}$ \emph{``Empty list"}.  $\mathtt{tl}$ $\mathtt{nil}^{T}\rightarrow\mathtt{wrong}^{[T]}$ \emph{``Empty list"}.  $\mathtt{hd}$ $(\mathtt{cons}$ $v_{M}^{1}$ $v_{M}^{2})\rightarrow v_{M}^{1}$.  $\mathtt{tl}$ $(\mathtt{cons}$ $v_{M}^{1}$ $v_{M}^{2})\rightarrow v_{M}^{2}$.  $\mathtt{hd}$ $({^{[T]}M}H^{[T]}$ $(\mathtt{cons}$ $e_{H}^{1}$ $e_{H}^{2}))\rightarrow{^{T}M}H^{T}$ $e_{H}^{1}$.  $\mathtt{tl}$ $({^{[T]}M}H^{[T]}$ $(\mathtt{cons}$ $e_{H}^{1}$ $e_{H}^{2}))\rightarrow{^{[T]}M}H^{[T]}$ $e_{H}^{2}$.  If $e_{M}^{1}\rightarrow e_{M}^{2}$ then $f$ $e_{M}^{1}\rightarrow f$ $e_{M}^{2}$.  If $e_{M}^{1}\rightarrow$ \emph{\textbf{Error}: string} then $f$ $e_{M}^{1}\rightarrow$ \emph{\textbf{Error}: string}.

\end{case}

% o e e

\begin{case}

$e_{A}=o$ $e_{A}^{1}$ $e_{A}^{2}$ where $A\in\lbrace H,M\rbrace$

$e_{A}^{1}$ is an unforced value or $e_{A}^{1}\rightarrow e_{A}^{3}$ or $e_{A}^{1}\rightarrow$ \emph{\textbf{Error}: string} by the induction hypothesis.  If $e_{A}^{1}\rightarrow e_{A}^{3}$ then $o$ $e_{A}^{1}$ $e_{A}^{2}\rightarrow o$ $e_{A}^{3}$ $e_{A}^{2}$.  If $e_{A}^{1}\rightarrow$ \emph{\textbf{Error}: string} then $o$ $e_{A}^{1}$ $e_{A}^{2}\rightarrow$ \emph{\textbf{Error}: string}.  $e_{A}^{2}$ is an unforced value or $e_{A}^{2}\rightarrow e_{A}^{4}$ or $e_{A}^{2}\rightarrow$ \emph{\textbf{Error}: string} by the induction hypothesis.  If $e_{A}^{2}\rightarrow e_{A}^{4}$ and $e_{A}^{1}$ is an unforced value then $o$ $e_{A}^{1}$ $e_{A}^{2}\rightarrow o$ $e_{A}^{1}$ $e_{A}^{4}$.  If $e_{A}^{2}\rightarrow$ \emph{\textbf{Error}: string} and $e_{A}^{1}$ is an unforced value then $o$ $e_{A}^{1}$ $e_{A}^{2}\rightarrow$ \emph{\textbf{Error}: string}.  $e_{A}^{1}$ and $e_{A}^{2}$ are unforced values otherwise.  $e_{A}^{1}:N$ and $e_{A}^{2}:N$ by inversion (Lemma \ref{i}) and uniqueness of types (Lemma \ref{uot}) and $e_{A}^{1}=\overline{n_{1}}$ and $e_{A}^{2}=\overline{n_{2}}$ by canonical forms (Lemma \ref{cf}).  $+$ $\overline{n_{1}}$ $\overline{n_{2}}\rightarrow\overline{n_{1}+n_{2}}$.  $-$ $\overline{n_{1}}$ $\overline{n_{2}}\rightarrow\overline{max(n_{1}-n_{2},0)}$.

\end{case}

% Haskell null? e

\begin{case}

$e_{H}=\mathtt{null?}$ $e_{H}^{1}$

$e_{H}^{1}$ is an unforced value or $e_{H}^{1}\rightarrow e_{H}^{2}$ or $e_{H}^{1}\rightarrow$ \emph{\textbf{Error}: string} by the induction hypothesis.  If $e_{H}^{1}$ is an unforced value then $e_{H}^{1}:[T]$ by inversion (Lemma \ref{i}) and uniqueness of types (Lemma \ref{uot}) and $e_{H}^{1}\in\lbrace\mathtt{nil}^{T},\mathtt{cons}$ $e_{H}^{1}$ $e_{H}^{2}\rbrace$ by canonical forms (Lemma \ref{cf}).  $\mathtt{null?}$ $\mathtt{nil}^{T}\rightarrow\overline{0}$.  $\mathtt{null?}$ $(\mathtt{cons}$ $e_{H}^{1}$ $e_{H}^{2})\rightarrow\overline{1}$.  If $e_{H}^{1}\rightarrow e_{H}^{2}$ then $\mathtt{null?}$ $e_{H}^{1}\rightarrow\mathtt{null?}$ $e_{H}^{2}$.  If $e_{H}^{1}\rightarrow$ \emph{\textbf{Error}: string} then $\mathtt{null?}$ $e_{H}^{1}\rightarrow$ \emph{\textbf{Error}: string}.

\end{case}

% ML null? e

\begin{case}

$e_{M}=\mathtt{null?}$ $e_{M}^{1}$

$e_{M}^{1}$ is an unforced value or $e_{M}^{1}\rightarrow e_{M}^{2}$ or $e_{M}^{1}\rightarrow$ \emph{\textbf{Error}: string} by the induction hypothesis.  If $e_{M}^{1}$ is an unforced value then $e_{M}^{1}:[T]$ by inversion (Lemma \ref{i}) and uniqueness of types (Lemma \ref{uot}) and $e_{M}^{1}\in\lbrace\mathtt{nil}^{T},\mathtt{cons}$ $v_{M}^{1}$ $v_{M}^{2},{^{[T]}M}H^{[T]}$ $(\mathtt{cons}$ $e_{H}^{1}$ $e_{H}^{2})\rbrace$ by canonical forms (Lemma \ref{cf}).  $\mathtt{null?}$ $\mathtt{nil}^{T}\rightarrow\overline{0}$.  If $e_{M}^{1}\in\lbrace\mathtt{cons}$ $v_{M}^{1}$ $v_{M}^{2},{^{[T]}M}H^{[T]}$ $(\mathtt{cons}$ $e_{H}^{1}$ $e_{H}^{2})\rbrace$ then $\mathtt{null?}$ $e_{M}^{1}\rightarrow\overline{1}$.  If $e_{M}^{1}\rightarrow e_{M}^{2}$ then $\mathtt{null?}$ $e_{M}^{1}\rightarrow\mathtt{null?}$ $e_{M}^{2}$.  If $e_{M}^{1}\rightarrow$ \emph{\textbf{Error}: string} then $\mathtt{null?}$ $e_{M}^{1}\rightarrow$ \emph{\textbf{Error}: string}.

\end{case}

% if0 e e e

\begin{case}

$e_{A}=\mathtt{if0}$ $e_{A}^{1}$ $e_{A}^{2}$ $e_{A}^{3}$ where $A\in\lbrace H,M\rbrace$

$e_{A}^{1}$ is an unforced value or $e_{A}^{1}\rightarrow e_{A}^{4}$ or $e_{A}^{1}\rightarrow$ \emph{\textbf{Error}: string} by the induction hypothesis.  If $e_{A}^{1}$ is an unforced value then $e_{A}^{1}:N$ by inversion (Lemma \ref{i}) and uniqueness of types (Lemma \ref{uot}) and $e_{A}^{1}=\overline{n}$ by canonical forms (Lemma \ref{cf}).  $\mathtt{if0}$ $\overline{0}$ $e_{A}^{2}$ $e_{A}^{3}\rightarrow e_{A}^{2}$.  $\mathtt{if0}$ $\overline{n}$ $e_{A}^{2}$ $e_{A}^{3}\rightarrow e_{A}^{3}$ $(n\neq 0)$.  If $e_{A}^{1}\rightarrow e_{A}^{4}$ then $\mathtt{if0}$ $e_{A}^{1}$ $e_{A}^{2}$ $e_{A}^{3}\rightarrow \mathtt{if0}$ $e_{A}^{4}$ $e_{A}^{2}$ $e_{A}^{3}$.  If $e_{A}^{1}\rightarrow$ \emph{\textbf{Error}: string} then $\mathtt{if0}$ $e_{A}^{1}$ $e_{A}^{2}$ $e_{A}^{3}\rightarrow$ \emph{\textbf{Error}: string}.

\end{case}

% wrong t string

\begin{case}

$e_{A}=\mathtt{wrong}^{T}$ \emph{string} where $A\in\lbrace H,M\rbrace$

$\mathtt{wrong}^{T}$ $\mathrm{string}\rightarrow$ \emph{\textbf{Error}: string}.

\end{case}

% hm t t e

\begin{case}

$e_{H}={^{T}H}M$ $e_{M}^{1}$

$e_{M}^{1}$ is an unforced value or $e_{M}^{1}\rightarrow e_{M}^{2}$ or $e_{M}^{1}\rightarrow$ \emph{\textbf{Error}: string} by the induction hypothesis.  If $e_{M}^{1}\rightarrow e_{M}^{2}$ then $^{T}HM$ $e_{M}^{1}\rightarrow{^{T}H}M$ $e_{M}^{2}$.  If $e_{M}^{1}\rightarrow$ \emph{\textbf{Error}: string} then $^{T}HM$ $e_{M}^{1}\rightarrow$ \emph{\textbf{Error}: string}.  If $e_{M}^{1}$ is an unforced value then $T$ determines the reduction of $^{T}HM$ $e_{M}^{1}$:

\begin{subcase}

$T=L$

$e_{M}^{1}={^{L}M}S$ $v_{S}$ by canonical forms (Lemma \ref{cf}).  $^{L}HM$ $(^{L}MS$ $v_{S})\rightarrow{^{L}H}S$ $v_{S}$.

\end{subcase}

\begin{subcase}

$T=N$

$e_{M}^{1}=\overline{n}$ by canonical forms (Lemma \ref{cf}).  $^{N}HM$ $\overline{n}\rightarrow\overline{n}$.

\end{subcase}

\begin{subcase}

$T=[T_{1}]$

$e_{M}^{1}\in\lbrace\mathtt{nil}^{T_{1}},\mathtt{cons}$ $v_{M}^{1}$ $v_{M}^{2},{^{[T_{1}]}M}H$ $(\mathtt{cons}$ $e_{H}^{1}$ $e_{H}^{2})\rbrace$ by canonical forms (Lemma \ref{cf}).  $^{[T_{1}]}HM$ $\mathtt{nil}^{T}\rightarrow\mathtt{nil}^{T}$.  $^{[T_{1}]}HM$ $(\mathtt{cons}$ $v_{M}^{1}$ $v_{M}^{2})\rightarrow\mathtt{cons}$ $(^{T_{1}}HM$ $v_{M}^{1})$ $(^{[T_{1}]}HM$ $v_{M}^{2})$.  $^{[T_{1}]}HM$ $(^{[T_{1}]}MH$ $(\mathtt{cons}$ $e_{H}^{1}$ $e_{H}^{2}))\rightarrow\mathtt{cons}$ $e_{H}^{1}$ $e_{H}^{2}$.

\end{subcase}

\begin{subcase}

$T=T_{1}^{a}$

Cannot occur because $T_{1}^{a}$ occurs only in $^{T_{1}^{a}}HS$ $e_{S}$.

\end{subcase}

\begin{subcase}

$T=T_{1}\rightarrow T_{2}$

$e_{M}^{1}=\lambda x_{1}:T_{1}.e_{M}^{3}$ by canonical forms (Lemma \ref{cf}).  $^{T_{1}\rightarrow T_{2}}HM$ $(\lambda x_{1}:T_{1}.e_{M}^{3})\rightarrow\lambda x_{2}:T_{1}.(^{T_{2}}HM$ $((\lambda x_{1}:T_{1}.e_{M}^{3})$ $(^{T_{1}}MH$ $x_{2})))$.

\end{subcase}

\begin{subcase}

$T=\forall X.T_{1}$

$e_{M}^{1}\in\lbrace\Lambda X.e_{M}^{3},{^{\forall X.T_{1}}M}S$ $v_{S}\rbrace$ by canonical forms (Lemma \ref{cf}).  $^{\forall X.T_{1}}HM$ $(\Lambda X.e_{M}^{3})\rightarrow\Lambda X.(^{T_{1}}HM$ $e_{M}^{3})$.  $^{\forall X.T_{1}}HM$ $(^{\forall X.T_{1}}MS$ $v_{S})\rightarrow{^{\forall X.T_{1}}H}S$ $v_{S}$.

\end{subcase}

\end{case}

% mh t t e

\begin{case}

$e_{M}={^{T}M}H$ $e_{H}$

$^{T}MH$ $e_{H}$ is an unforced value.

\end{case}

% hs k e

\begin{case}

$e_{A}={^{T}A}S$ $e_{S}^{1}$ where $A\in\lbrace H,M\rbrace$

$e_{S}^{1}$ is an unforced value or $e_{S}^{1}\rightarrow e_{S}^{2}$ or $e_{S}^{1}\rightarrow$ \emph{\textbf{Error}: string} by Scheme progress (Theorem \ref{sps}).  If $e_{S}^{1}\rightarrow e_{S}^{2}$ then $^{T}AS$ $e_{S}^{1}\rightarrow{^{T}A}S$ $e_{S}^{2}$.  If $e_{S}^{1}\rightarrow$ \emph{\textbf{Error}: string} then $^{T}AS$ $e_{S}^{1}\rightarrow$ \emph{\textbf{Error}: string}.  If $e_{S}^{1}$ is an unforced value then $T$ determines the reduction of $^{T}AS$ $e_{S}^{1}$:

\begin{subcase}

$T=L$

$^{L}AS$ $e_{S}^{1}$ is an unforced value.

\end{subcase}

\begin{subcase}

$T=N$

$^{N}AS$ $\overline{n}\rightarrow\overline{n}$.  $^{N}AS$ $e_{S}^{1}\rightarrow\mathtt{wrong}^{N}$ \emph{``Not a number"} $(e_{S}^{1}\neq\overline{n})$.

\end{subcase}

\begin{subcase}

$T=[T_{1}]$

$^{[T_{1}]}AS$ $\mathtt{nil}\rightarrow\mathtt{nil}^{T_{1}}$.  $^{[T_{1}]}AS$ $(\mathtt{cons}$ $v_{S}^{1}$ $v_{S}^{2})\rightarrow\mathtt{cons}$ $(^{T_{1}}AS$ $v_{S}^{1})$ $(^{[T_{1}]}AS$ $v_{S}^{2})$.  $^{[T_{1}]}HS$ $(SH^{[T_{1}]}$ $(\mathtt{cons}$ $e_{H}^{1}$ $e_{H}^{2}))\rightarrow\mathtt{cons}$ $e_{H}^{1}$ $e_{H}^{2}$.  $^{[T_{1}]}MS$ $(SH^{[T_{1}]}$ $(\mathtt{cons}$ $e_{H}^{1}$ $e_{H}^{2}))\rightarrow{^{[T_{1}]}M}H^{[T_{1}]}$ $(\mathtt{cons}$ $e_{H}^{1}$ $e_{H}^{2})$.  $^{[T_{1}]}AS$ $e_{S}^{1}\rightarrow\mathtt{wrong}^{[T_{1}[T_{i}/T_{i}^{a}]]}$ \emph{``Not a list"} $(e_{S}^{1}\not\in\lbrace\mathtt{nil},\mathtt{cons}$ $v_{S}^{1}$ $v_{S}^{2},SH^{[T_{1}]}$ $(\mathtt{cons}$ $e_{H}^{1}$ $e_{H}^{2})\rbrace)$.

\end{subcase}

\begin{subcase}

$T=T_{1}^{a}$

$^{T_{1}^{a}}HS$ $(SH^{T_{1}^{a}}$ $e_{H})\rightarrow e_{H}$.  $^{T_{1}^{a}}HS$ $e_{S}^{1}\rightarrow\mathtt{wrong}^{T_{1}}$ \emph{``Parametricity violated"} $(e_{S}^{1}\neq SH^{T_{1}^{a}}$ $e_{H})$.  $^{T_{1}^{a}}MS$ $(SM^{T_{1}^{a}}$ $v_{M})\rightarrow v_{M}$.  $^{T_{1}^{a}}MS$ $e_{S}^{1}\rightarrow\mathtt{wrong}^{T_{1}}$ \emph{``Parametricity violated"} $(e_{S}^{1}\neq SM^{T_{1}^{a}}$ $v_{M})$.

\end{subcase}

\begin{subcase}

$T=T_{1}\rightarrow T_{2}$

$^{T_{1}\rightarrow T_{2}}AS$ $(\lambda x_{1}.e_{S}^{3})\rightarrow\lambda x_{2}:T_{1}[T_{i}/T^{a}_{i}].(^{T_{2}}AS$ $((\lambda x_{1}.e_{S}^{3})$ $(SA^{T_{1}}$ $x_{2})))$.  $^{T_{1}\rightarrow T_{2}}AS$ $e_{S}^{1}\rightarrow\mathtt{wrong}^{(T_{1}\rightarrow T_{2})[T_{i}/T_{i}^{a}]]}$ \emph{``Not a function"} $(e_{S}^{1}\neq\lambda x_{1}.e_{S}^{3})$.

\end{subcase}

\begin{subcase}

$T=\forall X.T_{1}$

$^{\forall X.T_{1}}AS$ $e_{S}^{1}$ is an unforced value.

\end{subcase}

\end{case}

% ML

% cons v v

\newcommand{\psconsm}{\expcons{\first{\varvalum}}{\second{\varvalum}}\xspace}

\begin{case}

\psconsm

\psconsm is a \profv.

\end{case}

% ML e e

\begin{case}

$e_{M}=e_{M}^{1}$ $e_{M}^{2}$

$e_{M}^{1}$ is an unforced value or $e_{M}^{1}\rightarrow e_{M}^{3}$ or $e_{M}^{1}\rightarrow$ \emph{\textbf{Error}: string} by the induction hypothesis.  If $e_{M}^{1}$ is an unforced value then $e_{M}^{1}:T_{1}\rightarrow T_{2}$ by inversion (Lemma \ref{i}) and uniqueness of types (Lemma \ref{uot}) and $e_{M}^{1}=\lambda x:T_{1}.e_{M}^{4}$ by canonical forms (Lemma \ref{cf}).  If $e_{M}^{1}\rightarrow e_{M}^{3}$ then $e_{M}^{1}$ $e_{M}^{2}\rightarrow e_{M}^{3}$ $e_{M}^{2}$.  If $e_{M}^{1}\rightarrow$ \emph{\textbf{Error}: string} then $e_{M}^{1}$ $e_{M}^{2}\rightarrow$ \emph{\textbf{Error}: string}.  $e_{M}^{2}$ is an unforced value or $e_{M}^{2}\rightarrow e_{M}^{5}$ or $e_{M}^{2}\rightarrow$ \emph{\textbf{Error}: string} by the induction hypothesis.  If $e_{M}^{2}\rightarrow e_{M}^{5}$ and $e_{M}^{1}$ is an unforced value then $e_{M}^{1}$ $e_{M}^{2}\rightarrow e_{M}^{1}$ $e_{M}^{5}$.  If $e_{M}^{2}\rightarrow$ \emph{\textbf{Error}: string} and $e_{M}^{1}$ is an unforced value then $e_{M}^{1}$ $e_{M}^{2}\rightarrow$ \emph{\textbf{Error}: string}.  If $e_{M}^{1}$ is an unforced value and $e_{M}^{2}$ is an unforced value then $(\lambda x:T_{1}.e_{M}^{4})$ $e_{M}^{2}\rightarrow e_{M}^{4}[e_{M}^{2}/x]$.

\end{case}

\end{proof}

\end{theorem}


\subsection{Scheme Progress Theorem}

\begin{theorem}{Scheme Progress}

\label{thmpss}

If \judes{}{\first{\varexps}}{\tytst} then \pshyp{\first{\varexps}}{\second{\varexps}}.

\begin{proof}

By structural induction on \first{\varexps}.  Cases similar to Haskell and ML cases are omitted.

% cons w w

\begin{case}

$e_{S}=\mathtt{cons}$ $v_{S}^{1}$ $v_{S}^{2}$

$\mathtt{cons}$ $v_{S}^{1}$ $v_{S}^{2}$ is an unforced value.

\end{case}

% sh k e

\begin{case}

$e_{S}=SH^{T}$ $e_{H}$

$SH^{T}$ $e_{H}$ is an unforced value.

\end{case}

% e e

\begin{case}

$e_{S}=e_{S}^{1}$ $e_{S}^{2}$

$e_{S}^{1}$ is an unforced value or $e_{S}^{1}\rightarrow e_{S}^{3}$ or $e_{S}^{1}\rightarrow$ \emph{\textbf{Error}: string} by the induction hypothesis.  If $e_{S}^{1}\rightarrow e_{S}^{3}$ then $e_{S}^{1}$ $e_{S}^{2}\rightarrow e_{S}^{3}$ $e_{S}^{2}$.  If $e_{S}^{1}\rightarrow$ \emph{\textbf{Error}: string} then $e_{S}^{1}$ $e_{S}^{2}\rightarrow$ \emph{\textbf{Error}: string}.  $e_{S}^{2}$ is an unforced value or $e_{S}^{2}\rightarrow e_{S}^{4}$ or $e_{S}^{2}\rightarrow$ \emph{\textbf{Error}: string} by the induction hypothesis.  If $e_{S}^{2}\rightarrow e_{S}^{4}$ and $e_{S}^{1}$ is an unforced value then $e_{S}^{1}$ $e_{S}^{2}\rightarrow e_{S}^{1}$ $e_{S}^{4}$.  If $e_{S}^{2}\rightarrow$ \emph{\textbf{Error}: string} and $e_{S}^{1}$ is an unforced value then $e_{S}^{1}$ $e_{S}^{2}\rightarrow$ \emph{\textbf{Error}: string}.  $e_{S}^{1}$ and $e_{S}^{2}$ are unforced values otherwise.  $(\lambda x.e_{S}^{5})$ $e_{S}^{2}\rightarrow e_{S}^{5}[e_{S}^{2}/x]$.  $e_{S}^{1}$ $e_{S}^{2}\rightarrow\mathtt{wrong}$ \emph{``Not a function"} $(e_{S}^{1}\neq\lambda x.e_{S}^{5})$.

\end{case}

% cons e e

\begin{case}

$e_{S}=\mathtt{cons}$ $e_{S}^{1}$ $e_{S}^{2}$

$e_{S}^{1}$ is an unforced value or $e_{S}^{1}\rightarrow e_{S}^{3}$ or $e_{S}^{1}\rightarrow$ \emph{\textbf{Error}: string} by the induction hypothesis.  If $e_{S}^{1}\rightarrow e_{S}^{3}$ then $\mathtt{cons}$ $e_{S}^{1}$ $e_{S}^{2}\rightarrow\mathtt{cons}$ $e_{S}^{3}$ $e_{S}^{2}$.  If $e_{S}^{1}\rightarrow$ \emph{\textbf{Error}: string} then $\mathtt{cons}$ $e_{S}^{1}$ $e_{S}^{2}\rightarrow$ \emph{\textbf{Error}: string}.  $e_{S}^{2}$ is an unforced value or $e_{S}^{2}\rightarrow e_{S}^{4}$ or $e_{S}^{1}\rightarrow$ \emph{\textbf{Error}: string} by the induction hypothesis.  If $e_{S}^{2}\rightarrow e_{S}^{4}$ and $e_{M}^{1}$ is an unforced value then $\mathtt{cons}$ $e_{S}^{1}$ $e_{S}^{2}\rightarrow\mathtt{cons}$ $e_{S}^{1}$ $e_{S}^{4}$.  If $e_{S}^{2}\rightarrow$ \emph{\textbf{Error}: string} and $e_{S}^{1}$ is an unforced value then $\mathtt{cons}$ $e_{S}^{1}$ $e_{S}^{2}\rightarrow$ \emph{\textbf{Error}: string}.  If $e_{S}^{1}$ and $e_{S}^{2}$ are unforced values then $\mathtt{cons}$ $e_{S}^{1}$ $e_{S}^{2}$ is an unforced value.

\end{case}

% sm k e

\begin{case}

$e_{S}=SM^{T}$ $e_{M}^{1}$

$e_{M}^{1}$ is an unforced value or $e_{M}^{1}\rightarrow e_{M}^{2}$ or $e_{M}^{1}\rightarrow$ \emph{\textbf{Error}: string} by ML progress (Theorem \ref{hmps}).  If $e_{M}^{1}\rightarrow e_{M}^{2}$ then $SM^{T}$ $e_{M}^{1}\rightarrow SM^{T}$ $e_{M}^{2}$.  If $e_{M}^{1}\rightarrow$ \emph{\textbf{Error}: string} then $SM^{T}$ $e_{M}^{1}\rightarrow$ \emph{\textbf{Error}: string}.  If $e_{M}^{1}$ is an unforced value then $T$ determines the reduction of $SM^{T}$ $e_{M}^{1}$:

\begin{subcase}

$T=L$

$e_{M}^{1}={^{L}M}S$ $v_{S}$ by canonical forms (Lemma \ref{cf}).  $SM^{L}$ $(^{L}MS$ $v_{S})\rightarrow v_{S}$.

\end{subcase}

\begin{subcase}

$T=N$

$e_{M}^{1}=\overline{n}$ by canonical forms (Lemma \ref{cf}).  $SM^{N}$ $\overline{n}\rightarrow\overline{n}$.

\end{subcase}

\begin{subcase}

$T=[T_{1}]$

$e_{M}^{1}\in\lbrace\mathtt{nil}^{T_{1}[T_{i}/T_{i}^{a}]},\mathtt{cons}$ $v_{M}^{1}$ $v_{M}^{2},{^{[T_{1}]}M}H^{[T_{1}]}$ $(\mathtt{cons}$ $e_{H}^{1}$ $e_{H}^{2})\rbrace$ by canonical forms (Lemma \ref{cf}).  $SM^{T_{1}}$ $\mathtt{nil}^{T_{1}}\rightarrow\mathtt{nil}$.  $SM^{[T_{1}]}$ $(\mathtt{cons}$ $v_{M}^{1}$ $v_{M}^{2})\rightarrow\mathtt{cons}$ $(SM^{T_{1}}$ $v_{M}^{1})$ $(SM^{[T_{1}]}$ $v_{M}^{2})$.  $SM^{[T_{1}]}$ $({^{[T_{1}]}M}H^{[T_{1}]}$ $(\mathtt{cons}$ $e_{H}^{1}$ $e_{H}^{2}))\rightarrow SH^{[T_{1}]}$ $(\mathtt{cons}$ $e_{H}^{1}$ $e_{H}^{2})$.

\end{subcase}

\begin{subcase}

$T=T_{1}^{a}$

$SM^{T_{1}^{a}}$ $e_{M}^{1}$ is an unforced value.

\end{subcase}

\begin{subcase}

$T=T_{1}\rightarrow T_{2}$

$e_{M}^{1}=\lambda x_{1}:T_{1}[T_{i}/T_{i}^{a}].e_{M}^{3}$ by canonical forms (Lemma \ref{cf}).  $SM^{T_{1}\rightarrow T_{2}}$ $(\lambda x_{1}:T_{1}[T_{i}/T_{i}^{a}].e_{M}^{3})\rightarrow\lambda x_{2}.(SM^{T_{2}}$ $((\lambda x_{1}:T_{1}[T_{i}/T_{i}^{a}].e_{M}^{3})$ $(^{T_{1}}MS$ $x_{2})))$.

\end{subcase}

\begin{subcase}

$T=\forall X.T_{1}$

$e_{M}^{1}\in\lbrace\Lambda X.e_{M}^{3},{^{\forall X.T_{1}}M}S$ $v_{S}\rbrace$ by canonical forms (Lemma \ref{cf}).  $SM^{\forall X.T_{1}}$ $(\Lambda X.e_{M}^{3})\rightarrow SM^{T_{1}[L/X]}$ $((\Lambda X.e_{M}^{3})$ $\lbrace L\rbrace)$.  $SM^{\forall X.T_{1}}$ $(^{\forall X.T_{1}}MS$ $v_{S})\rightarrow v_{S}$.

\end{subcase}

\end{case}

\end{proof}

\end{theorem}


\section{Proof of Type Preservation}

Preservation will be proven by cases on the reduction rules.  In each case, the right side will be proven to be well-typed and have the same type as the left side.  Inversion (Lemma \ref{i}) and uniqueness of types (Lemma \ref{uot}) are used to determine the types of the left side and its subexpressions and the type of the right side.  Some reduction rules use expression and type substitutions.

\subsection{Expression Substitution Lemma}

If $e_{A}^{1}$ is substituted for free occurrences of $x$ within $e_{A}^{2}$, $e_{A}^{1}$ and $x$ have the same type, and the result has the same type as $e_{A}^{2}$, where $A\in\lbrace H,M,S\rbrace$.

\begin{lemma}{Expression Substitution Preservation}

\label{lemexp}

If \judeh{\envexte{\first{\varvarh}}{\first{\vartyh}}}{\first{\varexph}}{\second{\vartyh}} and \judeh{\env}{\second{\varexph}}{\first{\vartyh}} then \judeh{\env}{\expsubst{\first{\varexph}}{\second{\varexph}}{\first{\varvarh}}}{\second{\vartyh}}.  If \judem{\envexte{\first{\varvarm}}{\first{\vartym}}}{\first{\varexpm}}{\second{\vartym}} and \judem{\env}{\second{\varexpm}}{\first{\vartym}} then \judem{\env}{\expsubst{\first{\varexpm}}{\second{\varexpm}}{\first{\varvarm}}}{\second{\vartym}}.  If \judes{\envexte{\first{\varvars}}{\tytst}}{\first{\varexps}}{\tytst} and \judes{\env}{\second{\varexps}}{\tytst} then \judes{\env}{\expsubst{\first{\varexps}}{\second{\varexps}}{\first{\varvars}}}{\tytst}.

\begin{proof}

By structural induction.

\end{proof}

\end{lemma}


\subsection{Type Substitution Lemma}

If $T_{1}$ is substituted for free occurrences of $X$ within $e_{A}$ of type $T_{2}$, the type of the result is $T_{1}$ substituted for free occurrences of $X$ within $T_{2}$, where $A\in\lbrace H,M\rbrace$.

\begin{lemma}
\label{tes}
If $\Gamma,X\vdash_{HM}e_{HM}:T_{1}$ and $\vdash_{HM}T_{2}$ then $\Gamma\vdash_{HM}e_{HM}[T_{2}/X]:T_{1}[T_{2}/X]$.
\begin{proof}
By structural induction.
\end{proof}
\end{lemma}

\subsection{Evaluation Context Lemma}

\begin{lemma}{Evaluation Context Preservation}

\label{lemeva}

If $\Gamma\vdash_{A}e_{A}^{1}:T_{1}$, $\Gamma\vdash_{A}e_{A}^{2}:T_{1}$, and $\mathscr{E}[e_{A}^{1}]:T_{2}$ then $\mathscr{E}[e_{A}^{2}]:T_{2}$ where $A\in\lbrace H,M,S\rbrace$.

\begin{proof}

By structural induction.

\end{proof}

\end{lemma}


\subsection{Preservation Theorem}

\begin{theorem}
\label{pn}
\onehalfspacing
If $\Gamma\vdash_{HMS}e_{HMS}^{1}:T$ and $e_{HMS}^{1}\rightarrow e_{HMS}^{2}$ then $\Gamma\vdash_{HMS}e_{HMS}^{2}:T$.
\begin{proof}
By cases on the reduction $e_{HMS}^{1}\rightarrow e_{HMS}^{2}$.  Scheme reductions that do not contain Haskell or ML terms are omitted because demonstrating the preservation of $TST$ is straightforward.
\begin{case}
$(\lambda x:T_{1}.e_{HM}^{1})\;e_{HM}^{2}\rightarrow e_{HM}^{1}[e_{HM}^{2}/x]$

$\Gamma\vdash_{HM}(\lambda x:T_{1}.e_{HM}^{1})\;e_{HM}^{2}:T$ by the premise and uniqueness of types (Lemma \ref{uot}).  $\Gamma\vdash_{HM}\lambda x:T_{1}.e_{HM}^{1}:T_{1}\rightarrow T$, $\Gamma,x:T_{1}\vdash_{HM}e_{HM}^{1}:T$, $\Gamma\vdash_{HM}e_{HM}^{2}:T_{1}$, and $\Gamma,x:T_{1}\vdash_{HM}x:T_{1}$ by inversion (Lemma \ref{i}) and uniqueness of types.  $e_{HM}^{1}[e_{HM}^{2}/x]:T$ by term substitution (Lemma \ref{tms}).
\end{case}
\begin{case}
$(\Lambda X.e_{HM})\;\lbrace T_{1}\rbrace\rightarrow e_{HM}[T_{1}/X]$

$\Gamma\vdash_{HM}(\Lambda X.e_{H})\;\lbrace T_{1}\rbrace:T$ by premise and uniqueness of types (Lemma \ref{uot}).  $\Gamma\vdash_{HM}\Lambda X.e_{HM}:\forall X.T_{2}$, $\Gamma,X\vdash_{HM}e_{HM}:T_{2}$, and $T=T_{2}[T_{1}/X]$ by inversion (Lemma \ref{i}) and uniqueness of types.  $\Gamma\vdash_{HM}e_{HM}[T_{1}/X]:T_{2}[T_{1}/X]$ by type substitution (Lemma \ref{tes}).  $\Gamma\vdash_{HM}e_{HM}[T_{1}/X]:T$ because $T=T_{2}[T_{1}/X]$.
\end{case}
\begin{case}
$\mathtt{if0}\;\overline{0}\;e_{HM}^{1}\;e_{HM}^{2}\rightarrow e_{HM}^{1}$

$\Gamma\vdash_{HM}\mathtt{if0}\;\overline{0}\;e_{HM}^{1}\;e_{HM}^{2}:T$ by premise and uniqueness of types (Lemma \ref{uot}).  $\Gamma\vdash_{HM}e_{HM}^{1}:T$ by inversion (Lemma \ref{i}) and uniqueness of types.
\end{case}
\begin{case}
$\mathtt{if0}\;\overline{n}\;e_{HM}^{1}\;e_{HM}^{2}\rightarrow e_{HM}^{2}\;(n\neq0)$

$\Gamma\vdash_{HM}\mathtt{if0}\;\overline{n}\;e_{HM}^{1}\;e_{HM}^{2}:T$ by premise and uniqueness of types (Lemma \ref{uot}).  $\Gamma\vdash_{HM}e_{HM}^{2}:T$ by inversion (Lemma \ref{i}) and uniqueness of types.
\end{case}
\begin{case}
$+\;\overline{n_{1}}\;\overline{n_{2}}\rightarrow\overline{n_{1}+n_{2}}$

$\vdash_{HM}+\;\overline{n_{1}}\;\overline{n_{2}}:N$ by inversion (Lemma \ref{i}) and uniqueness of types (Lemma \ref{uot}).  $\vdash_{HM}\overline{n_{1}+n_{2}}:N$ by inversion and uniqueness of types.
\end{case}
\begin{case}
$-\;\overline{n_{1}}\;\overline{n_{2}}\rightarrow\overline{max(n_{1}-n_{2},0)}$ where $A\in\lbrace H,M\rbrace$

$\vdash_{A}-\;\overline{n_{1}}\;\overline{n_{2}}:N$ by inversion (Lemma \ref{i}) and uniqueness of types (Lemma \ref{uot}).  $\vdash_{A}\overline{max(n_{1}-n_{2},0)}:N$ by inversion and uniqueness of types.
\end{case}
\input{cases/preservation/head-cons.tex}
\begin{case}
$\mathtt{tl}\;(\mathtt{cons}\;e_{HM}^{1}\;e_{HM}^{2})\rightarrow e_{HM}^{2}$

$\Gamma\vdash_{HM}\mathtt{tl}\;(\mathtt{cons}\;e_{HM}^{1}\;e_{HM}^{2}):T$ by premise and uniqueness of types (Lemma \ref{uot}).  $\Gamma\vdash_{HM}e_{HM}^{2}:T$ by inversion and uniqueness of types (Lemma \ref{uot}).
\end{case}
\begin{case}
$\mathtt{hd}\;\mathtt{nil}^{T_{1}}\rightarrow\,^{T}HS\;(\mathtt{wrong}\;\mathrm{``Empty\;list"})$

$\Gamma\vdash_{HM}\mathtt{hd}\;\mathtt{nil}^{T_{1}}:T$ by premise and uniqueness of types (Lemma \ref{uot}).  $\Gamma\vdash_{HM}\,^{T}HS\;(\mathtt{wrong}\;\mathrm{``Empty\;list"}):T$ by inversion (Lemma \ref{i}) and uniqueness of types (Lemma \ref{uot}).
\end{case}
\begin{case}
$\mathtt{tl}\;\mathtt{nil}^{T_{1}}\rightarrow\mathtt{nil}^{T_{1}}$ where $A\in\lbrace H,M\rbrace$

$\Gamma\vdash_{A}\mathtt{tl}\;\mathtt{nil}^{T_{1}}:T$ by premise and uniqueness of types (Lemma \ref{uot}).  $\Gamma\vdash_{A}\mathtt{nil}^{T_{1}}:[T_{1}]$ and $T=[T_{1}]$ by inversion (Lemma \ref{i}) and uniqueness of types.  $\Gamma\vdash_{A}\mathtt{nil}^{T_{1}}:T$ because $[T_{1}]=T$.
\end{case}
\begin{case}
$\mathtt{fix}\;(\lambda x:T_{1}.e_{A})\rightarrow e_{A}[(\mathtt{fix}\;(\lambda x:T_{1}.e_{A}))/x]$ where $A\in\lbrace H,M\rbrace$

$\Gamma\vdash_{A}\mathtt{fix}\;(\lambda x:T_{1}.e_{A}):T$ by premise and uniqueness of types (Lemma \ref{uot}).  $T=T_{1}$, $\Gamma,x:T_{1}\vdash_{A}e_{A}:T_{1}$, and $\Gamma,x:T_{1}\vdash_{A}x:T_{1}$ by inversion (Lemma \ref{i}) and uniqueness of types.  $\Gamma\vdash_{A}e_{A}[(\mathtt{fix}\;(\lambda x:T_{1}.e_{A}))/x]:T_{1}$ by term substitution (Lemma \ref{tms}).  $\Gamma\vdash_{A}e_{A}[(\mathtt{fix}\;(\lambda x:T_{1}.e_{A}))/x]:T$ because $T_{1}=T$.
\end{case}
\begin{case}

$e_{A}=\overline{n}$ where $A\in\lbrace H,M\rbrace$

$\overline{n}$ is an unforced value.

\end{case}
\begin{case}
$^{\forall X_{1}.T}B^{\forall X_{1}.T}\;(\Lambda X_{1}.e_{HM})\rightarrow\Lambda X_{2}.(^{T[X_{2}/X_{1}]}B^{T[X_{2}/X_{1}]}\;((\Lambda X_{1}.e_{HM})\;\lbrace X_{2}\rbrace))$ where $B\in\lbrace HM,MH\rbrace$

$\Gamma\vdash_{HM}\,^{\forall X_{1}.T}B^{\forall X_{1}.T}\;(\Lambda X_{1}.e_{HM}):\forall X_{1}.T$ by premise and inversion (Lemma \ref{i}) and uniqueness of types (Lemma \ref{uot}).  NOT DONE.
\end{case}
\begin{case}
$e_{S}=\lambda x.e_{S}^{1}$

$\lambda x.e_{S}^{1}$ is a forced value.
\end{case}
\begin{case}
$^{[T]}HM^{[T]}\;(\mathtt{cons}\;v_{M}^{1}\;v_{M}^{2})\rightarrow\mathtt{cons}\;(^{T}HM^{T}\;v_{M}^{1})\;(^{[T]}HM^{[T]}\;v_{M}^{2})$

$\Gamma\vdash_{H}^{[T]}HM^{[T]}\;(\mathtt{cons}\;v_{M}^{1}\;v_{M}^{2}):[T]$ by premise and inversion (Lemma \ref{i}) and uniqueness of types (Lemma \ref{uot}).  $\Gamma\vdash_{M}v_{M}^{1}:T$ and $\Gamma\vdash_{M}v_{M}^{1}:[T]$ by inversion (Lemma \ref{i}) and uniqueness of types (Lemma \ref{uot}).  $\Gamma\vdash_{H}\,^{T}HM^{T}\;v_{M}^{1}:T$ and $\Gamma\vdash_{H}\,^{[T]}HM^{[T]}\;v_{M}^{2}:[T]$ by the induction hypothesis and uniqueness of types (Lemma \ref{uot}).  $\Gamma\vdash_{H}\mathtt{cons}\;(^{T}HM^{T}\;v_{M}^{1})\;(^{[T]}HM^{[T]}\;v_{M}^{2}):[T]$.
\end{case}
\input{cases/preservation/list-cons-m.tex}
\input{cases/preservation/list-nil.tex}
%\input{cases/preservation/.tex}
\end{proof}
\end{theorem}