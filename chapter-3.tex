\chapter{Proof of Type Soundness}

The soundness of the type system must be proven for the model to be useful.  Soundness is ensured if progress of terms and preservation of types are ensured.  Progress ensures that a well-typed, closed term either is a value, reduces to another term, or reduces to an error.  Preservation ensures that if a well-typed term reduces to another term, the other term is well-typed and has the same type.  Proving progress and preservation proves soundness.  The proof extends the proof in Kinghorn \cite{kinghorn07}, which was based on proofs by Pierce \cite{pierce02} and Matthews and Findler \cite{matthews07}.

\section{Progress}

Progress will be proven by structural induction on a term of each syntactic form.  In each case, the term will be proven to be either a value, reducible to another term, or reducible to an error.  If a subterm is reducible to another term, the reduction of the term substitutes the new subterm for the old subterm.  If a subterm is reducible to an error, the term reduces to the error.  If no subterms are reducible but the term is reducible, the syntactic forms of its subterms must be determined to perform the reduction.  Since the term is well-typed, its subterms are well-typed, and thus the typing relations can be inverted to calculate its type from the types of its subterms.

\begin{lemma}
\label{i}
\onehalfspacing
The type of a term of each syntactic form can be calculated from the types of its subterms.
\begin{enumerate}
\item If $\Gamma\vdash_{A}x:T$ then $x:T_{1}\in\Gamma$ and $T=T_{1}$ where $A\in\lbrace H,M\rbrace$.
\item If $\Gamma\vdash_{S}x:TST$ then $x:TST\in\Gamma$.
\item If $\vdash_{A}\overline{n}:T$ then $T=N$ where $A\in\lbrace H,M\rbrace$.
\item $\vdash_{S}\overline{n}:TST$.
\item If $\Gamma\vdash_{A}\lambda x:T_{1}.e_{A}:T$ then $\Gamma\vdash_{A}T_{1}$, $\Gamma,x:T_{1}\vdash_{A}e_{A}:T_{2}$, and $T=T_{1}\rightarrow T_{2}$ where $A\in\lbrace H,M\rbrace$.
\item If $\Gamma\vdash_{S}\lambda x.e_{S}:TST$ then $\Gamma,x:TST\vdash_{S}e_{S}:TST$.
\item If $\Gamma\vdash_{A}\Lambda X.e_{A}:T$ then $\Gamma,X\vdash_{A}e_{A}:T_{1}$ and $T=\forall X.T_{1}$ where $A\in\lbrace H,M\rbrace$.
\item If $\Gamma\vdash_{A}\mathtt{cons}\;e_{A}^{1}\;e_{A}^{2}:T$ then $\Gamma\vdash_{A}e_{A}^{1}:T_{1}$, $\Gamma\vdash_{A}e_{A}^{2}:[T_{1}]$, and $T=[T_{1}]$ where $A\in\lbrace H,M\rbrace$.
\item If $\Gamma\vdash_{S}\mathtt{cons}\;e_{S}^{1}\;e_{S}^{2}:TST$ then $\Gamma\vdash_{S}e_{S}^{1}:TST$ and $\Gamma\vdash_{S}e_{S}^{2}:TST$.
\item If $\Gamma\vdash_{A}\mathtt{nil}^{T_{1}}:T$ then $\Gamma\vdash_{A}T_{1}$ and $T=[T_{1}]$ where $A\in\lbrace H,M\rbrace$.
\item $\vdash_{S}\mathtt{nil}:TST$.
\item If $\Gamma\vdash_{A}e_{A}^{1}\;e_{A}^{2}:T$ then $\Gamma\vdash_{A}e_{A}^{1}:T_{1}\rightarrow T_{2}$, $\Gamma\vdash_{A}e_{A}^{2}:T_{1}$, and $T=T_{2}$ where $A\in\lbrace H,M\rbrace$.
\item If $\Gamma\vdash_{S}e_{S}^{1}\;e_{S}^{2}:TST$ then $\Gamma\vdash_{S}e_{S}^{1}:TST$ and $\Gamma\vdash_{S}e_{S}^{2}:TST$.
\item If $\Gamma\vdash_{A}e_{A}\;\lbrace T_{1}\rbrace:T$ then $\Gamma\vdash_{A}T_{1}$, $\Gamma\vdash_{A}e_{A}:\forall X.T_{2}$, and $T=T_{2}[T_{1}/X]$ where $A\in\lbrace H,M\rbrace$.
\item If $\Gamma\vdash_{A}\mathtt{if0}\;e_{A}^{1}\;e_{A}^{2}\;e_{A}^{3}:T$ then $\Gamma\vdash_{A}e_{A}^{1}:N$, $\Gamma\vdash_{A}e_{A}^{2}:T_{1}$, $\Gamma\vdash_{A}e_{A}^{3}:T_{1}$, and $T=T_{1}$ where $A\in\lbrace H,M\rbrace$.
\item If $\Gamma\vdash_{S}\mathtt{if0}\;e_{S}^{1}\;e_{S}^{2}\;e_{S}^{3}:TST$ then $\Gamma\vdash_{S}e_{S}^{1}:TST$, $\Gamma\vdash_{S}e_{S}^{2}:TST$, and $\Gamma\vdash_{S}e_{S}^{3}:TST$.
\item If $\Gamma\vdash_{A}o\;e_{A}^{1}\;e_{A}^{2}:T$ then $\Gamma\vdash_{A}e_{A}^{1}:N$, $\Gamma\vdash_{A}e_{A}^{2}:N$, and $T=N$ where $A\in\lbrace H,M\rbrace$.
\item If $\Gamma\vdash_{S}o\;e_{S}^{1}\;e_{S}^{2}:TST$ then $\Gamma\vdash_{S}e_{S}^{1}:TST$ and $\Gamma\vdash_{S}e_{S}^{2}:TST$.
\item If $\Gamma\vdash_{A}\mathtt{hd}\;e_{A}:T$ then $\Gamma\vdash_{A}e_{A}:[T_{1}]$ and $T=T_{1}$ where $A\in\lbrace H,M\rbrace$.
\item If $\Gamma\vdash_{A}\mathtt{tl}\;e_{A}:T$ then $\Gamma\vdash_{A}e_{A}:[T_{1}]$ and $T=[T_{1}]$ where $A\in\lbrace H,M\rbrace$.
\item If $\Gamma\vdash_{S}f\;e_{S}:TST$ then $\Gamma\vdash_{S}e_{S}:TST$.
\item If $\Gamma\vdash_{A}\mathtt{fix}\;e_{A}:T$ then $\Gamma\vdash_{A}e_{A}:T_{1}\rightarrow T_{1}$ and $T=T_{1}$ where $A\in\lbrace H,M\rbrace$.
\item If $\Gamma\vdash_{S}p\;e_{S}:TST$ then $\Gamma\vdash_{S}e_{S}:TST$.
\item $\vdash_{S}\mathtt{wrong}\;\mathrm{string}:TST$.
\item If $\Gamma\vdash_{A}{^{T_{1}}A}B^{T_{1}}\;e_{B}:T$ then $\Gamma\vdash_{A}T_{1}$, $\Gamma\vdash_{B}T_{1}$, $\Gamma\vdash_{B}e_{B}:T_{1}$, and $T=T_{1}$ where $(A,B)\in\lbrace(H,M),(M,H)\rbrace$.
\item If $\Gamma\vdash_{A}{^{T_{1}}A}S\;e_{S}:T$ then $\Gamma\vdash_{A}T_{1}$, $\Gamma\vdash_{S}e_{S}:TST$, and $T=T_{1}[T_{i}/T_{i}^{a}]$ where $A\in\lbrace H,M\rbrace$.
\item If $\Gamma\vdash_{S}SA^{T_{1}}\;e_{A}:T$ then $\Gamma\vdash_{A}T_{1}$, $\Gamma\vdash_{A}e_{A}:T_{1}[T_{i}/T_{i}^{a}]$, and $T=TST$ where $A\in\lbrace H,M\rbrace$.
\end{enumerate}
\begin{proof}
Immediate from the definitions of the typing relations.
\end{proof}
\end{lemma}

The type system ensures that a well-typed term has one unique type.

\begin{lemma}
\label{uot}
%\onehalfspacing
$e_{A}$ has at most one type $T$ for a given context $\Gamma$ where $A\in\lbrace H,M\rbrace$.
\begin{proof}
By structural induction on $e_{A}$ using inversion (Lemma \ref{i}).
\end{proof}
\end{lemma}

Once the type of an irreducible subterm is determined, its syntactic form can be determined.

\begin{lemma}
\label{cf}
%\onehalfspacing
The possible syntactic forms of values of various types.
\begin{enumerate}
\item If $v_{A}:N$ then $v_{A}=\overline{n}$ where $A\in\lbrace H,M\rbrace$.
\item If $v_{A}:T_{1}\rightarrow T_{2}$ then $v_{A}=\lambda x:T_{1}.e_{A}$ where $A\in\lbrace H,M\rbrace$.
\item If $v_{A}:\forall X.T$ then $v_{A}\in\lbrace\Lambda X.e_{A},{^{\forall X.T}A}S$ $v_{S}\rbrace$ where $A\in\lbrace H,M\rbrace$.
\item If $v_{H}:[T]$ then $v_{H}\in\lbrace\mathtt{cons}$ $e_{H}^{1}$ $e_{H}^{2},\mathtt{nil}^{T}\rbrace$.
\item If $v_{M}:[T]$ then $v_{M}\in\lbrace\mathtt{cons}$ $v_{M}^{1}$ $v_{M}^{2},\mathtt{nil}^{T},{^{[T]}M}H^{[T]}$ $(\mathtt{cons}$ $e_{H}$ $e_{H})\rbrace$.
\item If $v_{A}:L$ then $v_{A}={^{L}A}S$ $v_{S}$ where $A\in\lbrace H,M\rbrace$.
\end{enumerate}
\begin{proof}
Immediate from the definitions of values and the typing relations.
\end{proof}
\end{lemma}

If the term is irreducible, it is a value.

\begin{theorem}
\label{ps}
%\onehalfspacing
If $\vdash_{A}e_{A}:T$ then $e_{A}$ is a value or $e_{A}\rightarrow e_{A}'$ or $e_{A}\rightarrow$ \emph{\textbf{Error}:\;string} where $A\in\lbrace H,M,S\rbrace$.
\begin{proof}
By structural induction on $e_{A}$.
\begin{case}
$e_{A}=x$ where $A\in\lbrace H,M,S\rbrace$

Cannot occur because $e_{A}$ is closed.
\end{case}
\begin{case}
$e_{A}=\overline{n}$ where $A\in\lbrace H,M,S\rbrace$

$\overline{n}$ is a value.
\end{case}
\begin{case}
$e_{HM}=\lambda x:T.e_{HM}^{1}$

$\lambda x:T.e_{HM}^{1}$ is a value.
\end{case}
\begin{case}
$e_{S}=\lambda x.e_{S}^{1}$

$\lambda x.e_{S}^{1}$ is a value.
\end{case}
\begin{case}
$e_{A}=\Lambda X.e_{A}^{1}$ where $A\in\lbrace H,M\rbrace$

$\Lambda X.e_{A}^{1}$ is a value.
\end{case}
\begin{case}
$e_{H}=\mathtt{cons}\;e_{H}^{1}\;e_{H}^{2}$

$\mathtt{cons}\;e_{H}^{1}\;e_{H}^{2}$ is a value.
\end{case}
\begin{case}
$e_{A}=\mathtt{cons}\;v_{A}^{1}\;v_{A}^{2}$ where $A\in\lbrace M,S\rbrace$

$\mathtt{cons}\;v_{A}^{1}\;v_{A}^{2}$ is a value.
\end{case}
\begin{case}
$e_{HM}=\mathtt{nil}^{T}$

$\mathtt{nil}^{T}$ is a value.
\end{case}
\begin{case}
$e_{S}=\mathtt{nil}$

$\mathtt{nil}$ is a value.
\end{case}
\begin{case}
$e_{H}=e_{H}^{1}\;e_{H}^{2}$

$e_{H}^{1}$ is a value or $e_{H}^{1}\rightarrow e_{H}^{3}$ or $e_{H}^{1}\rightarrow$ \emph{\textbf{Error}:\;string} by the induction hypothesis.  If $e_{H}^{1}$ is a value then $e_{H}^{1}:T_{1}\rightarrow T_{2}$ by inversion (Lemma \ref{i}) and uniqueness of types (Lemma \ref{uot}) and $e_{H}^{1}=\lambda x:T_{1}.e_{H}^{4}$ by canonical forms (Lemma \ref{cf}).  $(\lambda x:T_{1}.e_{H}^{4})\;e_{H}^{2}\rightarrow e_{H}^{4}[e_{H}^{2}/x]$.  If $e_{H}^{1}\rightarrow e_{H}^{3}$ then $e_{H}^{1}\;e_{H}^{2}\rightarrow e_{H}^{3}\;e_{H}^{2}$.  If $e_{H}^{1}\rightarrow$ \emph{\textbf{Error}:\;string} then $e_{H}^{1}\;e_{H}^{2}\rightarrow$ \emph{\textbf{Error}:\;string}.
\end{case}
\begin{case}
$e_{M}=e_{M}^{1}\;e_{M}^{2}$

$e_{M}^{1}$ is a value or $e_{M}^{1}\rightarrow e_{M}^{3}$ or $e_{M}^{1}\rightarrow$ \emph{\textbf{Error}:\;string} by the induction hypothesis.  If $e_{M}^{1}$ is a value then $e_{M}^{1}:T_{1}\rightarrow T_{2}$ by inversion (Lemma \ref{i}) and uniqueness of types (Lemma \ref{uot}) and $e_{M}^{1}=\lambda x:T_{1}.e_{M}^{4}$ by canonical forms (Lemma \ref{cf}).  If $e_{M}^{1}\rightarrow e_{M}^{3}$ then $e_{M}^{1}\;e_{M}^{2}\rightarrow e_{M}^{3}\;e_{M}^{2}$.  If $e_{M}^{1}\rightarrow$ \emph{\textbf{Error}:\;string} then $e_{M}^{1}\;e_{M}^{2}\rightarrow$ \emph{\textbf{Error}:\;string}.  $e_{M}^{2}$ is a value or $e_{M}^{2}\rightarrow e_{M}^{5}$ or $e_{M}^{2}\rightarrow$ \emph{\textbf{Error}:\;string} by the induction hypothesis.  If $e_{M}^{2}\rightarrow e_{M}^{5}$ and $e_{M}^{1}$ is a value then $e_{M}^{1}\;e_{M}^{2}\rightarrow e_{M}^{1}\;e_{M}^{5}$.  If $e_{M}^{2}\rightarrow$ \emph{\textbf{Error}:\;string} and $e_{M}^{1}$ is a value then $e_{M}^{1}\;e_{M}^{2}\rightarrow$ \emph{\textbf{Error}:\;string}.  If $e_{M}^{1}$ and $e_{M}^{2}$ are values then $(\lambda x:T_{1}.e_{M}^{4})\;e_{M}^{2}\rightarrow e_{M}^{4}[e_{M}^{2}/x]$.
\end{case}
\begin{case}
$e_{S}=e_{S}^{1}\;e_{S}^{2}$

$e_{S}^{1}$ is a value or $e_{S}^{1}\rightarrow e_{S}^{3}$ or $e_{S}^{1}\rightarrow$ \emph{\textbf{Error}:\;string} by the induction hypothesis.  If $e_{S}^{1}\rightarrow e_{S}^{3}$ then $e_{S}^{1}\;e_{S}^{2}\rightarrow e_{S}^{3}\;e_{S}^{2}$.  If $e_{S}^{1}\rightarrow$ \emph{\textbf{Error}:\;string} then $e_{S}^{1}\;e_{S}^{2}\rightarrow$ \emph{\textbf{Error}:\;string}.  $e_{S}^{2}$ is a value or $e_{S}^{2}\rightarrow e_{S}^{4}$ or $e_{S}^{2}\rightarrow$ \emph{\textbf{Error}:\;string} by the induction hypothesis.  If $e_{S}^{2}\rightarrow e_{S}^{4}$ and $e_{S}^{1}$ is a value then $e_{S}^{1}\;e_{S}^{2}\rightarrow e_{S}^{1}\;e_{S}^{4}$.  If $e_{S}^{2}\rightarrow$ \emph{\textbf{Error}:\;string} and $e_{S}^{1}$ is a value then $e_{S}^{1}\;e_{S}^{2}\rightarrow$ \emph{\textbf{Error}:\;string}.  If $e_{S}^{1}$ and $e_{S}^{2}$ are values then $(\lambda x.e_{S}^{5})\;e_{S}^{2}\rightarrow e_{S}^{5}[e_{S}^{2}/x]$ if $e_{S}^{1}=\lambda x.e_{S}^{5}$ or $e_{S}^{1}\;e_{S}^{2}\rightarrow\mathtt{wrong}\;\mathrm{``Not\;a\;procedure"}$ otherwise.
\end{case}
\begin{case}
$e_{A}=e_{A}^{1}\;\lbrace T_{1}\rbrace$ where $A\in\lbrace H,M\rbrace$

$e_{A}^{1}$ is a value or $e_{A}^{1}\rightarrow e_{A}^{2}$ or $e_{A}^{1}\rightarrow$ \emph{\textbf{Error}:\;string} by the induction hypothesis.  If $e_{A}^{1}$ is a value then $e_{A}^{1}:\forall X.T_{2}$ by inversion (Lemma \ref{i}) and uniqueness of types (Lemma \ref{uot}) and $e_{A}^{1}=\Lambda X.e_{A}^{3}$ or $e_{A}^{1}=\,^{\forall X.T_{2}}AS\;v_{S}$ by canonical forms (Lemma \ref{cf}).
\begin{subcase}
$e_{A}^{1}=\Lambda X.e_{A}^{3}$

$(\Lambda X.e_{A}^{3})\;\lbrace T_{1}\rbrace\rightarrow e_{A}^{3}[T_{1}/X]$.
\end{subcase}
\begin{subcase}
$e_{A}^{1}={^{\forall X.T_{2}}A}S\;v_{S}$

$(^{\forall X.T_{2}}AS\;v_{S})\;\lbrace T_{1}\rbrace\rightarrow{^{T_{2}[T_{1}^{a}/X]}A}S\;v_{S}$.
\end{subcase}
If $e_{A}^{1}\rightarrow e_{A}^{2}$ then $e_{A}^{1}\;\lbrace T_{1}\rbrace\rightarrow e_{A}^{2}\;\lbrace T_{1}\rbrace$.  If $e_{A}^{1}\rightarrow$ \emph{\textbf{Error}:\;string} then $e_{A}^{1}\;\lbrace T_{1}\rbrace\rightarrow$ \emph{\textbf{Error}:\;string}.
\end{case}
\begin{case}
$e_{HM}=\mathtt{if0}\;e_{HM}^{1}\;e_{HM}^{2}\;e_{HM}^{3}$

$e_{HM}^{1}$ is a value or $e_{HM}^{1}\rightarrow e_{HM}^{4}$ or $e_{HM}^{1}\rightarrow$ \emph{\textbf{Error}:\;string} by the induction hypothesis.  If $e_{HM}^{1}$ is a value then $e_{HM}^{1}:N$ by inversion (Lemma \ref{i}) and uniqueness of types (Lemma \ref{uot}) and $e_{HM}^{1}=\overline{n}$ by canonical forms (Lemma \ref{cf}).  $\mathtt{if0}\;e_{HM}^{1}\;e_{HM}^{2}\;e_{HM}^{3}\rightarrow e_{HM}^{2}$ if $e_{HM}^{1}=\overline{0}$ or $\mathtt{if0}\;e_{HM}^{1}\;e_{HM}^{2}\;e_{HM}^{3}\rightarrow e_{HM}^{3}$ otherwise.  If $e_{HM}^{1}\rightarrow e_{HM}^{4}$ then $\mathtt{if0}\;e_{HM}^{1}\;e_{HM}^{2}\;e_{HM}^{3}\rightarrow \mathtt{if0}\;e_{HM}^{4}\;e_{HM}^{2}\;e_{HM}^{3}$.  If $e_{HM}^{1}\rightarrow$ \emph{\textbf{Error}:\;string} then $\mathtt{if0}\;e_{HM}^{1}\;e_{HM}^{2}\;e_{HM}^{3}\rightarrow$ \emph{\textbf{Error}:\;string}.
\end{case}
\begin{case}
$e_{S}=\mathtt{if0}\;e_{S}^{1}\;e_{S}^{2}\;e_{S}^{3}$

$e_{S}^{1}$ is a value or $e_{S}^{1}\rightarrow e_{S}^{4}$ or $e_{S}^{1}\rightarrow$ \emph{\textbf{Error}:\;string} by the induction hypothesis.  If $e_{S}^{1}$ is a value then $\mathtt{if0}\;e_{S}^{1}\;e_{S}^{2}\;e_{S}^{3}\rightarrow e_{S}^{2}$ if $e_{S}^{1}=\overline{0}$ or $\mathtt{if0}\;e_{S}^{1}\;e_{S}^{2}\;e_{S}^{3}\rightarrow e_{S}^{3}$ otherwise.  If $e_{S}^{1}\rightarrow e_{S}^{4}$ then $\mathtt{if0}\;e_{S}^{1}\;e_{S}^{2}\;e_{S}^{3}\rightarrow \mathtt{if0}\;e_{S}^{4}\;e_{S}^{2}\;e_{S}^{3}$.  If $e_{S}^{1}\rightarrow$ \emph{\textbf{Error}:\;string} then $\mathtt{if0}\;e_{S}^{1}\;e_{S}^{2}\;e_{S}^{3}\rightarrow$ \emph{\textbf{Error}:\;string}.
\end{case}
\begin{case}
$e_{A}=\mathtt{cons}\;e_{A}^{1}\;e_{A}^{2}$ where $A\in\lbrace M,S\rbrace$

$e_{A}^{1}$ is a value or $e_{A}^{1}\rightarrow e_{A}^{3}$ or $e_{A}^{1}\rightarrow$ \emph{\textbf{Error}:\;string} by the induction hypothesis.  If $e_{A}^{1}\rightarrow e_{A}^{3}$ then $\mathtt{cons}\;e_{A}^{1}\;e_{A}^{2}\rightarrow\mathtt{cons}\;e_{A}^{3}\;e_{A}^{2}$.  If $e_{A}^{1}\rightarrow$ \emph{\textbf{Error}:\;string} then $\mathtt{cons}\;e_{A}^{1}\;e_{A}^{2}\rightarrow$ \emph{\textbf{Error}:\;string}.  $e_{A}^{2}$ is a value or $e_{A}^{2}\rightarrow e_{A}^{4}$ or $e_{A}^{1}\rightarrow$ \emph{\textbf{Error}:\;string} by the induction hypothesis.  If $e_{A}^{2}\rightarrow e_{A}^{4}$ and $e_{M}^{1}$ is a value then $\mathtt{cons}\;e_{A}^{1}\;e_{A}^{2}\rightarrow\mathtt{cons}\;e_{A}^{1}\;e_{A}^{4}$.  If $e_{A}^{2}\rightarrow$ \emph{\textbf{Error}:\;string} then $\mathtt{cons}\;e_{A}^{1}\;e_{A}^{2}\rightarrow$ \emph{\textbf{Error}:\;string}.  If $e_{A}^{1}$ and $e_{A}^{2}$ are values then $\mathtt{cons}\;e_{A}^{1}\;e_{A}^{2}$ is a value.
\end{case}
\begin{case}
$e_{A}=o\;e_{A}^{1}\;e_{A}^{2}$ where $A\in\lbrace H,M\rbrace$

$e_{A}^{1}$ is a value or $e_{A}^{1}\rightarrow e_{A}^{3}$ or $e_{A}^{1}\rightarrow$ \emph{\textbf{Error}:\;string} by the induction hypothesis.  If $e_{A}^{1}$ is a value then $e_{A}^{1}:N$ by inversion (Lemma \ref{i}) and uniqueness of types (Lemma \ref{uot}) and $e_{A}^{1}=\overline{n_{1}}$ by canonical forms (Lemma \ref{cf}).  If $e_{A}^{1}\rightarrow e_{A}^{3}$ then $o\;e_{A}^{1}\;e_{A}^{2}\rightarrow o\;e_{A}^{3}\;e_{A}^{2}$.  If $e_{A}^{1}\rightarrow$ \emph{\textbf{Error}:\;string} then $o\;e_{A}^{1}\;e_{A}^{2}\rightarrow$ \emph{\textbf{Error}:\;string}.  $e_{A}^{2}$ is a value or $e_{A}^{2}\rightarrow e_{A}^{4}$ or $e_{A}^{2}\rightarrow$ \emph{\textbf{Error}:\;string} by the induction hypothesis.  If $e_{A}^{2}$ is a value then $e_{A}^{2}:N$ by inversion and uniqueness of types and $e_{A}^{2}=\overline{n_{2}}$ by canonical forms.  If $e_{A}^{2}\rightarrow e_{A}^{4}$ and $e_{A}^{1}$ is a value then $o\;e_{A}^{1}\;e_{A}^{2}\rightarrow o\;e_{A}^{1}\;e_{A}^{4}$.  If $e_{A}^{2}\rightarrow$ \emph{\textbf{Error}:\;string} and $e_{A}^{1}$ is a value then $o\;e_{A}^{1}\;e_{A}^{2}\rightarrow$ \emph{\textbf{Error}:\;string}.  If $e_{A}^{1}$ and $e_{A}^{2}$ are values then $+\;\overline{n_{1}}\;\overline{n_{2}}\rightarrow\overline{n_{1}+n_{2}}$ and $-\;\overline{n_{1}}\;\overline{n_{2}}\rightarrow\overline{max(n_{1}-n_{2},0)}$.
\end{case}
\begin{case}
$e_{S}=o\;e_{S}^{1}\;e_{S}^{2}$

$e_{S}^{1}$ is a value or $e_{S}^{1}\rightarrow e_{S}^{3}$ or $e_{S}^{1}\rightarrow$ \emph{\textbf{Error}:\;string} by the induction hypothesis.  If $e_{S}^{1}\rightarrow e_{S}^{3}$ then $o\;e_{S}^{1}\;e_{S}^{2}\rightarrow o\;e_{S}^{3}\;e_{S}^{2}$.  If $e_{S}^{1}\rightarrow$ \emph{\textbf{Error}:\;string} then $o\;e_{S}^{1}\;e_{S}^{2}\rightarrow$ \emph{\textbf{Error}:\;string}.  $e_{S}^{2}$ is a value or $e_{S}^{2}\rightarrow e_{S}^{4}$ or $e_{S}^{2}\rightarrow$ \emph{\textbf{Error}:\;string} by the induction hypothesis.  If $e_{S}^{2}\rightarrow e_{S}^{4}$ and $e_{S}^{1}$ is a value then $o\;e_{S}^{1}\;e_{S}^{2}\rightarrow o\;e_{S}^{1}\;e_{S}^{4}$.  If $e_{S}^{2}\rightarrow$ \emph{\textbf{Error}:\;string} and $e_{S}^{1}$ is a value then $o\;e_{S}^{1}\;e_{S}^{2}\rightarrow$ \emph{\textbf{Error}:\;string}.  If $e_{S}^{1}$ and $e_{S}^{2}$ are values then $o\;e_{S}^{1}\;e_{S}^{2}\rightarrow\overline{n_{1}+n_{2}}$ if $o=+$, $e_{S}^{1}=\overline{n_{1}}$, and $e_{S}^{2}=\overline{n_{2}}$ and $o\;e_{S}^{1}\;e_{S}^{2}\rightarrow\overline{max(n_{1}-n_{2},0)}$ if $o=-$, $e_{S}^{1}=\overline{n_{1}}$, and $e_{S}^{2}=\overline{n_{2}}$ and $o\;e_{S}^{1}\;e_{S}^{2}\rightarrow\mathtt{wrong}\;\mathrm{``Not\;a\;number"}$ otherwise.
\end{case}
\begin{case}
$e_{H}=f$ $e_{H}^{1}$

$e_{H}^{1}$ is a value or $e_{H}^{1}\rightarrow e_{H}^{2}$ or $e_{H}^{1}\rightarrow$ \emph{\textbf{Error}:\;string} by the induction hypothesis.  If $e_{H}^{1}$ is a value then $e_{H}^{1}:[T]$ by inversion (Lemma \ref{i}) and uniqueness of types (Lemma \ref{uot}) and $e_{H}^{1}\in\lbrace\mathtt{cons}$ $e_{H}^{3}$ $e_{H}^{4},\mathtt{nil}^{T}\rbrace$ by canonical forms (Lemma \ref{cf}).  If $e_{H}^{1}=\mathtt{cons}$ $e_{H}^{3}$ $e_{H}^{4}$ then $\mathtt{hd}$ $(\mathtt{cons}$ $e_{H}^{3}$ $e_{H}^{4})\rightarrow e_{H}^{3}$ and $\mathtt{tl}$ $(\mathtt{cons}$ $e_{H}^{3}$ $e_{H}^{4})\rightarrow e_{H}^{4}$.  If $e_{H}^{1}=\mathtt{nil}^{T}$ then $\mathtt{hd}$ $\mathtt{nil}^{T}\rightarrow{^{T}H}S$ $(\mathtt{wrong}$ \emph{``Empty list"}$)$ and $\mathtt{tl}$ $\mathtt{nil}^{T}\rightarrow\mathtt{nil}^{T}$.  If $e_{H}^{1}\rightarrow e_{H}^{2}$ then $f$ $e_{H}^{1}\rightarrow f$ $e_{H}^{2}$.  If $e_{H}^{1}\rightarrow$ \emph{\textbf{Error}:\;string} then $f$ $e_{H}^{1}\rightarrow$ \emph{\textbf{Error}:\;string}.
\end{case}
\begin{case}
$e_{M}=f\;e_{M}^{1}$

$e_{M}^{1}$ is a value or $e_{M}^{1}\rightarrow e_{M}^{2}$ or $e_{M}^{1}\rightarrow$ \emph{\textbf{Error}:\;string} by the induction hypothesis.  If $e_{M}^{1}$ is a value then $e_{M}^{1}:[T]$ by inversion (Lemma \ref{i}) and uniqueness of types (Lemma \ref{uot}) and $e_{M}^{1}\in\lbrace\mathtt{cons}\;e_{M}^{3}\;e_{M}^{4},\mathtt{nil}^{T},{^{[T]}M}H^{[T]}\;v_{H}\rbrace$ by canonical forms (Lemma \ref{cf}).  If $e_{M}^{1}=\mathtt{cons}\;e_{M}^{3}\;e_{M}^{4}$ then $\mathtt{hd}\;(\mathtt{cons}\;e_{M}^{3}\;e_{M}^{4})\rightarrow e_{M}^{3}$ or $\mathtt{tl}\;(\mathtt{cons}\;e_{M}^{3}\;e_{M}^{4})\rightarrow e_{M}^{4}$.  If $e_{M}^{1}=\mathtt{nil}^{T}$ then $\mathtt{hd}\;\mathtt{nil}^{T}\rightarrow{^{T}M}S\;(\mathtt{wrong}\;\mathrm{``Empty\;list"})$ and $\mathtt{tl}\;\mathtt{nil}^{T}\rightarrow\mathtt{nil}^{T}$.  If $e_{M}^{1}={^{[T]}M}H^{[T]}\;v_{H}$ then $v_{H}:[T]$ by inversion and uniqueness of types and $v_{H}\in\lbrace\mathtt{cons}\;e_{H}^{1}\;e_{H}^{2},\mathtt{nil}^{T}\rbrace$ by canonical forms.  If $v_{H}=\mathtt{cons}\;e_{H}^{1}\;e_{H}^{2}$ then $\mathtt{hd}\;(^{[T]}MH^{[T]}\;(\mathtt{cons}\;e_{H}^{1}\;e_{H}^{2}))\rightarrow{^{T}M}H^{T}\;e_{H}^{1}$ and $\mathtt{tl}\;(^{[T]}MH^{[T]}\;(\mathtt{cons}\;e_{H}^{1}\;e_{H}^{2}))\rightarrow{^{[T]}M}H^{[T]}\;e_{H}^{2}$.  If $v_{H}=\mathtt{nil}^{T}$ then $\mathtt{hd}\;\mathtt{nil}^{T}\rightarrow{^{T}M}S\;(\mathtt{wrong}\;\mathrm{``Empty\;list"})$ and $\mathtt{tl}\;\mathtt{nil}^{T}\rightarrow\mathtt{nil}^{T}$.  If $e_{M}^{1}\rightarrow e_{M}^{2}$ then $f\;e_{M}^{1}\rightarrow f\;e_{M}^{2}$.  If $e_{M}^{1}\rightarrow e_{M}^{2}$ then $f\;e_{M}^{1}\rightarrow f\;e_{M}^{2}$.  If $e_{M}^{1}\rightarrow$ \emph{\textbf{Error}:\;string} then $f\;e_{M}^{1}\rightarrow$ \emph{\textbf{Error}:\;string}.
\end{case}
\begin{case}
$e_{S}=f$ $e_{S}^{1}$

$e_{S}^{1}$ is a value or $e_{S}^{1}\rightarrow e_{S}^{2}$ or $e_{S}^{1}\rightarrow$ \emph{\textbf{Error}:\;string} by the induction hypothesis.  If $e_{S}^{1}\rightarrow e_{S}^{2}$ then $f$ $e_{S}^{1}\rightarrow f$ $e_{S}^{2}$.  If $e_{S}^{1}\rightarrow$ \emph{\textbf{Error}:\;string} then $f$ $e_{S}^{1}\rightarrow$ \emph{\textbf{Error}:\;string}.  $e_{S}^{1}$ is a value otherwise.  If $e_{S}^{1}=\mathtt{cons}$ $e_{S}^{3}$ $e_{S}^{4}$ then $f$ $(\mathtt{cons}$ $e_{S}^{3}$ $e_{S}^{4})\rightarrow e_{S}^{3}$ if $f=\mathtt{hd}$ and $f$ $(\mathtt{cons}$ $e_{S}^{3}$ $e_{S}^{4})\rightarrow e_{S}^{4}$ if $f=\mathtt{tl}$.  If $e_{S}^{1}=\mathtt{nil}$ then $f$ $\mathtt{nil}\rightarrow\mathtt{wrong}$ \emph{``Empty list"}.  If $e_{S}^{1}=SH^{[T]}$ $(\mathtt{cons}$ $e_{H}^{1}$ $e_{H}^{2})$ then $f$ $(SH^{[T]}$ $(\mathtt{cons}$ $e_{H}^{1}$ $e_{H}^{2}))\rightarrow SH^{T}$ $e_{H}^{1}$ if $f=\mathtt{hd}$ and $f$ $(SH^{[T]}$ $(\mathtt{cons}$ $e_{H}^{1}$ $e_{H}^{2}))\rightarrow SH^{[T]}$ $e_{H}^{2}$ if $f=\mathtt{tl}$.  $f$ $e_{S}^{1}\rightarrow\mathtt{wrong}$ \emph{``Not a list"} otherwise.
\end{case}
\begin{case}
$e_{A}=\mathtt{fix}\;e_{A}^{1}$ where $A\in\lbrace H,M\rbrace$

$e_{A}^{1}$ is a value or $e_{A}^{1}\rightarrow e_{A}^{2}$ or $e_{A}^{1}\rightarrow$ \emph{\textbf{Error}:\;string} by the induction hypothesis.  If $e_{A}^{1}$ is a value then $e_{A}^{1}:T_{1}\rightarrow T_{2}$ by inversion (Lemma \ref{i}) and uniqueness of types (Lemma \ref{uot}) and $e_{A}^{1}=\lambda x:T_{1}.e_{A}^{3}$ by canonical forms (Lemma \ref{cf}).  $\mathtt{fix}\;e_{A}^{1}\rightarrow e_{A}^{3}[\mathtt{fix}\;(\lambda x:T_{1}.e_{A}^{3})/x]$.  If $e_{A}^{1}\rightarrow e_{A}^{2}$ then $\mathtt{fix}\;e_{A}^{1}\rightarrow\mathtt{fix}\;e_{A}^{2}$.  If $e_{A}^{1}\rightarrow$ \emph{\textbf{Error}:\;string} then $\mathtt{fix}\;e_{A}^{1}\rightarrow$ \emph{\textbf{Error}:\;string}.
\end{case}
\begin{case}
$e_{S}=p\;e_{S}^{1}$

$e_{S}^{1}$ is a value or $e_{S}^{1}\rightarrow e_{S}^{2}$ or $e_{S}^{1}\rightarrow$ \emph{\textbf{Error}:\;string} by the induction hypothesis.  If $e_{S}^{1}\rightarrow e_{S}^{2}$ then $p\;e_{S}^{1}\rightarrow p\;e_{S}^{2}$.  If $e_{S}^{1}\rightarrow$ \emph{\textbf{Error}:\;string} then $p\;e_{S}^{1}\rightarrow$ \emph{\textbf{Error}:\;string}.  $e_{S}^{1}$ is a value otherwise.  If $p=\mathtt{nat?}$ then $p\;e_{S}^{1}\rightarrow\overline{0}$ if $e_{S}^{1}=\overline{n}$ and $p\;e_{S}^{1}\rightarrow\overline{1}$ otherwise.  If $p=\mathtt{list?}$ then $p\;e_{S}^{1}\rightarrow\overline{0}$ if $e_{S}^{1}\in\lbrace\mathtt{cons}\;e_{S}^{3}\;e_{S}^{4},\mathtt{nil}\rbrace$ and $p\;e_{S}^{1}\rightarrow\overline{1}$ otherwise.  If $p=\mathtt{proc?}$ then $p\;e_{S}^{1}\rightarrow\overline{0}$ if $e_{S}^{1}=\lambda x.e_{S}^{5}$ and $p\;e_{S}^{1}\rightarrow\overline{1}$ otherwise.
\end{case}
\begin{case}
$e_{S}=\mathtt{wrong}\;\mathrm{string}$

$e_{S}\rightarrow$ \emph{\textbf{Error}:\;string}.
\end{case}
\begin{case}
\label{ab}
$e_{A}={^{T}A}B^{T}$ $e_{B}^{1}$ where $(A,B,C)\in\lbrace(H,M,v),(M,H,e)\rbrace$

$e_{B}^{1}$ is a value or $e_{B}^{1}\rightarrow e_{B}^{2}$ or $e_{B}^{1}\rightarrow$ \emph{\textbf{Error}:\;string} by the induction hypothesis.  If $e_{B}^{1}$ is a value then $T$ determines its reduction.
\begin{subcase}
$T=N$

$e_{B}^{1}=\overline{n}$ by canonical forms (Lemma \ref{cf}).  $^{N}AB^{N}$ $\overline{n}\rightarrow\overline{n}$.
\end{subcase}
\begin{subcase}
$T=T_{1}\rightarrow T_{2}$

$e_{B}^{1}=\lambda x_{1}:T_{1}.e_{B}^{2}$ by canonical forms (Lemma \ref{cf}).  $^{T_{1}\rightarrow T_{2}}AB^{T_{1}\rightarrow T_{2}}$ $(\lambda x_{1}:T_{1}.e_{B}^{2})\rightarrow\lambda x_{2}:T_{1}[T_{i}/T^{a}_{i}].(^{T_{2}}AB^{T_{2}}$ $((\lambda x_{1}:T_{1}.e_{B}^{2})$ $(^{T_{1}}BA^{T_{1}}$ $x_{2})))$.
\end{subcase}
\begin{subcase}
$T=\forall X.T_{1}$

$e_{B}^{1}\in\lbrace\Lambda X.e_{B}^{2},{^{\forall X.T_{1}}B}S$ $v_{S}\rbrace$ by canonical forms (Lemma \ref{cf}).  If $e_{B}^{1}=\Lambda X.e_{B}^{2}$ then $^{\forall X.T_{1}}AB^{\forall X.T_{1}}$ $(\Lambda X.e_{B}^{2})\rightarrow\Lambda X.(^{T_{1}}AB^{T_{1}}$ $e_{B}^{2})$.  If $e_{B}^{1}={^{\forall X.T_{1}}B}S$ $v_{S}$ then $^{\forall X.T_{1}}AB^{\forall X.T_{1}}$ $(^{\forall X.T_{1}}BS$ $v_{S})\rightarrow{^{\forall X.T_{1}}A}S$ $v_{S}$.
\end{subcase}
\begin{subcase}
$T=[T_{1}]$

If $(A,B)=(H,M)$ then $e_{B}^{1}\in\lbrace\mathtt{cons}$ $v_{M}^{1}$ $v_{M}^{2},\mathtt{nil}^{T_{1}},{^{[T_{1}]}M}H^{[T_{1}]}$ $(\mathtt{cons}$ $e_{H}^{1}$ $e_{H}^{2})\rbrace$ by canonical forms (Lemma \ref{cf}).  If $e_{B}^{1}=\mathtt{cons}$ $v_{M}^{1}$ $v_{M}^{2}$ then $^{[T_{1}]}HM^{[T_{1}]}$ $(\mathtt{cons}$ $v_{M}^{1}$ $v_{M}^{2})\rightarrow\mathtt{cons}$ $(^{T_{1}}HM^{T_{1}}$ $v_{M}^{1})$ $(^{[T_{1}]}HM^{[T_{1}]}$ $v_{M}^{2})]$.  If $e_{B}^{1}=\mathtt{nil}^{T_{1}}$ then $^{[T_{1}]}HM^{[T_{1}]}$ $\mathtt{nil}^{T}\rightarrow\mathtt{nil}^{T}$.  If $e_{B}^{1}={^{[T_{1}]}M}H^{[T_{1}]}$ $(\mathtt{cons}$ $e_{H}^{1}$ $e_{H}^{2})$ then $^{[T_{1}]}HM^{[T_{1}]}$ $(^{[T_{1}]}MH^{[T_{1}]}$ $(\mathtt{cons}$ $e_{H}^{1}$ $e_{H}^{2}))\rightarrow\mathtt{cons}$ $e_{H}^{1}$ $e_{H}^{2}$.

If $(A,B)=(M,H)$ then $e_{B}^{1}\in\lbrace\mathtt{cons}$ $e_{H}^{3}$ $e_{H}^{4},\mathtt{nil}^{T_{1}}\rbrace$ by canonical forms.  If $e_{B}^{1}=\mathtt{cons}$ $e_{H}^{3}$ $e_{H}^{4}$ then $^{[T_{1}]}MH^{[T_{1}]}$ $(\mathtt{cons}$ $e_{H}^{3}$ $e_{H}^{4})$ is a value.  If $e_{B}^{1}=\mathtt{nil}^{T_{1}}$ then $^{[T_{1}]}MH^{[T_{1}]}$ $\mathtt{nil}^{T_{1}}\rightarrow\mathtt{nil}^{T_{1}}$.
\end{subcase}
\begin{subcase}
$T=L$

$e_{B}^{1}={^{L}B}S$ $v_{S}$ by canonical forms (Lemma \ref{cf}).  $^{L}AB^{L}$ $(^{L}BS$ $v_{S})\rightarrow{^{L}A}S$ $v_{S}$.
\end{subcase}
\begin{subcase}
$T=T_{1}^{a}$

Cannot occur because $T_{1}^{a}$ occurs only in $^{T_{1}^{a}}AS$ $e_{S}$.
\end{subcase}
If $e_{B}^{1}\rightarrow e_{B}^{2}$ then $^{T}AB^{T}$ $e_{B}^{1}\rightarrow{^{T}A}B$ $e_{B}^{2}$.  If $e_{B}^{1}\rightarrow$ \emph{\textbf{Error}:\;string} then $^{T}AB^{T}$ $e_{B}^{1}\rightarrow$ \emph{\textbf{Error}:\;string}.
\end{case}
\begin{case}
$e_{A}={^{T}A}S\;e_{S}^{1}$ where $A\in\lbrace H,M\rbrace$

$e_{S}^{1}$ is a value or $e_{S}^{1}\rightarrow e_{S}^{2}$ or $e_{S}^{1}\rightarrow$ \emph{\textbf{Error}:\;string} by the induction hypothesis.  If $e_{S}^{1}$ is a value then $T$ determines its reduction.
\begin{subcase}
$T=N$

If $e_{S}^{1}=\overline{n}$ then $^{N}AS\;\overline{n}\rightarrow\overline{n}$.  $^{N}AS\;e_{S}^{1}\rightarrow{^{N}A}S\;(\mathtt{wrong}\;\mathrm{``Not\;a\;number"})$ otherwise.
\end{subcase}
\begin{subcase}
$T=T_{1}\rightarrow T_{2}$

If $e_{S}^{1}=\lambda x_{1}.e_{S}^{2}$ then $^{T_{1}\rightarrow T_{2}}AS\;(\lambda x_{1}.e_{S}^{2})\rightarrow\lambda x_{2}:T_{1}[T_{i}/T^{a}_{i}].(^{T_{2}}AS\;((\lambda x_{1}.e_{S}^{2})\;(SA^{T_{1}}\;x_{2})))$.  $^{T_{1}\rightarrow T_{2}}AS\;e_{S}^{1}\rightarrow{^{T_{1}\rightarrow T_{2}}A}S\;(\mathtt{wrong}\;\mathrm{``Not\;a\;procedure"})$ otherwise.
\end{subcase}
\begin{subcase}
$T=\forall X.T_{1}$

$^{\forall X.T_{1}}AS\;e_{S}^{1}$ is a value.
\end{subcase}
\begin{subcase}
$T=[T_{1}]$

If $e_{S}^{1}=\mathtt{cons}\;v_{S}^{2}\;v_{S}^{3}$ then $^{[T_{1}]}AS\;(\mathtt{cons}\;v_{S}^{2}\;v_{S}^{3})\rightarrow\mathtt{cons}\;(^{T_{1}}AS\;v_{S}^{2})\;(^{[T_{1}]}AS\;v_{S}^{3})$.  $^{[T_{1}]}AS\;e_{S}^{1}\rightarrow{^{[T_{1}]}A}S\;(\mathtt{wrong}\;\mathrm{``Not\;a\;list"})$ otherwise.
\end{subcase}
\begin{subcase}
$T=L$

$^{L}AS\;e_{S}^{1}$ is a value.
\end{subcase}
\begin{subcase}
$T=T_{1}^{a}$

If $e_{S}^{1}=SA^{T_{1}^{a}}\;e_{A}^{1}$ then $^{T_{1}^{a}}AS\;(SA^{T_{1}^{a}}\;e_{A}^{1})\rightarrow e_{A}^{1}$.  $^{T_{1}^{a}}AS\;e_{S}^{1}\rightarrow{^{T_{1}^{a}}A}S\;(\mathtt{wrong}\;\mathrm{``Parametricity\;violated"})$ otherwise.
\end{subcase}
If $e_{S}^{1}\rightarrow e_{S}^{2}$ then $^{T}AS\;e_{S}^{1}\rightarrow\,^{T}AS\;e_{S}^{2}$.  If $e_{S}^{1}\rightarrow$ \emph{\textbf{Error}:\;string} then $^{T}AS\;e_{S}^{1}\rightarrow$ \emph{\textbf{Error}:\;string}.
\end{case}
\begin{case}
$e_{S}=SA^{T}$ $e_{A}^{1}$ where $(A,B)\in\lbrace(H,e),(M,v)\rbrace$

$e_{A}^{1}$ is a value or $e_{A}^{1}\rightarrow e_{A}^{2}$ or $e_{A}^{1}\rightarrow$ \emph{\textbf{Error}:\;string} by the induction hypothesis.  If $e_{A}^{1}$ is a value then $T$ determines its reduction.
\begin{subcase}
$T=N$

$e_{A}^{1}=\overline{n}$ by canonical forms (Lemma \ref{cf}).  $SA^{N}$ $\overline{n}\rightarrow\overline{n}$.
\end{subcase}
\begin{subcase}
$T=T_{1}\rightarrow T_{2}$

$e_{A}^{1}=\lambda x_{1}:T_{1}[T_{i}/T_{i}^{a}].e_{A}^{3}$ by canonical forms (Lemma \ref{cf}).  $SA^{T_{1}\rightarrow T_{2}}$ $(\lambda x_{1}:T_{1}[T_{i}/T_{i}^{a}].e_{A}^{3})\rightarrow\lambda x_{2}.(SA^{T_{2}}$ $((\lambda x_{1}:T_{1}[T_{i}/T_{i}^{a}].e_{A}^{3})$ $(^{T_{1}}AS$ $x_{2})))$.
\end{subcase}
\begin{subcase}
$T=\forall X_{1}.T_{1}$

$e_{A}^{1}\in\lbrace\Lambda X_{1}.e_{A}^{3},{^{\forall X_{1}.T_{1}}A}S$ $v_{S}\rbrace$ by canonical forms (Lemma \ref{cf}).  If $e_{A}^{1}=\Lambda X_{1}.e_{A}^{3}$ then $SA^{\forall X_{1}.T_{1}}$ $(\Lambda X_{1}.e_{A}^{3})\rightarrow\Lambda X_{2}.(SA^{T_{1}[X_{2}/X_{1}]}$ $((\Lambda X_{1}.e_{A}^{3})$ $\lbrace X_{2}\rbrace))$.  If $e_{A}^{1}={^{\forall X_{1}.T_{1}}A}S$ $v_{S}$ then it reduces by Case \ref{as}.
\end{subcase}
\begin{subcase}
$T=[T_{1}]$

If $A=H$ then $e_{H}^{1}\in\lbrace\mathtt{cons}$ $e_{H}^{1}$ $e_{H}^{2},\mathtt{nil}^{T_{1}}\rbrace$ by canonical forms (Lemma \ref{cf}).  If $e_{H}^{1}=\mathtt{cons}$ $e_{H}^{1}$ $e_{H}^{2}$ then $SH^{[T_{1}]}$ $(\mathtt{cons}$ $e_{H}^{1}$ $e_{H}^{2})$ is a value.  If $e_{H}^{1}=\mathtt{nil}^{T_{1}}$ then $SH^{T_{1}}$ $\mathtt{nil}^{T_{1}}\rightarrow\mathtt{nil}$.

If $A=M$ then $e_{M}^{1}\in\lbrace\mathtt{cons}$ $v_{M}^{1}$ $v_{M}^{2},\mathtt{nil}^{T_{1}},{^{[T_{1}]}M}H^{[T_{1}]}$ $(\mathtt{cons}$ $e_{H}^{1}$ $e_{H}^{2})\rbrace$ by canonical forms (Lemma \ref{cf}).  If $e_{M}^{1}=\mathtt{cons}$ $v_{M}^{1}$ $v_{M}^{2}$ then $SM^{[T_{1}]}$ $(\mathtt{cons}$ $v_{M}^{1}$ $v_{M}^{2})\rightarrow\mathtt{cons}$ $(SM^{T_{1}}$ $v_{M}^{1})$ $(SM^{[T_{1}]}$ $v_{M}^{2})$.  If $e_{M}^{1}=\mathtt{nil}^{T_{1}}$ then $SM^{T_{1}}$ $\mathtt{nil}^{T_{1}}\rightarrow\mathtt{nil}$.  If $e_{M}^{1}={^{[T_{1}]}M}H^{[T_{1}]}$ $(\mathtt{cons}$ $e_{H}^{1}$ $e_{H}^{2})$ then $SM^{[T_{1}]}$ $({^{[T_{1}]}M}H^{[T_{1}]}$ $(\mathtt{cons}$ $e_{H}^{1}$ $e_{H}^{2}))\rightarrow SH^{[T_{1}]}$ $(\mathtt{cons}$ $e_{H}^{1}$ $e_{H}^{2})$.
\end{subcase}
\begin{subcase}
$T=L$

$e_{A}^{1}={^{L}A}S$ $v_{S}$ by canonical forms (Lemma \ref{cf}).  $SA^{L}$ $(^{L}AS$ $v_{S})\rightarrow v_{S}$.
\end{subcase}
\begin{subcase}
$T=T_{1}^{a}$

$SA^{T_{1}^{a}}$ $B_{A}^{3}$ is a value.
\end{subcase}
If $e_{A}^{1}\rightarrow e_{A}^{2}$ then $SA^{T}$ $e_{A}^{1}\rightarrow SA^{T}$ $e_{A}^{2}$.  If $e_{A}^{1}\rightarrow$ \emph{\textbf{Error}:\;string} then $SA^{T}$ $e_{A}^{1}\rightarrow$ \emph{\textbf{Error}:\;string}.
\end{case}
\end{proof}
\end{theorem}

\section{Preservation}

Preservation will be proven by cases on the reduction rules.  In each case, the new term will be proven to be well-typed and have the same type as the old term.  Inversion (Lemma \ref{i}) and uniqueness of types (Lemma \ref{uot}) will be used to determine the type of the old term, the types of the subterms of the old term that occur within the new term, and the type of the new term.  Some reduction rules use substitution.

Term substitution substitutes one term for a second term within a third term.  The result of a term substitution has the same type as the third term.

\begin{lemma}
\label{tms}
%\onehalfspacing
If $\Gamma,x:T_{1}\vdash_{A}e_{A}:T_{2}$ and $\Gamma\vdash_{A}y:T_{1}$ then $\Gamma\vdash_{A}e_{A}[y/x]:T_{2}$ where $A\in\lbrace H,M\rbrace$.  If $\Gamma,x:TST\vdash_{S}e_{S}:TST$ and $\Gamma\vdash_{S}y:TST$ then $\Gamma\vdash_{S}e_{S}[y/x]:TST$.
\begin{proof}
By structural induction.
\end{proof}
\end{lemma}

Type substitution substitutes a type for a type variable within a term.  The type of the result is the type substituted for the type variable within the term.

\begin{lemma}
\label{tes}
\onehalfspacing
If $\Gamma,X\vdash_{HM}e_{HM}:T_{1}$ and $\vdash_{HM}T_{2}$ then $\Gamma\vdash_{HM}e_{HM}[T_{2}/X]:T_{1}[T_{2}/X]$.
\begin{proof}
By structural induction.
\end{proof}
\end{lemma}
\begin{lemma}
\label{ecp}
%\onehalfspacing
If $\Gamma\vdash_{A}e_{A}^{1}:T_{1}$, $\Gamma\vdash_{A}e_{A}^{2}:T_{2}$, and $\mathscr{E}[e_{A}^{1}]:T_{1}$ then $\mathscr{E}[e_{A}^{2}]:T_{2}$ where $A\in\lbrace H,M,S\rbrace$.
\begin{proof}
By structural induction.
\end{proof}
\end{lemma}
\begin{theorem}
\label{pn}
If $\Gamma\vdash_{A}e_{A}^{1}:T$ and $e_{A}^{1}\rightarrow e_{A}^{2}$ then $\Gamma\vdash_{A}e_{A}^{2}:T$ where $A\in\lbrace H,M\rbrace$.  If $\Gamma\vdash_{S}e_{S}^{1}:TST$ and $e_{S}^{1}\rightarrow e_{S}^{2}$ then $\Gamma\vdash_{S}e_{S}^{2}:TST$.
\begin{proof}
By cases on the reductions $e_{A}^{1}\rightarrow e_{A}^{2}$ and $e_{S}^{1}\rightarrow e_{S}^{2}$.  Straightforward cases of Scheme preservation are elided.
\begin{case}
$(\lambda x:T_{1}.e_{HM}^{1})\;e_{HM}^{2}\rightarrow e_{HM}^{1}[e_{HM}^{2}/x]$

$\Gamma\vdash_{HM}(\lambda x:T_{1}.e_{HM}^{1})\;e_{HM}^{2}:T$ by the premise and uniqueness of types (Lemma \ref{uot}).  $\Gamma\vdash_{HM}\lambda x:T_{1}.e_{HM}^{1}:T_{1}\rightarrow T$, $\Gamma,x:T_{1}\vdash_{HM}e_{HM}^{1}:T$, $\Gamma\vdash_{HM}e_{HM}^{2}:T_{1}$, and $\Gamma,x:T_{1}\vdash_{HM}x:T_{1}$ by inversion (Lemma \ref{i}) and uniqueness of types.  $e_{HM}^{1}[e_{HM}^{2}/x]:T$ by term substitution (Lemma \ref{tms}).
\end{case}
\begin{case}
$\mathscr{E}[(\Lambda X.e_{H})\;\lbrace T_{1}\rbrace]_{H}\rightarrow\mathscr{E}[e_{H}[T_{1}/X]]$

$(\Lambda X.e_{H})\;\lbrace T_{1}\rbrace:T$ by the induction hypothesis and uniqueness of types (Lemma \ref{uot}).

%\textbf{!!! NOT DONE !!!}
%$\Gamma\vdash_{H}e_{H}:\forall X.T_{2}$, $\Gamma\vdash_{H}T_{1}$, and $T=T_{2}[T_{1}/X]$.
\end{case}
\begin{case}
$\mathtt{if0}\;\overline{0}\;e_{HM}^{1}\;e_{HM}^{2}\rightarrow e_{HM}^{1}$

$\Gamma\vdash_{HM}\mathtt{if0}\;\overline{0}\;e_{HM}^{1}\;e_{HM}^{2}:T$ by premise and uniqueness of types (Lemma \ref{uot}).  $\Gamma\vdash_{HM}e_{HM}^{1}:T$ by inversion (Lemma \ref{i}) and uniqueness of types.
\end{case}
\begin{case}
$\mathtt{if0}\;\overline{n}\;e_{A}^{1}\;e_{A}^{2}\rightarrow e_{A}^{2}\;(n\neq0)$ where $A\in\lbrace H,M\rbrace$

$\Gamma\vdash_{A}\mathtt{if0}\;\overline{n}\;e_{A}^{1}\;e_{A}^{2}:T$ by premise and uniqueness of types (Lemma \ref{uot}).  $T=T_{1}$ and $\Gamma\vdash_{A}e_{A}^{2}:T_{1}$ by inversion (Lemma \ref{i}) and uniqueness of types.  $\Gamma\vdash_{A}e_{A}^{2}:T$ because $T_{1}=T$.
\end{case}
\begin{case}
$+\;\overline{n_{1}}\;\overline{n_{2}}\rightarrow\overline{n_{1}+n_{2}}$ where $A\in\lbrace H,M\rbrace$

$\vdash_{A}+\;\overline{n_{1}}\;\overline{n_{2}}:N$ by inversion (Lemma \ref{i}) and uniqueness of types (Lemma \ref{uot}).  $\vdash_{A}\overline{n_{1}+n_{2}}:N$ by inversion and uniqueness of types.
\end{case}
\begin{case}
$-\;\overline{n_{1}}\;\overline{n_{2}}\rightarrow\overline{max(n_{1}-n_{2},0)}$

$\vdash_{HM}-\;\overline{n_{1}}\;\overline{n_{2}}:N$ by inversion (Lemma \ref{i}) and uniqueness of types (Lemma \ref{uot}).  $\vdash_{HM}\overline{max(n_{1}-n_{2},0)}:N$ by inversion and uniqueness of types.
\end{case}
\begin{case}
$\mathtt{hd}$ $(\mathtt{cons}$ $e_{H}^{1}$ $e_{H}^{2})\rightarrow e_{H}^{1}$

$\Gamma\vdash_{H}\mathtt{hd}$ $(\mathtt{cons}$ $e_{H}^{1}$ $e_{H}^{2}):T$ by premise and uniqueness of types (Lemma \ref{uot}).  $\Gamma\vdash_{H}e_{H}^{1}:T_{1}$, $\Gamma\vdash_{H}\mathtt{cons}$ $e_{H}^{1}$ $e_{H}^{2}:[T_{1}]$, and $T=T_{1}$ by inversion (Lemma \ref{i}) and uniqueness of types.  $\Gamma\vdash_{H}e_{H}^{1}:T$ because $T_{1}=T$.
\end{case}
\begin{case}
$\mathtt{tl}\;(\mathtt{cons}\;e_{H}^{1}\;e_{H}^{2})\rightarrow e_{H}^{2}$

$\Gamma\vdash_{H}\mathtt{tl}\;(\mathtt{cons}\;e_{H}^{1}\;e_{H}^{2}):T$ by premise and uniqueness of types (Lemma \ref{uot}).  $\Gamma\vdash_{H}e_{H}^{2}:[T_{1}]$, $\Gamma\vdash_{H}\mathtt{cons}\;e_{H}^{1}\;e_{H}^{2}:[T_{1}]$, and $T=[T_{1}]$ by inversion (Lemma \ref{i}) and uniqueness of types.  $\Gamma\vdash_{H}e_{H}^{2}:T$ because $[T_{1}]=T$.
\end{case}
\begin{case}
$\mathtt{hd}\;(\mathtt{cons}\;v_{M}^{1}\;v_{M}^{2})\rightarrow v_{M}^{1}$

$\Gamma\vdash_{M}\mathtt{hd}\;(\mathtt{cons}\;v_{M}^{1}\;v_{M}^{2}):T$ by premise and uniqueness of types (Lemma \ref{uot}).  $\Gamma\vdash_{M}v_{M}^{1}:T_{1}$, $\Gamma\vdash_{M}\mathtt{cons}\;v_{M}^{1}\;v_{M}^{2}:[T_{1}]$, and $T=T_{1}$ by inversion (Lemma \ref{i}) and uniqueness of types.  $\Gamma\vdash_{M}v_{M}^{1}:T$ because $T_{1}=T$.
\end{case}
\begin{case}
$\mathtt{tl}\;(\mathtt{cons}\;v_{M}^{1}\;v_{M}^{2})\rightarrow v_{M}^{2}$

$\Gamma\vdash_{M}\mathtt{tl}\;(\mathtt{cons}\;v_{M}^{1}\;v_{M}^{2}):T$ by premise and uniqueness of types (Lemma \ref{uot}).  $T=[T_{1}]$, $\Gamma\vdash_{M}\mathtt{cons}\;v_{M}^{1}\;v_{M}^{2}:[T_{1}]$, and $\Gamma\vdash_{M}v_{M}^{2}:[T_{1}]$ by inversion (Lemma \ref{i}) and uniqueness of types.  $\Gamma\vdash_{M}v_{M}^{2}:T$ because $[T_{1}]=T$.
\end{case}
\begin{case}
$\mathtt{hd}\;\mathtt{nil}^{T}\rightarrow\,^{T}B\;(\mathtt{wrong}\;\mathrm{``Empty\;list"})$ where $B\in\lbrace HS,MS\rbrace$

$\Gamma\vdash_{HM}\mathtt{hd}\;\mathtt{nil}^{T}:T$ by premise and uniqueness of types (Lemma \ref{uot}).  $\Gamma\vdash_{S}\mathtt{wrong}\;\mathrm{``Empty\;list"}:TST$ and $\Gamma\vdash_{HM}{^{T}B}\;(\mathtt{wrong}\;\mathrm{``Empty\;list"}):T$ by inversion (Lemma \ref{i}) and uniqueness of types.
\end{case}
\begin{case}
$\mathtt{tl}$ $\mathtt{nil}^{T_{1}}\rightarrow\mathtt{nil}^{T_{1}}$ where $A\in\lbrace H,M\rbrace$

$\Gamma\vdash_{A}\mathtt{tl}$ $\mathtt{nil}^{T_{1}}:T$ by premise and uniqueness of types (Lemma \ref{uot}).  $\Gamma\vdash_{A}\mathtt{nil}^{T_{1}}:[T_{1}]$ and $T=[T_{1}]$ by inversion (Lemma \ref{i}) and uniqueness of types.  $\Gamma\vdash_{A}\mathtt{nil}^{T_{1}}:T$ because $[T_{1}]=T$.
\end{case}
\begin{case}
$\mathtt{fix}\;(\lambda x:T_{1}.e_{A})\rightarrow e_{A}[(\mathtt{fix}\;(\lambda x:T_{1}.e_{A}))/x]$ where $A\in\lbrace H,M\rbrace$

$\Gamma\vdash_{A}\mathtt{fix}\;(\lambda x:T_{1}.e_{A}):T$ by premise and uniqueness of types (Lemma \ref{uot}).  $\Gamma\vdash_{A}\lambda x:T_{1}.e_{A}:T_{1}\rightarrow T_{1}$, $\Gamma,x:T_{1}\vdash_{A}e_{A}:T_{1}$, and $T=T_{1}$ by inversion (Lemma \ref{i}) and uniqueness of types.  $\Gamma\vdash_{A}e_{A}[(\mathtt{fix}\;(\lambda x:(T\rightarrow T).e_{A}))/x]:T_{1}$ by term substitution (Lemma \ref{tms}).  $\Gamma\vdash_{A}e_{A}[(\mathtt{fix}\;(\lambda x:(T\rightarrow T).e_{A}))/x]:T$ because $T_{1}=T$.
\end{case}
\begin{case}
$^{N}AB^{N}$ $\overline{n}\rightarrow\overline{n}$ where $(A,B)\in\lbrace(H,M),(M,H)\rbrace$

$\vdash_{A}{^{N}A}B^{N}$ $\overline{n}:T$ by premise and uniqueness of types (Lemma \ref{uot}).  $\vdash_{A}\overline{n}:N$ and $T=N$ by inversion (Lemma \ref{i}) and uniqueness of types.
\end{case}
\begin{case}
$^{N}AS$ $\overline{n}\rightarrow\overline{n}$ where $A\in\lbrace H,M\rbrace$

$\vdash_{A}{^{N}A}S$ $\overline{n}:T$ by premise and uniqueness of types (Lemma \ref{uot}).  $\vdash_{A}\overline{n}:N$ and $T=N$ by inversion (Lemma \ref{i}) and uniqueness of types.
\end{case}
\begin{case}
$^{N}AS\;v_{S}\rightarrow{^{N}A}S\;(\mathtt{wrong}\;\mathrm{``Not\;a\;number"})\;(v_{S}\neq\overline{n})$ where $A\in\lbrace H,M\rbrace$

$\Gamma\vdash_{A}{^{N}AS}\;v_{S}:T$ by premise and uniqueness of types (Lemma \ref{uot}).  $T=N$ by inversion (Lemma \ref{i}) and uniqueness of types.  $\vdash_{S}\mathtt{wrong}\;\mathrm{``Not\;a\;number"}:TST$ by inversion.  $\vdash_{A}{^{N}A}S\;(\mathtt{wrong}\;\mathrm{``Not\;a\;number"}):N$ by inversion and uniqueness of types.  $\vdash_{A}{^{N}A}S\;(\mathtt{wrong}\;\mathrm{``Not\;a\;number"}):T$ because $N=T$.
\end{case}
\begin{case}
$^{T_{1}\rightarrow T_{2}}AB^{T_{1}\rightarrow T_{2}}\;(\lambda x_{1}:T_{1}.e_{B})\rightarrow\lambda x_{2}:T_{1}.(^{T_{2}}AB^{T_{2}}\;((\lambda x_{1}:T_{1}.e_{B})\;(^{T_{1}}BA^{T_{1}}\;x_{2})))$ where $(A,B)\in\lbrace(H,M),(M,H)\rbrace$

$\Gamma\vdash_{A}{^{T_{1}\rightarrow T_{2}}}AB^{T_{1}\rightarrow T_{2}}\;(\lambda x_{1}:T_{1}.e_{B}):T$ by premise and uniqueness of types (Lemma \ref{uot}).  $\Gamma\vdash_{B}\lambda x_{1}:T_{1}.e_{B}:T_{1}\rightarrow T_{2}$, $T=T_{1}\rightarrow T_{2}$, $\Gamma,x_{2}:T_{1}\vdash_{A}x_{2}:T_{1}$, $\Gamma,x_{2}:T_{1}\vdash_{B}{^{T_{1}}B}A^{T_{1}}\;x_{2}:T_{1}$, $\Gamma,x_{2}:T_{1}\vdash_{B}(\lambda x_{1}:T_{1}.e_{B})\;(^{T_{1}}BA^{T_{1}}\;x_{2}):T_{2}$, $\Gamma,x_{2}:T_{1}\vdash_{A}{^{T_{2}}A}B^{T_{2}}\;((\lambda x_{1}:T_{1}.e_{B})\;(^{T_{1}}BA^{T_{1}}\;x_{2})):T_{2}$, and $\Gamma\vdash_{A}\lambda x_{2}:T_{1}.(^{T_{2}}AB^{T_{2}}\;((\lambda x_{1}:T_{1}.e_{B})\;(^{T_{1}}BA^{T_{1}}\;x_{2}))):T_{1}\rightarrow T_{2}$ by inversion (Lemma \ref{i}) and uniqueness of types.  $\Gamma\vdash_{A}\lambda x_{2}:T_{1}.(^{T_{2}}AB^{T_{2}}\;((\lambda x_{1}:T_{1}.e_{B})\;(^{T_{1}}BA^{T_{1}}\;x_{2}))):T$ because $T_{1}\rightarrow T_{2}=T$.
\end{case}
\begin{case}
$^{T_{1}\rightarrow T_{2}}AS$ $(\lambda x_{1}.e_{S})\rightarrow\lambda x_{2}:T_{1}[T_{i}/T_{i}^{a}].(^{T_{2}}AS$ $((\lambda x_{1}.e_{S})$ $(SA^{T_{1}}$ $x_{2})))$ where $A\in\lbrace H,M\rbrace$

$\Gamma\vdash_{A}{^{T_{1}\rightarrow T_{2}}A}S$ $(\lambda x_{1}.e_{S}):T$ by premise and uniqueness of types (Lemma \ref{uot}).  $\Gamma\vdash_{S}\lambda x_{1}.e_{S}:TST$ by inversion (Lemma \ref{i}).  $T=(T_{1}\rightarrow T_{2})[T_{i}/T_{i}^{a}]$ by inversion and uniqueness of types.  $\Gamma,x_{2}:T_{1}[T_{i}/T_{i}^{a}]\vdash_{A}x_{2}:T_{1}[T_{i}/T_{i}^{a}]$ by inversion and uniqueness of types.  $\Gamma,x_{2}:T_{1}[T_{i}/T_{i}^{a}]\vdash_{S}SA^{T_{1}}$ $x_{2}:TST$ and $\Gamma,x_{2}:T_{1}[T_{i}/T_{i}^{a}]\vdash_{S}(\lambda x_{1}.e_{S})$ $(SA^{T_{1}}$ $x_{2}):TST$ by inversion.  $\Gamma,x_{2}:T_{1}[T_{i}/T_{i}^{a}]\vdash_{A}{^{T_{2}}A}S$ $((\lambda x_{1}.e_{S})$ $(SA^{T_{1}}$ $x_{2})):T_{2}[T_{i}/T_{i}^{a}]$ and $\Gamma\vdash_{A}\lambda x_{2}:T_{1}[T_{i}/T_{i}^{a}].(^{T_{2}}AS$ $((\lambda x_{1}.e_{S})$ $(SA^{T_{1}}$ $x_{2}))):T_{1}[T_{i}/T_{i}^{a}]\rightarrow T_{2}[T_{i}/T_{i}^{a}]$ by inversion and uniqueness of types.  $\Gamma\vdash_{A}\lambda x_{2}:T_{1}[T_{i}/T_{i}^{a}].(^{T_{2}}AS$ $((\lambda x_{1}.e_{S})$ $(SA^{T_{1}}$ $x_{2}))):T$ because $T_{1}[T_{i}/T_{i}^{a}]\rightarrow T_{2}[T_{i}/T_{i}^{a}]=(T_{1}\rightarrow T_{2})[T_{i}/T_{i}^{a}]=T$.
\end{case}
\begin{case}
$^{T_{1}\rightarrow T_{2}}AS$ $v_{S}\rightarrow{^{T_{1}\rightarrow T_{2}}A}S$ $(\mathtt{wrong}$ $\mathrm{``Not}$ $\mathrm{a}$ $\mathrm{function"})$ $(v_{S}\neq\lambda x.e_{S})$ where $A\in\lbrace H,M\rbrace$

$\Gamma\vdash_{A}{^{T_{1}\rightarrow T_{2}}A}S$ $v_{S}:T$ by premise and uniqueness of types (Lemma \ref{uot}).  $T=(T_{1}\rightarrow T_{2})[T_{i}/T_{i}^{a}]$ by inversion (Lemma \ref{i}) and uniqueness of types.  $\vdash_{S}\mathtt{wrong}$ $\mathrm{``Not}$ $\mathrm{a}$ $\mathrm{function"}:TST$ by inversion.  $\Gamma\vdash_{A}{^{T_{1}\rightarrow T_{2}}A}S$ $(\mathtt{wrong}$ $\mathrm{``Not}$ $\mathrm{a}$ $\mathrm{function"}):(T_{1}\rightarrow T_{2})[T_{i}/T_{i}^{a}]$ by inversion and uniqueness of types.  $\Gamma\vdash_{A}{^{T_{1}\rightarrow T_{2}}A}S$ $(\mathtt{wrong}$ $\mathrm{``Not}$ $\mathrm{a}$ $\mathrm{function"}):T$ because $(T_{1}\rightarrow T_{2})[T_{i}/T_{i}^{a}]=T$.
\end{case}
\begin{case}
$SA^{T_{1}\rightarrow T_{2}}\;(\lambda x_{1}:T_{1}[T_{i}/T_{i}^{a}].e_{A})\rightarrow\lambda x_{2}.(SA^{T_{2}}\;((\lambda x_{1}:T_{1}[T_{i}/T_{i}^{a}].e_{A})\;(^{T_{1}}AS\;x_{2})))$ where $A\in\lbrace H,M\rbrace$

$\Gamma\vdash_{S}SA^{T_{1}\rightarrow T_{2}}\;(\lambda x_{1}:T_{1}[T_{i}/T_{i}^{a}].e_{A}):TST$ by premise.  $\Gamma\vdash_{A}\lambda x_{1}:T_{1}[T_{i}/T_{i}^{a}].e_{A}:T_{1}[T_{i}/T_{i}^{a}]\rightarrow T_{2}[T_{i}/T_{i}^{a}]$ by inversion (Lemma \ref{i}) and uniqueness of types (Lemma \ref{uot}).  $\Gamma,x_{2}:TST\vdash_{S}x_{2}:TST$ by inversion.  $\Gamma,x_{2}:TST\vdash_{A}{^{T_{1}}A}S\;x_{2}:T_{1}[T_{i}/T_{i}^{a}]$ and $\Gamma,x_{2}:TST\vdash_{A}(\lambda x_{1}:T_{1}[T_{i}/T_{i}^{a}].e_{A})\;(^{T_{1}}AS\;x_{2}):T_{2}[T_{i}/T_{i}^{a}]$ by inversion and uniqueness of types.  $\Gamma,x_{2}:TST\vdash_{S}SA^{T_{2}}\;((\lambda x_{1}:T_{1}[T_{i}/T_{i}^{a}].e_{A})\;(^{T_{1}}AS\;x_{2})):TST$ and $\Gamma\vdash_{S}\lambda x_{2}.(SA^{T_{2}}\;((\lambda x_{1}:T_{1}[T_{i}/T_{i}^{a}].e_{A})\;(^{T_{1}}AS\;x_{2}))):TST$ by inversion.
\end{case}
\begin{case}
$^{\forall X.T_{1}}AB^{\forall X_{1}.T_{1}}$ $(\Lambda X.e_{B})\rightarrow\Lambda X.(^{T_{1}}AB^{T_{1}}$ $e_{B})$ where $(A,B)\in\lbrace(H,M),(M,H)\rbrace$

$\Gamma\vdash_{A}{^{\forall X.T_{1}}A}B^{\forall X.T_{1}}$ $(\Lambda X_{1}.e_{B}):T$ by premise and uniqueness of types (Lemma \ref{uot}).  $\Gamma,X\vdash_{B}e_{B}:T_{1}$, $\Gamma\vdash_{B}\Lambda X.e_{B}:\forall X.T_{1}$, $T=\forall X.T_{1}$, $\Gamma,X\vdash_{A}{^{T_{1}}A}B^{T_{1}}$ $e_{B}:T_{1}$, and $\Gamma\vdash_{A}\Lambda X.(^{T_{1}}AB^{T_{1}}$ $e_{B}):\forall X.T_{1}$ by inversion (Lemma \ref{i}) and uniqueness of types.  $\Gamma\vdash_{A}\Lambda X.(^{T_{1}}AB^{T_{1}}$ $e_{B}):T$ because $\forall X.T_{1}=T$.
\end{case}
\begin{case}
$^{\forall X.T_{1}}AB^{\forall X.T_{1}}$ $(^{\forall X.T_{1}}BS$ $v_{S})\rightarrow{^{\forall X.T_{1}}A}S$ $v_{S}$ where $(A,B)\in\lbrace(H,M),$ $(M,H)\rbrace$

$\Gamma\vdash_{A}{^{\forall X.T_{1}}A}B^{\forall X.T_{1}}$ $(^{\forall X.T_{1}}BS$ $v_{S}):T$ by premise and uniqueness of types (Lemma \ref{uot}).  $\Gamma\vdash_{S}v_{S}:TST$ by inversion (Lemma \ref{i}).  $\Gamma\vdash_{B}{^{\forall X.T_{1}}B}S$ $v_{S}:\forall X.T_{1}$, $T=\forall X.T_{1}$, and $\Gamma\vdash_{A}{^{\forall X.T_{1}}A}S$ $v_{S}:\forall X.T_{1}$ by inversion and uniqueness of types.  $\Gamma\vdash_{A}{^{\forall X.T_{1}}A}S$ $v_{S}:T$ because $\forall X.T_{1}=T$.
\end{case}
\begin{case}
$(^{\forall X.T_{1}}AS\;v_{S})\;\lbrace T_{2}\rbrace\rightarrow{^{T_{1}[T_{2}^{a}/X]}A}S\;v_{S}$ where $A\in\lbrace H,M\rbrace$

$\Gamma\vdash_{A}(^{\forall X.T_{1}}AS\;v_{S})\;\lbrace T_{2}\rbrace:T$ by premise and uniqueness of types (Lemma \ref{uot}).  $T=T_{1}[T_{2}/X]$ by inversion (Lemma \ref{i}) and uniqueness of types.  $\Gamma\vdash_{S}v_{S}:TST$ by inversion.  $\Gamma\vdash_{A}{^{T_{1}[T_{2}^{a}/X]}A}S\;v_{S}:T_{1}[T_{2}^{a}/X][T_{i}/T_{i}^{a}]$ by inversion and uniqueness of types.  $\Gamma\vdash_{A}{^{T_{1}[T_{2}^{a}/X]}A}S\;v_{S}:T$ because $T_{1}[T_{2}^{a}/X][T_{i}/T_{i}^{a}]=T_{1}[T_{2}/X]=T$.
\end{case}
\begin{case}
$SA^{\forall X.T_{1}}$ $(\Lambda X.e_{A})\rightarrow SA^{T_{1}[L/X]}$ $((\Lambda X.e_{A})$ $\lbrace L\rbrace)$ where $A\in\lbrace H,M\rbrace$

$\Gamma\vdash_{S}SA^{\forall X.T_{1}}$ $(\Lambda X.e_{A}):TST$ by premise.  $\Gamma\vdash_{A}\Lambda X.e_{A}:\forall X.T_{1}$ and $\Gamma\vdash_{A}(\Lambda X.e_{A})$ $\lbrace L\rbrace:T_{1}[L/X]$ by inversion (Lemma \ref{i}) and uniqueness of types (Lemma \ref{uot}).  $\Gamma\vdash_{S}SA^{T_{1}[L/X]}$ $((\Lambda X.e_{A})$ $\lbrace L\rbrace):TST$ by inversion.
\end{case}
\begin{case}
$^{[T_{1}]}HM^{[T_{1}]}\;(\mathtt{cons}\;v_{M}^{1}\;v_{M}^{2})\rightarrow\mathtt{cons}\;(^{T_{1}}HM^{T_{1}}\;v_{M}^{1})\;(^{[T_{1}]}HM^{[T_{1}]}\;v_{M}^{2})$

$^{[T_{1}]}HM^{[T_{1}]}\;(\mathtt{cons}\;v_{M}^{1}\;v_{M}^{2}):T$ by premise and uniqueness of types (Lemma \ref{uot}).  $T=[T_{1}]$, $\Gamma\vdash_{M}v_{M}^{1}:T_{1}$, $\Gamma\vdash_{M}v_{M}^{2}:[T_{1}]$, $\Gamma\vdash_{H}{^{T_{1}}H}M^{T_{1}}\;v_{M}^{1}:T_{1}$, $\Gamma\vdash_{H}{^{[T_{1}]}H}M^{[T_{1}]}\;v_{M}^{2}:[T_{1}]$, and $\Gamma\vdash_{H}\mathtt{cons}\;(^{T_{1}}HM^{T_{1}}\;v_{M}^{1})\;(^{[T_{1}]}HM^{[T_{1}]}\;v_{M}^{2}):[T_{1}]$ by inversion (Lemma \ref{i}) and uniqueness of types.  $\Gamma\vdash_{H}\mathtt{cons}\;(^{T_{1}}HM^{T_{1}}\;v_{M}^{1})\;(^{[T_{1}]}HM^{[T_{1}]}\;v_{M}^{2}):T$ because $[T_{1}]=T$.
\end{case}
\begin{case}
$^{[T_{1}]}HM^{[T_{1}]}\;(^{[T_{1}]}MH^{[T_{1}]}\;(\mathtt{cons}\;e_{H}^{1}\;e_{H}^{2}))\rightarrow\mathtt{cons}\;e_{H}^{1}\;e_{H}^{2}$

$^{[T_{1}]}HM^{[T_{1}]}\;(^{[T_{1}]}MH^{[T_{1}]}\;(\mathtt{cons}\;e_{H}^{1}\;e_{H}^{2})):T$ by premise and uniqueness of types (Lemma \ref{uot}).  $T=[T_{1}]$, $\Gamma\vdash_{H}{^{[T_{1}]}M}H^{[T_{1}]}\;(\mathtt{cons}\;e_{H}^{1}\;e_{H}^{2}):[T_{1}]$, and $\Gamma\vdash_{H}\mathtt{cons}\;e_{H}^{1}\;e_{H}^{2}:[T_{1}]$ by inversion (Lemma \ref{i}) and uniqueness of types.  $\Gamma\vdash_{H}\mathtt{cons}\;e_{H}^{1}\;e_{H}^{2}:T$ because $[T_{1}]=T$.
\end{case}
\begin{case}
$\mathtt{hd}\;(^{[T_{1}]}MH^{[T_{1}]}\;(\mathtt{cons}\;e_{H}^{1}\;e_{H}^{2}))\rightarrow{^{T_{1}}M}H^{T_{1}}\;e_{H}^{1}$

$\Gamma\vdash_{M}\mathtt{hd}\;(^{[T_{1}]}MH^{[T_{1}]}\;(\mathtt{cons}\;e_{H}^{1}\;e_{H}^{2})):T$ by premise and uniqueness of types (Lemma \ref{uot}).  $T=T_{1}$, $\Gamma\vdash_{H}e_{H}^{1}:T_{1}$, and $^{T_{1}}MH^{T_{1}}\;e_{H}^{1}:T_{1}$ by inversion (Lemma \ref{i}) and uniqueness of types (Lemma \ref{uot}).  $^{T_{1}}MH^{T_{1}}\;e_{H}^{1}:T$ because $T_{1}=T$.
\end{case}
\begin{case}
$\mathtt{hd}\;(^{[T_{1}]}MH^{[T_{1}]}\;(\mathtt{cons}\;e_{H}^{1}\;e_{H}^{2}))\rightarrow{^{[T_{1}]}M}H^{[T_{1}]}\;e_{H}^{2}$

$\Gamma\vdash_{M}\mathtt{hd}\;(^{[T_{1}]}MH^{[T_{1}]}\;(\mathtt{cons}\;e_{H}^{1}\;e_{H}^{2})):T$ by premise and uniqueness of types (Lemma \ref{uot}).  $T=T_{1}$, $\Gamma\vdash_{H}e_{H}^{1}:T_{1}$, and $^{[T_{1}]}MH^{[T_{1}]}\;e_{H}^{1}:[T_{1}]$ by inversion (Lemma \ref{i}) and uniqueness of types (Lemma \ref{uot}).  $^{[T_{1}]}MH^{[T_{1}]}\;e_{H}^{2}:T$ because $[T_{1}]=T$.
\end{case}
\begin{case}
$^{[T_{1}]}AS\;(\mathtt{cons}\;v_{S}^{1}\;v_{S}^{2})\rightarrow\mathtt{cons}\;(^{T_{1}}AS\;v_{S}^{1})\;(^{[T_{1}]}AS\;v_{S}^{2})$ where $A\in\lbrace H,M\rbrace$

$\Gamma\vdash_{A}{^{[T_{1}]}A}S\;(\mathtt{cons}\;v_{S}^{1}\;v_{S}^{2}):T$ by premise and uniqueness of types (Lemma \ref{uot}).  $T=[T_{1}]$ by inversion (Lemma \ref{i}) and uniqueness of types.  $\Gamma\vdash_{S}v_{S}^{1}:TST$, and $\Gamma\vdash_{S}v_{S}^{2}:TST$ by inversion.  $\Gamma\vdash_{A}{^{T_{1}}A}S\;v_{S}^{1}:T_{1}$, $\Gamma\vdash_{A}{^{[T_{1}]}A}S\;v_{S}^{2}:[T_{1}]$, and $\Gamma\vdash_{A}\mathtt{cons}\;(^{T_{1}}AS\;v_{S}^{1})\;(^{[T_{1}]}AS\;v_{S}^{2}):[T_{1}]$ by inversion and uniqueness of types.  $\Gamma\vdash_{A}\mathtt{cons}\;(^{T_{1}}AS\;v_{S}^{1})\;(^{[T_{1}]}AS\;v_{S}^{2}):T$ because $[T_{1}]=T$.
\end{case}
\begin{case}
$^{[T_{1}]}HS$ $(SH^{[T_{1}]}$ $(\mathtt{cons}$ $e_{H}^{1}$ $e_{H}^{2}))\rightarrow\mathtt{cons}$ $e_{H}^{1}$ $e_{H}^{2}$

$^{[T_{1}]}HS$ $(SH^{[T_{1}]}$ $(\mathtt{cons}$ $e_{H}^{1}$ $e_{H}^{2})):T$ by premise and uniqueness of types (Lemma \ref{uot}).  $\Gamma\vdash_{H}\mathtt{cons}$ $e_{H}^{1}$ $e_{H}^{2}:[T_{1}]$ by inversion (Lemma \ref{i}) and uniqueness of types.  $\Gamma\vdash_{S}SH^{[T_{1}]}$ $(\mathtt{cons}$ $e_{H}^{1}$ $e_{H}^{2}):TST$ by inversion.  $T=[T_{1}]$ by inversion and uniqueness of types.  $\Gamma\vdash_{H}\mathtt{cons}$ $e_{H}^{1}$ $e_{H}^{2}:T$ because $[T_{1}]=T$.
\end{case}
\begin{case}
$^{[T_{1}]}MS$ $(SH^{[T_{1}]}$ $(\mathtt{cons}$ $e_{H}^{1}$ $e_{H}^{2}))\rightarrow{^{[T_{1}]}M}H^{[T_{1}]}$ $(\mathtt{cons}$ $e_{H}^{1}$ $e_{H}^{2})$

$^{[T_{1}]}MS$ $(SH^{[T_{1}]}$ $(\mathtt{cons}$ $e_{H}^{1}$ $e_{H}^{2})):T$ by premise and uniqueness of types (Lemma \ref{i}).  $\Gamma\vdash_{H}\mathtt{cons}$ $e_{H}^{1}$ $e_{H}^{2}:[T_{1}]$ by inversion (Lemma \ref{i}) and uniqueness of types.  $\Gamma\vdash_{S}SH^{[T_{1}]}$ $(\mathtt{cons}$ $e_{H}^{1}$ $e_{H}^{2}):TST$ by inversion.  $T=[T_{1}]$ and $\Gamma\vdash_{M}{^{[T_{1}]}M}H^{[T_{1}]}$ $(\mathtt{cons}$ $e_{H}^{1}$ $e_{H}^{2}):[T_{1}]$ by inversion and uniqueness of types.  $\Gamma\vdash_{M}{^{[T_{1}]}M}H^{[T_{1}]}$ $(\mathtt{cons}$ $e_{H}^{1}$ $e_{H}^{2}):T$ because $[T_{1}]=T$.
\end{case}
\begin{case}
$\mathtt{hd}\;(SH^{[T_{1}]}\;(\mathtt{cons}\;e_{H}^{1}\;e_{H}^{2}))\rightarrow SH^{T_{1}}\;e_{H}^{1}$

$\Gamma\vdash_{S}\mathtt{hd}\;(SH^{[T_{1}]}\;(\mathtt{cons}\;e_{H}^{1}\;e_{H}^{2})):TST$ by premise.  $\Gamma\vdash_{H}e_{H}^{1}:T_{1}$ by inversion (Lemma \ref{i}) and uniqueness of types (Lemma \ref{uot}).  $\Gamma\vdash_{S}SH^{T_{1}}\;e_{H}^{1}:TST$ by inversion.
\end{case}
\begin{case}
$\mathtt{tl}\;(SH^{[T_{1}]}\;(\mathtt{cons}\;e_{H}^{1}\;e_{H}^{2}))\rightarrow SH^{[T_{1}]}\;e_{H}^{2}$

$\Gamma\vdash_{S}\mathtt{tl}\;(SH^{[T_{1}]}\;(\mathtt{cons}\;e_{H}^{1}\;e_{H}^{2})):TST$ by premise.  $\Gamma\vdash_{H}e_{H}^{2}:[T_{1}]$ by inversion (Lemma \ref{i}) and uniqueness of types (Lemma \ref{uot}).  $\Gamma\vdash_{S}SH^{[T_{1}]}\;e_{H}^{2}:TST$ by inversion.
\end{case}
\begin{case}
$SM^{[T_{1}]}\;(\mathtt{cons}\;v_{M}^{1}\;v_{M}^{2})\rightarrow\mathtt{cons}\;(SM^{T_{1}}\;v_{M}^{1})\;(SM^{[T_{1}]}\;v_{M}^{2})$

$\Gamma\vdash_{S}SM^{[T_{1}]}\;(\mathtt{cons}\;v_{M}^{1}\;v_{M}^{2}):TST$ by premise.  $\Gamma\vdash_{M}v_{M}^{1}:T_{1}$ and $\Gamma\vdash_{M}v_{M}^{2}:[T_{1}]$ by inversion (Lemma \ref{i}) and uniqueness of types (Lemma \ref{uot}).  $\Gamma\vdash_{S}SM^{T_{1}}\;v_{M}^{1}:TST$, $\Gamma\vdash_{S}SM^{[T_{1}]}\;v_{M}^{2}:TST$, and $\Gamma\vdash_{S}\mathtt{cons}\;(SM^{T_{1}}\;v_{M}^{1})\;(SM^{[T_{1}]}\;v_{M}^{2}):TST$ by inversion.
\end{case}
\begin{case}
$SM^{[T_{1}]}\;(^{[T_{1}]}MH^{[T_{1}]}\;(\mathtt{cons}\;e_{H}^{1}\;e_{H}^{2}))\rightarrow SH^{[T_{1}]}\;(\mathtt{cons}\;e_{H}^{1}\;e_{H}^{2})$

$SM^{[T_{1}]}\;(^{[T_{1}]}MH^{[T_{1}]}\;(\mathtt{cons}\;e_{H}^{1}\;e_{H}^{2})):TST$ by premise.  $\Gamma\vdash_{H}\mathtt{cons}\;e_{H}^{1}\;e_{H}^{2}:[T_{1}]$ and $\Gamma\vdash_{M}{^{[T_{1}]}M}H^{[T_{1}]}\;(\mathtt{cons}\;e_{H}^{1}\;e_{H}^{2}):[T_{1}]$ by inversion (Lemma \ref{i}) and uniqueness of types (Lemma \ref{uot}).  $\Gamma\vdash_{S}SH^{[T_{1}]}\;(\mathtt{cons}\;e_{H}^{1}\;e_{H}^{2}):TST$ by inversion.
\end{case}
\begin{case}
$^{[T_{1}]}AB^{[T_{1}]}\;\mathtt{nil}^{T_{1}}\rightarrow\mathtt{nil}^{T_{1}}$ where $(A,B)\in\lbrace(H,M),(M,H)\rbrace$

$\Gamma\vdash_{A}{^{[T_{1}]}A}B^{[T_{1}]}\;\mathtt{nil}^{T_{1}}:T$ by premise and uniqueness of types (Lemma \ref{uot}).  $\Gamma\vdash_{A}\mathtt{nil}^{T_{1}}:[T_{1}]$ and $T=[T_{1}]$ by inversion (Lemma \ref{i}) and uniqueness of types.  $\Gamma\vdash_{A}\mathtt{nil}^{T_{1}}:T$ because $[T_{1}]=T$.
\end{case}
\begin{case}
$^{[T_{1}]}AS$ $\mathtt{nil}\rightarrow\mathtt{nil}^{T_{1}}$ where $A\in\lbrace H,M\rbrace$

$\Gamma\vdash_{A}{^{[T_{1}]}A}S$ $\mathtt{nil}:T$ by premise and uniqueness of types (Lemma \ref{uot}).  $T=[T_{1}]$ and $\Gamma\vdash_{A}\mathtt{nil}^{T_{1}}:[T_{1}]$ by inversion (Lemma \ref{i}) and uniqueness of types.  $\Gamma\vdash_{A}\mathtt{nil}^{T_{1}}:T$ because $[T_{1}]=T$.
\end{case}
\begin{case}
$^{[T_{1}]}AS\;v_{S}^{1}\rightarrow{^{[T_{1}]}A}S\;(\mathtt{wrong}\;\mathrm{``Not\;a\;list"})$ $(v_{S}^{1}\neq\mathtt{cons}\;v_{S}^{2}\;v_{S}^{3}$ and $v_{S}^{1}\neq\mathtt{nil})$ where $A\in\lbrace H,M\rbrace$

$\Gamma\vdash_{A}{^{[T_{1}]}AS}\;v_{S}^{1}:T$ by premise and uniqueness of types (Lemma \ref{uot}).  $T=[T_{1}]$ by inversion (Lemma \ref{i}) and uniqueness of types.  $\vdash_{S}\mathtt{wrong}\;\mathrm{``Not\;a\;list"}:TST$ by inversion.  $\Gamma\vdash_{A}{^{[T_{1}]}A}S\;(\mathtt{wrong}\;\mathrm{``Not\;a\;list"}):[T_{1}]$ by inversion and uniqueness of types.  $\Gamma\vdash_{A}{^{[T_{1}]}A}S\;(\mathtt{wrong}\;\mathrm{``Not\;a\;list"}):T$ because $[T_{1}]=T$.
\end{case}
\begin{case}
$^{L}AB^{L}\;(^{L}BS\;v_{S})\rightarrow{^{L}A}S\;v_{S}$ where $(A,B)\in\lbrace(H,M),(M,H)\rbrace$

$\Gamma\vdash_{A}{^{L}A}B^{L}\;(^{L}BS\;v_{S}):T$ by premise and uniqueness of types (Lemma \ref{uot}).  $T=L$ by inversion (Lemma \ref{i}) and uniqueness of types.  $\Gamma\vdash_{S}v_{S}:TST$ by inversion.  $\Gamma\vdash_{A}{^{L}A}S\;v_{S}:L$ by inversion and uniqueness of types.  $\Gamma\vdash_{A}{^{L}A}S\;v_{S}:T$ because $L=T$.
\end{case}
\begin{case}
$^{T_{1}^{a}}AS$ $(SA^{T_{1}^{a}}$ $B_{A})\rightarrow B_{A}$ where $(A,B)\in\lbrace(H,e),(M,v)\rbrace$

$^{T_{1}^{a}}AS$ $(SA^{T_{1}^{a}}$ $B_{A}):T$ by premise and uniqueness of types (Lemma \ref{uot}).  $\Gamma\vdash_{A}B_{A}:T_{1}^{a}[T_{i}/T_{i}^{a}]$ by inversion (Lemma \ref{i}) and uniqueness of types.  $\Gamma\vdash_{S}SA^{T_{1}^{a}}$ $B_{A}:TST$ by inversion.  $T=T_{1}^{a}[T_{i}/T_{i}^{a}]$ by inversion and uniqueness of types.  $\Gamma\vdash_{A}B_{A}:T$ because $T_{1}^{a}[T_{i}/T_{i}^{a}]=T$.
\end{case}
\begin{case}
$^{T_{1}^{a}}AS\;v_{S}\rightarrow{^{T_{1}^{a}}A}S\;(\mathtt{wrong}\;\mathrm{``Parametricity\;violated"})$ $(v_{S}\neq SA^{T_{1}^{a}}\;B_{A})$ where $(A,B)\in\lbrace(H,e),(M,v)\rbrace$

$\Gamma\vdash_{A}{^{T_{1}^{a}}A}S\;v_{S}:T$ by premise and uniqueness of types (Lemma \ref{uot}).  $T=T_{1}^{a}[T_{i}/T_{i}^{a}]$ by inversion (Lemma \ref{i}) and uniqueness of types.  $\vdash_{S}\mathtt{wrong}\;\mathrm{``Parametricity\;violated"}:TST$ by inversion.  $^{T_{1}^{a}}AS\;(\mathtt{wrong}\;\mathrm{``Parametricity\;violated"}):T_{1}^{a}[T_{i}/T_{i}^{a}]$ by inversion and uniqueness of types.  $^{T_{1}^{a}}AS\;(\mathtt{wrong}\;\mathrm{``Parametricity\;violated"}):T$ because $T_{1}^{a}[T_{i}/T_{i}^{a}]=T$.
\end{case}
\end{proof}
\end{theorem}