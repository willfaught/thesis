\chapter{Proof of Type Soundness}

Proving the progress of expressions and the preservation of types proves the type soundness of the model of computation.  Progress ensures that a well-typed, closed expression is either a value, reducible to another expression, or reducible to an error.  Preservation ensures that if a well-typed expression reduces to another expression, the other expression is well-typed and has the same type.  The proof extends the proof by Kinghorn \cite{kinghorn07}, which was based on proofs by Pierce \cite{pierce02} and Matthews and Findler \cite{matthews07}.

\section{Progress}

Progress will be proven by structural induction on a well-typed, closed expression of each syntactic form.  In each case, the expression will be proven to be either a value, reducible to another expression, or reducible to an error.  The reduction of a subexpression is the reduction of its parent expression.  If a subexpression reduces to an error, its parent expression reduces to the error.  In some cases, the syntactic form of a subexpression must be determined to reduce its parent expression.  Determining the unique type of a subexpression determines its syntactic form.

\subsection{Inversion Lemma}

Inverting the typing relations enables the syntactic forms of well-typed expressions to determine the types of their subexpressions:

\begin{lemma}
\label{i}
\onehalfspacing
The type of a term of each syntactic form can be calculated from the types of its subterms.
\begin{enumerate}
\item If $\Gamma\vdash_{A}x:T$ then $x:T_{1}\in\Gamma$ and $T=T_{1}$ where $A\in\lbrace H,M\rbrace$.
\item If $\Gamma\vdash_{S}x:TST$ then $x:TST\in\Gamma$.
\item If $\vdash_{A}\overline{n}:T$ then $T=N$ where $A\in\lbrace H,M\rbrace$.
\item $\vdash_{S}\overline{n}:TST$.
\item If $\Gamma\vdash_{A}\lambda x:T_{1}.e_{A}:T$ then $\Gamma\vdash_{A}T_{1}$, $\Gamma,x:T_{1}\vdash_{A}e_{A}:T_{2}$, and $T=T_{1}\rightarrow T_{2}$ where $A\in\lbrace H,M\rbrace$.
\item If $\Gamma\vdash_{S}\lambda x.e_{S}:TST$ then $\Gamma,x:TST\vdash_{S}e_{S}:TST$.
\item If $\Gamma\vdash_{A}\Lambda X.e_{A}:T$ then $\Gamma,X\vdash_{A}e_{A}:T_{1}$ and $T=\forall X.T_{1}$ where $A\in\lbrace H,M\rbrace$.
\item If $\Gamma\vdash_{A}\mathtt{cons}\;e_{A}^{1}\;e_{A}^{2}:T$ then $\Gamma\vdash_{A}e_{A}^{1}:T_{1}$, $\Gamma\vdash_{A}e_{A}^{2}:[T_{1}]$, and $T=[T_{1}]$ where $A\in\lbrace H,M\rbrace$.
\item If $\Gamma\vdash_{S}\mathtt{cons}\;e_{S}^{1}\;e_{S}^{2}:TST$ then $\Gamma\vdash_{S}e_{S}^{1}:TST$ and $\Gamma\vdash_{S}e_{S}^{2}:TST$.
\item If $\Gamma\vdash_{A}\mathtt{nil}^{T_{1}}:T$ then $\Gamma\vdash_{A}T_{1}$ and $T=[T_{1}]$ where $A\in\lbrace H,M\rbrace$.
\item $\vdash_{S}\mathtt{nil}:TST$.
\item If $\Gamma\vdash_{A}e_{A}^{1}\;e_{A}^{2}:T$ then $\Gamma\vdash_{A}e_{A}^{1}:T_{1}\rightarrow T_{2}$, $\Gamma\vdash_{A}e_{A}^{2}:T_{1}$, and $T=T_{2}$ where $A\in\lbrace H,M\rbrace$.
\item If $\Gamma\vdash_{S}e_{S}^{1}\;e_{S}^{2}:TST$ then $\Gamma\vdash_{S}e_{S}^{1}:TST$ and $\Gamma\vdash_{S}e_{S}^{2}:TST$.
\item If $\Gamma\vdash_{A}e_{A}\;\lbrace T_{1}\rbrace:T$ then $\Gamma\vdash_{A}T_{1}$, $\Gamma\vdash_{A}e_{A}:\forall X.T_{2}$, and $T=T_{2}[T_{1}/X]$ where $A\in\lbrace H,M\rbrace$.
\item If $\Gamma\vdash_{A}\mathtt{if0}\;e_{A}^{1}\;e_{A}^{2}\;e_{A}^{3}:T$ then $\Gamma\vdash_{A}e_{A}^{1}:N$, $\Gamma\vdash_{A}e_{A}^{2}:T_{1}$, $\Gamma\vdash_{A}e_{A}^{3}:T_{1}$, and $T=T_{1}$ where $A\in\lbrace H,M\rbrace$.
\item If $\Gamma\vdash_{S}\mathtt{if0}\;e_{S}^{1}\;e_{S}^{2}\;e_{S}^{3}:TST$ then $\Gamma\vdash_{S}e_{S}^{1}:TST$, $\Gamma\vdash_{S}e_{S}^{2}:TST$, and $\Gamma\vdash_{S}e_{S}^{3}:TST$.
\item If $\Gamma\vdash_{A}o\;e_{A}^{1}\;e_{A}^{2}:T$ then $\Gamma\vdash_{A}e_{A}^{1}:N$, $\Gamma\vdash_{A}e_{A}^{2}:N$, and $T=N$ where $A\in\lbrace H,M\rbrace$.
\item If $\Gamma\vdash_{S}o\;e_{S}^{1}\;e_{S}^{2}:TST$ then $\Gamma\vdash_{S}e_{S}^{1}:TST$ and $\Gamma\vdash_{S}e_{S}^{2}:TST$.
\item If $\Gamma\vdash_{A}\mathtt{hd}\;e_{A}:T$ then $\Gamma\vdash_{A}e_{A}:[T_{1}]$ and $T=T_{1}$ where $A\in\lbrace H,M\rbrace$.
\item If $\Gamma\vdash_{A}\mathtt{tl}\;e_{A}:T$ then $\Gamma\vdash_{A}e_{A}:[T_{1}]$ and $T=[T_{1}]$ where $A\in\lbrace H,M\rbrace$.
\item If $\Gamma\vdash_{S}f\;e_{S}:TST$ then $\Gamma\vdash_{S}e_{S}:TST$.
\item If $\Gamma\vdash_{A}\mathtt{fix}\;e_{A}:T$ then $\Gamma\vdash_{A}e_{A}:T_{1}\rightarrow T_{1}$ and $T=T_{1}$ where $A\in\lbrace H,M\rbrace$.
\item If $\Gamma\vdash_{S}p\;e_{S}:TST$ then $\Gamma\vdash_{S}e_{S}:TST$.
\item $\vdash_{S}\mathtt{wrong}\;\mathrm{string}:TST$.
\item If $\Gamma\vdash_{A}{^{T_{1}}A}B^{T_{1}}\;e_{B}:T$ then $\Gamma\vdash_{A}T_{1}$, $\Gamma\vdash_{B}T_{1}$, $\Gamma\vdash_{B}e_{B}:T_{1}$, and $T=T_{1}$ where $(A,B)\in\lbrace(H,M),(M,H)\rbrace$.
\item If $\Gamma\vdash_{A}{^{T_{1}}A}S\;e_{S}:T$ then $\Gamma\vdash_{A}T_{1}$, $\Gamma\vdash_{S}e_{S}:TST$, and $T=T_{1}[T_{i}/T_{i}^{a}]$ where $A\in\lbrace H,M\rbrace$.
\item If $\Gamma\vdash_{S}SA^{T_{1}}\;e_{A}:T$ then $\Gamma\vdash_{A}T_{1}$, $\Gamma\vdash_{A}e_{A}:T_{1}[T_{i}/T_{i}^{a}]$, and $T=TST$ where $A\in\lbrace H,M\rbrace$.
\end{enumerate}
\begin{proof}
Immediate from the definitions of the typing relations.
\end{proof}
\end{lemma}

\subsection{Uniqueness of Types Lemma}

Well-typed Haskell and ML expressions have unique types:

\begin{lemma}
\label{uot}
%\onehalfspacing
$e_{A}$ has at most one type $T$ for a given context $\Gamma$ where $A\in\lbrace H,M\rbrace$.
\begin{proof}
By structural induction on $e_{A}$ using inversion (Lemma \ref{i}).
\end{proof}
\end{lemma}

\subsection{Canonical Forms Lemma}

The types of Haskell and ML values determine their syntactic forms:

\begin{lemma}
\label{cf}
%\onehalfspacing
The possible syntactic forms of values of various types.
\begin{enumerate}
\item If $v_{A}:N$ then $v_{A}=\overline{n}$ where $A\in\lbrace H,M\rbrace$.
\item If $v_{A}:T_{1}\rightarrow T_{2}$ then $v_{A}=\lambda x:T_{1}.e_{A}$ where $A\in\lbrace H,M\rbrace$.
\item If $v_{A}:\forall X.T$ then $v_{A}\in\lbrace\Lambda X.e_{A},{^{\forall X.T}A}S$ $v_{S}\rbrace$ where $A\in\lbrace H,M\rbrace$.
\item If $v_{H}:[T]$ then $v_{H}\in\lbrace\mathtt{cons}$ $e_{H}^{1}$ $e_{H}^{2},\mathtt{nil}^{T}\rbrace$.
\item If $v_{M}:[T]$ then $v_{M}\in\lbrace\mathtt{cons}$ $v_{M}^{1}$ $v_{M}^{2},\mathtt{nil}^{T},{^{[T]}M}H^{[T]}$ $(\mathtt{cons}$ $e_{H}$ $e_{H})\rbrace$.
\item If $v_{A}:L$ then $v_{A}={^{L}A}S$ $v_{S}$ where $A\in\lbrace H,M\rbrace$.
\end{enumerate}
\begin{proof}
Immediate from the definitions of values and the typing relations.
\end{proof}
\end{lemma}

\subsection{Haskell and ML Progress Theorem}

\newcommand{\pscases}[4]{If #1 is a \profv then #2 and #3 determine the reduction of #4.}

\newcommand{\pserr}[2]{If #1 \red \emph{\experr{\varstr}} then #2 \red \emph{\experr{\varstr}}.}
\newcommand{\pserrand}[3]{If #1 \red \emph{\experr{\varstr}} and #2 is a \profv then #3 \red \emph{\experr{\varstr}}.}

\newcommand{\pshyp}[2]{#1 is a \profv or #1 \red #2 or #1 \red \emph{\experr{\varstr}}}
\newcommand{\pshypby}[2]{\pshyp{#1}{#2} by the induction hypothesis.}

\newcommand{\pssub}[4]{If #1 \red #2 then #3 \red #4.}
\newcommand{\pssuband}[5]{If #1 \red #2 and #3 is a \profv then #4 \red #5.}

\newcommand{\psred}[2]{\redrule{#1}{#2}.}
\newcommand{\psrednote}[3]{\redrule{#1}{#2} $(#3)$.}

\newcommand{\psval}[2]{#1 by lemmas \ref{leminv} and \ref{lemuni} and #2 by lemma \ref{lemcan}.}
\newcommand{\psvalh}[3]{\psval{\judeh{}{#1}{#2}}{#3}}
\newcommand{\psvalm}[3]{\psval{\judem{}{#1}{#2}}{#3}}
\newcommand{\psvaleqh}[3]{\psvalh{#1}{#2}{#1 $=$ #3}}
\newcommand{\psvaleqm}[3]{\psvalm{#1}{#2}{#1 $=$ #3}}
\newcommand{\psvalinh}[3]{\psvalh{#1}{#2}{$#1 \in \lbrace #3 \rbrace$}}
\newcommand{\psvalinm}[3]{\psvalm{#1}{#2}{$#1 \in \lbrace #3 \rbrace$}}
\newcommand{\psvalif}[2]{If #1 is a \profv then #2}
\newcommand{\psvalifeqh}[3]{\psvalif{#1}{\psvaleqh{#1}{#2}{#3}}}
\newcommand{\psvalifeqm}[3]{\psvalif{#1}{\psvaleqm{#1}{#2}{#3}}}
\newcommand{\psvalifinh}[3]{\psvalif{#1}{\psvalinh{#1}{#2}{#3}}}
\newcommand{\psvalifinm}[3]{\psvalif{#1}{\psvalinm{#1}{#2}{#3}}}

\newcommand{\w}{}
\newcommand{\x}{}
\newcommand{\y}{}
\newcommand{\z}{}

\begin{theorem}{Haskell Progress}

\label{thmhps}

If \judeh{}{\varexph}{\vartyh} then \pshyp{\first{\varexph}}{\second{\varexph}}.

\begin{proof}

By structural induction on \varexph.

% Haskell

% \x:t.e

\newcommand{\psfabss}{\expfabss{\varvarh}{\vartyh}{\varexph}\xspace}

\begin{case}{\psfabss}

\psfabss is a \profv.

\end{case}

% \\u.e

\newcommand{\pstabs}{\exptabs{\tyvarh}{\varexph}\xspace}

\begin{case}{\pstabs}

\pstabs is a \profv.

\end{case}

% n

\newcommand{\psnum}{\expnum{\symnum}\xspace}

\begin{case}{\psnum}

\psnum is a \profv.

\end{case}

% nil t

\newcommand{\psnils}{\expnils{\vartyh}\xspace}

\begin{case}{\psnils}

\psnils is a \profv.

\end{case}

% cons e e

\newcommand{\psconsh}{\expcons{\first{\varexph}}{\second{\varexph}}\xspace}

\begin{case}{\psconsh}

\psconsh is a \profv.

\end{case}

% x

\newcommand{\psvar}{\varvarh\xspace}

\begin{case}{\psvar}

Cannot occur because \varexph is closed.

\end{case}

% e e

\newcommand{\psfapp}{\expfapp{\first{\varexph}}{\second{\varexph}}\xspace}
\renewcommand{\x}{\expfabss{\varvarh}{\first{\vartyh}}{\third{\varexph}}\xspace}

\begin{case}{\psfapp}

\pshypby
{\first{\varexph}}
{\third{\varexph}}
\psvalifeqh
{\first{\varexph}}
{\tyfun{\first{\vartyh}}{\second{\vartyh}}}
{\x}
\psred
{\expfapp{(\x)}{\second{\varexph}}}
{\expsubst{\third{\varexph}}{\second{\varexph}}{\varvarh}}
\pssub
{\first{\varexph}}
{\third{\varexph}}
{\psfapp}
{\expfapp{\third{\varexph}}{\second{\varexph}}}
\pserr
{\first{\varexph}}
{\psfapp}

\end{case}

% fix e

\newcommand{\psfix}{\expfix{\first{\varexph}}\xspace}
\renewcommand{\x}{\expfabss{\varvarh}{\vartyh}{\second{\varexph}}\xspace}
\renewcommand{\y}{\expfix{(\x)}}

\begin{case}{\psfix}

\pshypby
{\first{\varexph}}
{\second{\varexph}}
\psvalifeqh
{\first{\varexph}}
{\tyfun{\vartyh}{\vartyh}}
{\x}
\psred
{\y}
{\expsubst{\second{\varexph}}{\y}{\varvarh}}
\pssub
{\first{\varexph}}
{\second{\varexph}}
{\psfix}
{\expfix{\second{\varexph}}}
\pserr
{\first{\varexph}}
{\psfix}

\end{case}

% e<t>

\newcommand{\pstapp}{\exptapp{\first{\varexph}}{\first{\vartyh}}\xspace}
\renewcommand{\x}{\exptabs{\tyvarh}{\second{\varexph}}\xspace}

\begin{case}{\pstapp}

\pshypby
{\first{\varexph}}
{\second{\varexph}}
\psvalifeqh
{\first{\varexph}}
{\tyfor{\tyvarh}{\second{\vartyh}}}
{\x}
\psred
{\exptapp{(\x)}{\first{\vartyh}}}
{\expsubst{\second{\varexph}}{\first{\vartyh}}{\tyvarh}}
\pssub
{\first{\varexph}}
{\second{\varexph}}
{\pstapp}
{\exptapp{\second{\varexph}}{\first{\vartyh}}}
\pserr
{\first{\varexph}}
{\pstapp}

\end{case}

% f e

\newcommand{\psfield}{\expfield{\first{\varexph}}\xspace}
\renewcommand{\x}{\expnils{\vartyh}\xspace}
\renewcommand{\y}{\expcons{\second{\varexph}}{\third{\varexph}}\xspace}

\begin{case}{\psfield}

\pshypby
{\first{\varexph}}
{\second{\varexph}}
\psvalifinh
{\first{\varexph}}
{\tylist{\vartyh}}
{\x, \y}
\psred
{\exphd{(\x)}}
{\expwrongs{\vartyh}{\errempty}}
\psred
{\exptl{(\x)}}
{\expwrongs{\tylist{\vartyh}}{\errempty}}
\psred
{\exphd{(\y)}}
{\second{\varexph}}
\psred
{\exptl{(\y)}}
{\third{\varexph}}
\pssub
{\first{\varexph}}
{\second{\varexph}}
{\psfield}
{\expfield{\second{\varexph}}}
\pserr
{\first{\varexph}}
{\psfield}

\end{case}

% o e e

\newcommand{\psop}{\expop{\first{\varexph}}{\second{\varexph}}\xspace}
\renewcommand{\x}{\first{\expnum{\varnum}}\xspace}
\renewcommand{\y}{\second{\expnum{\varnum}}\xspace}

\begin{case}{\psop}

\pshypby
{\first{\varexph}}
{\third{\varexph}}
\psvalifeqh
{\first{\varexph}}
{\tynum}
{\x}
\pssub
{\first{\varexph}}
{\third{\varexph}}
{\psop}
{\expop{\third{\varexph}}{\second{\varexph}}}
\pserr
{\first{\varexph}}
{\psop}
\pshypby
{\second{\varexph}}
{\third{\varexph}}
\psvalifeqh
{\second{\varexph}}
{\tynum}
{\y}
\pssuband
{\second{\varexph}}
{\third{\varexph}}
{\first{\varexph}}
{\psop}
{\expop{\first{\varexph}}{\third{\varexph}}}
\pserrand
{\second{\varexph}}
{\first{\varexph}}
{\psop}
\psred
{\expadd{\x}{\y}}
{\expnum{\first{\varnum} + \second{\varnum}}}
\psred
{\expsub{\x}{\y}}
{\expnum{\formvar{max}(\first{\varnum} - \second{\varnum}, 0)}}

\end{case}

% null? e

\newcommand{\pspnull}{\exppnull{\first{\varexph}}\xspace}
\renewcommand{\x}{\expnils{\vartyh}\xspace}
\renewcommand{\y}{\expcons{\second{\varexph}}{\third{\varexph}}}

\begin{case}{\pspnull}

\pshypby
{\first{\varexph}}
{\second{\varexph}}
\psvalifinh
{\first{\varexph}}
{\tylist{\vartyh}}
{\x, \y}
\psred
{\exppnull{(\x)}}
{\expnum{0}}
\psred
{\exppnull{(\y)}}
{\expnum{1}}
\pssub
{\first{\varexph}}
{\second{\varexph}}
{\pspnull}
{\exppnull{\second{\varexph}}}
\pserr
{\first{\varexph}}
{\pspnull}

\end{case}

% if0 e e e

\newcommand{\psif}{\expif{\first{\varexph}}{\second{\varexph}}{\third{\varexph}}\xspace}
\renewcommand{\x}{\expnum{\varnum}\xspace}

\begin{case}{\psif}

\pshypby
{\first{\varexph}}
{\fourth{\varexph}}
\psvalifeqh
{\first{\varexph}}
{\tynum}
{\x}
\psred
{\expif{\expnum{0}}{\second{\varexph}}{\third{\varexph}}}
{\second{\varexph}}
\psrednote
{\expif{\x}{\second{\varexph}}{\third{\varexph}}}
{\third{\varexph}}
{n \neq 0}
\pssub
{\first{\varexph}}
{\fourth{\varexph}}
{\psif}
{\expif{\fourth{\varexph}}{\second{\varexph}}{\third{\varexph}}}
\pserr
{\first{\varexph}}
{\psif}

\end{case}

% wrong t string

\newcommand{\pswrongs}{\expwrongs{\vartyh}{\varstr}\xspace}

\begin{case}{\pswrongs}

\psred
{\pswrongs}
{\emph{\experr{\varstr}}}

\end{case}

% hm t t e

\newcommand{\pshm}{\exphm{\first{\vartyh}}{\first{\vartym}}{\first{\varexpm}}}

\begin{case}{\pshm}

\pshypby
{\first{\varexpm}}
{\second{\varexpm}}
\pscases
{\first{\varexpm}}
{\first{\vartyh}}
{\first{\vartym}}
{\pshm}

% L, *

\begin{subcase}{\first{\vartyh} $=$ \tylump}

\exphm{\tylump}{\first{\vartym}}{\first{\varexpm}} is a \profv.

\end{subcase}

% !L, L

\begin{subcase}{\first{\vartyh} $\neq$ \tylump and \first{\vartym} $=$ \tylump}

\psvalinh
{\first{\varexpm}}
{\tylump}
{\expmh{\tylump}{\second{\vartyh}}{\varexph}, \expms{\cslump}{\varvalfs}}
\psrednote
{\exphm{\first{\vartyh}}{\tylump}{(\expmh{\tylump}{\second{\vartyh}}{\varexph})}}
{\varexph}
{\first{\vartyh} = \second{\vartyh}}
\psrednote
{\exphm{\first{\vartyh}}{\tylump}{(\expmh{\tylump}{\second{\vartyh}}{\varexph})}}
{\varexph}
{\first{\vartyh} \neq \second{\vartyh}}
\psred
{\exphm{\first{\vartyh}}{\tylump}{(\expms{\cslump}{\varvalfs})}}
{\expwrongs{\first{\vartyh}}{\errvalue}}

\end{subcase}

% N, N

\begin{subcase}{\first{\vartyh} $=$ \tynum and \first{\vartym} $=$ \tynum}

\psvaleqm
{\first{\varexpm}}
{\tynum}
{\expnum{\varnum}}
\psred
{\exphm{\tynum}{\tynum}{\expnum{\varnum}}}
{\expnum{\varnum}}

\end{subcase}

% {t}, {t}

\renewcommand{\w}{\tylist{\second{\vartyh}}}
\renewcommand{\x}{\tylist{\second\vartym}}
\renewcommand{\y}{\expnils{\third{\vartym}}}
\renewcommand{\z}{\expcons{\first{\varvalum}}{\second{\varvalum}}}

\begin{subcase}{\first{\vartyh} $=$ \w and \first{\vartym} $=$ \x}

\psvalinm
{\first{\varexpm}}
{\tylist{\third{\vartym}}}
{\y, \z}
\psred
{\exphm{\w}{\x}{(\y)}}
{\expnils{\second{\vartyh}}}
\psred
{\exphm{\w}{\x}{(\z)}}
{\expcons{(\exphm{\second{\vartyh}}{\second{\vartym}}{\first{\varvalum}})}{(\exphm{\w}{\x}{\second{\varvalum}})}}

\end{subcase}

% t->t, t->t

\renewcommand{\x}{\tyfun{\second{\vartyh}}{\third{\vartyh}}}
\renewcommand{\y}{\tyfun{\second{\vartym}}{\third{\vartym}}}
\renewcommand{\z}{\expfabss{\varvarm}{\fourth{\vartym}}{\second{\varexpm}}}

\begin{subcase}{\first{\vartyh} $=$ \x and \first{\vartym} $=$ \y}

\psvaleqm
{\first{\varexpm}}
{\tyfun{\fourth{\vartym}}{\fifth{\vartym}}}
{\z}
\psred
{\exphm{(\x)}{(\y)}{(\z)}}
{\expfabss{\varvarh}{\second{\vartyh}}{\exphm{\third{\vartyh}}{\third{\vartym}}{(\expfapp{(\z)}{(\expmh{\second{\vartym}}{\second{\vartyh}}{\varvarh})})}}}

\end{subcase}

% Au.t, Au.t

\renewcommand{\x}{\tyfor{\first{\tyvarh}}{\second{\vartyh}}}
\renewcommand{\y}{\tyfor{\first{\tyvarm}}{\second{\vartym}}}
\renewcommand{\z}{\exptabs{\second{\tyvarm}}{\second{\varexpm}}}

\begin{subcase}{\first{\vartyh} $=$ \x and \first{\vartym} $=$ \y}

\psvaleqm
{\first{\varexpm}}
{\tyfor{\second{\tyvarm}}{\third{\vartym}}}
{\z}
\psred
{\exphm{(\x)}{(\y)}{(\z)}}
{\exptabs{\first{\tyvarh}}{\exphm{\second{\vartyh}}{\tysubst{\second{\vartym}}{\tylump}{\first{\tyvarm}}}{\expsubst{\second{\varexpm}}{\tylump}{\second{\tyvarm}}}}}

\end{subcase}

\pssub
{\first{\varexpm}}
{\second{\varexpm}}
{\pshm}
{\exphm{\first{\vartyh}}{\first{\vartym}}{\second{\varexpm}}}
\pserr
{\first{\varexpm}}
{\pshm}

\end{case}

% mh t t e

\begin{case}

$e_{M}={^{T}M}H$ $e_{H}$

$^{T}MH$ $e_{H}$ is an unforced value.

\end{case}

% hs k e

\begin{case}

$e_{A}={^{T}A}S$ $e_{S}^{1}$ where $A\in\lbrace H,M\rbrace$

$e_{S}^{1}$ is an unforced value or $e_{S}^{1}\rightarrow e_{S}^{2}$ or $e_{S}^{1}\rightarrow$ \emph{\textbf{Error}: string} by Scheme progress (Theorem \ref{sps}).  If $e_{S}^{1}\rightarrow e_{S}^{2}$ then $^{T}AS$ $e_{S}^{1}\rightarrow{^{T}A}S$ $e_{S}^{2}$.  If $e_{S}^{1}\rightarrow$ \emph{\textbf{Error}: string} then $^{T}AS$ $e_{S}^{1}\rightarrow$ \emph{\textbf{Error}: string}.  If $e_{S}^{1}$ is an unforced value then $T$ determines the reduction of $^{T}AS$ $e_{S}^{1}$:

\begin{subcase}

$T=L$

$^{L}AS$ $e_{S}^{1}$ is an unforced value.

\end{subcase}

\begin{subcase}

$T=N$

$^{N}AS$ $\overline{n}\rightarrow\overline{n}$.  $^{N}AS$ $e_{S}^{1}\rightarrow\mathtt{wrong}^{N}$ \emph{``Not a number"} $(e_{S}^{1}\neq\overline{n})$.

\end{subcase}

\begin{subcase}

$T=[T_{1}]$

$^{[T_{1}]}AS$ $\mathtt{nil}\rightarrow\mathtt{nil}^{T_{1}}$.  $^{[T_{1}]}AS$ $(\mathtt{cons}$ $v_{S}^{1}$ $v_{S}^{2})\rightarrow\mathtt{cons}$ $(^{T_{1}}AS$ $v_{S}^{1})$ $(^{[T_{1}]}AS$ $v_{S}^{2})$.  $^{[T_{1}]}HS$ $(SH^{[T_{1}]}$ $(\mathtt{cons}$ $e_{H}^{1}$ $e_{H}^{2}))\rightarrow\mathtt{cons}$ $e_{H}^{1}$ $e_{H}^{2}$.  $^{[T_{1}]}MS$ $(SH^{[T_{1}]}$ $(\mathtt{cons}$ $e_{H}^{1}$ $e_{H}^{2}))\rightarrow{^{[T_{1}]}M}H^{[T_{1}]}$ $(\mathtt{cons}$ $e_{H}^{1}$ $e_{H}^{2})$.  $^{[T_{1}]}AS$ $e_{S}^{1}\rightarrow\mathtt{wrong}^{[T_{1}[T_{i}/T_{i}^{a}]]}$ \emph{``Not a list"} $(e_{S}^{1}\not\in\lbrace\mathtt{nil},\mathtt{cons}$ $v_{S}^{1}$ $v_{S}^{2},SH^{[T_{1}]}$ $(\mathtt{cons}$ $e_{H}^{1}$ $e_{H}^{2})\rbrace)$.

\end{subcase}

\begin{subcase}

$T=T_{1}^{a}$

$^{T_{1}^{a}}HS$ $(SH^{T_{1}^{a}}$ $e_{H})\rightarrow e_{H}$.  $^{T_{1}^{a}}HS$ $e_{S}^{1}\rightarrow\mathtt{wrong}^{T_{1}}$ \emph{``Parametricity violated"} $(e_{S}^{1}\neq SH^{T_{1}^{a}}$ $e_{H})$.  $^{T_{1}^{a}}MS$ $(SM^{T_{1}^{a}}$ $v_{M})\rightarrow v_{M}$.  $^{T_{1}^{a}}MS$ $e_{S}^{1}\rightarrow\mathtt{wrong}^{T_{1}}$ \emph{``Parametricity violated"} $(e_{S}^{1}\neq SM^{T_{1}^{a}}$ $v_{M})$.

\end{subcase}

\begin{subcase}

$T=T_{1}\rightarrow T_{2}$

$^{T_{1}\rightarrow T_{2}}AS$ $(\lambda x_{1}.e_{S}^{3})\rightarrow\lambda x_{2}:T_{1}[T_{i}/T^{a}_{i}].(^{T_{2}}AS$ $((\lambda x_{1}.e_{S}^{3})$ $(SA^{T_{1}}$ $x_{2})))$.  $^{T_{1}\rightarrow T_{2}}AS$ $e_{S}^{1}\rightarrow\mathtt{wrong}^{(T_{1}\rightarrow T_{2})[T_{i}/T_{i}^{a}]]}$ \emph{``Not a function"} $(e_{S}^{1}\neq\lambda x_{1}.e_{S}^{3})$.

\end{subcase}

\begin{subcase}

$T=\forall X.T_{1}$

$^{\forall X.T_{1}}AS$ $e_{S}^{1}$ is an unforced value.

\end{subcase}

\end{case}

% ML

% cons v v

\newcommand{\psconsm}{\expcons{\first{\varvalum}}{\second{\varvalum}}\xspace}

\begin{case}

\psconsm

\psconsm is a \profv.

\end{case}

% ML e e

\begin{case}

$e_{M}=e_{M}^{1}$ $e_{M}^{2}$

$e_{M}^{1}$ is an unforced value or $e_{M}^{1}\rightarrow e_{M}^{3}$ or $e_{M}^{1}\rightarrow$ \emph{\textbf{Error}: string} by the induction hypothesis.  If $e_{M}^{1}$ is an unforced value then $e_{M}^{1}:T_{1}\rightarrow T_{2}$ by inversion (Lemma \ref{i}) and uniqueness of types (Lemma \ref{uot}) and $e_{M}^{1}=\lambda x:T_{1}.e_{M}^{4}$ by canonical forms (Lemma \ref{cf}).  If $e_{M}^{1}\rightarrow e_{M}^{3}$ then $e_{M}^{1}$ $e_{M}^{2}\rightarrow e_{M}^{3}$ $e_{M}^{2}$.  If $e_{M}^{1}\rightarrow$ \emph{\textbf{Error}: string} then $e_{M}^{1}$ $e_{M}^{2}\rightarrow$ \emph{\textbf{Error}: string}.  $e_{M}^{2}$ is an unforced value or $e_{M}^{2}\rightarrow e_{M}^{5}$ or $e_{M}^{2}\rightarrow$ \emph{\textbf{Error}: string} by the induction hypothesis.  If $e_{M}^{2}\rightarrow e_{M}^{5}$ and $e_{M}^{1}$ is an unforced value then $e_{M}^{1}$ $e_{M}^{2}\rightarrow e_{M}^{1}$ $e_{M}^{5}$.  If $e_{M}^{2}\rightarrow$ \emph{\textbf{Error}: string} and $e_{M}^{1}$ is an unforced value then $e_{M}^{1}$ $e_{M}^{2}\rightarrow$ \emph{\textbf{Error}: string}.  If $e_{M}^{1}$ is an unforced value and $e_{M}^{2}$ is an unforced value then $(\lambda x:T_{1}.e_{M}^{4})$ $e_{M}^{2}\rightarrow e_{M}^{4}[e_{M}^{2}/x]$.

\end{case}

% ML cons e e

\begin{case}

$e_{M}=\mathtt{cons}$ $e_{M}^{1}$ $e_{M}^{2}$

$e_{M}^{1}$ is an unforced value or $e_{M}^{1}\rightarrow e_{M}^{3}$ or $e_{M}^{1}\rightarrow$ \emph{\textbf{Error}: string} by the induction hypothesis.  If $e_{M}^{1}\rightarrow e_{M}^{3}$ then $\mathtt{cons}$ $e_{M}^{1}$ $e_{M}^{2}\rightarrow\mathtt{cons}$ $e_{M}^{3}$ $e_{M}^{2}$.  If $e_{M}^{1}\rightarrow$ \emph{\textbf{Error}: string} then $\mathtt{cons}$ $e_{M}^{1}$ $e_{M}^{2}\rightarrow$ \emph{\textbf{Error}: string}.  $e_{M}^{2}$ is an unforced value or $e_{M}^{2}\rightarrow e_{M}^{4}$ or $e_{M}^{2}\rightarrow$ \emph{\textbf{Error}: string} by the induction hypothesis.  If $e_{M}^{2}\rightarrow e_{M}^{4}$ and $e_{M}^{1}$ is an unforced value then $\mathtt{cons}$ $e_{M}^{1}$ $e_{M}^{2}\rightarrow\mathtt{cons}$ $e_{M}^{1}$ $e_{M}^{4}$.  If $e_{M}^{2}\rightarrow$ \emph{\textbf{Error}: string} and $e_{M}^{1}$ is an unforced value then $\mathtt{cons}$ $e_{M}^{1}$ $e_{M}^{2}\rightarrow$ \emph{\textbf{Error}: string}.  If $e_{M}^{1}$ and $e_{M}^{2}$ are unforced values then $\mathtt{cons}$ $e_{M}^{1}$ $e_{M}^{2}$ is an unforced value.

\end{case}

% ML f e

\begin{case}

$e_{M}=f$ $e_{M}^{1}$

$e_{M}^{1}$ is an unforced value or $e_{M}^{1}\rightarrow e_{M}^{2}$ or $e_{M}^{1}\rightarrow$ \emph{\textbf{Error}: string} by the induction hypothesis.  If $e_{M}^{1}$ is an unforced value then $e_{M}^{1}:[T]$ by inversion (Lemma \ref{i}) and uniqueness of types (Lemma \ref{uot}) and $e_{M}^{1}\in\lbrace\mathtt{nil}^{T},\mathtt{cons}$ $v_{M}^{1}$ $v_{M}^{2},{^{[T]}M}H^{[T]}$ $(\mathtt{cons}$ $e_{H}^{1}$ $e_{H}^{2})\rbrace$ by canonical forms (Lemma \ref{cf}).  $\mathtt{hd}$ $\mathtt{nil}^{T}\rightarrow\mathtt{wrong}^{T}$ \emph{``Empty list"}.  $\mathtt{tl}$ $\mathtt{nil}^{T}\rightarrow\mathtt{wrong}^{[T]}$ \emph{``Empty list"}.  $\mathtt{hd}$ $(\mathtt{cons}$ $v_{M}^{1}$ $v_{M}^{2})\rightarrow v_{M}^{1}$.  $\mathtt{tl}$ $(\mathtt{cons}$ $v_{M}^{1}$ $v_{M}^{2})\rightarrow v_{M}^{2}$.  $\mathtt{hd}$ $({^{[T]}M}H^{[T]}$ $(\mathtt{cons}$ $e_{H}^{1}$ $e_{H}^{2}))\rightarrow{^{T}M}H^{T}$ $e_{H}^{1}$.  $\mathtt{tl}$ $({^{[T]}M}H^{[T]}$ $(\mathtt{cons}$ $e_{H}^{1}$ $e_{H}^{2}))\rightarrow{^{[T]}M}H^{[T]}$ $e_{H}^{2}$.  If $e_{M}^{1}\rightarrow e_{M}^{2}$ then $f$ $e_{M}^{1}\rightarrow f$ $e_{M}^{2}$.  If $e_{M}^{1}\rightarrow$ \emph{\textbf{Error}: string} then $f$ $e_{M}^{1}\rightarrow$ \emph{\textbf{Error}: string}.

\end{case}

% ML null? e

\begin{case}

$e_{M}=\mathtt{null?}$ $e_{M}^{1}$

$e_{M}^{1}$ is an unforced value or $e_{M}^{1}\rightarrow e_{M}^{2}$ or $e_{M}^{1}\rightarrow$ \emph{\textbf{Error}: string} by the induction hypothesis.  If $e_{M}^{1}$ is an unforced value then $e_{M}^{1}:[T]$ by inversion (Lemma \ref{i}) and uniqueness of types (Lemma \ref{uot}) and $e_{M}^{1}\in\lbrace\mathtt{nil}^{T},\mathtt{cons}$ $v_{M}^{1}$ $v_{M}^{2},{^{[T]}M}H^{[T]}$ $(\mathtt{cons}$ $e_{H}^{1}$ $e_{H}^{2})\rbrace$ by canonical forms (Lemma \ref{cf}).  $\mathtt{null?}$ $\mathtt{nil}^{T}\rightarrow\overline{0}$.  If $e_{M}^{1}\in\lbrace\mathtt{cons}$ $v_{M}^{1}$ $v_{M}^{2},{^{[T]}M}H^{[T]}$ $(\mathtt{cons}$ $e_{H}^{1}$ $e_{H}^{2})\rbrace$ then $\mathtt{null?}$ $e_{M}^{1}\rightarrow\overline{1}$.  If $e_{M}^{1}\rightarrow e_{M}^{2}$ then $\mathtt{null?}$ $e_{M}^{1}\rightarrow\mathtt{null?}$ $e_{M}^{2}$.  If $e_{M}^{1}\rightarrow$ \emph{\textbf{Error}: string} then $\mathtt{null?}$ $e_{M}^{1}\rightarrow$ \emph{\textbf{Error}: string}.

\end{case}

\end{proof}

\end{theorem}


\subsection{Scheme Progress Theorem}

\begin{theorem}{Scheme Progress Theorem}

\label{sps}

If $\vdash_{S}e_{S}:TST$ then $e_{S}$ is an unforced value or $e_{S}\rightarrow e_{S}'$ or $e_{S}\rightarrow$ \emph{\textbf{Error}: string}.

\begin{proof}

By structural induction on $e_{S}$.

\begin{case}

$e_{S}=\lambda x.e_{S}^{1}$

$\lambda x.e_{S}^{1}$ is a value.

\end{case}

\begin{case}
$e_{A}=\overline{n}$ where $A\in\lbrace H,M,S\rbrace$

$\overline{n}$ is a value.
\end{case}

\begin{case}
$e_{S}=\mathtt{nil}$

$\mathtt{nil}$ is a value.
\end{case}

\begin{case}

$e_{S}=\mathtt{cons}$ $v_{S}^{1}$ $v_{S}^{2}$

$\mathtt{cons}$ $v_{S}^{1}$ $v_{S}^{2}$ is an unforced value.

\end{case}

\begin{case}

$e_{S}=SH^{T}$ $e_{H}^{1}$

$e_{H}^{1}$ is a value or $e_{H}^{1}\rightarrow e_{H}^{2}$ or $e_{H}^{1}\rightarrow$ \emph{\textbf{Error}: string} by Haskell progress (Theorem \ref{hps}).  If $e_{H}^{1}$ is a value then $T$ determines the reduction of $e_{S}$.

\begin{subcase}

$T=L$

$e_{H}^{1}={^{L}H}S$ $v_{S}$ by canonical forms (Lemma \ref{cf}).  $SH^{L}$ $(^{L}HS$ $v_{S})\rightarrow v_{S}$.

\end{subcase}

\begin{subcase}

$T=N$

$e_{H}^{1}=\overline{n}$ by canonical forms (Lemma \ref{cf}).  $SH^{N}$ $\overline{n}\rightarrow\overline{n}$.

\end{subcase}

\begin{subcase}

$T=[T_{1}]$

$e_{H}^{1}\in\lbrace\mathtt{nil}^{T_{1}},\mathtt{cons}$ $e_{H}^{3}$ $e_{H}^{4}\rbrace$ by canonical forms (Lemma \ref{cf}).  If $e_{H}^{1}=\mathtt{nil}^{T_{1}}$ then $SH^{T_{1}}$ $\mathtt{nil}^{T_{1}}\rightarrow\mathtt{nil}$.  If $e_{H}^{1}=\mathtt{cons}$ $e_{H}^{3}$ $e_{H}^{4}$ then $SH^{[T_{1}]}$ $(\mathtt{cons}$ $e_{H}^{3}$ $e_{H}^{4})$ is a forced value.

\end{subcase}

\begin{subcase}

$T=T_{1}^{a}$

$SH^{T_{1}^{a}}$ $e_{H}^{3}$ is a forced value.

\end{subcase}

\begin{subcase}

$T=T_{1}\rightarrow T_{2}$

$e_{H}^{1}=\lambda x_{1}:T_{1}[T_{i}/T_{i}^{a}].e_{H}^{3}$ by canonical forms (Lemma \ref{cf}).  $SH^{T_{1}\rightarrow T_{2}}$ $(\lambda x_{1}:T_{1}[T_{i}/T_{i}^{a}].e_{H}^{3})\rightarrow\lambda x_{2}.(SH^{T_{2}}$ $((\lambda x_{1}:T_{1}[T_{i}/T_{i}^{a}].e_{H}^{3})$ $(^{T_{1}}HS$ $x_{2})))$.

\end{subcase}

\begin{subcase}

$T=\forall X.T_{1}$

$e_{H}^{1}\in\lbrace\Lambda X.e_{H}^{3},{^{\forall X.T_{1}}H}S$ $v_{S}\rbrace$ by canonical forms (Lemma \ref{cf}).  If $e_{H}^{1}=\Lambda X.e_{H}^{3}$ then $SH^{\forall X.T_{1}}$ $(\Lambda X.e_{H}^{3})\rightarrow SH^{T_{1}[L/X]}$ $((\Lambda X.e_{H}^{3})$ $\lbrace L\rbrace)$.  If $e_{H}^{1}={^{\forall X.T_{1}}H}S$ $v_{S}$ then $SH^{\forall X.T_{1}}$ $(^{\forall X.T_{1}}HS$ $v_{S})\rightarrow v_{S}$.

\end{subcase}

If $e_{H}^{1}\rightarrow e_{H}^{2}$ then $SH^{T}$ $e_{H}^{1}\rightarrow SH^{T}$ $e_{H}^{2}$.  If $e_{H}^{1}\rightarrow$ \emph{\textbf{Error}: string} then $SH^{T}$ $e_{H}^{1}\rightarrow$ \emph{\textbf{Error}: string}.

\end{case}

\begin{case}
$e_{A}=x$ where $A\in\lbrace H,M,S\rbrace$

Cannot occur because $e_{A}$ is closed.
\end{case}

\begin{case}

$e_{S}=e_{S}^{1}$ $e_{S}^{2}$

$e_{S}^{1}$ is a value or $e_{S}^{1}\rightarrow e_{S}^{3}$ or $e_{S}^{1}\rightarrow$ \emph{\textbf{Error}: string} by the induction hypothesis.  If $e_{S}^{1}\rightarrow e_{S}^{3}$ then $e_{S}^{1}$ $e_{S}^{2}\rightarrow e_{S}^{3}$ $e_{S}^{2}$.  If $e_{S}^{1}\rightarrow$ \emph{\textbf{Error}: string} then $e_{S}^{1}$ $e_{S}^{2}\rightarrow$ \emph{\textbf{Error}: string}.  $e_{S}^{2}$ is a value or $e_{S}^{2}\rightarrow e_{S}^{4}$ or $e_{S}^{2}\rightarrow$ \emph{\textbf{Error}: string} by the induction hypothesis.  If $e_{S}^{2}\rightarrow e_{S}^{4}$ and $e_{S}^{1}$ is a value then $e_{S}^{1}$ $e_{S}^{2}\rightarrow e_{S}^{1}$ $e_{S}^{4}$.  If $e_{S}^{2}\rightarrow$ \emph{\textbf{Error}: string} and $e_{S}^{1}$ is a value then $e_{S}^{1}$ $e_{S}^{2}\rightarrow$ \emph{\textbf{Error}: string}.  If $e_{S}^{1}$ and $e_{S}^{2}$ are values then $(\lambda x.e_{S}^{5})$ $e_{S}^{2}\rightarrow e_{S}^{5}[e_{S}^{2}/x]$ if $e_{S}^{1}=\lambda x.e_{S}^{5}$ and $e_{S}^{1}$ $e_{S}^{2}\rightarrow\mathtt{wrong}$ \emph{``Not a function"} otherwise.

\end{case}

\begin{case}

$e_{S}=\mathtt{cons}$ $e_{S}^{1}$ $e_{S}^{2}$

$e_{S}^{1}$ is an unforced value or $e_{S}^{1}\rightarrow e_{S}^{3}$ or $e_{S}^{1}\rightarrow$ \emph{\textbf{Error}: string} by the induction hypothesis.  If $e_{S}^{1}\rightarrow e_{S}^{3}$ then $\mathtt{cons}$ $e_{S}^{1}$ $e_{S}^{2}\rightarrow\mathtt{cons}$ $e_{S}^{3}$ $e_{S}^{2}$.  If $e_{S}^{1}\rightarrow$ \emph{\textbf{Error}: string} then $\mathtt{cons}$ $e_{S}^{1}$ $e_{S}^{2}\rightarrow$ \emph{\textbf{Error}: string}.  $e_{S}^{2}$ is an unforced value or $e_{S}^{2}\rightarrow e_{S}^{4}$ or $e_{S}^{1}\rightarrow$ \emph{\textbf{Error}: string} by the induction hypothesis.  If $e_{S}^{2}\rightarrow e_{S}^{4}$ and $e_{M}^{1}$ is an unforced value then $\mathtt{cons}$ $e_{S}^{1}$ $e_{S}^{2}\rightarrow\mathtt{cons}$ $e_{S}^{1}$ $e_{S}^{4}$.  If $e_{S}^{2}\rightarrow$ \emph{\textbf{Error}: string} and $e_{S}^{1}$ is an unforced value then $\mathtt{cons}$ $e_{S}^{1}$ $e_{S}^{2}\rightarrow$ \emph{\textbf{Error}: string}.  If $e_{S}^{1}$ and $e_{S}^{2}$ are unforced values then $\mathtt{cons}$ $e_{S}^{1}$ $e_{S}^{2}$ is an unforced value.

\end{case}

\begin{case}
$e_{S}=f$ $e_{S}^{1}$

$e_{S}^{1}$ is a forced value or $e_{S}^{1}\rightarrow e_{S}^{2}$ or $e_{S}^{1}\rightarrow$ \emph{\textbf{Error}: string} by the induction hypothesis.  If $e_{S}^{1}\rightarrow e_{S}^{2}$ then $f$ $e_{S}^{1}\rightarrow f$ $e_{S}^{2}$.  If $e_{S}^{1}\rightarrow$ \emph{\textbf{Error}: string} then $f$ $e_{S}^{1}\rightarrow$ \emph{\textbf{Error}: string}.  $e_{S}^{1}$ is a forced value otherwise.  If $e_{S}^{1}=\mathtt{cons}$ $e_{S}^{3}$ $e_{S}^{4}$ then $f$ $(\mathtt{cons}$ $e_{S}^{3}$ $e_{S}^{4})\rightarrow e_{S}^{3}$ if $f=\mathtt{hd}$ and $f$ $(\mathtt{cons}$ $e_{S}^{3}$ $e_{S}^{4})\rightarrow e_{S}^{4}$ if $f=\mathtt{tl}$.  If $e_{S}^{1}=\mathtt{nil}$ then $f$ $\mathtt{nil}\rightarrow\mathtt{wrong}$ \emph{``Empty list"}.  If $e_{S}^{1}=SH^{[T]}$ $(\mathtt{cons}$ $e_{H}^{1}$ $e_{H}^{2})$ then $f$ $(SH^{[T]}$ $(\mathtt{cons}$ $e_{H}^{1}$ $e_{H}^{2}))\rightarrow SH^{T}$ $e_{H}^{1}$ if $f=\mathtt{hd}$ and $f$ $(SH^{[T]}$ $(\mathtt{cons}$ $e_{H}^{1}$ $e_{H}^{2}))\rightarrow SH^{[T]}$ $e_{H}^{2}$ if $f=\mathtt{tl}$.  $f$ $e_{S}^{1}\rightarrow\mathtt{wrong}$ \emph{``Not a list"} otherwise.
\end{case}

\begin{case}
$e_{S}=o\;e_{S}^{1}\;e_{S}^{2}$

$e_{S}^{1}$ is a value or $e_{S}^{1}\rightarrow e_{S}^{3}$ or $e_{S}^{1}\rightarrow$ \emph{\textbf{Error}:\;string} by the induction hypothesis.  If $e_{S}^{1}\rightarrow e_{S}^{3}$ then $o\;e_{S}^{1}\;e_{S}^{2}\rightarrow o\;e_{S}^{3}\;e_{S}^{2}$.  If $e_{S}^{1}\rightarrow$ \emph{\textbf{Error}:\;string} then $o\;e_{S}^{1}\;e_{S}^{2}\rightarrow$ \emph{\textbf{Error}:\;string}.  $e_{S}^{2}$ is a value or $e_{S}^{2}\rightarrow e_{S}^{4}$ or $e_{S}^{2}\rightarrow$ \emph{\textbf{Error}:\;string} by the induction hypothesis.  If $e_{S}^{2}\rightarrow e_{S}^{4}$ and $e_{S}^{1}$ is a value then $o\;e_{S}^{1}\;e_{S}^{2}\rightarrow o\;e_{S}^{1}\;e_{S}^{4}$.  If $e_{S}^{2}\rightarrow$ \emph{\textbf{Error}:\;string} and $e_{S}^{1}$ is a value then $o\;e_{S}^{1}\;e_{S}^{2}\rightarrow$ \emph{\textbf{Error}:\;string}.  If $e_{S}^{1}$ and $e_{S}^{2}$ are values and $e_{S}^{1}=\overline{n_{1}}$ and $e_{S}^{2}=\overline{n_{2}}$ then $+\;\overline{n_{1}}\;\overline{n_{2}}\rightarrow\overline{n_{1}+n_{2}}$ and $-\;\overline{n_{1}}\;\overline{n_{2}}\rightarrow\overline{max(n_{1}-n_{2},0)}$.  If $e_{S}^{1}$ and $e_{S}^{2}$ are values and $e_{S}^{1}\neq\overline{n_{1}}$ or $e_{S}^{2}\neq\overline{n_{2}}$ then $o\;e_{S}^{1}\;e_{S}^{2}\rightarrow\mathtt{wrong}\;\mathrm{``Not\;a\;number"}$ otherwise.
\end{case}

\begin{case}
$e_{S}=p$ $e_{S}^{1}$

$e_{S}^{1}$ is a forced value or $e_{S}^{1}\rightarrow e_{S}^{2}$ or $e_{S}^{1}\rightarrow$ \emph{\textbf{Error}: string} by the induction hypothesis.  If $e_{S}^{1}\rightarrow e_{S}^{2}$ then $p$ $e_{S}^{1}\rightarrow p$ $e_{S}^{2}$.  If $e_{S}^{1}\rightarrow$ \emph{\textbf{Error}: string} then $p$ $e_{S}^{1}\rightarrow$ \emph{\textbf{Error}: string}.  $e_{S}^{1}$ is a forced value otherwise.  If $p=\mathtt{fun?}$ then $\mathtt{fun?}$ $e_{S}^{1}\rightarrow\overline{0}$ if $e_{S}^{1}=\lambda x.e_{S}^{3}$ and $\mathtt{fun?}$ $e_{S}^{1}\rightarrow\overline{1}$ otherwise.  If $p=\mathtt{list?}$ then $\mathtt{list?}$ $e_{S}^{1}\rightarrow\overline{0}$ if $e_{S}^{1}\in\lbrace\mathtt{nil},\mathtt{cons}$ $v_{S}^{1}$ $v_{S}^{2},SH^{[T]}$ $(\mathtt{cons}$ $e_{H}^{1}$ $e_{H}^{2})\rbrace$ and $\mathtt{list?}$ $e_{S}^{1}\rightarrow\overline{1}$ otherwise.  If $p=\mathtt{null?}$ then $\mathtt{null?}$ $e_{S}^{1}\rightarrow\overline{0}$ if $e_{S}^{1}=\mathtt{nil}$ and $\mathtt{null?}$ $e_{S}^{1}\rightarrow\overline{1}$ if $e_{S}^{1}\in\lbrace\mathtt{cons}$ $v_{S}^{1}$ $v_{S}^{2},SH^{[T]}$ $(\mathtt{cons}$ $e_{H}^{1}$ $e_{H}^{2})\rbrace$ and $\mathtt{null?}$ $e_{S}^{1}\rightarrow\mathtt{wrong}$ \emph{``Not a list"} otherwise.  If $p=\mathtt{num?}$ then $\mathtt{num?}$ $e_{S}^{1}\rightarrow\overline{0}$ if $e_{S}^{1}=\overline{n}$ and $\mathtt{num?}$ $e_{S}^{1}\rightarrow\overline{1}$ otherwise.
\end{case}

\begin{case}

$e_{S}=\mathtt{if0}$ $e_{S}^{1}$ $e_{S}^{2}$ $e_{S}^{3}$

$e_{S}^{1}$ is an unforced value or $e_{S}^{1}\rightarrow e_{S}^{4}$ or $e_{S}^{1}\rightarrow$ \emph{\textbf{Error}: string} by the induction hypothesis.  If $e_{S}^{1}\rightarrow e_{S}^{4}$ then $\mathtt{if0}$ $e_{S}^{1}$ $e_{S}^{2}$ $e_{S}^{3}\rightarrow \mathtt{if0}$ $e_{S}^{4}$ $e_{S}^{2}$ $e_{S}^{3}$.  If $e_{S}^{1}\rightarrow$ \emph{\textbf{Error}: string} then $\mathtt{if0}$ $e_{S}^{1}$ $e_{S}^{2}$ $e_{S}^{3}\rightarrow$ \emph{\textbf{Error}: string}.  $e_{S}^{1}$ is an unforced value otherwise.  $\mathtt{if0}$ $\overline{0}$ $e_{S}^{2}$ $e_{S}^{3}\rightarrow e_{S}^{2}$.  $\mathtt{if0}$ $\overline{n}$ $e_{S}^{2}$ $e_{S}^{3}\rightarrow e_{S}^{3}$ $(n\neq 0)$.  $\mathtt{if0}$ $e_{S}^{1}$ $e_{S}^{2}$ $e_{S}^{3}\rightarrow\mathtt{wrong}$ \emph{``Not a number"} $(e_{S}^{1}\neq\overline{n})$.

\end{case}

\begin{case}
$e_{S}=\mathtt{wrong}$ $\mathrm{string}$

$\mathtt{wrong}$ $\mathrm{string}\rightarrow$ \emph{\textbf{Error}:\;string}.
\end{case}

\begin{case}

$e_{S}=SM^{T}$ $e_{M}^{1}$

$e_{M}^{1}$ is a value or $e_{M}^{1}\rightarrow e_{M}^{2}$ or $e_{M}^{1}\rightarrow$ \emph{\textbf{Error}: string} by ML progress (Theorem \ref{mps}).  If $e_{M}^{1}\rightarrow e_{M}^{2}$ then $SM^{T}$ $e_{M}^{1}\rightarrow SM^{T}$ $e_{M}^{2}$.  If $e_{M}^{1}\rightarrow$ \emph{\textbf{Error}: string} then $SM^{T}$ $e_{M}^{1}\rightarrow$ \emph{\textbf{Error}: string}.  If $e_{M}^{1}$ is a value then $T$ determines the reduction of $e_{S}$.

\begin{subcase}

$T=L$

$e_{M}^{1}={^{L}M}S$ $v_{S}$ by canonical forms (Lemma \ref{cf}).  $SM^{L}$ $(^{L}MS$ $v_{S})\rightarrow v_{S}$.

\end{subcase}

\begin{subcase}

$T=N$

$e_{M}^{1}=\overline{n}$ by canonical forms (Lemma \ref{cf}).  $SM^{N}$ $\overline{n}\rightarrow\overline{n}$.

\end{subcase}

\begin{subcase}

$T=[T_{1}]$

$e_{M}^{1}\in\lbrace\mathtt{nil}^{T_{1}},\mathtt{cons}$ $v_{M}^{1}$ $v_{M}^{2},{^{[T_{1}]}M}H^{[T_{1}]}$ $(\mathtt{cons}$ $e_{H}^{1}$ $e_{H}^{2})\rbrace$ by canonical forms (Lemma \ref{cf}).  If $e_{M}^{1}=\mathtt{nil}^{T_{1}}$ then $SM^{T_{1}}$ $\mathtt{nil}^{T_{1}}\rightarrow\mathtt{nil}$.  If $e_{M}^{1}=\mathtt{cons}$ $v_{M}^{1}$ $v_{M}^{2}$ then $SM^{[T_{1}]}$ $(\mathtt{cons}$ $v_{M}^{1}$ $v_{M}^{2})\rightarrow\mathtt{cons}$ $(SM^{T_{1}}$ $v_{M}^{1})$ $(SM^{[T_{1}]}$ $v_{M}^{2})$.  If $e_{M}^{1}={^{[T_{1}]}M}H^{[T_{1}]}$ $(\mathtt{cons}$ $e_{H}^{1}$ $e_{H}^{2})$ then $SM^{[T_{1}]}$ $({^{[T_{1}]}M}H^{[T_{1}]}$ $(\mathtt{cons}$ $e_{H}^{1}$ $e_{H}^{2}))\rightarrow SH^{[T_{1}]}$ $(\mathtt{cons}$ $e_{H}^{1}$ $e_{H}^{2})$.

\end{subcase}

\begin{subcase}

$T=T_{1}^{a}$

$SM^{T_{1}^{a}}$ $e_{M}^{1}$ is a value.

\end{subcase}

\begin{subcase}

$T=T_{1}\rightarrow T_{2}$

$e_{M}^{1}=\lambda x_{1}:T_{1}[T_{i}/T_{i}^{a}].e_{M}^{3}$ by canonical forms (Lemma \ref{cf}).  $SM^{T_{1}\rightarrow T_{2}}$ $(\lambda x_{1}:T_{1}[T_{i}/T_{i}^{a}].e_{M}^{3})\rightarrow\lambda x_{2}.(SM^{T_{2}}$ $((\lambda x_{1}:T_{1}[T_{i}/T_{i}^{a}].e_{M}^{3})$ $(^{T_{1}}MS$ $x_{2})))$.

\end{subcase}

\begin{subcase}

$T=\forall X.T_{1}$

$e_{M}^{1}\in\lbrace\Lambda X.e_{M}^{3},{^{\forall X.T_{1}}M}S$ $v_{S}\rbrace$ by canonical forms (Lemma \ref{cf}).  If $e_{M}^{1}=\Lambda X.e_{M}^{3}$ then $SM^{\forall X.T_{1}}$ $(\Lambda X.e_{M}^{3})\rightarrow SM^{T_{1}[L/X]}$ $((\Lambda X.e_{M}^{3})$ $\lbrace L\rbrace)$.  If $e_{M}^{1}={^{\forall X.T_{1}}M}S$ $v_{S}$ then $SM^{\forall X.T_{1}}$ $(^{\forall X.T_{1}}MS$ $v_{S})\rightarrow v_{S}$.

\end{subcase}

\end{case}

\end{proof}

\end{theorem}


\section{Preservation}

Preservation will be proven by cases on the rewrite rules.  In each case, the right side is be proven to be well-typed and have the same type as the left side.  Inversion (Lemma \ref{i}) and uniqueness of types (Lemma \ref{uot}) are used to determine the types of the left side and its subexpressions and the type of the right side.  Some rewrite rules use expression and type substitutions.

\subsection{Expression Substitution Lemma}

If $e_{A}^{1}$ is substituted for free occurrences of $x$ within $e_{A}^{2}$, $e_{A}^{1}$ and $x$ have the same type, and the result has the same type as $e_{A}^{2}$, where $A\in\lbrace H,M,S\rbrace$:

\begin{lemma}
\label{tms}
If $\Gamma,x_{1}:T_{1}\vdash_{A}e_{A}:T_{2}$ and $\Gamma\vdash_{A}x_{2}:T_{1}$ then $\Gamma\vdash_{A}e_{A}[x_{2}/x_{1}]:T_{2}$ where $A\in\lbrace H,M\rbrace$.  If $\Gamma,x_{1}:TST\vdash_{S}e_{S}:TST$ and $\Gamma\vdash_{S}x_{2}:TST$ then $\Gamma\vdash_{S}e_{S}[x_{2}/x_{1}]:TST$.
\begin{proof}
By structural induction.
\end{proof}
\end{lemma}

\subsection{Type Substitution Lemma}

If $T_{1}$ is substituted for free occurrences of $X$ within $e_{A}$ of type $T_{2}$, the type of the result is $T_{1}$ substituted for free occurrences of $X$ within $T_{2}$, where $A\in\lbrace H,M\rbrace$:

\begin{lemma}
\label{tes}
\onehalfspacing
If $\Gamma,X\vdash_{HM}e_{HM}:T_{1}$ and $\vdash_{HM}T_{2}$ then $\Gamma\vdash_{HM}e_{HM}[T_{2}/X]:T_{1}[T_{2}/X]$.
\begin{proof}
By structural induction.
\end{proof}
\end{lemma}

\subsection{Evaluation Context Lemma}

\begin{lemma}
\label{ec}
If $\Gamma\vdash_{A}e_{A}^{1}:T_{1}$, $\Gamma\vdash_{A}e_{A}^{2}:T_{1}$, and $\mathscr{E}[e_{A}^{1}]:T_{2}$ then $\mathscr{E}[e_{A}^{2}]:T_{2}$ where $A\in\lbrace H,M,S\rbrace$.
\begin{proof}
By structural induction.
\end{proof}
\end{lemma}

\subsection{Preservation Theorem}

\begin{theorem}
\label{pn}
If $\Gamma\vdash_{A}e_{A}^{1}:T$ and $e_{A}^{1}\rightarrow e_{A}^{2}$ then $\Gamma\vdash_{A}e_{A}^{2}:T$ where $A\in\lbrace H,M\rbrace$.  If $\Gamma\vdash_{S}e_{S}^{1}:TST$ and $e_{S}^{1}\rightarrow e_{S}^{2}$ then $\Gamma\vdash_{S}e_{S}^{2}:TST$.
\begin{proof}
By cases on the reductions $e_{A}^{1}\rightarrow e_{A}^{2}$ and $e_{S}^{1}\rightarrow e_{S}^{2}$.  Straightforward cases of Scheme preservation are elided.
\begin{case}
$(\lambda x:T_{1}.e_{HM}^{1})\;e_{HM}^{2}\rightarrow e_{HM}^{1}[e_{HM}^{2}/x]$

$\Gamma\vdash_{HM}(\lambda x:T_{1}.e_{HM}^{1})\;e_{HM}^{2}:T$ by the premise and uniqueness of types (Lemma \ref{uot}).  $\Gamma\vdash_{HM}\lambda x:T_{1}.e_{HM}^{1}:T_{1}\rightarrow T$, $\Gamma,x:T_{1}\vdash_{HM}e_{HM}^{1}:T$, $\Gamma\vdash_{HM}e_{HM}^{2}:T_{1}$, and $\Gamma,x:T_{1}\vdash_{HM}x:T_{1}$ by inversion (Lemma \ref{i}) and uniqueness of types.  $e_{HM}^{1}[e_{HM}^{2}/x]:T$ by term substitution (Lemma \ref{tms}).
\end{case}
\begin{case}
$\mathscr{E}[(\Lambda X.e_{H})\;\lbrace T_{1}\rbrace]_{H}\rightarrow\mathscr{E}[e_{H}[T_{1}/X]]$

$(\Lambda X.e_{H})\;\lbrace T_{1}\rbrace:T$ by the induction hypothesis and uniqueness of types (Lemma \ref{uot}).

%\textbf{!!! NOT DONE !!!}
%$\Gamma\vdash_{H}e_{H}:\forall X.T_{2}$, $\Gamma\vdash_{H}T_{1}$, and $T=T_{2}[T_{1}/X]$.
\end{case}
\begin{case}
$\mathtt{if0}\;\overline{0}\;e_{HM}^{1}\;e_{HM}^{2}\rightarrow e_{HM}^{1}$

$\Gamma\vdash_{HM}\mathtt{if0}\;\overline{0}\;e_{HM}^{1}\;e_{HM}^{2}:T$ by premise and uniqueness of types (Lemma \ref{uot}).  $\Gamma\vdash_{HM}e_{HM}^{1}:T$ by inversion (Lemma \ref{i}) and uniqueness of types.
\end{case}
\begin{case}
$\mathtt{if0}\;\overline{n}\;e_{A}^{1}\;e_{A}^{2}\rightarrow e_{A}^{2}\;(n\neq0)$ where $A\in\lbrace H,M\rbrace$

$\Gamma\vdash_{A}\mathtt{if0}\;\overline{n}\;e_{A}^{1}\;e_{A}^{2}:T$ by premise and uniqueness of types (Lemma \ref{uot}).  $T=T_{1}$ and $\Gamma\vdash_{A}e_{A}^{2}:T_{1}$ by inversion (Lemma \ref{i}) and uniqueness of types.  $\Gamma\vdash_{A}e_{A}^{2}:T$ because $T_{1}=T$.
\end{case}
\begin{case}
$+\;\overline{n_{1}}\;\overline{n_{2}}\rightarrow\overline{n_{1}+n_{2}}$ where $A\in\lbrace H,M\rbrace$

$\vdash_{A}+\;\overline{n_{1}}\;\overline{n_{2}}:N$ by inversion (Lemma \ref{i}) and uniqueness of types (Lemma \ref{uot}).  $\vdash_{A}\overline{n_{1}+n_{2}}:N$ by inversion and uniqueness of types.
\end{case}
\begin{case}
$-\;\overline{n_{1}}\;\overline{n_{2}}\rightarrow\overline{max(n_{1}-n_{2},0)}$

$\vdash_{HM}-\;\overline{n_{1}}\;\overline{n_{2}}:N$ by inversion (Lemma \ref{i}) and uniqueness of types (Lemma \ref{uot}).  $\vdash_{HM}\overline{max(n_{1}-n_{2},0)}:N$ by inversion and uniqueness of types.
\end{case}
\begin{case}
$\mathtt{hd}$ $(\mathtt{cons}$ $e_{H}^{1}$ $e_{H}^{2})\rightarrow e_{H}^{1}$

$\Gamma\vdash_{H}\mathtt{hd}$ $(\mathtt{cons}$ $e_{H}^{1}$ $e_{H}^{2}):T$ by premise and uniqueness of types (Lemma \ref{uot}).  $\Gamma\vdash_{H}e_{H}^{1}:T_{1}$, $\Gamma\vdash_{H}\mathtt{cons}$ $e_{H}^{1}$ $e_{H}^{2}:[T_{1}]$, and $T=T_{1}$ by inversion (Lemma \ref{i}) and uniqueness of types.  $\Gamma\vdash_{H}e_{H}^{1}:T$ because $T_{1}=T$.
\end{case}
\begin{case}
$\mathtt{tl}\;(\mathtt{cons}\;e_{H}^{1}\;e_{H}^{2})\rightarrow e_{H}^{2}$

$\Gamma\vdash_{H}\mathtt{tl}\;(\mathtt{cons}\;e_{H}^{1}\;e_{H}^{2}):T$ by premise and uniqueness of types (Lemma \ref{uot}).  $\Gamma\vdash_{H}e_{H}^{2}:[T_{1}]$, $\Gamma\vdash_{H}\mathtt{cons}\;e_{H}^{1}\;e_{H}^{2}:[T_{1}]$, and $T=[T_{1}]$ by inversion (Lemma \ref{i}) and uniqueness of types.  $\Gamma\vdash_{H}e_{H}^{2}:T$ because $[T_{1}]=T$.
\end{case}
\begin{case}
$\mathtt{hd}\;(\mathtt{cons}\;v_{M}^{1}\;v_{M}^{2})\rightarrow v_{M}^{1}$

$\Gamma\vdash_{M}\mathtt{hd}\;(\mathtt{cons}\;v_{M}^{1}\;v_{M}^{2}):T$ by premise and uniqueness of types (Lemma \ref{uot}).  $\Gamma\vdash_{M}v_{M}^{1}:T_{1}$, $\Gamma\vdash_{M}\mathtt{cons}\;v_{M}^{1}\;v_{M}^{2}:[T_{1}]$, and $T=T_{1}$ by inversion (Lemma \ref{i}) and uniqueness of types.  $\Gamma\vdash_{M}v_{M}^{1}:T$ because $T_{1}=T$.
\end{case}
\begin{case}
$\mathtt{tl}\;(\mathtt{cons}\;v_{M}^{1}\;v_{M}^{2})\rightarrow v_{M}^{2}$

$\Gamma\vdash_{M}\mathtt{tl}\;(\mathtt{cons}\;v_{M}^{1}\;v_{M}^{2}):T$ by premise and uniqueness of types (Lemma \ref{uot}).  $T=[T_{1}]$, $\Gamma\vdash_{M}\mathtt{cons}\;v_{M}^{1}\;v_{M}^{2}:[T_{1}]$, and $\Gamma\vdash_{M}v_{M}^{2}:[T_{1}]$ by inversion (Lemma \ref{i}) and uniqueness of types.  $\Gamma\vdash_{M}v_{M}^{2}:T$ because $[T_{1}]=T$.
\end{case}
\begin{case}
$\mathtt{hd}\;\mathtt{nil}^{T}\rightarrow\,^{T}B\;(\mathtt{wrong}\;\mathrm{``Empty\;list"})$ where $B\in\lbrace HS,MS\rbrace$

$\Gamma\vdash_{HM}\mathtt{hd}\;\mathtt{nil}^{T}:T$ by premise and uniqueness of types (Lemma \ref{uot}).  $\Gamma\vdash_{S}\mathtt{wrong}\;\mathrm{``Empty\;list"}:TST$ and $\Gamma\vdash_{HM}{^{T}B}\;(\mathtt{wrong}\;\mathrm{``Empty\;list"}):T$ by inversion (Lemma \ref{i}) and uniqueness of types.
\end{case}
\begin{case}
$\mathtt{tl}$ $\mathtt{nil}^{T_{1}}\rightarrow\mathtt{nil}^{T_{1}}$ where $A\in\lbrace H,M\rbrace$

$\Gamma\vdash_{A}\mathtt{tl}$ $\mathtt{nil}^{T_{1}}:T$ by premise and uniqueness of types (Lemma \ref{uot}).  $\Gamma\vdash_{A}\mathtt{nil}^{T_{1}}:[T_{1}]$ and $T=[T_{1}]$ by inversion (Lemma \ref{i}) and uniqueness of types.  $\Gamma\vdash_{A}\mathtt{nil}^{T_{1}}:T$ because $[T_{1}]=T$.
\end{case}
\begin{case}
$\mathtt{fix}\;(\lambda x:T_{1}.e_{A})\rightarrow e_{A}[(\mathtt{fix}\;(\lambda x:T_{1}.e_{A}))/x]$ where $A\in\lbrace H,M\rbrace$

$\Gamma\vdash_{A}\mathtt{fix}\;(\lambda x:T_{1}.e_{A}):T$ by premise and uniqueness of types (Lemma \ref{uot}).  $\Gamma\vdash_{A}\lambda x:T_{1}.e_{A}:T_{1}\rightarrow T_{1}$, $\Gamma,x:T_{1}\vdash_{A}e_{A}:T_{1}$, and $T=T_{1}$ by inversion (Lemma \ref{i}) and uniqueness of types.  $\Gamma\vdash_{A}e_{A}[(\mathtt{fix}\;(\lambda x:(T\rightarrow T).e_{A}))/x]:T_{1}$ by term substitution (Lemma \ref{tms}).  $\Gamma\vdash_{A}e_{A}[(\mathtt{fix}\;(\lambda x:(T\rightarrow T).e_{A}))/x]:T$ because $T_{1}=T$.
\end{case}
\begin{case}
$^{N}AB^{N}$ $\overline{n}\rightarrow\overline{n}$ where $(A,B)\in\lbrace(H,M),(M,H)\rbrace$

$\vdash_{A}{^{N}A}B^{N}$ $\overline{n}:T$ by premise and uniqueness of types (Lemma \ref{uot}).  $\vdash_{A}\overline{n}:N$ and $T=N$ by inversion (Lemma \ref{i}) and uniqueness of types.
\end{case}
\begin{case}
$^{N}AS$ $\overline{n}\rightarrow\overline{n}$ where $A\in\lbrace H,M\rbrace$

$\vdash_{A}{^{N}A}S$ $\overline{n}:T$ by premise and uniqueness of types (Lemma \ref{uot}).  $\vdash_{A}\overline{n}:N$ and $T=N$ by inversion (Lemma \ref{i}) and uniqueness of types.
\end{case}
\begin{case}
$^{N}AS\;v_{S}\rightarrow{^{N}A}S\;(\mathtt{wrong}\;\mathrm{``Not\;a\;number"})\;(v_{S}\neq\overline{n})$ where $A\in\lbrace H,M\rbrace$

$\Gamma\vdash_{A}{^{N}AS}\;v_{S}:T$ by premise and uniqueness of types (Lemma \ref{uot}).  $T=N$ by inversion (Lemma \ref{i}) and uniqueness of types.  $\vdash_{S}\mathtt{wrong}\;\mathrm{``Not\;a\;number"}:TST$ by inversion.  $\vdash_{A}{^{N}A}S\;(\mathtt{wrong}\;\mathrm{``Not\;a\;number"}):N$ by inversion and uniqueness of types.  $\vdash_{A}{^{N}A}S\;(\mathtt{wrong}\;\mathrm{``Not\;a\;number"}):T$ because $N=T$.
\end{case}
\begin{case}
$^{T_{1}\rightarrow T_{2}}AB^{T_{1}\rightarrow T_{2}}\;(\lambda x_{1}:T_{1}.e_{B})\rightarrow\lambda x_{2}:T_{1}.(^{T_{2}}AB^{T_{2}}\;((\lambda x_{1}:T_{1}.e_{B})\;(^{T_{1}}BA^{T_{1}}\;x_{2})))$ where $(A,B)\in\lbrace(H,M),(M,H)\rbrace$

$\Gamma\vdash_{A}{^{T_{1}\rightarrow T_{2}}}AB^{T_{1}\rightarrow T_{2}}\;(\lambda x_{1}:T_{1}.e_{B}):T$ by premise and uniqueness of types (Lemma \ref{uot}).  $\Gamma\vdash_{B}\lambda x_{1}:T_{1}.e_{B}:T_{1}\rightarrow T_{2}$, $T=T_{1}\rightarrow T_{2}$, $\Gamma,x_{2}:T_{1}\vdash_{A}x_{2}:T_{1}$, $\Gamma,x_{2}:T_{1}\vdash_{B}{^{T_{1}}B}A^{T_{1}}\;x_{2}:T_{1}$, $\Gamma,x_{2}:T_{1}\vdash_{B}(\lambda x_{1}:T_{1}.e_{B})\;(^{T_{1}}BA^{T_{1}}\;x_{2}):T_{2}$, $\Gamma,x_{2}:T_{1}\vdash_{A}{^{T_{2}}A}B^{T_{2}}\;((\lambda x_{1}:T_{1}.e_{B})\;(^{T_{1}}BA^{T_{1}}\;x_{2})):T_{2}$, and $\Gamma\vdash_{A}\lambda x_{2}:T_{1}.(^{T_{2}}AB^{T_{2}}\;((\lambda x_{1}:T_{1}.e_{B})\;(^{T_{1}}BA^{T_{1}}\;x_{2}))):T_{1}\rightarrow T_{2}$ by inversion (Lemma \ref{i}) and uniqueness of types.  $\Gamma\vdash_{A}\lambda x_{2}:T_{1}.(^{T_{2}}AB^{T_{2}}\;((\lambda x_{1}:T_{1}.e_{B})\;(^{T_{1}}BA^{T_{1}}\;x_{2}))):T$ because $T_{1}\rightarrow T_{2}=T$.
\end{case}
\begin{case}
$^{T_{1}\rightarrow T_{2}}AS$ $(\lambda x_{1}.e_{S})\rightarrow\lambda x_{2}:T_{1}[T_{i}/T_{i}^{a}].(^{T_{2}}AS$ $((\lambda x_{1}.e_{S})$ $(SA^{T_{1}}$ $x_{2})))$ where $A\in\lbrace H,M\rbrace$

$\Gamma\vdash_{A}{^{T_{1}\rightarrow T_{2}}A}S$ $(\lambda x_{1}.e_{S}):T$ by premise and uniqueness of types (Lemma \ref{uot}).  $\Gamma\vdash_{S}\lambda x_{1}.e_{S}:TST$ by inversion (Lemma \ref{i}).  $T=(T_{1}\rightarrow T_{2})[T_{i}/T_{i}^{a}]$ by inversion and uniqueness of types.  $\Gamma,x_{2}:T_{1}[T_{i}/T_{i}^{a}]\vdash_{A}x_{2}:T_{1}[T_{i}/T_{i}^{a}]$ by inversion and uniqueness of types.  $\Gamma,x_{2}:T_{1}[T_{i}/T_{i}^{a}]\vdash_{S}SA^{T_{1}}$ $x_{2}:TST$ and $\Gamma,x_{2}:T_{1}[T_{i}/T_{i}^{a}]\vdash_{S}(\lambda x_{1}.e_{S})$ $(SA^{T_{1}}$ $x_{2}):TST$ by inversion.  $\Gamma,x_{2}:T_{1}[T_{i}/T_{i}^{a}]\vdash_{A}{^{T_{2}}A}S$ $((\lambda x_{1}.e_{S})$ $(SA^{T_{1}}$ $x_{2})):T_{2}[T_{i}/T_{i}^{a}]$ and $\Gamma\vdash_{A}\lambda x_{2}:T_{1}[T_{i}/T_{i}^{a}].(^{T_{2}}AS$ $((\lambda x_{1}.e_{S})$ $(SA^{T_{1}}$ $x_{2}))):T_{1}[T_{i}/T_{i}^{a}]\rightarrow T_{2}[T_{i}/T_{i}^{a}]$ by inversion and uniqueness of types.  $\Gamma\vdash_{A}\lambda x_{2}:T_{1}[T_{i}/T_{i}^{a}].(^{T_{2}}AS$ $((\lambda x_{1}.e_{S})$ $(SA^{T_{1}}$ $x_{2}))):T$ because $T_{1}[T_{i}/T_{i}^{a}]\rightarrow T_{2}[T_{i}/T_{i}^{a}]=(T_{1}\rightarrow T_{2})[T_{i}/T_{i}^{a}]=T$.
\end{case}
\begin{case}
$^{T_{1}\rightarrow T_{2}}AS$ $v_{S}\rightarrow{^{T_{1}\rightarrow T_{2}}A}S$ $(\mathtt{wrong}$ $\mathrm{``Not}$ $\mathrm{a}$ $\mathrm{function"})$ $(v_{S}\neq\lambda x.e_{S})$ where $A\in\lbrace H,M\rbrace$

$\Gamma\vdash_{A}{^{T_{1}\rightarrow T_{2}}A}S$ $v_{S}:T$ by premise and uniqueness of types (Lemma \ref{uot}).  $T=(T_{1}\rightarrow T_{2})[T_{i}/T_{i}^{a}]$ by inversion (Lemma \ref{i}) and uniqueness of types.  $\vdash_{S}\mathtt{wrong}$ $\mathrm{``Not}$ $\mathrm{a}$ $\mathrm{function"}:TST$ by inversion.  $\Gamma\vdash_{A}{^{T_{1}\rightarrow T_{2}}A}S$ $(\mathtt{wrong}$ $\mathrm{``Not}$ $\mathrm{a}$ $\mathrm{function"}):(T_{1}\rightarrow T_{2})[T_{i}/T_{i}^{a}]$ by inversion and uniqueness of types.  $\Gamma\vdash_{A}{^{T_{1}\rightarrow T_{2}}A}S$ $(\mathtt{wrong}$ $\mathrm{``Not}$ $\mathrm{a}$ $\mathrm{function"}):T$ because $(T_{1}\rightarrow T_{2})[T_{i}/T_{i}^{a}]=T$.
\end{case}
\begin{case}
$SA^{T_{1}\rightarrow T_{2}}\;(\lambda x_{1}:T_{1}[T_{i}/T_{i}^{a}].e_{A})\rightarrow\lambda x_{2}.(SA^{T_{2}}\;((\lambda x_{1}:T_{1}[T_{i}/T_{i}^{a}].e_{A})\;(^{T_{1}}AS\;x_{2})))$ where $A\in\lbrace H,M\rbrace$

$\Gamma\vdash_{S}SA^{T_{1}\rightarrow T_{2}}\;(\lambda x_{1}:T_{1}[T_{i}/T_{i}^{a}].e_{A}):TST$ by premise.  $\Gamma\vdash_{A}\lambda x_{1}:T_{1}[T_{i}/T_{i}^{a}].e_{A}:T_{1}[T_{i}/T_{i}^{a}]\rightarrow T_{2}[T_{i}/T_{i}^{a}]$ by inversion (Lemma \ref{i}) and uniqueness of types (Lemma \ref{uot}).  $\Gamma,x_{2}:TST\vdash_{S}x_{2}:TST$ by inversion.  $\Gamma,x_{2}:TST\vdash_{A}{^{T_{1}}A}S\;x_{2}:T_{1}[T_{i}/T_{i}^{a}]$ and $\Gamma,x_{2}:TST\vdash_{A}(\lambda x_{1}:T_{1}[T_{i}/T_{i}^{a}].e_{A})\;(^{T_{1}}AS\;x_{2}):T_{2}[T_{i}/T_{i}^{a}]$ by inversion and uniqueness of types.  $\Gamma,x_{2}:TST\vdash_{S}SA^{T_{2}}\;((\lambda x_{1}:T_{1}[T_{i}/T_{i}^{a}].e_{A})\;(^{T_{1}}AS\;x_{2})):TST$ and $\Gamma\vdash_{S}\lambda x_{2}.(SA^{T_{2}}\;((\lambda x_{1}:T_{1}[T_{i}/T_{i}^{a}].e_{A})\;(^{T_{1}}AS\;x_{2}))):TST$ by inversion.
\end{case}
\begin{case}
$^{\forall X.T_{1}}AB^{\forall X_{1}.T_{1}}$ $(\Lambda X.e_{B})\rightarrow\Lambda X.(^{T_{1}}AB^{T_{1}}$ $e_{B})$ where $(A,B)\in\lbrace(H,M),(M,H)\rbrace$

$\Gamma\vdash_{A}{^{\forall X.T_{1}}A}B^{\forall X.T_{1}}$ $(\Lambda X_{1}.e_{B}):T$ by premise and uniqueness of types (Lemma \ref{uot}).  $\Gamma,X\vdash_{B}e_{B}:T_{1}$, $\Gamma\vdash_{B}\Lambda X.e_{B}:\forall X.T_{1}$, $T=\forall X.T_{1}$, $\Gamma,X\vdash_{A}{^{T_{1}}A}B^{T_{1}}$ $e_{B}:T_{1}$, and $\Gamma\vdash_{A}\Lambda X.(^{T_{1}}AB^{T_{1}}$ $e_{B}):\forall X.T_{1}$ by inversion (Lemma \ref{i}) and uniqueness of types.  $\Gamma\vdash_{A}\Lambda X.(^{T_{1}}AB^{T_{1}}$ $e_{B}):T$ because $\forall X.T_{1}=T$.
\end{case}
\begin{case}
$^{\forall X.T_{1}}AB^{\forall X.T_{1}}$ $(^{\forall X.T_{1}}BS$ $v_{S})\rightarrow{^{\forall X.T_{1}}A}S$ $v_{S}$ where $(A,B)\in\lbrace(H,M),$ $(M,H)\rbrace$

$\Gamma\vdash_{A}{^{\forall X.T_{1}}A}B^{\forall X.T_{1}}$ $(^{\forall X.T_{1}}BS$ $v_{S}):T$ by premise and uniqueness of types (Lemma \ref{uot}).  $\Gamma\vdash_{S}v_{S}:TST$ by inversion (Lemma \ref{i}).  $\Gamma\vdash_{B}{^{\forall X.T_{1}}B}S$ $v_{S}:\forall X.T_{1}$, $T=\forall X.T_{1}$, and $\Gamma\vdash_{A}{^{\forall X.T_{1}}A}S$ $v_{S}:\forall X.T_{1}$ by inversion and uniqueness of types.  $\Gamma\vdash_{A}{^{\forall X.T_{1}}A}S$ $v_{S}:T$ because $\forall X.T_{1}=T$.
\end{case}
\begin{case}
$(^{\forall X.T_{1}}AS\;v_{S})\;\lbrace T_{2}\rbrace\rightarrow{^{T_{1}[T_{2}^{a}/X]}A}S\;v_{S}$ where $A\in\lbrace H,M\rbrace$

$\Gamma\vdash_{A}(^{\forall X.T_{1}}AS\;v_{S})\;\lbrace T_{2}\rbrace:T$ by premise and uniqueness of types (Lemma \ref{uot}).  $T=T_{1}[T_{2}/X]$ by inversion (Lemma \ref{i}) and uniqueness of types.  $\Gamma\vdash_{S}v_{S}:TST$ by inversion.  $\Gamma\vdash_{A}{^{T_{1}[T_{2}^{a}/X]}A}S\;v_{S}:T_{1}[T_{2}^{a}/X][T_{i}/T_{i}^{a}]$ by inversion and uniqueness of types.  $\Gamma\vdash_{A}{^{T_{1}[T_{2}^{a}/X]}A}S\;v_{S}:T$ because $T_{1}[T_{2}^{a}/X][T_{i}/T_{i}^{a}]=T_{1}[T_{2}/X]=T$.
\end{case}
\begin{case}
$SA^{\forall X.T_{1}}$ $(\Lambda X.e_{A})\rightarrow SA^{T_{1}[L/X]}$ $((\Lambda X.e_{A})$ $\lbrace L\rbrace)$ where $A\in\lbrace H,M\rbrace$

$\Gamma\vdash_{S}SA^{\forall X.T_{1}}$ $(\Lambda X.e_{A}):TST$ by premise.  $\Gamma\vdash_{A}\Lambda X.e_{A}:\forall X.T_{1}$ and $\Gamma\vdash_{A}(\Lambda X.e_{A})$ $\lbrace L\rbrace:T_{1}[L/X]$ by inversion (Lemma \ref{i}) and uniqueness of types (Lemma \ref{uot}).  $\Gamma\vdash_{S}SA^{T_{1}[L/X]}$ $((\Lambda X.e_{A})$ $\lbrace L\rbrace):TST$ by inversion.
\end{case}
\begin{case}
$^{[T_{1}]}HM^{[T_{1}]}\;(\mathtt{cons}\;v_{M}^{1}\;v_{M}^{2})\rightarrow\mathtt{cons}\;(^{T_{1}}HM^{T_{1}}\;v_{M}^{1})\;(^{[T_{1}]}HM^{[T_{1}]}\;v_{M}^{2})$

$^{[T_{1}]}HM^{[T_{1}]}\;(\mathtt{cons}\;v_{M}^{1}\;v_{M}^{2}):T$ by premise and uniqueness of types (Lemma \ref{uot}).  $T=[T_{1}]$, $\Gamma\vdash_{M}v_{M}^{1}:T_{1}$, $\Gamma\vdash_{M}v_{M}^{2}:[T_{1}]$, $\Gamma\vdash_{H}{^{T_{1}}H}M^{T_{1}}\;v_{M}^{1}:T_{1}$, $\Gamma\vdash_{H}{^{[T_{1}]}H}M^{[T_{1}]}\;v_{M}^{2}:[T_{1}]$, and $\Gamma\vdash_{H}\mathtt{cons}\;(^{T_{1}}HM^{T_{1}}\;v_{M}^{1})\;(^{[T_{1}]}HM^{[T_{1}]}\;v_{M}^{2}):[T_{1}]$ by inversion (Lemma \ref{i}) and uniqueness of types.  $\Gamma\vdash_{H}\mathtt{cons}\;(^{T_{1}}HM^{T_{1}}\;v_{M}^{1})\;(^{[T_{1}]}HM^{[T_{1}]}\;v_{M}^{2}):T$ because $[T_{1}]=T$.
\end{case}
\begin{case}
$^{[T_{1}]}HM^{[T_{1}]}\;(^{[T_{1}]}MH^{[T_{1}]}\;(\mathtt{cons}\;e_{H}^{1}\;e_{H}^{2}))\rightarrow\mathtt{cons}\;e_{H}^{1}\;e_{H}^{2}$

$^{[T_{1}]}HM^{[T_{1}]}\;(^{[T_{1}]}MH^{[T_{1}]}\;(\mathtt{cons}\;e_{H}^{1}\;e_{H}^{2})):T$ by premise and uniqueness of types (Lemma \ref{uot}).  $T=[T_{1}]$, $\Gamma\vdash_{H}{^{[T_{1}]}M}H^{[T_{1}]}\;(\mathtt{cons}\;e_{H}^{1}\;e_{H}^{2}):[T_{1}]$, and $\Gamma\vdash_{H}\mathtt{cons}\;e_{H}^{1}\;e_{H}^{2}:[T_{1}]$ by inversion (Lemma \ref{i}) and uniqueness of types.  $\Gamma\vdash_{H}\mathtt{cons}\;e_{H}^{1}\;e_{H}^{2}:T$ because $[T_{1}]=T$.
\end{case}
\begin{case}
$\mathtt{hd}\;(^{[T_{1}]}MH^{[T_{1}]}\;(\mathtt{cons}\;e_{H}^{1}\;e_{H}^{2}))\rightarrow{^{T_{1}}M}H^{T_{1}}\;e_{H}^{1}$

$\Gamma\vdash_{M}\mathtt{hd}\;(^{[T_{1}]}MH^{[T_{1}]}\;(\mathtt{cons}\;e_{H}^{1}\;e_{H}^{2})):T$ by premise and uniqueness of types (Lemma \ref{uot}).  $T=T_{1}$, $\Gamma\vdash_{H}e_{H}^{1}:T_{1}$, and $^{T_{1}}MH^{T_{1}}\;e_{H}^{1}:T_{1}$ by inversion (Lemma \ref{i}) and uniqueness of types (Lemma \ref{uot}).  $^{T_{1}}MH^{T_{1}}\;e_{H}^{1}:T$ because $T_{1}=T$.
\end{case}
\begin{case}
$\mathtt{hd}\;(^{[T_{1}]}MH^{[T_{1}]}\;(\mathtt{cons}\;e_{H}^{1}\;e_{H}^{2}))\rightarrow{^{[T_{1}]}M}H^{[T_{1}]}\;e_{H}^{2}$

$\Gamma\vdash_{M}\mathtt{hd}\;(^{[T_{1}]}MH^{[T_{1}]}\;(\mathtt{cons}\;e_{H}^{1}\;e_{H}^{2})):T$ by premise and uniqueness of types (Lemma \ref{uot}).  $T=T_{1}$, $\Gamma\vdash_{H}e_{H}^{1}:T_{1}$, and $^{[T_{1}]}MH^{[T_{1}]}\;e_{H}^{1}:[T_{1}]$ by inversion (Lemma \ref{i}) and uniqueness of types (Lemma \ref{uot}).  $^{[T_{1}]}MH^{[T_{1}]}\;e_{H}^{2}:T$ because $[T_{1}]=T$.
\end{case}
\begin{case}
$^{[T_{1}]}AS\;(\mathtt{cons}\;v_{S}^{1}\;v_{S}^{2})\rightarrow\mathtt{cons}\;(^{T_{1}}AS\;v_{S}^{1})\;(^{[T_{1}]}AS\;v_{S}^{2})$ where $A\in\lbrace H,M\rbrace$

$\Gamma\vdash_{A}{^{[T_{1}]}A}S\;(\mathtt{cons}\;v_{S}^{1}\;v_{S}^{2}):T$ by premise and uniqueness of types (Lemma \ref{uot}).  $T=[T_{1}]$ by inversion (Lemma \ref{i}) and uniqueness of types.  $\Gamma\vdash_{S}v_{S}^{1}:TST$, and $\Gamma\vdash_{S}v_{S}^{2}:TST$ by inversion.  $\Gamma\vdash_{A}{^{T_{1}}A}S\;v_{S}^{1}:T_{1}$, $\Gamma\vdash_{A}{^{[T_{1}]}A}S\;v_{S}^{2}:[T_{1}]$, and $\Gamma\vdash_{A}\mathtt{cons}\;(^{T_{1}}AS\;v_{S}^{1})\;(^{[T_{1}]}AS\;v_{S}^{2}):[T_{1}]$ by inversion and uniqueness of types.  $\Gamma\vdash_{A}\mathtt{cons}\;(^{T_{1}}AS\;v_{S}^{1})\;(^{[T_{1}]}AS\;v_{S}^{2}):T$ because $[T_{1}]=T$.
\end{case}
\begin{case}
$^{[T_{1}]}HS$ $(SH^{[T_{1}]}$ $(\mathtt{cons}$ $e_{H}^{1}$ $e_{H}^{2}))\rightarrow\mathtt{cons}$ $e_{H}^{1}$ $e_{H}^{2}$

$^{[T_{1}]}HS$ $(SH^{[T_{1}]}$ $(\mathtt{cons}$ $e_{H}^{1}$ $e_{H}^{2})):T$ by premise and uniqueness of types (Lemma \ref{uot}).  $\Gamma\vdash_{H}\mathtt{cons}$ $e_{H}^{1}$ $e_{H}^{2}:[T_{1}]$ by inversion (Lemma \ref{i}) and uniqueness of types.  $\Gamma\vdash_{S}SH^{[T_{1}]}$ $(\mathtt{cons}$ $e_{H}^{1}$ $e_{H}^{2}):TST$ by inversion.  $T=[T_{1}]$ by inversion and uniqueness of types.  $\Gamma\vdash_{H}\mathtt{cons}$ $e_{H}^{1}$ $e_{H}^{2}:T$ because $[T_{1}]=T$.
\end{case}
\begin{case}
$^{[T_{1}]}MS$ $(SH^{[T_{1}]}$ $(\mathtt{cons}$ $e_{H}^{1}$ $e_{H}^{2}))\rightarrow{^{[T_{1}]}M}H^{[T_{1}]}$ $(\mathtt{cons}$ $e_{H}^{1}$ $e_{H}^{2})$

$^{[T_{1}]}MS$ $(SH^{[T_{1}]}$ $(\mathtt{cons}$ $e_{H}^{1}$ $e_{H}^{2})):T$ by premise and uniqueness of types (Lemma \ref{i}).  $\Gamma\vdash_{H}\mathtt{cons}$ $e_{H}^{1}$ $e_{H}^{2}:[T_{1}]$ by inversion (Lemma \ref{i}) and uniqueness of types.  $\Gamma\vdash_{S}SH^{[T_{1}]}$ $(\mathtt{cons}$ $e_{H}^{1}$ $e_{H}^{2}):TST$ by inversion.  $T=[T_{1}]$ and $\Gamma\vdash_{M}{^{[T_{1}]}M}H^{[T_{1}]}$ $(\mathtt{cons}$ $e_{H}^{1}$ $e_{H}^{2}):[T_{1}]$ by inversion and uniqueness of types.  $\Gamma\vdash_{M}{^{[T_{1}]}M}H^{[T_{1}]}$ $(\mathtt{cons}$ $e_{H}^{1}$ $e_{H}^{2}):T$ because $[T_{1}]=T$.
\end{case}
\begin{case}
$\mathtt{hd}\;(SH^{[T_{1}]}\;(\mathtt{cons}\;e_{H}^{1}\;e_{H}^{2}))\rightarrow SH^{T_{1}}\;e_{H}^{1}$

$\Gamma\vdash_{S}\mathtt{hd}\;(SH^{[T_{1}]}\;(\mathtt{cons}\;e_{H}^{1}\;e_{H}^{2})):TST$ by premise.  $\Gamma\vdash_{H}e_{H}^{1}:T_{1}$ by inversion (Lemma \ref{i}) and uniqueness of types (Lemma \ref{uot}).  $\Gamma\vdash_{S}SH^{T_{1}}\;e_{H}^{1}:TST$ by inversion.
\end{case}
\begin{case}
$\mathtt{tl}\;(SH^{[T_{1}]}\;(\mathtt{cons}\;e_{H}^{1}\;e_{H}^{2}))\rightarrow SH^{[T_{1}]}\;e_{H}^{2}$

$\Gamma\vdash_{S}\mathtt{tl}\;(SH^{[T_{1}]}\;(\mathtt{cons}\;e_{H}^{1}\;e_{H}^{2})):TST$ by premise.  $\Gamma\vdash_{H}e_{H}^{2}:[T_{1}]$ by inversion (Lemma \ref{i}) and uniqueness of types (Lemma \ref{uot}).  $\Gamma\vdash_{S}SH^{[T_{1}]}\;e_{H}^{2}:TST$ by inversion.
\end{case}
\begin{case}
$SM^{[T_{1}]}\;(\mathtt{cons}\;v_{M}^{1}\;v_{M}^{2})\rightarrow\mathtt{cons}\;(SM^{T_{1}}\;v_{M}^{1})\;(SM^{[T_{1}]}\;v_{M}^{2})$

$\Gamma\vdash_{S}SM^{[T_{1}]}\;(\mathtt{cons}\;v_{M}^{1}\;v_{M}^{2}):TST$ by premise.  $\Gamma\vdash_{M}v_{M}^{1}:T_{1}$ and $\Gamma\vdash_{M}v_{M}^{2}:[T_{1}]$ by inversion (Lemma \ref{i}) and uniqueness of types (Lemma \ref{uot}).  $\Gamma\vdash_{S}SM^{T_{1}}\;v_{M}^{1}:TST$, $\Gamma\vdash_{S}SM^{[T_{1}]}\;v_{M}^{2}:TST$, and $\Gamma\vdash_{S}\mathtt{cons}\;(SM^{T_{1}}\;v_{M}^{1})\;(SM^{[T_{1}]}\;v_{M}^{2}):TST$ by inversion.
\end{case}
\begin{case}
$SM^{[T_{1}]}\;(^{[T_{1}]}MH^{[T_{1}]}\;(\mathtt{cons}\;e_{H}^{1}\;e_{H}^{2}))\rightarrow SH^{[T_{1}]}\;(\mathtt{cons}\;e_{H}^{1}\;e_{H}^{2})$

$SM^{[T_{1}]}\;(^{[T_{1}]}MH^{[T_{1}]}\;(\mathtt{cons}\;e_{H}^{1}\;e_{H}^{2})):TST$ by premise.  $\Gamma\vdash_{H}\mathtt{cons}\;e_{H}^{1}\;e_{H}^{2}:[T_{1}]$ and $\Gamma\vdash_{M}{^{[T_{1}]}M}H^{[T_{1}]}\;(\mathtt{cons}\;e_{H}^{1}\;e_{H}^{2}):[T_{1}]$ by inversion (Lemma \ref{i}) and uniqueness of types (Lemma \ref{uot}).  $\Gamma\vdash_{S}SH^{[T_{1}]}\;(\mathtt{cons}\;e_{H}^{1}\;e_{H}^{2}):TST$ by inversion.
\end{case}
\begin{case}
$^{[T_{1}]}AB^{[T_{1}]}\;\mathtt{nil}^{T_{1}}\rightarrow\mathtt{nil}^{T_{1}}$ where $(A,B)\in\lbrace(H,M),(M,H)\rbrace$

$\Gamma\vdash_{A}{^{[T_{1}]}A}B^{[T_{1}]}\;\mathtt{nil}^{T_{1}}:T$ by premise and uniqueness of types (Lemma \ref{uot}).  $\Gamma\vdash_{A}\mathtt{nil}^{T_{1}}:[T_{1}]$ and $T=[T_{1}]$ by inversion (Lemma \ref{i}) and uniqueness of types.  $\Gamma\vdash_{A}\mathtt{nil}^{T_{1}}:T$ because $[T_{1}]=T$.
\end{case}
\begin{case}
$^{[T_{1}]}AS$ $\mathtt{nil}\rightarrow\mathtt{nil}^{T_{1}}$ where $A\in\lbrace H,M\rbrace$

$\Gamma\vdash_{A}{^{[T_{1}]}A}S$ $\mathtt{nil}:T$ by premise and uniqueness of types (Lemma \ref{uot}).  $T=[T_{1}]$ and $\Gamma\vdash_{A}\mathtt{nil}^{T_{1}}:[T_{1}]$ by inversion (Lemma \ref{i}) and uniqueness of types.  $\Gamma\vdash_{A}\mathtt{nil}^{T_{1}}:T$ because $[T_{1}]=T$.
\end{case}
\begin{case}
$^{[T_{1}]}AS\;v_{S}^{1}\rightarrow{^{[T_{1}]}A}S\;(\mathtt{wrong}\;\mathrm{``Not\;a\;list"})$ $(v_{S}^{1}\neq\mathtt{cons}\;v_{S}^{2}\;v_{S}^{3}$ and $v_{S}^{1}\neq\mathtt{nil})$ where $A\in\lbrace H,M\rbrace$

$\Gamma\vdash_{A}{^{[T_{1}]}AS}\;v_{S}^{1}:T$ by premise and uniqueness of types (Lemma \ref{uot}).  $T=[T_{1}]$ by inversion (Lemma \ref{i}) and uniqueness of types.  $\vdash_{S}\mathtt{wrong}\;\mathrm{``Not\;a\;list"}:TST$ by inversion.  $\Gamma\vdash_{A}{^{[T_{1}]}A}S\;(\mathtt{wrong}\;\mathrm{``Not\;a\;list"}):[T_{1}]$ by inversion and uniqueness of types.  $\Gamma\vdash_{A}{^{[T_{1}]}A}S\;(\mathtt{wrong}\;\mathrm{``Not\;a\;list"}):T$ because $[T_{1}]=T$.
\end{case}
\begin{case}
$^{L}AB^{L}\;(^{L}BS\;v_{S})\rightarrow{^{L}A}S\;v_{S}$ where $(A,B)\in\lbrace(H,M),(M,H)\rbrace$

$\Gamma\vdash_{A}{^{L}A}B^{L}\;(^{L}BS\;v_{S}):T$ by premise and uniqueness of types (Lemma \ref{uot}).  $T=L$ by inversion (Lemma \ref{i}) and uniqueness of types.  $\Gamma\vdash_{S}v_{S}:TST$ by inversion.  $\Gamma\vdash_{A}{^{L}A}S\;v_{S}:L$ by inversion and uniqueness of types.  $\Gamma\vdash_{A}{^{L}A}S\;v_{S}:T$ because $L=T$.
\end{case}
\begin{case}
$^{T_{1}^{a}}AS$ $(SA^{T_{1}^{a}}$ $B_{A})\rightarrow B_{A}$ where $(A,B)\in\lbrace(H,e),(M,v)\rbrace$

$^{T_{1}^{a}}AS$ $(SA^{T_{1}^{a}}$ $B_{A}):T$ by premise and uniqueness of types (Lemma \ref{uot}).  $\Gamma\vdash_{A}B_{A}:T_{1}^{a}[T_{i}/T_{i}^{a}]$ by inversion (Lemma \ref{i}) and uniqueness of types.  $\Gamma\vdash_{S}SA^{T_{1}^{a}}$ $B_{A}:TST$ by inversion.  $T=T_{1}^{a}[T_{i}/T_{i}^{a}]$ by inversion and uniqueness of types.  $\Gamma\vdash_{A}B_{A}:T$ because $T_{1}^{a}[T_{i}/T_{i}^{a}]=T$.
\end{case}
\begin{case}
$^{T_{1}^{a}}AS\;v_{S}\rightarrow{^{T_{1}^{a}}A}S\;(\mathtt{wrong}\;\mathrm{``Parametricity\;violated"})$ $(v_{S}\neq SA^{T_{1}^{a}}\;B_{A})$ where $(A,B)\in\lbrace(H,e),(M,v)\rbrace$

$\Gamma\vdash_{A}{^{T_{1}^{a}}A}S\;v_{S}:T$ by premise and uniqueness of types (Lemma \ref{uot}).  $T=T_{1}^{a}[T_{i}/T_{i}^{a}]$ by inversion (Lemma \ref{i}) and uniqueness of types.  $\vdash_{S}\mathtt{wrong}\;\mathrm{``Parametricity\;violated"}:TST$ by inversion.  $^{T_{1}^{a}}AS\;(\mathtt{wrong}\;\mathrm{``Parametricity\;violated"}):T_{1}^{a}[T_{i}/T_{i}^{a}]$ by inversion and uniqueness of types.  $^{T_{1}^{a}}AS\;(\mathtt{wrong}\;\mathrm{``Parametricity\;violated"}):T$ because $T_{1}^{a}[T_{i}/T_{i}^{a}]=T$.
\end{case}
\end{proof}
\end{theorem}