\chapter{Proof of Type Soundness}

% TODO: Well-typed terms are closed, no need to specify both.

Proving the progress of expressions and the preservation of types proves the type soundness of the model of computation.  Progress ensures that a well-typed, closed expression is either an unforced value, reducible to another expression, or reducible to an error.  Preservation ensures that if a well-typed expression reduces to another expression, the other expression is well-typed and has the same type.  The proof extends the proof by Kinghorn \cite{kinghorn07}, which was based on proofs by Pierce \cite{pierce02} and Matthews and Findler \cite{matthews07}.  Cases common to two or more languages are elided for brevity.

\section{Proof of Expression Progress}

Progress will be proven by structural induction on a well-typed, closed expression of each syntactic form.  In each case, the expression will be proven to be either an unforced value, reducible to another expression, or reducible to an error.  The reduction of a subexpression is the reduction of its parent expression.  If a subexpression reduces to an error, its parent expression reduces to the error.  In some cases, the syntactic form of a subexpression must be determined to reduce its parent expression.  Determining the unique type of a subexpression determines its syntactic form.

Inverting the typing rules enables the syntactic forms of well-typed expressions to determine the types of their subexpressions.

\begin{lemma}{Inversion of the Typing Relation}

\label{leminv}

The syntactic forms of well-typed expressions determine the types of their subexpressions.  ML cases are omitted because they mirror those of Haskell.  Straightforward Scheme cases are omitted because in those cases well-typed Scheme expressions and subexpressions have the type \tytst.

\begin{enumerate}

% Haskell

% \x:t.e

\item If \judeh{\env}{\expfabss{\varvarh}{\first{\vartyh}}{\varexph}}{\second{\vartyh}} then $\second{\vartyh} = \tyfun{\first{\vartyh}}{\third{\vartyh}}$, \judth{\env}{\first{\vartyh}}, and \judeh{\envexte{\varvarh}{\first{\vartyh}}}{\varexph}{\third{\vartyh}}.

% \\u.e

\item If \judeh{\env}{\exptabs{\tyvarh}{\varexph}}{\first{\vartyh}} then $\first{\vartyh} = \tyfor{\tyvarh}{\second{\vartyh}}$ and \judeh{\envextt{\tyvarh}}{\varexph}{\second{\vartyh}}.

% n

\item If \judeh{}{\expnum{\symnum}}{\vartyh} then $\vartyh = \tynum$.

% nil t

\item If \judeh{\env}{\expnils{\first{\vartyh}}}{\second{\vartyh}} then $\second{\vartyh} = \tylist{\first{\vartyh}}$ and \judth{\env}{\first{\vartyh}}.

% cons e e

\item If \judeh{\env}{\expcons{\first{\varexph}}{\second{\varexph}}}{\first{\vartyh}} then $\first{\vartyh} = \tylist{\second{\vartyh}}$, \judeh{\env}{\first{\varexph}}{\second{\vartyh}}, and \judeh{\env}{\second{\varexph}}{\tylist{\second{\vartyh}}}.

% x

\item \judeh{\envexte{\varvarh}{\vartyh}}{\varvarh}{\vartyh}.

% e e

\item If \judeh{\env}{\expfapp{\first{\varexph}}{\second{\varexph}}}{\first{\vartyh}} then \judeh{\env}{\first{\varexph}}{\tyfun{\second{\vartyh}}{\first{\vartyh}}} and \judeh{\env}{\second{\varexph}}{\second{\vartyh}}.

% fix e

\item If \judeh{\env}{\expfix{\varexph}}{\vartyh} then \judeh{\env}{\varexph}{\tyfun{\vartyh}{\vartyh}}.

% e<t>

\item If \judeh{\env}{\exptapp{\varexph}{\first{\vartyh}}}{\second{\vartyh}} then $\second{\vartyh} = \tysubst{\third{\vartyh}}{\first{\vartyh}}{\tyvarh}$, \judth{\env}{\vartyh}, and \judeh{\env}{\varexph}{\tyfor{\tyvarh}{\third{\vartyh}}}.

% hd e

\item If \judeh{\env}{\exphd{\varexph}}{\vartyh} then \judeh{\env}{\varexph}{\tylist{\vartyh}}.

% tl e

\item If \judeh{\env}{\exptl{\varexph}}{\first{\vartyh}} then $\first{\vartyh} = \tylist{\second{\vartyh}}$ and \judeh{\env}{\varexph}{\tylist{\second{\vartyh}}}.

% o e e

\item If \judeh{\env}{\expop{\first{\varexph}}{\second{\varexph}}}{\first{\vartyh}} then $\first{\vartyh} = \tynum$, \judeh{\env}{\first{\varexph}}{\tynum}, and \judeh{\env}{\second{\varexph}}{\tynum}.

% null? e

\item If \judeh{\env}{\exppnull{\varexph}}{\first{\vartyh}} then $\first{\vartyh} = \tynum$ and \judeh{\env}{\varexph}{\tylist{\second{\vartyh}}}.

% if0 e e e

\item If \judeh{\env}{\expif{\first{\varexph}}{\second{\varexph}}{\third{\varexph}}}{\vartyh} then \judeh{\env}{\first{\varexph}}{\tynum}, \judeh{\env}{\second{\varexph}}{\vartyh}, \judeh{\env}{\third{\varexph}}{\vartyh}.

% wrong t string

\item If \judeh{\env}{\expwrongs{\first{\vartyh}}{\formvar{string}}}{\second{\vartyh}} then $\second{\vartyh} = \first{\vartyh}$.

% hm t t e

\item If \judeh{\env}{\exphm{\first{\vartyh}}{\first{\vartym}}{\varexpm}}{\second{\vartyh}}, $\first{\vartyh} = \first{\vartym}$, and $\first{\vartyh} \neq \tylump$ then $\second{\vartyh} = \first{\vartyh}$, \judth{\env}{\first{\vartyh}}, \judtm{\env}{\first{\vartym}}, \judem{\env}{\varexpm}{\second{\vartym}}, and $\first{\vartym} = \second{\vartym}$.

% hm L t e

\item If \judeh{\env}{\exphm{\tylump}{\first{\vartym}}{\varexpm}}{\tylump} and $\first{\vartym} \neq \tylump$ then \judtm{\env}{\first{\vartym}}, \judem{\env}{\varexpm}{\second{\vartym}}, and $\first{\vartym} = \second{\vartym}$.

% hm t L e

\item If \judeh{\env}{\exphm{\first{\vartyh}}{\tylump}{\varexpm}}{\second{\vartyh}} and $\first{\vartyh} \neq \tylump$ then \judth{\env}{\vartyh} and \judem{\env}{\varexpm}{\tylump}.

% hs k k e

\item If \judeh{\env}{\exphs{\varcsh}{\varexps}}{\vartyh} then $\vartyh = \tyunbrand{\varcsh}$, \judth{\env}{\tyunbrand{\varcsh}}, and \judes{\env}{\varexps}{\tytst}.

% Scheme

% sh k e

\item If \judes{\env}{\expsh{\varcsh}{\varexph}}{\tytst} then \judth{\env}{\tyunbrand{\varcsh}}, \judeh{\env}{\varexph}{\vartyh}, and $\tyunbrand{\varcsh} = \vartyh$.

% sm k e

\item If \judes{\env}{\expsm{\varcsm}{\varexpm}}{\tytst} then \judtm{\env}{\tyunbrand{\varcsm}}, \judem{\env}{\varexpm}{\vartym}, and $\tyunbrand{\varcsm} = \vartym$.

\end{enumerate}

\begin{proof}

Immediate from the typing rules.

\end{proof}

\end{lemma}


Well-typed Haskell and ML expressions have unique types.

\begin{lemma}{Uniqueness of Types}

\label{lemuni}

If \varexph, \varexpm, and \varexps are well-typed then they have only one type.

\begin{proof}

Straightforward structural induction on \varexph, \varexpm, and \varexps using the induction hypothesis and the \proinv.

\end{proof}

\end{lemma}


The types of Haskell and ML values determine their syntactic forms.

\begin{lemma}{Canonical Forms}

\label{lemcan}

The syntactic forms of \prouvs for each type.

\begin{enumerate}

% Haskell

% L

\item If \judeh{\env}{\varvalfh}{\tylump} then $\varvalfh \in \lbrace \exphm{\tylump}{\vartym}{\varvalfm}, \exphs{\cslump}{\varvalfs} \rbrace$.

% N

\item If \judeh{\env}{\varvalfh}{\tynum} then $\varvalfh = \expnum{\symnum}$.

% {t}

\item If \judeh{\env}{\varvalfh}{\tylist{\vartyh}} then $\varvalfh \in \lbrace \expnils{\vartyh}, \expcons{\first{\varexph}}{\second{\varexph}} \rbrace$.

% t->t

\item If \judeh{\env}{\varvalfh}{\tyfun{\first{\vartyh}}{\second{\vartyh}}} then $\varvalfh = \expfabss{\varvarh}{\first{\vartyh}}{\varexph}$.

% Au.t

\item If \judeh{\env}{\varvalfh}{\tyfor{\tyvarh}{\vartyh}} then $\varvalfh = \exptabs{\tyvarh}{\varexph}$.

% ML

% L

\item If \judem{\env}{\varvalfm}{\tylump} then $\varvalfm \in \lbrace \expmh{\tylump}{\vartym}{\varvalfm}, \expms{\cslump}{\varvalfs} \rbrace$.

% N

\item If \judem{\env}{\varvalfm}{\tynum} then $\varvalfm = \expnum{\symnum}$.

% {t}

\item If \judem{\env}{\varvalfm}{\tylist{\vartym}} then $\varvalfm \in \lbrace \expnils{\vartym}, \expcons{\first{\varvalum}}{\second{\varvalum}} \rbrace$.

% t->t

\item If \judem{\env}{\varvalfm}{\tyfun{\first{\vartym}}{\second{\vartym}}} then $\varvalfm = \expfabss{\varvarh}{\first{\vartym}}{\varexph}$.

% Au.t

\item If \judem{\env}{\varvalfm}{\tyfor{\tyvarm}{\vartym}} then $\varvalfm = \exptabs{\tyvarm}{\varexph}$.

\end{enumerate}

\begin{proof}

Immediate from the definitions of unforced values and the typing relations.

\end{proof}

\end{lemma}


\begin{theorem}{Haskell Progress}

\label{thmpsh}

If \judeh{}{\first{\varexph}}{\vartyh} then \pshyp{\first{\varexph}}{\second{\varexph}}.

\begin{proof}

By structural induction on \first{\varexph}.

% x

\newcommand{\psvar}{\varvarh\xspace}

\begin{case}{\psvar}

Cannot occur because \varexph is closed.

\end{case}

% \x:t.e

\newcommand{\psfabss}{\expfabss{\varvarh}{\vartyh}{\varexph}\xspace}

\begin{case}{\psfabss}

\psfabss is an \prouv.

\end{case}

% \\u.e

\newcommand{\pstabs}{\exptabs{\tyvarh}{\varexph}\xspace}

\begin{case}{\pstabs}

\pstabs is an \prouv.

\end{case}

% n

\newcommand{\psnum}{\expnum{\symnum}\xspace}

\begin{case}{\psnum}

\psnum is an \prouv.

\end{case}

% nil t

\newcommand{\psnils}{\expnils{\vartyh}\xspace}

\begin{case}{\psnils}

\psnils is an \prouv.

\end{case}

% cons e e

\newcommand{\psconsh}{\expcons{\first{\varexph}}{\second{\varexph}}\xspace}

\begin{case}{\psconsh}

\psconsh is an \prouv.

\end{case}

% e e

\newcommand{\psfapph}{\expfapp{\first{\varexph}}{\second{\varexph}}\xspace}
\renewcommand{\x}{\expfabss{\varvarh}{\first{\vartyh}}{\third{\varexph}}\xspace}

\begin{case}{\psfapph}

\pshypby
{\first{\varexph}}
{\third{\varexph}}
\psvalifeqh
{\first{\varexph}}
{\tyfun{\first{\vartyh}}{\second{\vartyh}}}
{\x}
\psred
{\expfapp{(\x)}{\second{\varexph}}}
{\expsubst{\third{\varexph}}{\second{\varexph}}{\varvarh}}
\pssub
{\first{\varexph}}
{\third{\varexph}}
{\psfapph}
{\expfapp{\third{\varexph}}{\second{\varexph}}}
\pserr
{\first{\varexph}}
{\psfapph}

\end{case}

% e<t>

\newcommand{\pstapp}{\exptapp{\first{\varexph}}{\first{\vartyh}}\xspace}
\renewcommand{\x}{\exptabs{\tyvarh}{\second{\varexph}}\xspace}

\begin{case}{\pstapp}

\pshypby
{\first{\varexph}}
{\second{\varexph}}
\psvalifeqh
{\first{\varexph}}
{\tyfor{\tyvarh}{\second{\vartyh}}}
{\x}
\psred
{\exptapp{(\x)}{\first{\vartyh}}}
{\expsubst{\second{\varexph}}{\first{\vartyh}}{\tyvarh}}
\pssub
{\first{\varexph}}
{\second{\varexph}}
{\pstapp}
{\exptapp{\second{\varexph}}{\first{\vartyh}}}
\pserr
{\first{\varexph}}
{\pstapp}

\end{case}

% fix e

\newcommand{\psfix}{\expfix{\first{\varexph}}\xspace}
\renewcommand{\x}{\expfabss{\varvarh}{\vartyh}{\second{\varexph}}\xspace}
\renewcommand{\y}{\expfix{(\x)}}

\begin{case}{\psfix}

\pshypby
{\first{\varexph}}
{\second{\varexph}}
\psvalifeqh
{\first{\varexph}}
{\tyfun{\vartyh}{\vartyh}}
{\x}
\psred
{\y}
{\expsubst{\second{\varexph}}{\y}{\varvarh}}
\pssub
{\first{\varexph}}
{\second{\varexph}}
{\psfix}
{\expfix{\second{\varexph}}}
\pserr
{\first{\varexph}}
{\psfix}

\end{case}

% o e e

\newcommand{\psop}{\expop{\first{\varexph}}{\second{\varexph}}\xspace}
\renewcommand{\x}{\first{\expnum{\varnum}}\xspace}
\renewcommand{\y}{\second{\expnum{\varnum}}\xspace}

\begin{case}{\psop}

\pshypby
{\first{\varexph}}
{\third{\varexph}}
\psvalifeqh
{\first{\varexph}}
{\tynum}
{\x}
\pssub
{\first{\varexph}}
{\third{\varexph}}
{\psop}
{\expop{\third{\varexph}}{\second{\varexph}}}
\pserr
{\first{\varexph}}
{\psop}
\pshypby
{\second{\varexph}}
{\third{\varexph}}
\psvalifeqh
{\second{\varexph}}
{\tynum}
{\y}
\pssuband
{\second{\varexph}}
{\third{\varexph}}
{\first{\varexph}}
{\psop}
{\expop{\first{\varexph}}{\third{\varexph}}}
\pserrand
{\second{\varexph}}
{\first{\varexph}}
{\psop}
\psred
{\expadd{\x}{\y}}
{\expnum{\first{\varnum} + \second{\varnum}}}
\psred
{\expsub{\x}{\y}}
{\expnum{\formvar{max}(\first{\varnum} - \second{\varnum}, 0)}}

\end{case}

% if0 e e e

\newcommand{\psif}{\expif{\first{\varexph}}{\second{\varexph}}{\third{\varexph}}\xspace}
\renewcommand{\x}{\expnum{\varnum}\xspace}

\begin{case}{\psif}

\pshypby
{\first{\varexph}}
{\fourth{\varexph}}
\psvalifeqh
{\first{\varexph}}
{\tynum}
{\x}
\psred
{\expif{\expnum{0}}{\second{\varexph}}{\third{\varexph}}}
{\second{\varexph}}
\psrednote
{\expif{\x}{\second{\varexph}}{\third{\varexph}}}
{\third{\varexph}}
{n \neq 0}
\pssub
{\first{\varexph}}
{\fourth{\varexph}}
{\psif}
{\expif{\fourth{\varexph}}{\second{\varexph}}{\third{\varexph}}}
\pserr
{\first{\varexph}}
{\psif}

\end{case}

% f e

\newcommand{\psfield}{\expfield{\first{\varexph}}\xspace}
\renewcommand{\x}{\expnils{\vartyh}\xspace}
\renewcommand{\y}{\expcons{\second{\varexph}}{\third{\varexph}}\xspace}

\begin{case}{\psfield}

\pshypby
{\first{\varexph}}
{\second{\varexph}}
\psvalifinh
{\first{\varexph}}
{\tylist{\vartyh}}
{\x, \y}
\psred
{\exphd{(\x)}}
{\expwrongs{\vartyh}{\errempty}}
\psred
{\exptl{(\x)}}
{\expwrongs{\tylist{\vartyh}}{\errempty}}
\psred
{\exphd{(\y)}}
{\second{\varexph}}
\psred
{\exptl{(\y)}}
{\third{\varexph}}
\pssub
{\first{\varexph}}
{\second{\varexph}}
{\psfield}
{\expfield{\second{\varexph}}}
\pserr
{\first{\varexph}}
{\psfield}

\end{case}

% null? e

\newcommand{\pspnull}{\exppnull{\first{\varexph}}\xspace}
\renewcommand{\x}{\expnils{\vartyh}\xspace}
\renewcommand{\y}{\expcons{\second{\varexph}}{\third{\varexph}}}

\begin{case}{\pspnull}

\pshypby
{\first{\varexph}}
{\second{\varexph}}
\psvalifinh
{\first{\varexph}}
{\tylist{\vartyh}}
{\x, \y}
\psred
{\exppnull{(\x)}}
{\expnum{0}}
\psred
{\exppnull{(\y)}}
{\expnum{1}}
\pssub
{\first{\varexph}}
{\second{\varexph}}
{\pspnull}
{\exppnull{\second{\varexph}}}
\pserr
{\first{\varexph}}
{\pspnull}

\end{case}


% wrong t string

\newcommand{\pswrongs}{\expwrongs{\vartyh}{\varstr}\xspace}

\begin{case}{\pswrongs}

\psred
{\pswrongs}
{\emph{\experr{\varstr}}}

\end{case}

% hm t t e

\newcommand{\pshm}{\exphm{\first{\vartyh}}{\first{\vartym}}{\first{\varexpm}}}

\begin{case}{\pshm}

\pshypby
{\first{\varexpm}}
{\second{\varexpm}}
\pscasestwo
{\first{\varexpm}}
{\first{\vartyh}}
{\first{\vartym}}
{\pshm}

% L, *

\begin{subcase}{\first{\vartyh} $=$ \tylump}

\exphm{\tylump}{\first{\vartym}}{\first{\varexpm}} is an \prouv.

\end{subcase}

% !L, L

\begin{subcase}{\first{\vartyh} $\neq$ \tylump and \first{\vartym} $=$ \tylump}

\psvalinh
{\first{\varexpm}}
{\tylump}
{\expmh{\tylump}{\second{\vartyh}}{\varexph}, \expms{\cslump}{\varvalfs}}
\psrednote
{\exphm{\first{\vartyh}}{\tylump}{(\expmh{\tylump}{\second{\vartyh}}{\varexph})}}
{\varexph}
{\first{\vartyh} = \second{\vartyh}}
\psrednote
{\exphm{\first{\vartyh}}{\tylump}{(\expmh{\tylump}{\second{\vartyh}}{\varexph})}}
{\varexph}
{\first{\vartyh} \neq \second{\vartyh}}
\psred
{\exphm{\first{\vartyh}}{\tylump}{(\expms{\cslump}{\varvalfs})}}
{\expwrongs{\first{\vartyh}}{\errvalue}}

\end{subcase}

% N, N

\begin{subcase}{\first{\vartyh} $=$ \tynum and \first{\vartym} $=$ \tynum}

\psvaleqm
{\first{\varexpm}}
{\tynum}
{\expnum{\varnum}}
\psred
{\exphm{\tynum}{\tynum}{\expnum{\varnum}}}
{\expnum{\varnum}}

\end{subcase}

% {t}, {t}

\renewcommand{\w}{\tylist{\second{\vartyh}}}
\renewcommand{\x}{\tylist{\second\vartym}}
\renewcommand{\y}{\expnils{\third{\vartym}}}
\renewcommand{\z}{\expcons{\first{\varvalum}}{\second{\varvalum}}}

\begin{subcase}{\first{\vartyh} $=$ \w and \first{\vartym} $=$ \x}

\psvalinm
{\first{\varexpm}}
{\tylist{\third{\vartym}}}
{\y, \z}
\psred
{\exphm{\w}{\x}{(\y)}}
{\expnils{\second{\vartyh}}}
\psred
{\exphm{\w}{\x}{(\z)}}
{\expcons{(\exphm{\second{\vartyh}}{\second{\vartym}}{\first{\varvalum}})}{(\exphm{\w}{\x}{\second{\varvalum}})}}

\end{subcase}

% t->t, t->t

\renewcommand{\x}{\tyfun{\second{\vartyh}}{\third{\vartyh}}}
\renewcommand{\y}{\tyfun{\second{\vartym}}{\third{\vartym}}}
\renewcommand{\z}{\expfabss{\varvarm}{\fourth{\vartym}}{\second{\varexpm}}}

\begin{subcase}{\first{\vartyh} $=$ \x and \first{\vartym} $=$ \y}

\psvaleqm
{\first{\varexpm}}
{\tyfun{\fourth{\vartym}}{\fifth{\vartym}}}
{\z}
\psred
{\exphm{(\x)}{(\y)}{(\z)}}
{\expfabss{\varvarh}{\second{\vartyh}}{\exphm{\third{\vartyh}}{\third{\vartym}}{(\expfapp{(\z)}{(\expmh{\second{\vartym}}{\second{\vartyh}}{\varvarh})})}}}

\end{subcase}

% Au.t, Au.t

\renewcommand{\x}{\tyfor{\first{\tyvarh}}{\second{\vartyh}}}
\renewcommand{\y}{\tyfor{\first{\tyvarm}}{\second{\vartym}}}
\renewcommand{\z}{\exptabs{\second{\tyvarm}}{\second{\varexpm}}}

\begin{subcase}{\first{\vartyh} $=$ \x and \first{\vartym} $=$ \y}

\psvaleqm
{\first{\varexpm}}
{\tyfor{\second{\tyvarm}}{\third{\vartym}}}
{\z}
\psred
{\exphm{(\x)}{(\y)}{(\z)}}
{\exptabs{\first{\tyvarh}}{\exphm{\second{\vartyh}}{\tysubst{\second{\vartym}}{\tylump}{\first{\tyvarm}}}{\expsubst{\second{\varexpm}}{\tylump}{\second{\tyvarm}}}}}

\end{subcase}

\pssub
{\first{\varexpm}}
{\second{\varexpm}}
{\pshm}
{\exphm{\first{\vartyh}}{\first{\vartym}}{\second{\varexpm}}}
\pserr
{\first{\varexpm}}
{\pshm}

\end{case}

% hs k e

\newcommand{\pshs}{\exphs{\first{\varcsh}}{\first{\varexps}}}

\begin{case}{\pshs}

\pshypby
{\first{\varexps}}
{\second{\varexps}}
\pscasesone
{\first{\varexps}}
{\first{\varcsh}}
{\pshs}

% L

\begin{subcase}{\cslump}

\exphs{\cslump}{\first{\varexps}} is an \prouv.

\end{subcase}

% N

\begin{subcase}{\csnum}

\psred
{\exphs{\csnum}{\expnum{\varnum}}}
{{\expnum{\varnum}}}
\psrednote
{\exphs{\csnum}{\first{\varexps}}}
{\expwrongs{\tynum}{\errnum}}
{\first{\varexps} \neq \expnum{\varnum}}

\end{subcase}

% {k}

\renewcommand{\x}{\cslist{\second{\varcsh}}}

\begin{subcase}{\x}

\psred
{\exphs{\x}{\expnild}}
{\expnils{\tyunbrand{\second{\varcsh}}}}
\psred
{\exphs{\x}{(\expcons{\first{\varvalus}}{\second{\varvalus}})}}
{\expcons{(\exphs{\varcsh}{\first{\varvalus}})}{(\exphs{\x}{\second{\varvalus}})}}
\psrednote
{\exphs{\x}{\first{\varexps}}}
{\expwrongs{\tyunbrand{\x}}{\errlist}}
{\first{\varexps} \neq \expnild$ and $\first{\varexps} \neq \expcons{\first{\varvalus}}{\second{\varvalus}}}

\end{subcase}

\renewcommand{\x}{\csbrand{\varbrand}{\vartyh}}

\begin{subcase}{\x}

% b.t

\psred
{\exphs{(\x)}{(\expsh{(\x)}{\varexph})}}
{\varexph}
\psrednote
{\exphs{(\x)}{\first{\varexps}}}
{\expwrongs{\vartyh}{\errbrand}}
{\first{\varexps} \neq \expsh{(\x)}{\varexph}}

\end{subcase}

% k->k

\renewcommand{\x}{\csfun{\second{\varcsh}}{\third{\varcsh}}}

\begin{subcase}{\x}

\psred
{\exphs{(\x)}{(\expfabsd{\varvars}{\varexps})}}
{\expfabss{\varvarh}{\tyunbrand{\second{\varcsh}}}{\exphs{\third{\varcsh}}{(\expfapp{(\expfabsd{\varvars}{\varexps})}{(\expsh{\second{\varcsh}}{\varvarh})})}}}
\psrednote
{\exphs{(\x)}{\first{\first{\varexps}}}}
{\expwrongs{\tyunbrand{\x}}{\errfun}}
{\first{\varexps} \neq \expfabsd{\varvars}{\varexps}}

\end{subcase}

% Au.k

\renewcommand{\x}{\csfor{\csvarh}{\second{\varcsh}}}

\begin{subcase}{\x}

\psred
{\exphs{(\x)}{\first{\varexps}}}
{\exptabs{\tyvarh}{\exphs{\second{\varcsh}}{\first{\varexps}}}}

\end{subcase}

\pssub
{\first{\varexps}}
{\second{\varexps}}
{\pshs}
{\exphs{\first{\varcsh}}{\second{\varexps}}}
\pserr
{\first{\varexps}}
{\pshs}

\end{case}

\end{proof}

\end{theorem}


\begin{theorem}{ML Progress}

\label{thmpsm}

If \judem{}{\first{\varexpm}}{\vartym} then \pshyp{\first{\varexpm}}{\second{\varexpm}}.

\begin{proof}

By structural induction on \first{\varexpm}.  Cases similar to Haskell cases are omitted.

% cons w w

\newcommand{\psconswm}{\expcons{\first{\varvalum}}{\second{\varvalum}}\xspace}

\begin{case}{\psconswm}

\psconswm is an \prouv.

\end{case}

% e e

\newcommand{\psfappm}{\expfapp{\first{\varexpm}}{\second{\varexpm}}}
\renewcommand{\x}{\expfabss{\varvarm}{\first{\vartym}}{\third{\varexpm}}}

\begin{case}{\psfappm}

\pshypby
{\first{\varexpm}}
{\third{\varexpm}}
\psvalifeqm
{\first{\varexpm}}
{\tyfun{\first{\vartym}}{\second{\vartym}}}
{\x}
\pssub
{\first{\varexpm}}
{\third{\varexpm}}
{\psfappm}
{\expfapp{\third{\varexpm}}{\second{\varexpm}}}
\pserr
{\first{\varexpm}}
{\psfappm}
\pshypby
{\second{\varexpm}}
{\third{\varexpm}}
\pssuband
{\second{\varexpm}}
{\third{\varexpm}}
{\first{\varexpm}}
{\psfappm}
{\expfapp{\first{\varexpm}}{\third{\varexpm}}}
\pserrand
{\second{\varexpm}}
{\first{\varexpm}}
{\psfappm}
\psred
{\expfapp{(\x)}{\second{\varexpm}}}
{\expsubst{\third{\varexpm}}{\second{\varexpm}}{\varvarm}}

\end{case}

% cons e e

\newcommand{\psconsem}{\expcons{\first{\varexpm}}{\second{\varexpm}}\xspace}

\begin{case}{\psconsem}

\pshypby
{\first{\varexpm}}
{\third{\varexpm}}
\pssub
{\first{\varexpm}}
{\third{\varexpm}}
{\psconsem}
{\expcons{\third{\varexpm}}{\second{\varexpm}}}
\pserr
{\first{\varexpm}}
{\psconsem}
\pssuband
{\second{\varexpm}}
{\third{\varexpm}}
{\first{\varexpm}}
{\psconsem}
{\expcons{\third{\varexpm}}{\second{\varexpm}}}
\pserrand
{\second{\varexpm}}
{\first{\varexpm}}
{\psconsem}
\psvaliftwo
{\first{\varexpm}}
{\second{\varexpm}}
{\psconsem is an \prouv.}

\end{case}

\end{proof}

\end{theorem}


\begin{theorem}{Scheme Progress Theorem}

\label{sps}

If $\vdash_{S}e_{S}:TST$ then $e_{S}$ is an unforced value or $e_{S}\rightarrow e_{S}'$ or $e_{S}\rightarrow$ \emph{\textbf{Error}: string}.

\begin{proof}

By structural induction on $e_{S}$.

\begin{case}

$e_{S}=\lambda x.e_{S}^{1}$

$\lambda x.e_{S}^{1}$ is a value.

\end{case}

\begin{case}
$e_{A}=\overline{n}$ where $A\in\lbrace H,M,S\rbrace$

$\overline{n}$ is a value.
\end{case}

\begin{case}
$e_{S}=\mathtt{nil}$

$\mathtt{nil}$ is a value.
\end{case}

\begin{case}

$e_{S}=\mathtt{cons}$ $v_{S}^{1}$ $v_{S}^{2}$

$\mathtt{cons}$ $v_{S}^{1}$ $v_{S}^{2}$ is an unforced value.

\end{case}

\begin{case}

$e_{S}=SH^{T}$ $e_{H}^{1}$

$e_{H}^{1}$ is a value or $e_{H}^{1}\rightarrow e_{H}^{2}$ or $e_{H}^{1}\rightarrow$ \emph{\textbf{Error}: string} by Haskell progress (Theorem \ref{hps}).  If $e_{H}^{1}$ is a value then $T$ determines the reduction of $e_{S}$.

\begin{subcase}

$T=L$

$e_{H}^{1}={^{L}H}S$ $v_{S}$ by canonical forms (Lemma \ref{cf}).  $SH^{L}$ $(^{L}HS$ $v_{S})\rightarrow v_{S}$.

\end{subcase}

\begin{subcase}

$T=N$

$e_{H}^{1}=\overline{n}$ by canonical forms (Lemma \ref{cf}).  $SH^{N}$ $\overline{n}\rightarrow\overline{n}$.

\end{subcase}

\begin{subcase}

$T=[T_{1}]$

$e_{H}^{1}\in\lbrace\mathtt{nil}^{T_{1}},\mathtt{cons}$ $e_{H}^{3}$ $e_{H}^{4}\rbrace$ by canonical forms (Lemma \ref{cf}).  If $e_{H}^{1}=\mathtt{nil}^{T_{1}}$ then $SH^{T_{1}}$ $\mathtt{nil}^{T_{1}}\rightarrow\mathtt{nil}$.  If $e_{H}^{1}=\mathtt{cons}$ $e_{H}^{3}$ $e_{H}^{4}$ then $SH^{[T_{1}]}$ $(\mathtt{cons}$ $e_{H}^{3}$ $e_{H}^{4})$ is a forced value.

\end{subcase}

\begin{subcase}

$T=T_{1}^{a}$

$SH^{T_{1}^{a}}$ $e_{H}^{3}$ is a forced value.

\end{subcase}

\begin{subcase}

$T=T_{1}\rightarrow T_{2}$

$e_{H}^{1}=\lambda x_{1}:T_{1}[T_{i}/T_{i}^{a}].e_{H}^{3}$ by canonical forms (Lemma \ref{cf}).  $SH^{T_{1}\rightarrow T_{2}}$ $(\lambda x_{1}:T_{1}[T_{i}/T_{i}^{a}].e_{H}^{3})\rightarrow\lambda x_{2}.(SH^{T_{2}}$ $((\lambda x_{1}:T_{1}[T_{i}/T_{i}^{a}].e_{H}^{3})$ $(^{T_{1}}HS$ $x_{2})))$.

\end{subcase}

\begin{subcase}

$T=\forall X.T_{1}$

$e_{H}^{1}\in\lbrace\Lambda X.e_{H}^{3},{^{\forall X.T_{1}}H}S$ $v_{S}\rbrace$ by canonical forms (Lemma \ref{cf}).  If $e_{H}^{1}=\Lambda X.e_{H}^{3}$ then $SH^{\forall X.T_{1}}$ $(\Lambda X.e_{H}^{3})\rightarrow SH^{T_{1}[L/X]}$ $((\Lambda X.e_{H}^{3})$ $\lbrace L\rbrace)$.  If $e_{H}^{1}={^{\forall X.T_{1}}H}S$ $v_{S}$ then $SH^{\forall X.T_{1}}$ $(^{\forall X.T_{1}}HS$ $v_{S})\rightarrow v_{S}$.

\end{subcase}

If $e_{H}^{1}\rightarrow e_{H}^{2}$ then $SH^{T}$ $e_{H}^{1}\rightarrow SH^{T}$ $e_{H}^{2}$.  If $e_{H}^{1}\rightarrow$ \emph{\textbf{Error}: string} then $SH^{T}$ $e_{H}^{1}\rightarrow$ \emph{\textbf{Error}: string}.

\end{case}

\begin{case}
$e_{A}=x$ where $A\in\lbrace H,M,S\rbrace$

Cannot occur because $e_{A}$ is closed.
\end{case}

\begin{case}

$e_{S}=e_{S}^{1}$ $e_{S}^{2}$

$e_{S}^{1}$ is a value or $e_{S}^{1}\rightarrow e_{S}^{3}$ or $e_{S}^{1}\rightarrow$ \emph{\textbf{Error}: string} by the induction hypothesis.  If $e_{S}^{1}\rightarrow e_{S}^{3}$ then $e_{S}^{1}$ $e_{S}^{2}\rightarrow e_{S}^{3}$ $e_{S}^{2}$.  If $e_{S}^{1}\rightarrow$ \emph{\textbf{Error}: string} then $e_{S}^{1}$ $e_{S}^{2}\rightarrow$ \emph{\textbf{Error}: string}.  $e_{S}^{2}$ is a value or $e_{S}^{2}\rightarrow e_{S}^{4}$ or $e_{S}^{2}\rightarrow$ \emph{\textbf{Error}: string} by the induction hypothesis.  If $e_{S}^{2}\rightarrow e_{S}^{4}$ and $e_{S}^{1}$ is a value then $e_{S}^{1}$ $e_{S}^{2}\rightarrow e_{S}^{1}$ $e_{S}^{4}$.  If $e_{S}^{2}\rightarrow$ \emph{\textbf{Error}: string} and $e_{S}^{1}$ is a value then $e_{S}^{1}$ $e_{S}^{2}\rightarrow$ \emph{\textbf{Error}: string}.  If $e_{S}^{1}$ and $e_{S}^{2}$ are values then $(\lambda x.e_{S}^{5})$ $e_{S}^{2}\rightarrow e_{S}^{5}[e_{S}^{2}/x]$ if $e_{S}^{1}=\lambda x.e_{S}^{5}$ and $e_{S}^{1}$ $e_{S}^{2}\rightarrow\mathtt{wrong}$ \emph{``Not a function"} otherwise.

\end{case}

\begin{case}

$e_{S}=\mathtt{cons}$ $e_{S}^{1}$ $e_{S}^{2}$

$e_{S}^{1}$ is an unforced value or $e_{S}^{1}\rightarrow e_{S}^{3}$ or $e_{S}^{1}\rightarrow$ \emph{\textbf{Error}: string} by the induction hypothesis.  If $e_{S}^{1}\rightarrow e_{S}^{3}$ then $\mathtt{cons}$ $e_{S}^{1}$ $e_{S}^{2}\rightarrow\mathtt{cons}$ $e_{S}^{3}$ $e_{S}^{2}$.  If $e_{S}^{1}\rightarrow$ \emph{\textbf{Error}: string} then $\mathtt{cons}$ $e_{S}^{1}$ $e_{S}^{2}\rightarrow$ \emph{\textbf{Error}: string}.  $e_{S}^{2}$ is an unforced value or $e_{S}^{2}\rightarrow e_{S}^{4}$ or $e_{S}^{1}\rightarrow$ \emph{\textbf{Error}: string} by the induction hypothesis.  If $e_{S}^{2}\rightarrow e_{S}^{4}$ and $e_{M}^{1}$ is an unforced value then $\mathtt{cons}$ $e_{S}^{1}$ $e_{S}^{2}\rightarrow\mathtt{cons}$ $e_{S}^{1}$ $e_{S}^{4}$.  If $e_{S}^{2}\rightarrow$ \emph{\textbf{Error}: string} and $e_{S}^{1}$ is an unforced value then $\mathtt{cons}$ $e_{S}^{1}$ $e_{S}^{2}\rightarrow$ \emph{\textbf{Error}: string}.  If $e_{S}^{1}$ and $e_{S}^{2}$ are unforced values then $\mathtt{cons}$ $e_{S}^{1}$ $e_{S}^{2}$ is an unforced value.

\end{case}

\begin{case}
$e_{S}=f$ $e_{S}^{1}$

$e_{S}^{1}$ is a forced value or $e_{S}^{1}\rightarrow e_{S}^{2}$ or $e_{S}^{1}\rightarrow$ \emph{\textbf{Error}: string} by the induction hypothesis.  If $e_{S}^{1}\rightarrow e_{S}^{2}$ then $f$ $e_{S}^{1}\rightarrow f$ $e_{S}^{2}$.  If $e_{S}^{1}\rightarrow$ \emph{\textbf{Error}: string} then $f$ $e_{S}^{1}\rightarrow$ \emph{\textbf{Error}: string}.  $e_{S}^{1}$ is a forced value otherwise.  If $e_{S}^{1}=\mathtt{cons}$ $e_{S}^{3}$ $e_{S}^{4}$ then $f$ $(\mathtt{cons}$ $e_{S}^{3}$ $e_{S}^{4})\rightarrow e_{S}^{3}$ if $f=\mathtt{hd}$ and $f$ $(\mathtt{cons}$ $e_{S}^{3}$ $e_{S}^{4})\rightarrow e_{S}^{4}$ if $f=\mathtt{tl}$.  If $e_{S}^{1}=\mathtt{nil}$ then $f$ $\mathtt{nil}\rightarrow\mathtt{wrong}$ \emph{``Empty list"}.  If $e_{S}^{1}=SH^{[T]}$ $(\mathtt{cons}$ $e_{H}^{1}$ $e_{H}^{2})$ then $f$ $(SH^{[T]}$ $(\mathtt{cons}$ $e_{H}^{1}$ $e_{H}^{2}))\rightarrow SH^{T}$ $e_{H}^{1}$ if $f=\mathtt{hd}$ and $f$ $(SH^{[T]}$ $(\mathtt{cons}$ $e_{H}^{1}$ $e_{H}^{2}))\rightarrow SH^{[T]}$ $e_{H}^{2}$ if $f=\mathtt{tl}$.  $f$ $e_{S}^{1}\rightarrow\mathtt{wrong}$ \emph{``Not a list"} otherwise.
\end{case}

\begin{case}
$e_{S}=o\;e_{S}^{1}\;e_{S}^{2}$

$e_{S}^{1}$ is a value or $e_{S}^{1}\rightarrow e_{S}^{3}$ or $e_{S}^{1}\rightarrow$ \emph{\textbf{Error}:\;string} by the induction hypothesis.  If $e_{S}^{1}\rightarrow e_{S}^{3}$ then $o\;e_{S}^{1}\;e_{S}^{2}\rightarrow o\;e_{S}^{3}\;e_{S}^{2}$.  If $e_{S}^{1}\rightarrow$ \emph{\textbf{Error}:\;string} then $o\;e_{S}^{1}\;e_{S}^{2}\rightarrow$ \emph{\textbf{Error}:\;string}.  $e_{S}^{2}$ is a value or $e_{S}^{2}\rightarrow e_{S}^{4}$ or $e_{S}^{2}\rightarrow$ \emph{\textbf{Error}:\;string} by the induction hypothesis.  If $e_{S}^{2}\rightarrow e_{S}^{4}$ and $e_{S}^{1}$ is a value then $o\;e_{S}^{1}\;e_{S}^{2}\rightarrow o\;e_{S}^{1}\;e_{S}^{4}$.  If $e_{S}^{2}\rightarrow$ \emph{\textbf{Error}:\;string} and $e_{S}^{1}$ is a value then $o\;e_{S}^{1}\;e_{S}^{2}\rightarrow$ \emph{\textbf{Error}:\;string}.  If $e_{S}^{1}$ and $e_{S}^{2}$ are values and $e_{S}^{1}=\overline{n_{1}}$ and $e_{S}^{2}=\overline{n_{2}}$ then $+\;\overline{n_{1}}\;\overline{n_{2}}\rightarrow\overline{n_{1}+n_{2}}$ and $-\;\overline{n_{1}}\;\overline{n_{2}}\rightarrow\overline{max(n_{1}-n_{2},0)}$.  If $e_{S}^{1}$ and $e_{S}^{2}$ are values and $e_{S}^{1}\neq\overline{n_{1}}$ or $e_{S}^{2}\neq\overline{n_{2}}$ then $o\;e_{S}^{1}\;e_{S}^{2}\rightarrow\mathtt{wrong}\;\mathrm{``Not\;a\;number"}$ otherwise.
\end{case}

\begin{case}
$e_{S}=p$ $e_{S}^{1}$

$e_{S}^{1}$ is a forced value or $e_{S}^{1}\rightarrow e_{S}^{2}$ or $e_{S}^{1}\rightarrow$ \emph{\textbf{Error}: string} by the induction hypothesis.  If $e_{S}^{1}\rightarrow e_{S}^{2}$ then $p$ $e_{S}^{1}\rightarrow p$ $e_{S}^{2}$.  If $e_{S}^{1}\rightarrow$ \emph{\textbf{Error}: string} then $p$ $e_{S}^{1}\rightarrow$ \emph{\textbf{Error}: string}.  $e_{S}^{1}$ is a forced value otherwise.  If $p=\mathtt{fun?}$ then $\mathtt{fun?}$ $e_{S}^{1}\rightarrow\overline{0}$ if $e_{S}^{1}=\lambda x.e_{S}^{3}$ and $\mathtt{fun?}$ $e_{S}^{1}\rightarrow\overline{1}$ otherwise.  If $p=\mathtt{list?}$ then $\mathtt{list?}$ $e_{S}^{1}\rightarrow\overline{0}$ if $e_{S}^{1}\in\lbrace\mathtt{nil},\mathtt{cons}$ $v_{S}^{1}$ $v_{S}^{2},SH^{[T]}$ $(\mathtt{cons}$ $e_{H}^{1}$ $e_{H}^{2})\rbrace$ and $\mathtt{list?}$ $e_{S}^{1}\rightarrow\overline{1}$ otherwise.  If $p=\mathtt{null?}$ then $\mathtt{null?}$ $e_{S}^{1}\rightarrow\overline{0}$ if $e_{S}^{1}=\mathtt{nil}$ and $\mathtt{null?}$ $e_{S}^{1}\rightarrow\overline{1}$ if $e_{S}^{1}\in\lbrace\mathtt{cons}$ $v_{S}^{1}$ $v_{S}^{2},SH^{[T]}$ $(\mathtt{cons}$ $e_{H}^{1}$ $e_{H}^{2})\rbrace$ and $\mathtt{null?}$ $e_{S}^{1}\rightarrow\mathtt{wrong}$ \emph{``Not a list"} otherwise.  If $p=\mathtt{num?}$ then $\mathtt{num?}$ $e_{S}^{1}\rightarrow\overline{0}$ if $e_{S}^{1}=\overline{n}$ and $\mathtt{num?}$ $e_{S}^{1}\rightarrow\overline{1}$ otherwise.
\end{case}

\begin{case}

$e_{S}=\mathtt{if0}$ $e_{S}^{1}$ $e_{S}^{2}$ $e_{S}^{3}$

$e_{S}^{1}$ is an unforced value or $e_{S}^{1}\rightarrow e_{S}^{4}$ or $e_{S}^{1}\rightarrow$ \emph{\textbf{Error}: string} by the induction hypothesis.  If $e_{S}^{1}\rightarrow e_{S}^{4}$ then $\mathtt{if0}$ $e_{S}^{1}$ $e_{S}^{2}$ $e_{S}^{3}\rightarrow \mathtt{if0}$ $e_{S}^{4}$ $e_{S}^{2}$ $e_{S}^{3}$.  If $e_{S}^{1}\rightarrow$ \emph{\textbf{Error}: string} then $\mathtt{if0}$ $e_{S}^{1}$ $e_{S}^{2}$ $e_{S}^{3}\rightarrow$ \emph{\textbf{Error}: string}.  $e_{S}^{1}$ is an unforced value otherwise.  $\mathtt{if0}$ $\overline{0}$ $e_{S}^{2}$ $e_{S}^{3}\rightarrow e_{S}^{2}$.  $\mathtt{if0}$ $\overline{n}$ $e_{S}^{2}$ $e_{S}^{3}\rightarrow e_{S}^{3}$ $(n\neq 0)$.  $\mathtt{if0}$ $e_{S}^{1}$ $e_{S}^{2}$ $e_{S}^{3}\rightarrow\mathtt{wrong}$ \emph{``Not a number"} $(e_{S}^{1}\neq\overline{n})$.

\end{case}

\begin{case}
$e_{S}=\mathtt{wrong}$ $\mathrm{string}$

$\mathtt{wrong}$ $\mathrm{string}\rightarrow$ \emph{\textbf{Error}:\;string}.
\end{case}

\begin{case}

$e_{S}=SM^{T}$ $e_{M}^{1}$

$e_{M}^{1}$ is a value or $e_{M}^{1}\rightarrow e_{M}^{2}$ or $e_{M}^{1}\rightarrow$ \emph{\textbf{Error}: string} by ML progress (Theorem \ref{mps}).  If $e_{M}^{1}\rightarrow e_{M}^{2}$ then $SM^{T}$ $e_{M}^{1}\rightarrow SM^{T}$ $e_{M}^{2}$.  If $e_{M}^{1}\rightarrow$ \emph{\textbf{Error}: string} then $SM^{T}$ $e_{M}^{1}\rightarrow$ \emph{\textbf{Error}: string}.  If $e_{M}^{1}$ is a value then $T$ determines the reduction of $e_{S}$.

\begin{subcase}

$T=L$

$e_{M}^{1}={^{L}M}S$ $v_{S}$ by canonical forms (Lemma \ref{cf}).  $SM^{L}$ $(^{L}MS$ $v_{S})\rightarrow v_{S}$.

\end{subcase}

\begin{subcase}

$T=N$

$e_{M}^{1}=\overline{n}$ by canonical forms (Lemma \ref{cf}).  $SM^{N}$ $\overline{n}\rightarrow\overline{n}$.

\end{subcase}

\begin{subcase}

$T=[T_{1}]$

$e_{M}^{1}\in\lbrace\mathtt{nil}^{T_{1}},\mathtt{cons}$ $v_{M}^{1}$ $v_{M}^{2},{^{[T_{1}]}M}H^{[T_{1}]}$ $(\mathtt{cons}$ $e_{H}^{1}$ $e_{H}^{2})\rbrace$ by canonical forms (Lemma \ref{cf}).  If $e_{M}^{1}=\mathtt{nil}^{T_{1}}$ then $SM^{T_{1}}$ $\mathtt{nil}^{T_{1}}\rightarrow\mathtt{nil}$.  If $e_{M}^{1}=\mathtt{cons}$ $v_{M}^{1}$ $v_{M}^{2}$ then $SM^{[T_{1}]}$ $(\mathtt{cons}$ $v_{M}^{1}$ $v_{M}^{2})\rightarrow\mathtt{cons}$ $(SM^{T_{1}}$ $v_{M}^{1})$ $(SM^{[T_{1}]}$ $v_{M}^{2})$.  If $e_{M}^{1}={^{[T_{1}]}M}H^{[T_{1}]}$ $(\mathtt{cons}$ $e_{H}^{1}$ $e_{H}^{2})$ then $SM^{[T_{1}]}$ $({^{[T_{1}]}M}H^{[T_{1}]}$ $(\mathtt{cons}$ $e_{H}^{1}$ $e_{H}^{2}))\rightarrow SH^{[T_{1}]}$ $(\mathtt{cons}$ $e_{H}^{1}$ $e_{H}^{2})$.

\end{subcase}

\begin{subcase}

$T=T_{1}^{a}$

$SM^{T_{1}^{a}}$ $e_{M}^{1}$ is a value.

\end{subcase}

\begin{subcase}

$T=T_{1}\rightarrow T_{2}$

$e_{M}^{1}=\lambda x_{1}:T_{1}[T_{i}/T_{i}^{a}].e_{M}^{3}$ by canonical forms (Lemma \ref{cf}).  $SM^{T_{1}\rightarrow T_{2}}$ $(\lambda x_{1}:T_{1}[T_{i}/T_{i}^{a}].e_{M}^{3})\rightarrow\lambda x_{2}.(SM^{T_{2}}$ $((\lambda x_{1}:T_{1}[T_{i}/T_{i}^{a}].e_{M}^{3})$ $(^{T_{1}}MS$ $x_{2})))$.

\end{subcase}

\begin{subcase}

$T=\forall X.T_{1}$

$e_{M}^{1}\in\lbrace\Lambda X.e_{M}^{3},{^{\forall X.T_{1}}M}S$ $v_{S}\rbrace$ by canonical forms (Lemma \ref{cf}).  If $e_{M}^{1}=\Lambda X.e_{M}^{3}$ then $SM^{\forall X.T_{1}}$ $(\Lambda X.e_{M}^{3})\rightarrow SM^{T_{1}[L/X]}$ $((\Lambda X.e_{M}^{3})$ $\lbrace L\rbrace)$.  If $e_{M}^{1}={^{\forall X.T_{1}}M}S$ $v_{S}$ then $SM^{\forall X.T_{1}}$ $(^{\forall X.T_{1}}MS$ $v_{S})\rightarrow v_{S}$.

\end{subcase}

\end{case}

\end{proof}

\end{theorem}


\section{Proof of Type Preservation}

Preservation will be proven by cases on the reduction rules.  In each case, the right side will be proven to be well-typed and have the same type as the left side.  Inversion (Lemma \ref{i}) and uniqueness of types (Lemma \ref{uot}) are used to determine the types of the left side and its subexpressions and the type of the right side.  Some reduction rules use expression and type substitutions.

If $e_{A}^{1}$ is substituted for free occurrences of $x$ within $e_{A}^{2}$, $e_{A}^{1}$ and $x$ have the same type, and the result has the same type as $e_{A}^{2}$, where $A\in\lbrace H,M,S\rbrace$.

\begin{lemma}{Expression Substitution Preservation}

\label{lemexp}

If $\Gamma,x_{1}:T_{1}\vdash_{A}e_{A}:T_{2}$ and $\Gamma\vdash_{A}x_{2}:T_{1}$ then $\Gamma\vdash_{A}e_{A}[x_{2}/x_{1}]:T_{2}$ where $A\in\lbrace H,M\rbrace$.  If $\Gamma,x_{1}:TST\vdash_{S}e_{S}:TST$ and $\Gamma\vdash_{S}x_{2}:TST$ then $\Gamma\vdash_{S}e_{S}[x_{2}/x_{1}]:TST$.

\begin{proof}

By structural induction.

\end{proof}

\end{lemma}


If $T_{1}$ is substituted for free occurrences of $X$ within $e_{A}$ of type $T_{2}$, the type of the result is $T_{1}$ substituted for free occurrences of $X$ within $T_{2}$, where $A\in\lbrace H,M\rbrace$.

\begin{lemma}{Type Substitution Preservation}

\label{lemtyp}

If \judeh{\envextt{\first{\tyvarh}}}{\first{\varexph}}{\first{\vartyh}} and \judth{\env}{\second{\vartyh}} then \judeh{\env}{\expsubst{\first{\varexph}}{\second{\vartyh}}{\first{\tyvarh}}}{\tysubst{\first{\vartyh}}{\second{\vartyh}}{\first{\tyvarh}}}.  If \judem{\envextt{\first{\tyvarm}}}{\first{\varexpm}}{\first{\vartym}} and \judtm{\env}{\second{\vartym}} then \judem{\env}{\expsubst{\first{\varexpm}}{\second{\vartym}}{\first{\tyvarm}}}{\tysubst{\first{\vartym}}{\second{\vartym}}{\first{\tyvarm}}}.

\begin{proof}

By structural induction.

\end{proof}

\end{lemma}


\begin{lemma}{Evaluation Context Preservation}

\label{lemeva}

If \judeh{}{\first{\varexph}}{\first{\vartyh}}, \judeh{}{\second{\varexph}}{\first{\vartyh}}, and \judeh{}{\redconh{\first{\varexph}}}{\second{\vartyh}} then \judeh{}{\redconh{\second{\varexph}}}{\second{\vartyh}}.
If \judem{}{\first{\varexpm}}{\first{\vartym}}, \judem{}{\second{\varexpm}}{\first{\vartym}}, and \judem{}{\redconm{\first{\varexpm}}}{\second{\vartym}} then \judem{}{\redconm{\second{\varexpm}}}{\second{\vartym}}.
If \judes{}{\first{\varexps}}{\tytst}, \judes{}{\second{\varexps}}{\tytst}, and \judes{}{\redcons{\first{\varexps}}}{\tytst} then \judes{}{\redcons{\second{\varexps}}}{\tytst}.

\begin{proof}

By structural induction.

\end{proof}

\end{lemma}


\begin{theorem}
\label{pn}
If $\Gamma\vdash_{A}e_{A}^{1}:T$ and $e_{A}^{1}\rightarrow e_{A}^{2}$ then $\Gamma\vdash_{A}e_{A}^{2}:T$ where $A\in\lbrace H,M\rbrace$.  If $\Gamma\vdash_{S}e_{S}^{1}:TST$ and $e_{S}^{1}\rightarrow e_{S}^{2}$ then $\Gamma\vdash_{S}e_{S}^{2}:TST$.
\begin{proof}
By cases on the reductions $e_{A}^{1}\rightarrow e_{A}^{2}$ and $e_{S}^{1}\rightarrow e_{S}^{2}$.  Straightforward cases of Scheme preservation are elided.
\begin{case}
$(\lambda x:T_{1}.e_{HM}^{1})\;e_{HM}^{2}\rightarrow e_{HM}^{1}[e_{HM}^{2}/x]$

$\Gamma\vdash_{HM}(\lambda x:T_{1}.e_{HM}^{1})\;e_{HM}^{2}:T$ by the premise and uniqueness of types (Lemma \ref{uot}).  $\Gamma\vdash_{HM}\lambda x:T_{1}.e_{HM}^{1}:T_{1}\rightarrow T$, $\Gamma,x:T_{1}\vdash_{HM}e_{HM}^{1}:T$, $\Gamma\vdash_{HM}e_{HM}^{2}:T_{1}$, and $\Gamma,x:T_{1}\vdash_{HM}x:T_{1}$ by inversion (Lemma \ref{i}) and uniqueness of types.  $e_{HM}^{1}[e_{HM}^{2}/x]:T$ by term substitution (Lemma \ref{tms}).
\end{case}
\begin{case}
$\mathscr{E}[(\Lambda X.e_{H})\;\lbrace T_{1}\rbrace]_{H}\rightarrow\mathscr{E}[e_{H}[T_{1}/X]]$

$(\Lambda X.e_{H})\;\lbrace T_{1}\rbrace:T$ by the induction hypothesis and uniqueness of types (Lemma \ref{uot}).

%\textbf{!!! NOT DONE !!!}
%$\Gamma\vdash_{H}e_{H}:\forall X.T_{2}$, $\Gamma\vdash_{H}T_{1}$, and $T=T_{2}[T_{1}/X]$.
\end{case}
\begin{case}
$\mathtt{if0}\;\overline{0}\;e_{HM}^{1}\;e_{HM}^{2}\rightarrow e_{HM}^{1}$

$\Gamma\vdash_{HM}\mathtt{if0}\;\overline{0}\;e_{HM}^{1}\;e_{HM}^{2}:T$ by premise and uniqueness of types (Lemma \ref{uot}).  $\Gamma\vdash_{HM}e_{HM}^{1}:T$ by inversion (Lemma \ref{i}) and uniqueness of types.
\end{case}
\begin{case}
$\mathtt{if0}\;\overline{n}\;e_{A}^{1}\;e_{A}^{2}\rightarrow e_{A}^{2}\;(n\neq0)$ where $A\in\lbrace H,M\rbrace$

$\Gamma\vdash_{A}\mathtt{if0}\;\overline{n}\;e_{A}^{1}\;e_{A}^{2}:T$ by premise and uniqueness of types (Lemma \ref{uot}).  $T=T_{1}$ and $\Gamma\vdash_{A}e_{A}^{2}:T_{1}$ by inversion (Lemma \ref{i}) and uniqueness of types.  $\Gamma\vdash_{A}e_{A}^{2}:T$ because $T_{1}=T$.
\end{case}
\begin{case}
$+\;\overline{n_{1}}\;\overline{n_{2}}\rightarrow\overline{n_{1}+n_{2}}$ where $A\in\lbrace H,M\rbrace$

$\vdash_{A}+\;\overline{n_{1}}\;\overline{n_{2}}:N$ by inversion (Lemma \ref{i}) and uniqueness of types (Lemma \ref{uot}).  $\vdash_{A}\overline{n_{1}+n_{2}}:N$ by inversion and uniqueness of types.
\end{case}
\begin{case}
$-\;\overline{n_{1}}\;\overline{n_{2}}\rightarrow\overline{max(n_{1}-n_{2},0)}$

$\vdash_{HM}-\;\overline{n_{1}}\;\overline{n_{2}}:N$ by inversion (Lemma \ref{i}) and uniqueness of types (Lemma \ref{uot}).  $\vdash_{HM}\overline{max(n_{1}-n_{2},0)}:N$ by inversion and uniqueness of types.
\end{case}
\begin{case}
$\mathtt{hd}$ $(\mathtt{cons}$ $e_{H}^{1}$ $e_{H}^{2})\rightarrow e_{H}^{1}$

$\Gamma\vdash_{H}\mathtt{hd}$ $(\mathtt{cons}$ $e_{H}^{1}$ $e_{H}^{2}):T$ by premise and uniqueness of types (Lemma \ref{uot}).  $\Gamma\vdash_{H}e_{H}^{1}:T_{1}$, $\Gamma\vdash_{H}\mathtt{cons}$ $e_{H}^{1}$ $e_{H}^{2}:[T_{1}]$, and $T=T_{1}$ by inversion (Lemma \ref{i}) and uniqueness of types.  $\Gamma\vdash_{H}e_{H}^{1}:T$ because $T_{1}=T$.
\end{case}
\begin{case}
$\mathtt{tl}\;(\mathtt{cons}\;e_{H}^{1}\;e_{H}^{2})\rightarrow e_{H}^{2}$

$\Gamma\vdash_{H}\mathtt{tl}\;(\mathtt{cons}\;e_{H}^{1}\;e_{H}^{2}):T$ by premise and uniqueness of types (Lemma \ref{uot}).  $\Gamma\vdash_{H}e_{H}^{2}:[T_{1}]$, $\Gamma\vdash_{H}\mathtt{cons}\;e_{H}^{1}\;e_{H}^{2}:[T_{1}]$, and $T=[T_{1}]$ by inversion (Lemma \ref{i}) and uniqueness of types.  $\Gamma\vdash_{H}e_{H}^{2}:T$ because $[T_{1}]=T$.
\end{case}
\begin{case}
$\mathtt{hd}\;(\mathtt{cons}\;v_{M}^{1}\;v_{M}^{2})\rightarrow v_{M}^{1}$

$\Gamma\vdash_{M}\mathtt{hd}\;(\mathtt{cons}\;v_{M}^{1}\;v_{M}^{2}):T$ by premise and uniqueness of types (Lemma \ref{uot}).  $\Gamma\vdash_{M}v_{M}^{1}:T_{1}$, $\Gamma\vdash_{M}\mathtt{cons}\;v_{M}^{1}\;v_{M}^{2}:[T_{1}]$, and $T=T_{1}$ by inversion (Lemma \ref{i}) and uniqueness of types.  $\Gamma\vdash_{M}v_{M}^{1}:T$ because $T_{1}=T$.
\end{case}
\begin{case}
$\mathtt{tl}\;(\mathtt{cons}\;v_{M}^{1}\;v_{M}^{2})\rightarrow v_{M}^{2}$

$\Gamma\vdash_{M}\mathtt{tl}\;(\mathtt{cons}\;v_{M}^{1}\;v_{M}^{2}):T$ by premise and uniqueness of types (Lemma \ref{uot}).  $T=[T_{1}]$, $\Gamma\vdash_{M}\mathtt{cons}\;v_{M}^{1}\;v_{M}^{2}:[T_{1}]$, and $\Gamma\vdash_{M}v_{M}^{2}:[T_{1}]$ by inversion (Lemma \ref{i}) and uniqueness of types.  $\Gamma\vdash_{M}v_{M}^{2}:T$ because $[T_{1}]=T$.
\end{case}
\begin{case}
$\mathtt{hd}\;\mathtt{nil}^{T}\rightarrow\,^{T}B\;(\mathtt{wrong}\;\mathrm{``Empty\;list"})$ where $B\in\lbrace HS,MS\rbrace$

$\Gamma\vdash_{HM}\mathtt{hd}\;\mathtt{nil}^{T}:T$ by premise and uniqueness of types (Lemma \ref{uot}).  $\Gamma\vdash_{S}\mathtt{wrong}\;\mathrm{``Empty\;list"}:TST$ and $\Gamma\vdash_{HM}{^{T}B}\;(\mathtt{wrong}\;\mathrm{``Empty\;list"}):T$ by inversion (Lemma \ref{i}) and uniqueness of types.
\end{case}
\begin{case}
$\mathtt{tl}$ $\mathtt{nil}^{T_{1}}\rightarrow\mathtt{nil}^{T_{1}}$ where $A\in\lbrace H,M\rbrace$

$\Gamma\vdash_{A}\mathtt{tl}$ $\mathtt{nil}^{T_{1}}:T$ by premise and uniqueness of types (Lemma \ref{uot}).  $\Gamma\vdash_{A}\mathtt{nil}^{T_{1}}:[T_{1}]$ and $T=[T_{1}]$ by inversion (Lemma \ref{i}) and uniqueness of types.  $\Gamma\vdash_{A}\mathtt{nil}^{T_{1}}:T$ because $[T_{1}]=T$.
\end{case}
\begin{case}
$\mathtt{fix}\;(\lambda x:T_{1}.e_{A})\rightarrow e_{A}[(\mathtt{fix}\;(\lambda x:T_{1}.e_{A}))/x]$ where $A\in\lbrace H,M\rbrace$

$\Gamma\vdash_{A}\mathtt{fix}\;(\lambda x:T_{1}.e_{A}):T$ by premise and uniqueness of types (Lemma \ref{uot}).  $\Gamma\vdash_{A}\lambda x:T_{1}.e_{A}:T_{1}\rightarrow T_{1}$, $\Gamma,x:T_{1}\vdash_{A}e_{A}:T_{1}$, and $T=T_{1}$ by inversion (Lemma \ref{i}) and uniqueness of types.  $\Gamma\vdash_{A}e_{A}[(\mathtt{fix}\;(\lambda x:(T\rightarrow T).e_{A}))/x]:T_{1}$ by term substitution (Lemma \ref{tms}).  $\Gamma\vdash_{A}e_{A}[(\mathtt{fix}\;(\lambda x:(T\rightarrow T).e_{A}))/x]:T$ because $T_{1}=T$.
\end{case}
\begin{case}
$^{N}AB^{N}$ $\overline{n}\rightarrow\overline{n}$ where $(A,B)\in\lbrace(H,M),(M,H)\rbrace$

$\vdash_{A}{^{N}A}B^{N}$ $\overline{n}:T$ by premise and uniqueness of types (Lemma \ref{uot}).  $\vdash_{A}\overline{n}:N$ and $T=N$ by inversion (Lemma \ref{i}) and uniqueness of types.
\end{case}
\begin{case}
$^{N}AS$ $\overline{n}\rightarrow\overline{n}$ where $A\in\lbrace H,M\rbrace$

$\vdash_{A}{^{N}A}S$ $\overline{n}:T$ by premise and uniqueness of types (Lemma \ref{uot}).  $\vdash_{A}\overline{n}:N$ and $T=N$ by inversion (Lemma \ref{i}) and uniqueness of types.
\end{case}
\begin{case}
$^{N}AS\;v_{S}\rightarrow{^{N}A}S\;(\mathtt{wrong}\;\mathrm{``Not\;a\;number"})\;(v_{S}\neq\overline{n})$ where $A\in\lbrace H,M\rbrace$

$\Gamma\vdash_{A}{^{N}AS}\;v_{S}:T$ by premise and uniqueness of types (Lemma \ref{uot}).  $T=N$ by inversion (Lemma \ref{i}) and uniqueness of types.  $\vdash_{S}\mathtt{wrong}\;\mathrm{``Not\;a\;number"}:TST$ by inversion.  $\vdash_{A}{^{N}A}S\;(\mathtt{wrong}\;\mathrm{``Not\;a\;number"}):N$ by inversion and uniqueness of types.  $\vdash_{A}{^{N}A}S\;(\mathtt{wrong}\;\mathrm{``Not\;a\;number"}):T$ because $N=T$.
\end{case}
\begin{case}
$^{T_{1}\rightarrow T_{2}}AB^{T_{1}\rightarrow T_{2}}\;(\lambda x_{1}:T_{1}.e_{B})\rightarrow\lambda x_{2}:T_{1}.(^{T_{2}}AB^{T_{2}}\;((\lambda x_{1}:T_{1}.e_{B})\;(^{T_{1}}BA^{T_{1}}\;x_{2})))$ where $(A,B)\in\lbrace(H,M),(M,H)\rbrace$

$\Gamma\vdash_{A}{^{T_{1}\rightarrow T_{2}}}AB^{T_{1}\rightarrow T_{2}}\;(\lambda x_{1}:T_{1}.e_{B}):T$ by premise and uniqueness of types (Lemma \ref{uot}).  $\Gamma\vdash_{B}\lambda x_{1}:T_{1}.e_{B}:T_{1}\rightarrow T_{2}$, $T=T_{1}\rightarrow T_{2}$, $\Gamma,x_{2}:T_{1}\vdash_{A}x_{2}:T_{1}$, $\Gamma,x_{2}:T_{1}\vdash_{B}{^{T_{1}}B}A^{T_{1}}\;x_{2}:T_{1}$, $\Gamma,x_{2}:T_{1}\vdash_{B}(\lambda x_{1}:T_{1}.e_{B})\;(^{T_{1}}BA^{T_{1}}\;x_{2}):T_{2}$, $\Gamma,x_{2}:T_{1}\vdash_{A}{^{T_{2}}A}B^{T_{2}}\;((\lambda x_{1}:T_{1}.e_{B})\;(^{T_{1}}BA^{T_{1}}\;x_{2})):T_{2}$, and $\Gamma\vdash_{A}\lambda x_{2}:T_{1}.(^{T_{2}}AB^{T_{2}}\;((\lambda x_{1}:T_{1}.e_{B})\;(^{T_{1}}BA^{T_{1}}\;x_{2}))):T_{1}\rightarrow T_{2}$ by inversion (Lemma \ref{i}) and uniqueness of types.  $\Gamma\vdash_{A}\lambda x_{2}:T_{1}.(^{T_{2}}AB^{T_{2}}\;((\lambda x_{1}:T_{1}.e_{B})\;(^{T_{1}}BA^{T_{1}}\;x_{2}))):T$ because $T_{1}\rightarrow T_{2}=T$.
\end{case}
\begin{case}
$^{T_{1}\rightarrow T_{2}}AS$ $(\lambda x_{1}.e_{S})\rightarrow\lambda x_{2}:T_{1}[T_{i}/T_{i}^{a}].(^{T_{2}}AS$ $((\lambda x_{1}.e_{S})$ $(SA^{T_{1}}$ $x_{2})))$ where $A\in\lbrace H,M\rbrace$

$\Gamma\vdash_{A}{^{T_{1}\rightarrow T_{2}}A}S$ $(\lambda x_{1}.e_{S}):T$ by premise and uniqueness of types (Lemma \ref{uot}).  $\Gamma\vdash_{S}\lambda x_{1}.e_{S}:TST$ by inversion (Lemma \ref{i}).  $T=(T_{1}\rightarrow T_{2})[T_{i}/T_{i}^{a}]$ by inversion and uniqueness of types.  $\Gamma,x_{2}:T_{1}[T_{i}/T_{i}^{a}]\vdash_{A}x_{2}:T_{1}[T_{i}/T_{i}^{a}]$ by inversion and uniqueness of types.  $\Gamma,x_{2}:T_{1}[T_{i}/T_{i}^{a}]\vdash_{S}SA^{T_{1}}$ $x_{2}:TST$ and $\Gamma,x_{2}:T_{1}[T_{i}/T_{i}^{a}]\vdash_{S}(\lambda x_{1}.e_{S})$ $(SA^{T_{1}}$ $x_{2}):TST$ by inversion.  $\Gamma,x_{2}:T_{1}[T_{i}/T_{i}^{a}]\vdash_{A}{^{T_{2}}A}S$ $((\lambda x_{1}.e_{S})$ $(SA^{T_{1}}$ $x_{2})):T_{2}[T_{i}/T_{i}^{a}]$ and $\Gamma\vdash_{A}\lambda x_{2}:T_{1}[T_{i}/T_{i}^{a}].(^{T_{2}}AS$ $((\lambda x_{1}.e_{S})$ $(SA^{T_{1}}$ $x_{2}))):T_{1}[T_{i}/T_{i}^{a}]\rightarrow T_{2}[T_{i}/T_{i}^{a}]$ by inversion and uniqueness of types.  $\Gamma\vdash_{A}\lambda x_{2}:T_{1}[T_{i}/T_{i}^{a}].(^{T_{2}}AS$ $((\lambda x_{1}.e_{S})$ $(SA^{T_{1}}$ $x_{2}))):T$ because $T_{1}[T_{i}/T_{i}^{a}]\rightarrow T_{2}[T_{i}/T_{i}^{a}]=(T_{1}\rightarrow T_{2})[T_{i}/T_{i}^{a}]=T$.
\end{case}
\begin{case}
$^{T_{1}\rightarrow T_{2}}AS$ $v_{S}\rightarrow{^{T_{1}\rightarrow T_{2}}A}S$ $(\mathtt{wrong}$ $\mathrm{``Not}$ $\mathrm{a}$ $\mathrm{function"})$ $(v_{S}\neq\lambda x.e_{S})$ where $A\in\lbrace H,M\rbrace$

$\Gamma\vdash_{A}{^{T_{1}\rightarrow T_{2}}A}S$ $v_{S}:T$ by premise and uniqueness of types (Lemma \ref{uot}).  $T=(T_{1}\rightarrow T_{2})[T_{i}/T_{i}^{a}]$ by inversion (Lemma \ref{i}) and uniqueness of types.  $\vdash_{S}\mathtt{wrong}$ $\mathrm{``Not}$ $\mathrm{a}$ $\mathrm{function"}:TST$ by inversion.  $\Gamma\vdash_{A}{^{T_{1}\rightarrow T_{2}}A}S$ $(\mathtt{wrong}$ $\mathrm{``Not}$ $\mathrm{a}$ $\mathrm{function"}):(T_{1}\rightarrow T_{2})[T_{i}/T_{i}^{a}]$ by inversion and uniqueness of types.  $\Gamma\vdash_{A}{^{T_{1}\rightarrow T_{2}}A}S$ $(\mathtt{wrong}$ $\mathrm{``Not}$ $\mathrm{a}$ $\mathrm{function"}):T$ because $(T_{1}\rightarrow T_{2})[T_{i}/T_{i}^{a}]=T$.
\end{case}
\begin{case}
$SA^{T_{1}\rightarrow T_{2}}\;(\lambda x_{1}:T_{1}[T_{i}/T_{i}^{a}].e_{A})\rightarrow\lambda x_{2}.(SA^{T_{2}}\;((\lambda x_{1}:T_{1}[T_{i}/T_{i}^{a}].e_{A})\;(^{T_{1}}AS\;x_{2})))$ where $A\in\lbrace H,M\rbrace$

$\Gamma\vdash_{S}SA^{T_{1}\rightarrow T_{2}}\;(\lambda x_{1}:T_{1}[T_{i}/T_{i}^{a}].e_{A}):TST$ by premise.  $\Gamma\vdash_{A}\lambda x_{1}:T_{1}[T_{i}/T_{i}^{a}].e_{A}:T_{1}[T_{i}/T_{i}^{a}]\rightarrow T_{2}[T_{i}/T_{i}^{a}]$ by inversion (Lemma \ref{i}) and uniqueness of types (Lemma \ref{uot}).  $\Gamma,x_{2}:TST\vdash_{S}x_{2}:TST$ by inversion.  $\Gamma,x_{2}:TST\vdash_{A}{^{T_{1}}A}S\;x_{2}:T_{1}[T_{i}/T_{i}^{a}]$ and $\Gamma,x_{2}:TST\vdash_{A}(\lambda x_{1}:T_{1}[T_{i}/T_{i}^{a}].e_{A})\;(^{T_{1}}AS\;x_{2}):T_{2}[T_{i}/T_{i}^{a}]$ by inversion and uniqueness of types.  $\Gamma,x_{2}:TST\vdash_{S}SA^{T_{2}}\;((\lambda x_{1}:T_{1}[T_{i}/T_{i}^{a}].e_{A})\;(^{T_{1}}AS\;x_{2})):TST$ and $\Gamma\vdash_{S}\lambda x_{2}.(SA^{T_{2}}\;((\lambda x_{1}:T_{1}[T_{i}/T_{i}^{a}].e_{A})\;(^{T_{1}}AS\;x_{2}))):TST$ by inversion.
\end{case}
\begin{case}
$^{\forall X.T_{1}}AB^{\forall X_{1}.T_{1}}$ $(\Lambda X.e_{B})\rightarrow\Lambda X.(^{T_{1}}AB^{T_{1}}$ $e_{B})$ where $(A,B)\in\lbrace(H,M),(M,H)\rbrace$

$\Gamma\vdash_{A}{^{\forall X.T_{1}}A}B^{\forall X.T_{1}}$ $(\Lambda X_{1}.e_{B}):T$ by premise and uniqueness of types (Lemma \ref{uot}).  $\Gamma,X\vdash_{B}e_{B}:T_{1}$, $\Gamma\vdash_{B}\Lambda X.e_{B}:\forall X.T_{1}$, $T=\forall X.T_{1}$, $\Gamma,X\vdash_{A}{^{T_{1}}A}B^{T_{1}}$ $e_{B}:T_{1}$, and $\Gamma\vdash_{A}\Lambda X.(^{T_{1}}AB^{T_{1}}$ $e_{B}):\forall X.T_{1}$ by inversion (Lemma \ref{i}) and uniqueness of types.  $\Gamma\vdash_{A}\Lambda X.(^{T_{1}}AB^{T_{1}}$ $e_{B}):T$ because $\forall X.T_{1}=T$.
\end{case}
\begin{case}
$^{\forall X.T_{1}}AB^{\forall X.T_{1}}$ $(^{\forall X.T_{1}}BS$ $v_{S})\rightarrow{^{\forall X.T_{1}}A}S$ $v_{S}$ where $(A,B)\in\lbrace(H,M),$ $(M,H)\rbrace$

$\Gamma\vdash_{A}{^{\forall X.T_{1}}A}B^{\forall X.T_{1}}$ $(^{\forall X.T_{1}}BS$ $v_{S}):T$ by premise and uniqueness of types (Lemma \ref{uot}).  $\Gamma\vdash_{S}v_{S}:TST$ by inversion (Lemma \ref{i}).  $\Gamma\vdash_{B}{^{\forall X.T_{1}}B}S$ $v_{S}:\forall X.T_{1}$, $T=\forall X.T_{1}$, and $\Gamma\vdash_{A}{^{\forall X.T_{1}}A}S$ $v_{S}:\forall X.T_{1}$ by inversion and uniqueness of types.  $\Gamma\vdash_{A}{^{\forall X.T_{1}}A}S$ $v_{S}:T$ because $\forall X.T_{1}=T$.
\end{case}
\begin{case}
$(^{\forall X.T_{1}}AS\;v_{S})\;\lbrace T_{2}\rbrace\rightarrow{^{T_{1}[T_{2}^{a}/X]}A}S\;v_{S}$ where $A\in\lbrace H,M\rbrace$

$\Gamma\vdash_{A}(^{\forall X.T_{1}}AS\;v_{S})\;\lbrace T_{2}\rbrace:T$ by premise and uniqueness of types (Lemma \ref{uot}).  $T=T_{1}[T_{2}/X]$ by inversion (Lemma \ref{i}) and uniqueness of types.  $\Gamma\vdash_{S}v_{S}:TST$ by inversion.  $\Gamma\vdash_{A}{^{T_{1}[T_{2}^{a}/X]}A}S\;v_{S}:T_{1}[T_{2}^{a}/X][T_{i}/T_{i}^{a}]$ by inversion and uniqueness of types.  $\Gamma\vdash_{A}{^{T_{1}[T_{2}^{a}/X]}A}S\;v_{S}:T$ because $T_{1}[T_{2}^{a}/X][T_{i}/T_{i}^{a}]=T_{1}[T_{2}/X]=T$.
\end{case}
\begin{case}
$SA^{\forall X.T_{1}}$ $(\Lambda X.e_{A})\rightarrow SA^{T_{1}[L/X]}$ $((\Lambda X.e_{A})$ $\lbrace L\rbrace)$ where $A\in\lbrace H,M\rbrace$

$\Gamma\vdash_{S}SA^{\forall X.T_{1}}$ $(\Lambda X.e_{A}):TST$ by premise.  $\Gamma\vdash_{A}\Lambda X.e_{A}:\forall X.T_{1}$ and $\Gamma\vdash_{A}(\Lambda X.e_{A})$ $\lbrace L\rbrace:T_{1}[L/X]$ by inversion (Lemma \ref{i}) and uniqueness of types (Lemma \ref{uot}).  $\Gamma\vdash_{S}SA^{T_{1}[L/X]}$ $((\Lambda X.e_{A})$ $\lbrace L\rbrace):TST$ by inversion.
\end{case}
\begin{case}
$^{[T_{1}]}HM^{[T_{1}]}\;(\mathtt{cons}\;v_{M}^{1}\;v_{M}^{2})\rightarrow\mathtt{cons}\;(^{T_{1}}HM^{T_{1}}\;v_{M}^{1})\;(^{[T_{1}]}HM^{[T_{1}]}\;v_{M}^{2})$

$^{[T_{1}]}HM^{[T_{1}]}\;(\mathtt{cons}\;v_{M}^{1}\;v_{M}^{2}):T$ by premise and uniqueness of types (Lemma \ref{uot}).  $T=[T_{1}]$, $\Gamma\vdash_{M}v_{M}^{1}:T_{1}$, $\Gamma\vdash_{M}v_{M}^{2}:[T_{1}]$, $\Gamma\vdash_{H}{^{T_{1}}H}M^{T_{1}}\;v_{M}^{1}:T_{1}$, $\Gamma\vdash_{H}{^{[T_{1}]}H}M^{[T_{1}]}\;v_{M}^{2}:[T_{1}]$, and $\Gamma\vdash_{H}\mathtt{cons}\;(^{T_{1}}HM^{T_{1}}\;v_{M}^{1})\;(^{[T_{1}]}HM^{[T_{1}]}\;v_{M}^{2}):[T_{1}]$ by inversion (Lemma \ref{i}) and uniqueness of types.  $\Gamma\vdash_{H}\mathtt{cons}\;(^{T_{1}}HM^{T_{1}}\;v_{M}^{1})\;(^{[T_{1}]}HM^{[T_{1}]}\;v_{M}^{2}):T$ because $[T_{1}]=T$.
\end{case}
\begin{case}
$^{[T_{1}]}HM^{[T_{1}]}\;(^{[T_{1}]}MH^{[T_{1}]}\;(\mathtt{cons}\;e_{H}^{1}\;e_{H}^{2}))\rightarrow\mathtt{cons}\;e_{H}^{1}\;e_{H}^{2}$

$^{[T_{1}]}HM^{[T_{1}]}\;(^{[T_{1}]}MH^{[T_{1}]}\;(\mathtt{cons}\;e_{H}^{1}\;e_{H}^{2})):T$ by premise and uniqueness of types (Lemma \ref{uot}).  $T=[T_{1}]$, $\Gamma\vdash_{H}{^{[T_{1}]}M}H^{[T_{1}]}\;(\mathtt{cons}\;e_{H}^{1}\;e_{H}^{2}):[T_{1}]$, and $\Gamma\vdash_{H}\mathtt{cons}\;e_{H}^{1}\;e_{H}^{2}:[T_{1}]$ by inversion (Lemma \ref{i}) and uniqueness of types.  $\Gamma\vdash_{H}\mathtt{cons}\;e_{H}^{1}\;e_{H}^{2}:T$ because $[T_{1}]=T$.
\end{case}
\begin{case}
$\mathtt{hd}\;(^{[T_{1}]}MH^{[T_{1}]}\;(\mathtt{cons}\;e_{H}^{1}\;e_{H}^{2}))\rightarrow{^{T_{1}}M}H^{T_{1}}\;e_{H}^{1}$

$\Gamma\vdash_{M}\mathtt{hd}\;(^{[T_{1}]}MH^{[T_{1}]}\;(\mathtt{cons}\;e_{H}^{1}\;e_{H}^{2})):T$ by premise and uniqueness of types (Lemma \ref{uot}).  $T=T_{1}$, $\Gamma\vdash_{H}e_{H}^{1}:T_{1}$, and $^{T_{1}}MH^{T_{1}}\;e_{H}^{1}:T_{1}$ by inversion (Lemma \ref{i}) and uniqueness of types (Lemma \ref{uot}).  $^{T_{1}}MH^{T_{1}}\;e_{H}^{1}:T$ because $T_{1}=T$.
\end{case}
\begin{case}
$\mathtt{hd}\;(^{[T_{1}]}MH^{[T_{1}]}\;(\mathtt{cons}\;e_{H}^{1}\;e_{H}^{2}))\rightarrow{^{[T_{1}]}M}H^{[T_{1}]}\;e_{H}^{2}$

$\Gamma\vdash_{M}\mathtt{hd}\;(^{[T_{1}]}MH^{[T_{1}]}\;(\mathtt{cons}\;e_{H}^{1}\;e_{H}^{2})):T$ by premise and uniqueness of types (Lemma \ref{uot}).  $T=T_{1}$, $\Gamma\vdash_{H}e_{H}^{1}:T_{1}$, and $^{[T_{1}]}MH^{[T_{1}]}\;e_{H}^{1}:[T_{1}]$ by inversion (Lemma \ref{i}) and uniqueness of types (Lemma \ref{uot}).  $^{[T_{1}]}MH^{[T_{1}]}\;e_{H}^{2}:T$ because $[T_{1}]=T$.
\end{case}
\begin{case}
$^{[T_{1}]}AS\;(\mathtt{cons}\;v_{S}^{1}\;v_{S}^{2})\rightarrow\mathtt{cons}\;(^{T_{1}}AS\;v_{S}^{1})\;(^{[T_{1}]}AS\;v_{S}^{2})$ where $A\in\lbrace H,M\rbrace$

$\Gamma\vdash_{A}{^{[T_{1}]}A}S\;(\mathtt{cons}\;v_{S}^{1}\;v_{S}^{2}):T$ by premise and uniqueness of types (Lemma \ref{uot}).  $T=[T_{1}]$ by inversion (Lemma \ref{i}) and uniqueness of types.  $\Gamma\vdash_{S}v_{S}^{1}:TST$, and $\Gamma\vdash_{S}v_{S}^{2}:TST$ by inversion.  $\Gamma\vdash_{A}{^{T_{1}}A}S\;v_{S}^{1}:T_{1}$, $\Gamma\vdash_{A}{^{[T_{1}]}A}S\;v_{S}^{2}:[T_{1}]$, and $\Gamma\vdash_{A}\mathtt{cons}\;(^{T_{1}}AS\;v_{S}^{1})\;(^{[T_{1}]}AS\;v_{S}^{2}):[T_{1}]$ by inversion and uniqueness of types.  $\Gamma\vdash_{A}\mathtt{cons}\;(^{T_{1}}AS\;v_{S}^{1})\;(^{[T_{1}]}AS\;v_{S}^{2}):T$ because $[T_{1}]=T$.
\end{case}
\begin{case}
$^{[T_{1}]}HS$ $(SH^{[T_{1}]}$ $(\mathtt{cons}$ $e_{H}^{1}$ $e_{H}^{2}))\rightarrow\mathtt{cons}$ $e_{H}^{1}$ $e_{H}^{2}$

$^{[T_{1}]}HS$ $(SH^{[T_{1}]}$ $(\mathtt{cons}$ $e_{H}^{1}$ $e_{H}^{2})):T$ by premise and uniqueness of types (Lemma \ref{uot}).  $\Gamma\vdash_{H}\mathtt{cons}$ $e_{H}^{1}$ $e_{H}^{2}:[T_{1}]$ by inversion (Lemma \ref{i}) and uniqueness of types.  $\Gamma\vdash_{S}SH^{[T_{1}]}$ $(\mathtt{cons}$ $e_{H}^{1}$ $e_{H}^{2}):TST$ by inversion.  $T=[T_{1}]$ by inversion and uniqueness of types.  $\Gamma\vdash_{H}\mathtt{cons}$ $e_{H}^{1}$ $e_{H}^{2}:T$ because $[T_{1}]=T$.
\end{case}
\begin{case}
$^{[T_{1}]}MS$ $(SH^{[T_{1}]}$ $(\mathtt{cons}$ $e_{H}^{1}$ $e_{H}^{2}))\rightarrow{^{[T_{1}]}M}H^{[T_{1}]}$ $(\mathtt{cons}$ $e_{H}^{1}$ $e_{H}^{2})$

$^{[T_{1}]}MS$ $(SH^{[T_{1}]}$ $(\mathtt{cons}$ $e_{H}^{1}$ $e_{H}^{2})):T$ by premise and uniqueness of types (Lemma \ref{i}).  $\Gamma\vdash_{H}\mathtt{cons}$ $e_{H}^{1}$ $e_{H}^{2}:[T_{1}]$ by inversion (Lemma \ref{i}) and uniqueness of types.  $\Gamma\vdash_{S}SH^{[T_{1}]}$ $(\mathtt{cons}$ $e_{H}^{1}$ $e_{H}^{2}):TST$ by inversion.  $T=[T_{1}]$ and $\Gamma\vdash_{M}{^{[T_{1}]}M}H^{[T_{1}]}$ $(\mathtt{cons}$ $e_{H}^{1}$ $e_{H}^{2}):[T_{1}]$ by inversion and uniqueness of types.  $\Gamma\vdash_{M}{^{[T_{1}]}M}H^{[T_{1}]}$ $(\mathtt{cons}$ $e_{H}^{1}$ $e_{H}^{2}):T$ because $[T_{1}]=T$.
\end{case}
\begin{case}
$\mathtt{hd}\;(SH^{[T_{1}]}\;(\mathtt{cons}\;e_{H}^{1}\;e_{H}^{2}))\rightarrow SH^{T_{1}}\;e_{H}^{1}$

$\Gamma\vdash_{S}\mathtt{hd}\;(SH^{[T_{1}]}\;(\mathtt{cons}\;e_{H}^{1}\;e_{H}^{2})):TST$ by premise.  $\Gamma\vdash_{H}e_{H}^{1}:T_{1}$ by inversion (Lemma \ref{i}) and uniqueness of types (Lemma \ref{uot}).  $\Gamma\vdash_{S}SH^{T_{1}}\;e_{H}^{1}:TST$ by inversion.
\end{case}
\begin{case}
$\mathtt{tl}\;(SH^{[T_{1}]}\;(\mathtt{cons}\;e_{H}^{1}\;e_{H}^{2}))\rightarrow SH^{[T_{1}]}\;e_{H}^{2}$

$\Gamma\vdash_{S}\mathtt{tl}\;(SH^{[T_{1}]}\;(\mathtt{cons}\;e_{H}^{1}\;e_{H}^{2})):TST$ by premise.  $\Gamma\vdash_{H}e_{H}^{2}:[T_{1}]$ by inversion (Lemma \ref{i}) and uniqueness of types (Lemma \ref{uot}).  $\Gamma\vdash_{S}SH^{[T_{1}]}\;e_{H}^{2}:TST$ by inversion.
\end{case}
\begin{case}
$SM^{[T_{1}]}\;(\mathtt{cons}\;v_{M}^{1}\;v_{M}^{2})\rightarrow\mathtt{cons}\;(SM^{T_{1}}\;v_{M}^{1})\;(SM^{[T_{1}]}\;v_{M}^{2})$

$\Gamma\vdash_{S}SM^{[T_{1}]}\;(\mathtt{cons}\;v_{M}^{1}\;v_{M}^{2}):TST$ by premise.  $\Gamma\vdash_{M}v_{M}^{1}:T_{1}$ and $\Gamma\vdash_{M}v_{M}^{2}:[T_{1}]$ by inversion (Lemma \ref{i}) and uniqueness of types (Lemma \ref{uot}).  $\Gamma\vdash_{S}SM^{T_{1}}\;v_{M}^{1}:TST$, $\Gamma\vdash_{S}SM^{[T_{1}]}\;v_{M}^{2}:TST$, and $\Gamma\vdash_{S}\mathtt{cons}\;(SM^{T_{1}}\;v_{M}^{1})\;(SM^{[T_{1}]}\;v_{M}^{2}):TST$ by inversion.
\end{case}
\begin{case}
$SM^{[T_{1}]}\;(^{[T_{1}]}MH^{[T_{1}]}\;(\mathtt{cons}\;e_{H}^{1}\;e_{H}^{2}))\rightarrow SH^{[T_{1}]}\;(\mathtt{cons}\;e_{H}^{1}\;e_{H}^{2})$

$SM^{[T_{1}]}\;(^{[T_{1}]}MH^{[T_{1}]}\;(\mathtt{cons}\;e_{H}^{1}\;e_{H}^{2})):TST$ by premise.  $\Gamma\vdash_{H}\mathtt{cons}\;e_{H}^{1}\;e_{H}^{2}:[T_{1}]$ and $\Gamma\vdash_{M}{^{[T_{1}]}M}H^{[T_{1}]}\;(\mathtt{cons}\;e_{H}^{1}\;e_{H}^{2}):[T_{1}]$ by inversion (Lemma \ref{i}) and uniqueness of types (Lemma \ref{uot}).  $\Gamma\vdash_{S}SH^{[T_{1}]}\;(\mathtt{cons}\;e_{H}^{1}\;e_{H}^{2}):TST$ by inversion.
\end{case}
\begin{case}
$^{[T_{1}]}AB^{[T_{1}]}\;\mathtt{nil}^{T_{1}}\rightarrow\mathtt{nil}^{T_{1}}$ where $(A,B)\in\lbrace(H,M),(M,H)\rbrace$

$\Gamma\vdash_{A}{^{[T_{1}]}A}B^{[T_{1}]}\;\mathtt{nil}^{T_{1}}:T$ by premise and uniqueness of types (Lemma \ref{uot}).  $\Gamma\vdash_{A}\mathtt{nil}^{T_{1}}:[T_{1}]$ and $T=[T_{1}]$ by inversion (Lemma \ref{i}) and uniqueness of types.  $\Gamma\vdash_{A}\mathtt{nil}^{T_{1}}:T$ because $[T_{1}]=T$.
\end{case}
\begin{case}
$^{[T_{1}]}AS$ $\mathtt{nil}\rightarrow\mathtt{nil}^{T_{1}}$ where $A\in\lbrace H,M\rbrace$

$\Gamma\vdash_{A}{^{[T_{1}]}A}S$ $\mathtt{nil}:T$ by premise and uniqueness of types (Lemma \ref{uot}).  $T=[T_{1}]$ and $\Gamma\vdash_{A}\mathtt{nil}^{T_{1}}:[T_{1}]$ by inversion (Lemma \ref{i}) and uniqueness of types.  $\Gamma\vdash_{A}\mathtt{nil}^{T_{1}}:T$ because $[T_{1}]=T$.
\end{case}
\begin{case}
$^{[T_{1}]}AS\;v_{S}^{1}\rightarrow{^{[T_{1}]}A}S\;(\mathtt{wrong}\;\mathrm{``Not\;a\;list"})$ $(v_{S}^{1}\neq\mathtt{cons}\;v_{S}^{2}\;v_{S}^{3}$ and $v_{S}^{1}\neq\mathtt{nil})$ where $A\in\lbrace H,M\rbrace$

$\Gamma\vdash_{A}{^{[T_{1}]}AS}\;v_{S}^{1}:T$ by premise and uniqueness of types (Lemma \ref{uot}).  $T=[T_{1}]$ by inversion (Lemma \ref{i}) and uniqueness of types.  $\vdash_{S}\mathtt{wrong}\;\mathrm{``Not\;a\;list"}:TST$ by inversion.  $\Gamma\vdash_{A}{^{[T_{1}]}A}S\;(\mathtt{wrong}\;\mathrm{``Not\;a\;list"}):[T_{1}]$ by inversion and uniqueness of types.  $\Gamma\vdash_{A}{^{[T_{1}]}A}S\;(\mathtt{wrong}\;\mathrm{``Not\;a\;list"}):T$ because $[T_{1}]=T$.
\end{case}
\begin{case}
$^{L}AB^{L}\;(^{L}BS\;v_{S})\rightarrow{^{L}A}S\;v_{S}$ where $(A,B)\in\lbrace(H,M),(M,H)\rbrace$

$\Gamma\vdash_{A}{^{L}A}B^{L}\;(^{L}BS\;v_{S}):T$ by premise and uniqueness of types (Lemma \ref{uot}).  $T=L$ by inversion (Lemma \ref{i}) and uniqueness of types.  $\Gamma\vdash_{S}v_{S}:TST$ by inversion.  $\Gamma\vdash_{A}{^{L}A}S\;v_{S}:L$ by inversion and uniqueness of types.  $\Gamma\vdash_{A}{^{L}A}S\;v_{S}:T$ because $L=T$.
\end{case}
\begin{case}
$^{T_{1}^{a}}AS$ $(SA^{T_{1}^{a}}$ $B_{A})\rightarrow B_{A}$ where $(A,B)\in\lbrace(H,e),(M,v)\rbrace$

$^{T_{1}^{a}}AS$ $(SA^{T_{1}^{a}}$ $B_{A}):T$ by premise and uniqueness of types (Lemma \ref{uot}).  $\Gamma\vdash_{A}B_{A}:T_{1}^{a}[T_{i}/T_{i}^{a}]$ by inversion (Lemma \ref{i}) and uniqueness of types.  $\Gamma\vdash_{S}SA^{T_{1}^{a}}$ $B_{A}:TST$ by inversion.  $T=T_{1}^{a}[T_{i}/T_{i}^{a}]$ by inversion and uniqueness of types.  $\Gamma\vdash_{A}B_{A}:T$ because $T_{1}^{a}[T_{i}/T_{i}^{a}]=T$.
\end{case}
\begin{case}
$^{T_{1}^{a}}AS\;v_{S}\rightarrow{^{T_{1}^{a}}A}S\;(\mathtt{wrong}\;\mathrm{``Parametricity\;violated"})$ $(v_{S}\neq SA^{T_{1}^{a}}\;B_{A})$ where $(A,B)\in\lbrace(H,e),(M,v)\rbrace$

$\Gamma\vdash_{A}{^{T_{1}^{a}}A}S\;v_{S}:T$ by premise and uniqueness of types (Lemma \ref{uot}).  $T=T_{1}^{a}[T_{i}/T_{i}^{a}]$ by inversion (Lemma \ref{i}) and uniqueness of types.  $\vdash_{S}\mathtt{wrong}\;\mathrm{``Parametricity\;violated"}:TST$ by inversion.  $^{T_{1}^{a}}AS\;(\mathtt{wrong}\;\mathrm{``Parametricity\;violated"}):T_{1}^{a}[T_{i}/T_{i}^{a}]$ by inversion and uniqueness of types.  $^{T_{1}^{a}}AS\;(\mathtt{wrong}\;\mathrm{``Parametricity\;violated"}):T$ because $T_{1}^{a}[T_{i}/T_{i}^{a}]=T$.
\end{case}
\end{proof}
\end{theorem}
