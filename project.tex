\documentclass[12pt]{article}

\usepackage{amsmath}
\usepackage{amssymb}
\usepackage{amsthm}
\usepackage[letterpaper]{geometry}
\usepackage{mathrsfs}
\usepackage{setspace}
\usepackage[overload]{textcase}
\usepackage{array}
\usepackage{xspace}
\usepackage{color}
\usepackage{epic}

\newtheorem{theorem}{Theorem}
\newtheorem{lemma}{Lemma}
\newtheorem{case}{Case}[theorem]
\newtheorem{subcase}{Case}[case]

\begin{document}

\title{Interoperation for Lazy and Eager Evaluation}
\author{William Faught}
\date{May 10, 2011}
\maketitle

\begin{abstract}
Programmers forgo existing solutions to problems in other programming languages where software interoperation proves too cumbersome; they remake solutions, rather than reuse them. To facilitate reuse, interoperation must resolve language incompatibilities transparently. To address part of this problem, we present a model of computation that resolves lazy and eager evaluation strategies using dual notions of evaluation contexts and values to mirror the lazy evaluation strategy in the eager one. This method could be extended to resolve incompatible evaluation contexts for expressions common to any pair of languages.
\end{abstract}

% Symbols

\newcommand{\symadd}{+}
\newcommand{\symcons}{cons}
\newcommand{\symconu}{E}
\newcommand{\symconf}{F}
\newcommand{\symexp}{e}
\newcommand{\symfield}{f}
\newcommand{\symfix}{fix}
\newcommand{\symhd}{hd}
\newcommand{\symhm}{hm}
\newcommand{\symholeh}{\ensuremath{[\,]\langh}\xspace}
\newcommand{\symholem}{\ensuremath{[\,]\langm}\xspace}
\newcommand{\symholes}{\ensuremath{[\,]\langs}\xspace}
\newcommand{\symhs}{hs}
\newcommand{\symif}{if0}
\newcommand{\symlangh}{h}
\newcommand{\symlangm}{m}
\newcommand{\symlangs}{s}
\newcommand{\symlump}{L}
\newcommand{\symmh}{mh}
\newcommand{\symms}{ms}
\newcommand{\symnat}{N}
\newcommand{\symnil}{nil}
\newcommand{\symnull}{null?}
\newcommand{\symnum}{n}
\newcommand{\symop}{o}
\newcommand{\symsh}{sh}
\newcommand{\symsm}{sm}
\newcommand{\symstr}{string}
\newcommand{\symsub}{-}
\newcommand{\symtl}{tl}
\newcommand{\symtype}{t}
\newcommand{\symvalf}{w}
\newcommand{\symvalu}{v}
\newcommand{\symvar}{x}
\newcommand{\symwrong}{wrong}

% Languages

\newcommand{\langh}{\formlang{\symlangh}}
\newcommand{\langm}{\formlang{\symlangm}}
\newcommand{\langs}{\formlang{\symlangs}}

% Variables

\newcommand{\varexph}{\ensuremath{\symexp\langh}\xspace}
\newcommand{\varexpm}{\ensuremath{\symexp\langm}\xspace}
\newcommand{\varexps}{\ensuremath{\symexp\langs}\xspace}
\newcommand{\varvarh}{\ensuremath{\symvar\langh}\xspace}
\newcommand{\varvarm}{\ensuremath{\symvar\langm}\xspace}
\newcommand{\varvars}{\ensuremath{\symvar\langs}\xspace}
\newcommand{\varvaluh}{\ensuremath{\symvalu\langh}\xspace}
\newcommand{\varvalum}{\ensuremath{\symvalu\langm}\xspace}
\newcommand{\varvalus}{\ensuremath{\symvalu\langs}\xspace}
\newcommand{\varvalfm}{\ensuremath{\symvalf\langm}\xspace}
\newcommand{\varvalfs}{\ensuremath{\symvalf\langs}\xspace}
\newcommand{\vartyh}{\ensuremath{\symtype\langh}\xspace}
\newcommand{\vartym}{\ensuremath{\symtype\langm}\xspace}
\newcommand{\vartys}{\ensuremath{\symtype\langs}\xspace}
\newcommand{\varconm}{\ensuremath{\mathscr{\symconu}\xspace}}
\newcommand{\varconuh}{\ensuremath{\symconu\langh}\xspace}
\newcommand{\varconum}{\ensuremath{\symconu\langm}\xspace}
\newcommand{\varconus}{\ensuremath{\symconu\langs}\xspace}
\newcommand{\varconfm}{\ensuremath{\symconf\langm}\xspace}
\newcommand{\varconfs}{\ensuremath{\symconf\langs}\xspace}

% Expressions

\newcommand{\expfabs}[3]{\ensuremath{\lambda #1 : #2 . #3}}
\newcommand{\exptabs}[3]{\ensuremath{\Lambda #1 . #2}}
\newcommand{\expnum}[1]{\ensuremath{\formnum{#1}}}
\newcommand{\expnil}[1]{\ensuremath{\formsym{\symnil} \; #1}}
\newcommand{\expcons}[2]{\ensuremath{\formsym{\symcons} \; #1 \; #2}}
\newcommand{\expfapp}[2]{\ensuremath{#1 \; #2}}
\newcommand{\exptapp}[2]{\ensuremath{#1 \; \lbrace #2 \rbrace}}
\newcommand{\expfix}[1]{\ensuremath{\formsym{\symfix} \; #1}}
\newcommand{\expop}[2]{\ensuremath{\formvar{\symop} \; #1 \; #2}}
\newcommand{\expadd}[2]{\ensuremath{\formsym{\symadd} \; #1 \; #2}}
\newcommand{\expsub}[2]{\ensuremath{\formsym{\symsub} \; #1 \; #2}}
\newcommand{\expifzero}[3]{\ensuremath{\formsym{\symif} \; #1 \; #2 \; #3}}
\newcommand{\expfield}[1]{\ensuremath{\formvar{\symfield} \; #1}}
\newcommand{\exphd}[1]{\ensuremath{\formsym{\symhd} \; #1}}
\newcommand{\exptl}[1]{\ensuremath{\formsym{\symtl} \; #1}}
\newcommand{\expnull}[1]{\ensuremath{\formsym{\symnull} \; #1}}
\newcommand{\expwrongs}[2]{\ensuremath{\formsym{\symwrong} \; #1 \; #2}}
\newcommand{\expwrongd}[1]{\ensuremath{\formsym{\symwrong} \; #1}}
\newcommand{\exphm}[2]{\ensuremath{\formsym{\symhm} \; #1 \; #2}}
\newcommand{\exphs}[2]{\ensuremath{\formsym{\symhs} \; #1 \; #2}}
\newcommand{\expmh}[2]{\ensuremath{\formsym{\symmh} \; #1 \; #2}}
\newcommand{\expms}[2]{\ensuremath{\formsym{\symms} \; #1 \; #2}}
\newcommand{\expsh}[2]{\ensuremath{\formsym{\symsh} \; #1 \; #2}}
\newcommand{\expsm}[2]{\ensuremath{\formsym{\symsm} \; #1 \; #2}}

% Types

\newcommand{\tylump}{\ensuremath{\formsym{L}}\xspace}
\newcommand{\tynat}{\ensuremath{\formsym{N}}\xspace}
\newcommand{\tyvarh}{\ensuremath{\symvar\langh}\xspace}
\newcommand{\tyvarm}{\ensuremath{\symvar\langm}\xspace}
\newcommand{\tylist}[1]{\ensuremath{[#1]}\xspace}
\newcommand{\tylabel}[2]{\ensuremath{#1 . #2}\xspace}
\newcommand{\tyfun}[2]{\ensuremath{#1 \rightarrow #2}\xspace}
\newcommand{\tyfor}[2]{\ensuremath{\forall #1 . #2}\xspace}

% Formats

\newcommand{\formvar}[1]{\ensuremath{#1}}
\newcommand{\formsym}[1]{\ensuremath{\mathtt{#1}}}
\newcommand{\formnum}[1]{\ensuremath{\overline{#1}}}
\newcommand{\formstr}[1]{\ensuremath{\mathrm{#1}}}
\newcommand{\formlang}[1]{\ensuremath{^{#1}}}
\newcommand{\formindex}[2]{\ensuremath{#1_{#2}}}

% Indices

\newcommand{\numa}[1]{\formindex{#1}{1}}
\newcommand{\numb}[1]{\formindex{#1}{2}}
\newcommand{\numc}[1]{\formindex{#1}{3}}
\newcommand{\numi}[1]{\formindex{#1}{i}}

% Miscellaneous

\newcommand{\str}[1]{\ensuremath{\formstr{``#1"}}}
\section{Introduction}

Programmers forgo existing solutions to problems in other programming languages where software interoperation proves too cumbersome; they remake solutions, rather than reuse them. To facilitate reuse, interoperation must resolve incompatible language features transparently at the boundaries between languages. To address part of this problem, this paper presents a model of computation that resolves lazy and eager evaluation strategies.




Matthews and Findler presented \cite{matthews07} two approaches to resolving static and dynamic type systems for interoperation. The first approach, called lump embedding, introduced a new expression called a boundary and a new type called a lump. In the lump embedding, interoperation is represented in syntax by embedding an expression of one language in a boundary of another language. Boundaries have type annotations for the embedded expression that indicate its expected type in the outer language and its actual type in the inner language. 



languages interoperate by exchanging values. A sending language  by the sending language between two languages is the conversion of a value of the sending language to an equal value of the receiving language. This is represented in syntax by embedding an expression of the sending language within a special expression called a boundary of the receiving language. When the boundary is evaluated by the receiving language, the embedded expression is evaluated to a value by the sending language, then the value is converted from the sending language to the receiving language. They presented two ways to 

They resolved dynamic and static type systems, but all languages had to have the same evaluation strategy. If one language was lazy and another eager, then their approach would not work in certain cases.

In interoperation, converted values always pass from inner

 This is represented as an expression of the sending language nested in a special expression called a boundary in the receiving language. 

 expressed as an expression of the  language 
 expressed by nesting expression of one language within 

In their approach, languages are tied together to express interoperation by nesting expressions of one language in expressions of another language in special cross-language conversion expressions called boundaries.

 a system of interoperation for languages with incompatible type systems.

 in which expressions called boundaries wrap expressions of other languages. A boundary represents the conversion of the value to which the wrapped expression evaluates from the inner language to the outer language. Type annotations in a boundary specify the actual type of the wrapped expression of the inner language and the expected type of the converted value of the outer language. For example, $\formsym{xy}\;\varty_x\;\varty_y\;\varexp_y$ represents the conversion of the expression $\varexp_y$ with the type $\varty_y$ of the language \formsym{y} to an equal value with the type $\varty_x$ of the language \formsym{x}. \formsym{xy} can be read as ``language \emph{x} on the outside, language \emph{y} on the inside," where \formsym{xy} is an initialism of the two language names.

Boundaries evaluate to the values they convert. Number conversion is straightforward because numbers can be coerced between languages: \redruleh{\exphm{\tynum}{\tynum}{\expnum{0}}}{\expnum{0}}.


 called boundary expressions that represent the exchange of values between languages. Boundaries contain expressions of other languages that evaluate to values 

 in which expressions from other languages can be nested. Boundaries , where the nested value of the inner language is converted into an equal value of the outer language. Boundaries have type annotations for the inner and outer languages that are type checked and determine the conversion of nested values.



. a model of computation in which interoperation between two languages was expressed as expressions from one language nested within that of the other. using special boundary expressions likeThe boundaryy expressions contained types for the original value on one side and the type for the converte result  in the other language. When the inner expression is a value, it can be converted according to the type in the boundary. The inner expression is reduced when the guard is reduced.  Their model In their model, the two interoperating languages are statically typed, so the boundaries had types for the inner and outer expressions. The system of interoperation ensured that converted values were equal to their original values . Since the languages they used were all eagerly evaluated, there were no evaluation strategy incompatibilities.d If a lazy language were added to the mix, interoperation would change the behavior o f functions or diverge when converted data of infinite size like infinite lists. 

Given a model like that of Kinghorn, there are two ways interoperation can go wrong: converting data and functions. 

Functions are converted by composing them with boundaries around the arguments and results and wrapping it in a native function. For example, the Haskell constant function for natural numbers is converted to an equal ML function like so:

%HM (N->(N->N)) (\x_M:N.(\y_M:(N->N).x_M)) --> \x_H:N.HM (N->N) ((\x_M:N.(\y_M:(N->N).x_M)) (MH N x_H))

In ML, K (wrong N "Wrong") 2 --> Error: Wrong. In Haskell, K (wrong N "Wrong") 2 --> 2. However, if, in Haskell, K were a conversion of the ML K, then K (wrong N "Wrong") 2 --> Error: Wrong.

Lists are converted by wrapping the heads and tails in boundaries and reducing those. If the tail is equal to the outer list, in the case of a list of infinite length, then this process will recurse forever and diverge.

%zeroes = fix (\x_H:([N]->[N]).cons 0 x_H)
%MH [N] zeroes --> cons (MH N 0) (MH [N] zeroes) /-->


\setlength{\unitlength}{0.75mm}
\begin{picture}(60,40)
\put(30,20){\vector(1,0){30}}
\put(30,20){\vector(4,1){20}}
\put(30,20){\vector(3,1){25}}
\put(30,20){\vector(2,1){30}}
\put(30,20){\vector(1,2){10}}
\thicklines
\put(30,20){\vector(-4,1){30}}
\put(30,20){\vector(-1,4){5}}
\thinlines
\put(30,20){\vector(-1,-1){5}}
\put(30,20){\vector(-1,-4){5}}
\end{picture}



The systems of interoperation presented by Matthews and Findler \cite{matthews07} preserved the equivalence of values converted between languages that have incompatible type systems. Since the languages they used were all eager, there were no evaluation strategy incompatibilities to resolve. If a lazy language is introduced to their systems, then interoperation does not preserve the equivalence of values converted between the lazy language and the eager languages. For example, since the application of a converted function involves applications in both the outer and inner languages, the argument is subject to both the outer and inner evaluation strategies. If the outer language is lazy and the inner language is eager, then the argument may be evaluated by the inner language but not the outer language. In this case, the converted function is not equivalent to the original function. Futhermore, the conversion of composite types like lists from lazy languages to eager ones may diverge or cause an error because eager evaluation will convert the entire value, which may be of infinite size or contain expressions assumed by lazy languages not to be immediately evaluated.

Lazy and eager evaluation take opposite approaches: lazy evaluation evaluates expressions as needed, and eager evaluation evaluates expressions immediately. As such, for common expressions, lazy evaluation evaluates a proper subset of the expressions that eager evaluation does. In other words, the set of lazy evaluation strictness points is a proper subset of that of eager evaluation. The difference between these two sets is the set of incompatible strictness points that may change the meaning of values converted from eager languages to lazy ones or may cause a divergence or an error for values converted from lazy languages to eager ones. Where boundaries that contain expressions of lazy languages are at these points, the original lazy evaluation strategy must be followed, and the guards not evaluated. This requires introducing a dual notion of values where \emph{forced} values force the evaluation of guarded expressions of lazy languages and \emph{unforced} values prevent their evaluation.
\section{Model of Computation}

The model of computation extends the model presented by Kinghorn \cite{kinghorn07} with a third language identical to the ML model except it uses lazy evaluation, and as such is named after Haskell, to which it is more similar. Hereafter, the names Haskell, ML, and Scheme refer to their corresponding models in this paper. Lists are added to all three languages. Being lazy, Haskell does not evaluate function arguments or list construction operands. These three points constitute the set of incompatible strictness points between Haskell and ML and Haskell and Scheme. At these points in ML and Scheme, reducible expressions in Haskell boundaries must not be evaluated.

Since values are irreducible at all points, and since the expressions in Haskell boundaries are irreducible at some points and not others, Haskell boundaries are a new kind of value called an \emph{unforced value}. Like thunks, unforced values can be forced to evaluate to values. The Haskell expressions in Haskell boundaries are forced to evaluate to Haskell values, then the Haskell values are converted to ML or Scheme values. ML and Scheme values are called \emph{forced values} because any might be the result of forcing an unforced value. Forced values are a proper subset of unforced values because unforced values can only be at points where forced values can also be, but forced values can be at points where unforced values cannot. ML and Scheme reduction rules and evaluation contexts use unforced values at the incompatible strictness points to match against Haskell boundaries, and their evaluation contexts prevent evaluation within Haskell boundaries at those points.

Figure \ref{figunf} illustrates forced and unforced values at work for the cases explained in the introduction. The reductions for lines 1-4 show that the outer Haskell argument $zeroes$ is not forced by the application of the inner Scheme function. The reductions for lines 4-8 show that the conversion of $zeroes$ from Haskell to Scheme did not diverge, despite $zeroes$ being a list of infinite size.

\begin{figure}[tb]
\onehalfspacing
\centering
$zeroes = \expfix{(\expfabss{\varvarh}{\tylist{\tynum}}{\expcons{\expnum{0}}{\varvarh}})}$

\begin{tabular}{lll}
% (hs ({N}->{N}) (\x.x)) zeroes
\expfapp
{
	(
	\exphs
	{
		(
		\csfun
		{
			\cslist
			{
				\csnum
			}
		}
		{
			\cslist
			{
				\csnum
			}
		}
		)
	}
	{
		(
		\expfabsd
		{
			\first
			{
				\varvars
			}
		}
		{
			\first
			{
				\varvars
			}
		}
		)
	}
	)
}
{
	\formvar
	{
		zeroes
	}
}
& \red \\
% (\x:{N}.hs {N} ((\x.x) (sh {N} x))) zeroes
\expfapp
{
	(
	\expfabss
	{
		\second
		{
			\varvarh
		}
	}
	{
		\tylist
		{
			\tynum
		}
	}
	{
		\exphs
		{
			\cslist
			{
				\csnum
			}
		}
		{
			(
			\expfapp
			{
				(
				\expfabsd
				{
					\first
					{
						\varvars
					}
				}
				{
					\first
					{
						\varvars
					}
				}
				)
			}
			{
				(
				\expsh
				{
					\cslist
					{
						\csnum
					}
				}
				{
					\second
					{
						\varvarh
					}
				}
				)
			}
			)
		}
	}
	)
}
{
	\formvar
	{
		zeroes
	}
}
& \red \\
% hs {N} ((\x.x) (sh {N} zeroes))
\exphs
{
	\cslist
	{
		\csnum
	}
}
{
	(
	\expfapp
	{
		(
		\expfabsd
		{
			\first
			{
				\varvars
			}
		}
		{
			\first
			{
				\varvars
			}
		}
		)
	}
	{
		(
		\expsh
		{
			\cslist
			{
				\csnum
			}
		}
		{
			\formvar
			{
				zeroes
			}
		}
		)
	}
	)
}
& \red \\
% hs {N} (sh {N} zeroes)
\exphs
{
	\cslist
	{
		\csnum
	}
}
{
	(
	\expsh
	{
		\cslist
		{
			\csnum
		}
	}
	{
		\formvar
		{
			zeroes
		}
	}
	)
}
& \red \\
% hs {N} (sh {N} (cons 0 zeroes))
\exphs
{
	\cslist
	{
		\csnum
	}
}
{
	(
	\expsh
	{
		\cslist
		{
			\csnum
		}
	}
	{
		(
		\expcons
		{
			\expnum
			{
				0
			}
		}
		{
			\formvar
			{
				zeroes
			}
		}
		)
	}
	)
}
& \red \\
% hs {N} (cons (sh N 0) (sh {N} zeroes))
\exphs
{
	\cslist
	{
		\csnum
	}
}
{
	(
	\expcons
	{
		(
		\expsh
		{
			\csnum
		}
		{
			\expnum
			{
				0
			}
		}
		)
	}
	{
		(
		\expsh
		{
			\cslist
			{
				\csnum
			}
		}
		{
			\formvar
			{
				zeroes
			}
		}
		)
	}
	)
}
& \red \\
% hs {N} (cons 0 (sh {N} zeroes))
\exphs
{
	\cslist
	{
		\csnum
	}
}
{
	(
	\expcons
	{
		\expnum
		{
			0
		}
	}
	{
		(
		\expsh
		{
			\cslist
			{
				\csnum
			}
		}
		{
			\formvar
			{
				zeroes
			}
		}
		)
	}
	)
}
& \red \\
% cons (hs N 0) (hs {N} (sh {N} zeroes))
\expcons
{
	(
	\exphs
	{
		\csnum
	}
	{
		\expnum
		{
			0
		}
	}
	)
}
{
	(
	\exphs
	{
		\cslist
		{
			\csnum
		}
	}
	{
		(
		\expsh
		{
			\cslist
			{
				\csnum
			}
		}
		{
			\formvar
			{
				zeroes
			}
		}
		)
	}
	)
}
& \\
\end{tabular}
\caption{Haskell argument and list conversions.}
\label{figunf}
\end{figure}

\begin{theorem}{Evaluation Strategy Preservation}

\label{thmstr}
$\varexph = \expmh{\vartym}{\vartyh}{\varexph} = \expsh{\vartyh}{\varexph}$.
$\varexpm = \exphm{\vartyh}{\vartym}{\varexpm} = \expsm{\vartym}{\varexpm}$.
$\varexps = \exphs{\vartyh}{\varexps} = \expms{\vartym}{\varexps}$.
\begin{proof}
By structural induction.
\end{proof}
\end{theorem}

The interoperation of Haskell and ML posed another problem: the conversion of type abstractions. The application of a converted type abstraction cannot substitute the type argument into the inner language directly, since the inner language has no notion of the types of the outer language. Instead, conversion substitutes lumps in a boundary's inner type. The application of a converted type abstraction substitutes the type argument in the boundary's outer type. Since the natural embedding \cite{matthews07} requires the boundary's outer and inner types to be equal, a new equality relation called lump equality is used here that allows lumps within the boundary's inner type to match any corresponding type in the boundary's outer type.

Legends of symbol and syntax names are presented in figures \ref{figsymbols}-\ref{figsyntax2}; Haskell is presented in figures \ref{fighs}-\ref{fighsos}; ML is presented in figures \ref{figms}-\ref{figmsos}; Scheme is presented in figures \ref{figss}-\ref{figsmos}; the unbrand function is presented in figure \ref{figunbrand}; and the lump equality relation is presented in figure \ref{figequality}.

\clearpage

\begin{figure}[p]
%\onehalfspacing
\centering
\begin{tabular}{cl}

\textbf{Symbol} & \textbf{Name} \\

\varbrand & Brand \\
\varcs & Conversion scheme \\
\varexp & Expression \\
\varconf & Forced evaluation context \\
\varvalf & Forced value \\
\tylump & Lump \\
\eq & Lump equality relation \\
\varconm & Meta evaluation context \\
\expnum{\varnum} & Natural number \\
\tynum & Natural number \\
\red & Reduction relation \\
\varty & Type \\
\tyvar & Type variable \\
\env & Typing environment \\
\jud & Typing relation \\
\varconu & Unforced evaluation context \\
\varvalu & Unforced value \\
\varvar & Variable \\

\end{tabular}
\caption{Symbol names}
\label{figsymbols}
\end{figure}

\clearpage

\begin{figure}[p]
%\onehalfspacing
\centering
\begin{tabular}{rl}

\textbf{Syntax} & \textbf{Name} \\

\expadd{\varexp}{\varexp} & Addition \\
\expif{\varexp}{\varexp}{\varexp} & Condition \\\expnils{\varty} & Empty list \\
\expnild & Empty list \\
\exppnull{\varexp} & Empty list predicate \\
\expwrongs{\varty}{\formvar{string}} & Error \\
\expwrongd{\formvar{string}} & Error \\
\expfix{\varexp} & Fixed-point operation \\
\expfabss{\varvar}{\varty}{\varexp} & Function abstraction \\
\expfabsd{\varvars}{\varexps} & Function abstraction \\
\exppfun{\varexps} & Function abstraction predicate \\
\expfapp{\varexp}{\varexp} & Function application \\
\exphm{\vartyh}{\vartym}{\varexpm} & Haskell-ML guard \\
\exphs{\varcsh}{\varexps} & Haskell-Scheme guard \\
\expcons{\varexp}{\varexp} & List construction \\
\exphd{\varexp} & List head \\
\expplist{\varexps} & List predicate \\
\exptl{\varexp} & List tail \\
\expmh{\vartym}{\vartyh}{\varexph} & ML-Haskell guard \\
\expms{\varcsm}{\varexps} & ML-Scheme guard \\
\exppnum{\varexps} & Number predicate \\
\expsh{\varcsh}{\varexph} & Scheme-Haskell guard \\
\expsm{\varcsm}{\varexpm} & Scheme-ML guard \\
\expsub{\varexp}{\varexp} & Subtraction \\
\exptabs{\tyvar}{\varexp} & Type abstraction \\
\exptapp{\varexp}{\varty} & Type application \\

\end{tabular}
\caption{Syntax names}
\label{figsyntax1}
\end{figure}

\begin{figure}[p]
%\onehalfspacing
\centering
\begin{tabular}{rl}

\textbf{Syntax} & \textbf{Name} \\

\csbrand{\varbrand}{\varty} & Branded type \\
\tyfor{\tyvar}{\varty} & Universally quantified type \\
\csfor{\csvar}{\varcs} & Universally quantified conversion scheme \\
\tyfun{\varty}{\varty} & Function abstraction \\
\csfun{\varcs}{\varcs} & Function abstraction \\
\tylist{\varty} & List \\
\cslist{\varcs} & List \\

\end{tabular}
\caption{Syntax names}
\label{figsyntax2}
\end{figure}

\clearpage

\begin{figure}[p]
\caption{The Haskell syntax and evaluation contexts.}
\centering
\begin{tabular}{rcl}

\varexph & $=$ & \varvarh $|$ \varvalfh $|$ \expfapp{\varexph}{\varexph} $|$ \exptapp{\varexph}{\vartyh} $|$ \expop{\varexph}{\varexph} $|$ \expif{\varexph}{\varexph}{\varexph} $|$ \expfield{\varexph} \\

&& \exppnull{\varexph} $|$ \expwrongs{\vartyh}{\formvar{string}} $|$ \exphm{\vartyh}{\vartym}{\varexpm} $|$ \exphs{\varcsh}{\varexps} \\

\varvalfh & $=$ & \expfabss{\varvarh}{\vartyh}{\varexph} $|$ \exptabs{\tyvarh}{\varexph} $|$ \expnum{\varnum} $|$ \expnils{\vartyh} $|$ \expcons{\varexph}{\varexph} $|$ \exphm{\tylump}{\vartym}{\varvalfm} \\

&& \exphs{\cslump}{\varvalfs} \\

\vartyh & $=$ & \tylump $|$ \tynum $|$ \tyvarh $|$ \tylist{\vartyh} $|$ \tyfun{\vartyh}{\vartyh} $|$ \tyfor{\tyvarh}{\vartyh} \\

\varcsh & $=$ & \cslump $|$ \csnum $|$ \csvarh $|$ \cslist{\varcsh} $|$ \csfun{\varcsh}{\varcsh} $|$ \csfor{\csvarh}{\varcsh} $|$ \csbrand{\varbrand}{\vartyh} \\

\formvar{\symop} & $=$ & \formsym{\symadd} $|$ \formsym{\symsub} \\

\formvar{\symfield} & $=$ & \formsym{\symhd} $|$ \formsym{\symtl} \\

\varconfh & $=$ & \symholeh $|$ \expfapp{\varconfh}{\varexph} $|$ \exptapp{\varconfh}{\vartyh} $|$ \expop{\varconfh}{\varexph} $|$ \expop{\varvalfh}{\varconfh} $|$ \expif{\varconfh}{\varexph}{\varexph} \\

&& \expfield{\varconfh} $|$ \exppnull{\varconfh} $|$ \exphm{\vartyh}{\vartym}{\varconfm} $|$ \exphs{\varcsh}{\varconfs}

\end{tabular}
\label{fighs}
\end{figure}

\clearpage

\begin{figure}[p]
\[
% L
\frac
{}
{\judth{}{\tylump}}
\quad
% N
\frac
{}
{\judth{}{\tynum}}
\quad
% u
\frac
{}
{\judth{\envextt{\env}{\tyvarh}}{\tyvarh}}
\]
\[
% {t}
\frac
{\judth{\env}{\vartyh}}
{\judth{\env}{\tylist{\vartyh}}}
\quad
% t->t
\frac
{\judth{\env}{\first{\vartyh}} \quad \judth{\env}{\second{\vartyh}}}
{\judth{\env}{\tyfun{\first{\vartyh}}{\second{\vartyh}}}}
\quad
% Au.t
\frac
{\judth{\env, \tyvarh}{\vartyh}}
{\judth{\env}{\tyfor{\tyvarh}{\vartyh}}}
\]
\bigskip
\[
% \x:t.e
\frac
{\judth{\env}{\first{\vartyh}} \quad \judeh{\envexte{\env}{\varvarh}{\first{\vartyh}}}{\varexph}{\second{\vartyh}}}
{\judeh{\env}{(\expfabss{\varvarh}{\first{\vartyh}}{\varexph})}{\tyfun{\first{\vartyh}}{\second{\vartyh}}}}
\quad
% \\u.e
\frac
{\judeh{\envextt{\env}{\tyvarh}}{\varexph}{\vartyh}}
{\judeh{\env}{\exptabs{\tyvarh}{\varexph}}{\tyfor{\tyvarh}{\vartyh}}}
\quad
% n
\frac
{}
{\judeh{}{\expnum{n}}{\tynum}}
\]
\[
% nil t
\frac
{\judeh{\env}{\vartyh}}
{\judeh{\env}{\expnils{\vartyh}}{\tylist{\vartyh}}}
\quad
% cons e e
\frac
{\judeh{\env}{\first{\varexph}}{\vartyh} \quad \judeh{\env}{\second{\varexph}}{\tylist{\vartyh}}}
{\judeh{\env}{\expcons{\first{\varexph}}{\second{\varexph}}}{\tylist{\vartyh}}}
\quad
% x
\frac
{}
{\judeh{\envexte{\env}{\varvarh}{\vartyh}}{\varvarh}{\vartyh}}
\]
\[
% e e
\frac
{\judeh{\env}{\first{\varexph}}{\tyfun{\first{\vartyh}}{\second{\vartyh}}} \quad \judeh{\env}{\second{\varexph}}{\first{\vartyh}}}
{\env\symjudh\expfapp{\first{\varexph}}{\second{\varexph}}:\second{\vartyh}}
\quad
% fix e
\frac
{\judeh{\env}{\varexph}{\tyfun{\vartyh}{\vartyh}}}
{\judeh{\env}{\expfix{\varexph}}{\vartyh}}
\]
\[
% e<t>
\frac
{\judth{\env}{\first{\vartyh}} \quad \judeh{\env}{\varexph}{\tyfor{\tyvarh}{\second{\vartyh}}}}
{\judeh{\env}{\exptapp{\varexph}{\first{\vartyh}}}{\tysubst{\second{\vartyh}}{\first{\vartyh}}{\tyvarh}}}
\quad
% hd e
\frac
{\judeh{\env}{\varexph}{\tylist{\vartyh}}}
{\judeh{\env}{\exphd{\varexph}}{\vartyh}}
\quad
% tl e
\frac
{\judeh{\env}{\varexph}{\tylist{\vartyh}}}
{\judeh{\env}{\exptl{\varexph}}{\tylist{\vartyh}}}
\]
\[
% o e e
\frac
{\judeh{\env}{\first{\varexph}}{\tynum} \quad \judeh{\env}{\second{\varexph}}{\tynum}}
{\judeh{\env}{\expop{\first{\varexph}}{\second{\varexph}}}{\tynum}}
\quad
% null? e
\frac
{\judeh{\env}{\varexph}{\tylist{\vartyh}}}
{\judeh{\env}{\exppnull{\varexph}}{\tynum}}
\quad
% hs k e
\frac
{\judth{\env}{\tyunbrand{\varcsh}} \quad \judes{\env}{\varexps}{\tytst}}
{\judeh{\env}{\exphs{\varcsh}{\varexps}}{\tyunbrand{\varcsh}}}
\]
\[
% if0 e e e
\frac
{\judeh{\env}{\first{\varexph}}{\tynum} \quad \judeh{\env}{\second{\varexph}}{\vartyh} \quad \judeh{\env}{\third{\varexph}}{\vartyh}}
{\judeh{\env}{\expif{\first{\varexph}}{\second{\varexph}}{\third{\varexph}}}{\vartyh}}
\quad
% wrong t s
\frac
{\judth{\env}{\vartyh}}
{\judeh{\env}{\expwrongs{\vartyh}{\formvar{string}}}{\vartyh}}
\]
\[
% hm k k e
\frac
{\judth{\env}{\vartyh} \quad \judtm{\env}{\first{\vartym}} \quad \judem{\env}{\varexpm}{\second{\vartym}} \quad \vartyh \eq \vartym \quad \first{\vartym} = \second{\vartym}}
{\judeh{\env}{\exphm{\vartyh}{\first{\vartym}}{\varexpm}}{\vartyh}}
\]
\caption{Haskell typing rules}
\label{htr}
\end{figure}


\clearpage

\begin{figure}[p]
\centering
\begin{tabular}{l}

% Function application

\redruleh
{\expfapp{(\expfabss{\varvarh}{\vartyh}{\first{\varexph}})}{\second{\varexph}}}
{\expsubst{\first{\varexph}}{\second{\varexph}}{\varvarh}} \\

% Type application

\redruleh
{\exptapp{(\exptabs{\tyvarh}{\varexph})}{\vartyh}}
{\expsubst{\varexph}{\vartyh}{\tyvarh}} \\

% Fix

\redruleh
{\expfix{(\expfabss{\varvarh}{\vartyh}{\varexph})}}
{\expsubst{\varexph}{\expfix{(\expfabss{\varvarh}{\vartyh}{\varexph})}}{\varvarh}} \\

% Add

\redruleh
{\expadd{\first{\expnum{\varnum}}}{\second{\expnum{\varnum}}}}
{\expnum{\first{\varnum} + \second{\varnum}}} \\

% Subtract

\redruleh
{\expsub{\first{\expnum{\varnum}}}{\second{\expnum{\varnum}}}}
{\expnum{\formvar{max}(\first{\varnum} - \second{\varnum}, 0)}} \\

% If0 true

\redruleh
{\expif{\expnum{0}}{\first{\varexph}}{\second{\varexph}}}
{\first{\varexph}} \\

% If0 false

\redruleh
{\expif{\expnum{\varnum}}{\first{\varexph}}{\second{\varexph}}}
{\second{\varexph}}
$(\varnum \neq 0)$ \\

% Head nil

\redruleh
{\exphd{(\expnils{\vartyh})}}
{\expwrongs{\vartyh}{\str{Empty\;list}}} \\

% Tail nil

\redruleh
{\exptl{(\expnils{\vartyh})}}
{\expwrongs{\tylist{\vartyh}}{\str{Empty\;list}}} \\

% Head cons

\redruleh
{\exphd{(\expcons{\first{\varexph}}{\second{\varexph}})}}
{\first{\varexph}} \\

% Tail cons

\redruleh
{\exptl{(\expcons{\first{\varexph}}{\second{\varexph}})}}
{\second{\varexph}} \\

% Null nil

\redruleh
{\expnull{(\expnils{\vartyh})}}
{\expnum{0}} \\

% Null cons

\redruleh
{\expnull{(\expcons{\first{\varexph}}{\second{\varexph}})}}
{\expnum{1}} \\

% Wrong

\redrule
{\redenvh{\expwrongs{\vartyh}{\formvar{string}}}}
{\experr{\varstr}}

\end{tabular}
\caption{Haskell operational semantics}
\label{hos}
\end{figure}


\clearpage

\begin{figure}[p]
\centering
\begin{tabular}{l}

% hm L (ms L v)

\redruleh
{\exphm{\tylump}{(\expms{\tylump}{\varvalfs})}}
{\exphs{\tylump}{\varvalfs}} \\

% hm N n

\redruleh
{\exphm{\tynum}{\expnum{\varnum}}}
{\expnum{\varnum}} \\

% hm [t] (nil t)

\redruleh
{\exphm{\tylist{\varcsh}}{(\expnils{\vartym})}}
{\expnils{\tyunbrand{\varcsh}}} \\

% hm [t] (cons v v)

\redruleh
{\exphm{\tylist{\varcsh}}{(\expcons{\first{\varvalum}}{\second{\varvalum}})}}
{\expcons{(\exphm{\tyunbrand{\varcsh}}{\first{\varvalum}})}{(\exphm{\tylist{\tyunbrand{\varcsh}}}{\second{\varvalum}})}} \\

% hm (t->t) (\x:t.e)

\redrule
{\redconh{\exphm{(\tyfun{\first{\varcsh}}{\second{\varcsh}})}{(\expfabss{\varvarm}{\vartym}{\varexpm})}}}
{} \\

\redsp \redcon{\expfabss{\varvarh}{\tyunbrand{\first{\varcsh}}}{\exphm{\second{\varcsh}}{\expfapp{((\expfabss{\varvarm}{\vartym}{\varexpm})}{(\expmh{\vartym}{\varvarh})})}}} \\

% hm (Ax.t) (\\x.e)

\redruleh
{\exphm{(\csfor{\csvarh}{\varcsh})}{(\exptabs{\tyvarm}{\varexpm})}}
{\exptabs{\tyvarh}{\exphm{\varcsh}{(\exptapp{(\exptabs{\tyvarm}{\varexpm})}{\tyconv{\tyvarh}})}}} \\

\end{tabular}
\caption{Haskell-ML operational semantics}
\label{hmos}
\end{figure}

\clearpage

\begin{figure}[p]
\onehalfspacing
\centering
\begin{tabular}{l}

% hs N n

\redruleh
{\exphs{\csnum}{\expnum{\varnum}}}
{{\expnum{\varnum}}} \\

% hs N v

\redruleh
{\exphs{\csnum}{\varvalfs}}
{\expwrongs{\tynum}{\errnum}}
$(\varvalfs \neq \expnum{\varnum})$ \\

% hs {k} nil

\redruleh
{\exphs{\cslist{\varcsh}}{\expnild}}
{\expnils{\tyunbrand{\varcsh}}} \\

% hs {k} (cons v v)

\redruleh
{\exphs{\cslist{\varcsh}}{(\expcons{\first{\varvalus}}{\second{\varvalus}})}}
{\expcons{(\exphs{\varcsh}{\first{\varvalus}})}{(\exphs{\cslist{\varcsh}}{\second{\varvalus}})}} \\

% hs {k} v

\redruleh
{\exphs{\cslist{\varcsh}}{\varvalfs}}
{\expwrongs{\tyunbrand{\cslist{\varcsh}}}{\errlist}} \\

\redsp $(\varvalfs \neq \expnild$ and $\varvalfs \neq \expcons{\first{\varvalus}}{\second{\varvalus}})$ \\

% hs (b.t) (sh (b.t) e)

\redruleh
{\exphs{(\csbrand{\varbrand}{\vartyh})}{(\expsh{(\csbrand{\varbrand}{\vartyh})}{\varexph})}}
{\varexph} \\

% hs (b.t) v

\redruleh
{\exphs{(\csbrand{\varbrand}{\vartyh})}{\varvalfs}}
{\expwrongs{\vartyh}{\errbrand}}
$(\varvalfs \neq \expsh{(\csbrand{\varbrand}{\vartyh})}{\varexph})$ \\

% hs (k->k) (\x.e)

\redruleh
{\exphs{(\csfun{\first{\varcsh}}{\second{\varcsh}})}{(\expfabsd{\varvars}{\varexps})}}
{\expfabss{\varvarh}{\tyunbrand{\first{\varcsh}}}{\exphs{\second{\varcsh}}{(\expfapp{(\expfabsd{\varvars}{\varexps})}{(\expsh{\first{\varcsh}}{\varvarh})})}}} \\

% hs (k->k) v

\redruleh
{\exphs{(\csfun{\first{\varcsh}}{\second{\varcsh}})}{\varvalfs}}
{\expwrongs{\tyunbrand{\csfun{\first{\varcsh}}{\second{\varcsh}}}}{\errfun}} \\

\redsp $(\varvalfs \neq \expfabsd{\varvars}{\varexps})$ \\

% hs (Au.k) w

\redruleh
{\exphs{(\csfor{\csvarh}{\varcsh})}{\varvalfs}}
{\exptabs{\tyvarh}{\exphs{\varcsh}{\varvalfs}}} \\

\end{tabular}
\caption{Haskell-Scheme operational semantics}
\label{fighsos}
\end{figure}

\clearpage

\begin{figure}[p]
\onehalfspacing
\centering
\begin{tabular}{rcl}

\varexpm & $=$ & \varvarm $|$ \varvalum $|$ \expfapp{\varexpm}{\varexpm} $|$ \exptapp{\varexpm}{\vartym} $|$ \expfix{\varexpm} $|$ \expop{\varexpm}{\varexpm} $|$ \expif{\varexpm}{\varexpm}{\varexpm} \\

&& \expcons{\varexpm}{\varexpm} $|$ \expfield{\varexpm} $|$ \exppnull{\varexpm} $|$ \expwrongs{\vartym}{\formvar{string}} $|$ \expms{\varcsm}{\varexps} \\

\varvalum & $=$ & \varvalfm $|$ \expmh{\vartym}{\vartyh}{\varexph} \\

\varvalfm & $=$ & \expfabss{\varvarm}{\vartym}{\varexpm} $|$ \exptabs{\tyvarm}{\varexpm} $|$ \expnum{\varnum} $|$ \expnils{\vartym} $|$ \expcons{\varvalum}{\varvalum} $|$ \expmh{\tylump}{\vartyh}{\varexph} \\

&& \expms{\cslump}{\varvalfs} \\

\vartym & $=$ & \tylump $|$ \tynum $|$ \tyvarm $|$ \tylist{\vartym} $|$ \tyfun{\vartym}{\vartym} $|$ \tyfor{\tyvarm}{\vartym} \\

\varcsm & $=$ & \cslump $|$ \csnum $|$ \csvarm $|$ \cslist{\varcsm} $|$ \csfun{\varcsm}{\varcsm} $|$ \csfor{\csvarm}{\varcsm} $|$ \csbrand{\varbrand}{\vartym} \\

\formvar{\symop} & $=$ & \formsym{\symadd} $|$ \formsym{\symsub} \\

\formvar{\symfield} & $=$ & \formsym{\symhd} $|$ \formsym{\symtl} \\

\varconfm & $=$ & \varconum $|$ \expmh{\vartym}{\vartyh}{\varconfh} \\

\varconum & $=$ & \symholem $|$ \expfapp{\varconfm}{\varexpm} $|$ \expfapp{\varvalfm}{\varconum} $|$ \exptapp{\varconfm}{\vartym} $|$ \expfix{\varconfm} $|$ \expop{\varconfm}{\varexpm} $|$ \expop{\varvalfm}{\varconfm} \\

&& \expif{\varconfm}{\varexpm}{\varexpm} $|$ \expcons{\varconum}{\varexpm} $|$ \expcons{\varvalum}{\varconum} $|$ \expfield{\varconfm} $|$ \exppnull{\varconfm} \\

&& \expms{\varcsm}{\varconfs}

\end{tabular}
\caption{ML syntax and evaluation contexts}
\label{figms}
\end{figure}

\clearpage

\begin{figure}[p]
\[
% Lump
\frac
{}
{\judtm{}{\tylump}}
\quad
% Number
\frac
{}
{\judtm{}{\tynum}}
\quad
% Variable
\frac
{}
{\judtm{\envextt{\tyvarm}}{\tyvarm}}
\]
\[
% List
\frac
{\judtm{\env}{\vartym}}
{\judtm{\env}{\tylist{\vartym}}}
\quad
% Label
\frac
{\judtm{\env}{\vartym}}
{\judtm{\env}{\tylabel{\vartym}{\tyvarm}}}
\quad
% Function
\frac
{\judtm{\env}{\first{\vartym}} \quad \judtm{\env}{\second{\vartym}}}
{\judtm{\env}{\tyfun{\first{\vartym}}{\second{\vartym}}}}
\quad
% Forall
\frac
{\judtm{\env, \tyvarm}{\vartym}}
{\judtm{\env}{\tyfor{\tyvarm}{\vartym}}}
\]
\bigskip
\[
% Function abstraction
\frac
{\judtm{}{\first{\vartym}} \quad \judem{\envexte{\varvarm}{\first{\vartym}}}{\varexpm}{\second{\vartym}}}
{\judem{\env}{(\expfabss{\varvarm}{\first{\vartym}}{\varexpm})}{\tyfor{\first{\vartym}}{\second{\vartym}}}}
\quad
% Type abstraction
\frac
{\judem{\envextt{\tyvarm}}{\varexpm}{\vartym}}
{\judem{\env}{\exptabs{\tyvarm}{\varexpm}}{\tyfor{\tyvarm}{\vartym}}}
\quad
% Number
\frac
{}
{\judem{}{\expnum{n}}{\tynum}}
\]
\[
% Nil
\frac
{\judem{\env}{\vartym}}
{\judem{\env}{\expnils{\vartym}}{\tylist{\vartym}}}
\quad
% Cons
\frac
{\judem{\env}{\first{\varexpm}}{\vartym} \quad \judem{\env}{\second{\varexpm}}{\tylist{\vartym}}}
{\judem{\env}{\expcons{\first{\varexpm}}{\second{\varexpm}}}{\tylist{\vartym}}}
\quad
% Variable
\frac
{}
{\judem{\envexte{\varvarm}{\vartym}}{\varvarm}{\vartym}}
\]
\[
% Function application
\frac
{\judem{\env}{\first{\varexpm}}{\tyfun{\first{\vartym}}{\second{\vartym}}} \quad \judem{\env}{\second{\varexpm}}{\first{\vartym}}}
{\env\symjudh\expfapp{\first{\varexpm}}{\second{\varexpm}}:\second{\vartym}}
\quad
% Fix
\frac
{\judem{\env}{\varexpm}{\tyfun{\vartym}{\vartym}}}
{\judem{\env}{\expfix{\varexpm}}{\vartym}}
\]
\[
% Type application
\frac
{\judtm{\env}{\first{\vartym}} \quad \judem{\env}{\varexpm}{\tyfor{\tyvarm}{\second{\vartym}}}}
{\judem{\env}{\exptapp{\varexpm}{\first{\vartym}}}{\tysubst{\second{\vartym}}{\first{\vartym}}{\tyvarm}}}
\quad
% Head
\frac
{\judem{\env}{\varexpm}{\tylist{\vartym}}}
{\judem{\env}{\exphd{\varexpm}}{\vartym}}
\quad
% Tail
\frac
{\judem{\env}{\varexpm}{\tylist{\vartym}}}
{\judem{\env}{\exptl{\varexpm}}{\tylist{\vartym}}}
\]
\[
% Arithmetic
\frac
{\judem{\env}{\first{\varexpm}}{\tynum} \quad \judem{\env}{\second{\varexpm}}{\tynum}}
{\judem{\env}{\expop{\first{\varexpm}}{\second{\varexpm}}}{\tynum}}
\quad
% Null
\frac
{\judem{\env}{\varexpm}{\tylist{\vartym}}}
{\judem{\env}{\exppnull{\varexpm}}{\tynum}}
\]
\[
% If0
\frac
{\judem{\env}{\first{\varexpm}}{\tynum} \quad \judem{\env}{\second{\varexpm}}{\vartym} \quad \judem{\env}{\third{\varexpm}}{\vartym}}
{\judem{\env}{\expif{\first{\varexpm}}{\second{\varexpm}}{\third{\varexpm}}}{\vartym}}
\quad
% Wrong
\frac
{\judtm{\env}{\vartym}}
{\judem{\env}{\expwrongs{\vartym}{\formvar{string}}}{\vartym}}
\]
\[
% ML
\frac
{\judtm{\env}{\vartym} \quad \judeh{\env}{\varexph}{\vartyh} \quad \vartym=\vartyh}
{\judem{\env}{\expmh{\vartym}{\varexph}}{\vartym}}
\quad
% Scheme
\frac
{\judtm{\env}{\vartym} \quad \judes{\env}{\varexps}{\tytst}}
{\judem{\env}{\expms{\vartym}{\varexps}}{\tyunlabm{\vartym}}}
\]
\caption{ML typing rules}
\label{mtr}
\end{figure}

\clearpage

\begin{figure}[p]
\centering
\begin{tabular}{l}

$\mathscr{E}[(\lambda x^m:t.e^m)$ $v^m]^m\rightarrow\mathscr{E}[e^m[v^m/x^m]]$ \\

$\mathscr{E}[(\Lambda x^m.e^m)$ $\lbrace t\rbrace]]^m\rightarrow\mathscr{E}[e^m[t/x^m]]$ \\

$\mathscr{E}[\mathtt{fix}$ $(\lambda x^m:t.e^m)]^m\rightarrow\mathscr{E}[e^m[(\mathtt{fix}$ $(\lambda x^m:t.e^m))/x^m]]$ \\

$\mathscr{E}[+$ $\overline{n_{1}}$ $\overline{n_{2}}]^m\rightarrow\mathscr{E}[\overline{n_{1}+n_{2}}]$ \\

$\mathscr{E}[-$ $\overline{n_{1}}$ $\overline{n_{2}}]^m\rightarrow\mathscr{E}[\overline{max^m(n_{1}-n_{2},0)}]$ \\

$\mathscr{E}[\mathtt{if0}$ $\overline{0}$ $e^m_1$ $e^m_2]^m\rightarrow\mathscr{E}[e^m_1]$ \\

$\mathscr{E}[\mathtt{if0}$ $\overline{n}$ $e^m_1$ $e^m_2]^m\rightarrow\mathscr{E}[e^m_2]$ $(n\neq0)$ \\

$\mathscr{E}[\mathtt{hd}$ $(\mathtt{nil}$ $t)]^m\rightarrow\mathscr{E}[\mathtt{wrong}$ $t$ ``Empty list"$]$ \\

$\mathscr{E}[\mathtt{tl}$ $(\mathtt{nil}$ $t)]^m\rightarrow\mathscr{E}[\mathtt{wrong}$ $[t]$ ``Empty list"$]$ \\

$\mathscr{E}[\mathtt{hd}$ $(\mathtt{cons}$ $v^m_1$ $v^m_2)]^m\rightarrow\mathscr{E}[v^m_1]$ \\

$\mathscr{E}[\mathtt{tl}$ $(\mathtt{cons}$ $v^m_1$ $v^m_2)]^m\rightarrow\mathscr{E}[v^m_2]$ \\

$\mathscr{E}[\mathtt{hd}$ $(\mathtt{mh}$ $[t]$ $(\mathtt{cons}$ $e^h_1$ $e^h_2))]^m\rightarrow\mathscr{E}[\mathtt{mh}$ $t$ $e^h_1]$ \\

$\mathscr{E}[\mathtt{tl}$ $(\mathtt{mh}$ $[t]$ $(\mathtt{cons}$ $e^h_1$ $e^h_2))]^m\rightarrow\mathscr{E}[\mathtt{mh}$ $[t]$ $e^h_2]$ \\

$\mathscr{E}[\mathtt{null?}$ $(\mathtt{nil}$ $t)]^m\rightarrow\mathscr{E}[\overline{0}]$ \\

$\mathscr{E}[\mathtt{null?}$ $(\mathtt{cons}$ $v^m_1$ $v^m_2)]^m\rightarrow\mathscr{E}[\overline{1}]$ \\

$\mathscr{E}[\mathtt{null?}$ $(\mathtt{mh}$ $[t]$ $(\mathtt{cons}$ $e^h_1$ $e^h_2))]^m\rightarrow\mathscr{E}[\overline{1}]$ \\

$\mathscr{E}[\mathtt{wrong}$ $t$ string$]^m\rightarrow$ \textbf{Error}: string

\end{tabular}
\caption{ML operational semantics}
\label{mos}
\end{figure}

\clearpage

\begin{figure}[p]
\centering
\begin{tabular}{l}

$\mathscr{E}[\mathtt{mh}$ $\mathtt{L}$ $(\mathtt{hs}$ $\mathtt{L}$ $v^s)]^m\rightarrow\mathscr{E}[\mathtt{ms}$ $\mathtt{L}$ $v^s]$ \\

$\mathscr{E}[\mathtt{mh}$ $\mathtt{N}$ $\overline{n}]^m\rightarrow\mathscr{E}[\overline{n}]$ \\

$\mathscr{E}[\mathtt{mh}$ $[t]$ $(\mathtt{nil}$ $t)]^m\rightarrow\mathscr{E}[\mathtt{nil}$ $t]$ \\

$\mathscr{E}[\mathtt{mh}$ $[t]$ $(\mathtt{cons}$ $e^h_1$ $e^h_2)]^m\rightarrow\mathscr{E}[\mathtt{cons}$ $(\mathtt{mh}$ $t$ $e^h_1)$ $(\mathtt{mh}$ $[t]$ $e^h_2)]$ \\

$\mathscr{E}[\mathtt{mh}$ $(t_1\rightarrow t_2)$ $(\lambda x^h:t_1.e^h)]^m\rightarrow\mathscr{E}[\lambda x^m:t_1.\mathtt{mh}$ $t_2$ $((\lambda x^h:t_1.e^h)$ $(\mathtt{hm}$ $t_1$ $x^m))]$ \\

$\mathscr{E}[\mathtt{mh}$ $(\forall x^m.t)$ $(\Lambda x^m.e^h)]^m\rightarrow\mathscr{E}[\Lambda x^m.\mathtt{mh}$ $t$ $e^h]$ \\

$\mathscr{E}[\mathtt{mh}$ $(\forall x^m.t)$ $(\mathtt{hs}$ $(\forall x^h.t)$ $v^s)]^m\rightarrow\mathscr{E}[\mathtt{ms}$ $(\forall x^m.t)$ $v^s]$ \\

\end{tabular}
\caption{ML-Haskell operational semantics}
\label{mhos}
\end{figure}

\clearpage

\begin{figure}[p]
\centering
\begin{tabular}{l}

% ms N n

\redrulem
{\expms{\tynum}{\expnum{\varnum}}}
{{\expnum{\varnum}}} \\

% ms N v

\redrulem
{\expms{\tynum}{\varvalus}}
{\expwrongs{\tynum}{\str{Not \; a \; number}}}
$(\varvalus \neq \expnum{\varnum})$ \\

% ms [t] nil

\redrulem
{\expms{\tylist{\vartym}}{\expnild}}
{\expnils{\tyunlabm{\vartym}}} \\

% ms [t] (cons v v)

\redrulem
{\expms{\tylist{\vartym}}{(\expcons{\first{\varvalus}}{\second{\varvalus}})}}
{\expcons{(\expms{\vartym}{\first{\varvalus}})}{(\expms{\tylist{\vartym}}{\second{\varvalus}})}} \\

% ms [t] v

\redrulem
{\expms{\tylist{\vartym}}{\first{\varvalus}}}
{\expwrongs{\tyunlabm{\vartym}}{\str{Not \; a \; list}}} \\

\redsp $(\first{\varvalus} \neq \expnild$ and $\first{\varvalus} \neq \expcons{\second{\varvalus}}{\third{\varvalus}})$ \\

% ms (t.x) (sm (t.x) v)

\redrulem
{\expms{(\tylabel{\vartym}{\tyvarm})}{(\expsm{(\tylabel{\vartyh}{\tyvarh})}{\varvalum})}}
{\varvalum} \\

% ms (t.x) v

\redrulem
{\expms{(\tylabel{\vartym}{\tyvarm})}{\varvalus}}
{\expwrongs{\vartym}{\str{Parametricity \; violated}}}
$(\varvalus \neq \expsm{(\tylabel{\vartym}{\tyvarm})}{\varexpm})$ \\

% ms (t->t) (\x.e)

\redrulem
{\expms{(\tyfun{\first{\vartym}}{\second{\vartym}})}{(\expfabsd{\varvars}{\varexps})}}
{\expfabss{\varvarm}{\tyunlabm{\first{\vartym}}}{\expms{\second{\vartym}}{(\expfapp{(\expfabsd{\varvars}{\varexps})}{(\expsm{\first{\vartym}}{\varvarm})})}}} \\

% ms (t->t) v

\redrulem
{\expms{(\tyfun{\first{\vartym}}{\second{\vartym}})}{\varvalus}}
{\expwrongs{\tyunlabm{(\tyfun{\first{\vartym}}{\second{\vartym}})}}{\str{Not \; a \; function}}} \\

\redsp $(\varvalus \neq \expfabsd{\varvars}{\varexps})$ \\

% (ms (Ax.t) v) {t}

\redrulem
{\exptapp{(\expms{(\tyfor{\tyvarm}{\first{\vartym}})}{\varvalus})}{\second{\vartym}}}
{\expms{\tysubst{\first{\vartym}}{\tylabel{\second{\vartym}}{\tyvarm}}{\tyvarm}}{\varvalus}} \\

\end{tabular}
\caption{ML-Scheme operational semantics}
\label{msos}
\end{figure}

\clearpage

\begin{figure}[p]
\centering
\begin{tabular}{lcl}

\varexps & $=$ & \varvars $|$ \varvalus $|$ \expfapp{\varexps}{\varexps} $|$ \expop{\varexps}{\varexps} $|$ \exppred{\varexps} $|$ \expif{\varexps}{\varexps}{\varexps} $|$ \expcons{\varexps}{\varexps} $|$ \expfield{\varexps} \\

&& \expwrongd{\formvar{string}} $|$ \expsm{\varcsm}{\varexpm} \\

\varvalus & $=$ & \varvalfs $|$ \expsh{\varcsh}{\varexph} \\

\varvalfs & $=$ & \expfabsd{\varvars}{\varexps} $|$ \expnum{\varnum} $|$ \expnild $|$ \expcons{\varvalus}{\varvalus} $|$ \expsh{(\csbrand{\varbrand}{\vartyh})}{\varexph} $|$ \expsm{(\csbrand{\varbrand}{\vartym})}{\varvalfm} \\

\formvar{\symop} & $=$ & \formsym{\symadd} $|$ \formsym{\symsub} \\

\formvar{\symfield} & $=$ & \formsym{\symhd} $|$ \formsym{\symtl} \\

\formvar{\sympred} & $=$ & \formsym{\sympfun} $|$ \formsym{\symplist} $|$ \formsym{\sympnull} $|$ \formsym{\sympnum} \\

\varconfs & $=$ & \varconus $|$ \expsh{\varcsh}{\varconfh} \\

\varconus & $=$ & \symholes $|$ \expfapp{\varconfs}{\varexps} $|$ \expfapp{\varvalfs}{\varconus} $|$ \expop{\varconfs}{\varexps} $|$ \expop{\varvalfs}{\varconfs} $|$ \exppred{\varconfs} $|$ \expif{\varconfs}{\varexps}{\varexps} \\

&& \expcons{\varconus}{\varexps} $|$ \expcons{\varvalus}{\varconus} $|$ \expfield{\varconfs} $|$ \expsm{\varcsm}{\varconfm}

\end{tabular}
\caption{Scheme syntax and evaluation contexts}
\label{figss}
\end{figure}

\clearpage

\begin{figure}[p]
\[
\frac{}{\vdash^s\mathtt{TST}}
\]
\bigskip
\[
\frac{\Gamma,x^s:\mathtt{TST}\vdash^se^s:\mathtt{TST}}{\Gamma\vdash^s\lambda x^s.e^s:\mathtt{TST}}
\quad
\frac{}{\vdash^s\overline{n}:\mathtt{TST}}
\quad
\frac{}{\vdash^s\mathtt{nil}:\mathtt{TST}}
\]
\[
\frac{\Gamma\vdash^se^s_1:\mathtt{TST}\quad\Gamma\vdash^se^s_2:\mathtt{TST}}{\Gamma\vdash^s\mathtt{cons}\;e^s_1\;e^s_2:\mathtt{TST}}
\quad
\frac{}{\Gamma,x^s:\mathtt{TST}\vdash^sx^s:\mathtt{TST}}
\]
\[
\frac{\Gamma\vdash^se^s_1:\mathtt{TST}\quad\Gamma\vdash^se^s_2:\mathtt{TST}}{\Gamma\vdash^se^s_1\;e^s_2:\mathtt{TST}}
\quad
\frac{\Gamma\vdash^se^s:\mathtt{TST}}{\Gamma\vdash^sf\;e^s:\mathtt{TST}}
\]
\[
\frac{\Gamma\vdash^se^s_1:\mathtt{TST}\quad\Gamma\vdash^se^s_2:\mathtt{TST}}{\Gamma\vdash^so\;e^s_1\;e^s_2:\mathtt{TST}}
\quad
\frac{\Gamma\vdash^se^s:\mathtt{TST}}{\Gamma\vdash^sp\;e^s:\mathtt{TST}}
\]
\[
\frac{\Gamma\vdash^se^s_1:\mathtt{TST}\quad\Gamma\vdash^se^s_2:\mathtt{TST}\quad\Gamma\vdash^se^s_3:\mathtt{TST}}{\Gamma\vdash^s\mathtt{if0}\;e^s_1\;e^s_2\;e^s_3:\mathtt{TST}}
\quad
\frac{}{\vdash^s\mathtt{wrong}\;\mathrm{string}:\mathtt{TST}}
\]
\[
\frac{\Gamma\vdash^ht\quad\Gamma\vdash^he^h:t[t_i/t_i.x^h]}{\Gamma\vdash^s\mathtt{sh}\;t\;e^h:\mathtt{TST}}
\quad
\frac{\Gamma\vdash^mt\quad\Gamma\vdash^me^m:t[t_i/t_i.x^m]}{\Gamma\vdash^s\mathtt{sm}\;t\;e^m:\mathtt{TST}}
\]
\caption{Scheme typing rules}
\label{str}
\end{figure}

\clearpage

\begin{figure}[p]
\centering
\begin{tabular}{l}

% (\x.e) v

\redrules
{\expfapp{(\expfabsd{\varvars}{\varexps})}{\varvalus}}
{\expsubst{\varexps}{\varvalus}{\varvars}} \\

% w v

\redrules
{\expfapp{\varvalfs}{\varvalus}}
{\expwrongd{\str{Not \; a \; function}}}
$(\varvalfs \neq \expfabsd{\varvars}{\varexps})$ \\

% + n n

\redrules
{\expadd{\first{\expnum{\varnum}}}{\second{\expnum{\varnum}}}}
{\expnum{\first{\varnum} + \second{\varnum}}} \\

% - n n

\redrules
{\expsub{\first{\expnum{\varnum}}}{\second{\expnum{\varnum}}}}
{\expnum{\formvar{max}(\first{\varnum} - \second{\varnum}, 0)}} \\

% o w w

\redrules
{\expop{\first{\varvalfs}}{\second{\varvalfs}}}
{\expwrongd{\str{Not \; a \; number}}}
$(\first{\varvalfs} \neq \expnum{\varnum}$ or $\second{\varvalfs} \neq \expnum{\varnum})$ \\

% if0 0 e e

\redrules
{\expif{\expnum{0}}{\first{\varexps}}{\second{\varexps}}}
{\first{\varexps}} \\

% if0 n e e

\redrules
{\expif{\expnum{\varnum}}{\first{\varexps}}{\second{\varexps}}}
{\second{\varexps}}
$(\varnum \neq 0)$ \\

% if0 w e e

\redrules
{\expif{\varvalfs}{\first{\varexps}}{\second{\varexps}}}
{\expwrongd{\str{Not \; a \; number}}}
$(\varvalfs \neq \expnum{\varnum})$ \\

% f nil

\redrules
{\expfield{\expnild}}
{\expwrongd{\str{Empty \; list}}} \\

% hd (cons v v)

\redrules
{\exphd{(\expcons{\first{\varvalus}}{\second{\varvalus}})}}
{\first{\varvalus}} \\

% tl (cons v v)

\redrules
{\exptl{(\expcons{\first{\varvalus}}{\second{\varvalus}})}}
{\second{\varvalus}} \\

% f w

\redrules
{\expfield{\varvalfs}}
{\expwrongd{\str{Not \; a \; list}}}
$(\varvalfs \neq \expnild$ and $\varvalfs \neq \expcons{\first{\varvalus}}{\second{\varvalus}})$ \\

% fun? (\x.e)

\redrules
{\exppfun{(\expfabsd{\varvars}{\varexps})}}
{\expnum{0}} \\

% fun? w

\redrules
{\exppfun{\varvalfs}}
{\expnum{1}}
$(\varvalfs \neq \expfabsd{\varvars}{\varexps})$ \\

% list? nil

\redrules
{\expplist{\expnild}}
{\expnum{0}} \\

% list? (cons v v)

\redrules
{\expplist{(\expcons{\first{\varvalus}}{\second{\varvalus}})}}
{\expnum{0}} \\

% list? w

\redrules
{\expplist{\varvalfs}}
{\expnum{1}}
$(\varvalfs \neq \expnild$ and $\varvalfs \neq \expcons{\first{\varvalus}}{\second{\varvalus}})$ \\

% null? nil

\redrules
{\exppnull{\expnild}}
{\expnum{0}} \\

% null? w

\redrules
{\exppnull{\varvalfs}}
{\expnum{1}}
$(\varvalfs \neq \expnild)$ \\

% num? n

\redrules
{\exppnum{\expnum{\varnum}}}
{\expnum{0}} \\

% num? w

\redrules
{\exppnum{\varvalfs}}
{\expnum{1}}
$(\varvalfs \neq \expnum{\varnum})$ \\

% wrong string

\redrule
{\redcons{\expwrongd{\formvar{string}}}}
{\experr{\varstr}}

\end{tabular}
\caption{Scheme operational semantics}
\label{sos}
\end{figure}

\clearpage

\begin{figure}[p]
\centering
\begin{tabular}{l}
\vspace{5pt}

% sh - lump
$\mathscr{E}[SH^{L}$ $(^{L}HS$ $v_{S})]_{F}\rightarrow\mathscr{E}[v_{S}]$ \\

\vspace{5pt}

% sh - number
$\mathscr{E}[SH^{N}$ $\overline{n}]_{F}\rightarrow\mathscr{E}[\overline{n}]$ \\

\vspace{5pt}

% sh - list - nil
$\mathscr{E}[SH^{[T]}$ $\mathtt{nil}^{T[T_{i}/T_{i}^{a}]}]_{F}\rightarrow\mathscr{E}[\mathtt{nil}]$ \\

\vspace{5pt}

% sh - function
$\mathscr{E}[SH^{T_{1}\rightarrow T_{2}}$ $(\lambda x_{1}:T_{1}[T_{i}/T_{i}^{a}].e_{H})]_{F}\rightarrow$ \\

\vspace{5pt}

$\quad\mathscr{E}[\lambda x_{2}.(SH^{T_{2}}$ $((\lambda x_{1}:T_{1}[T_{i}/T_{i}^{a}].e_{H})$ $(^{T_{1}}HS$ $x_{2})))]$ \\

\vspace{5pt}

% sh - universal
$\mathscr{E}[SH^{\forall X.T}$ $(\Lambda X.e_{H})]_{F}\rightarrow\mathscr{E}[SH^{T[L/X]}$ $((\Lambda X.e_{H})$ $\lbrace L\rbrace)]$ \\

\vspace{5pt}

% sh - universal - hs
$\mathscr{E}[SH^{\forall X.T}$ $(^{\forall X.T}HS$ $v_{S})]_{F}\rightarrow\mathscr{E}[v_{S}]$ \\
\end{tabular}
\caption{Scheme-Haskell operational semantics}
\label{isos}
\end{figure}

\clearpage

\begin{figure}[p]
\centering
\begin{tabular}{l}
\vspace{5pt}

% sm - lump
$\mathscr{E}[SM^{L}$ $(^{L}MS$ $v_{S})]_{S}\rightarrow\mathscr{E}[v_{S}]$ \\

\vspace{5pt}

% sm - number
$\mathscr{E}[SM^{N}$ $\overline{n}]_{S}\rightarrow\mathscr{E}[\overline{n}]$ \\

\vspace{5pt}

% sm - list - nil
$\mathscr{E}[SM^{[T]}$ $\mathtt{nil}^{T[T_{i}/T_{i}^{a}]}]_{S}\rightarrow\mathscr{E}[\mathtt{nil}]$ \\

\vspace{5pt}

% sm - list - cons
$\mathscr{E}[SM^{[T]}$ $(\mathtt{cons}$ $v_{M}^{1}$ $v_{M}^{2})]_{S}\rightarrow\mathscr{E}[\mathtt{cons}$ $(SM^{T}$ $v_{M}^{1})$ $(SM^{[T]}$ $v_{M}^{2})]$ \\

\vspace{5pt}

% sm - list - mh cons
$\mathscr{E}[SM^{[T]}$ $(^{[T]}MH^{[T]}$ $(\mathtt{cons}$ $e_{H}^{1}$ $e_{H}^{2}))]_{S}\rightarrow\mathscr{E}[SH^{[T]}$ $(\mathtt{cons}$ $e_{H}^{1}$ $e_{H}^{2})]$ \\

\vspace{5pt}

% sm - function
$\mathscr{E}[SM^{T_{1}\rightarrow T_{2}}$ $(\lambda x_{1}:T_{1}[T_{i}/T_{i}^{a}].e_{M})]_{S}\rightarrow$ \\

\vspace{5pt}

$\quad\mathscr{E}[\lambda x_{2}.(SM^{T_{2}}$ $((\lambda x_{1}:T_{1}[T_{i}/T_{i}^{a}].e_{M})$ $(^{T_{1}}MS$ $x_{2})))]$ \\

\vspace{5pt}

% sm - universal
$\mathscr{E}[SM^{\forall X.T}$ $(\Lambda X.e_{M})]_{S}\rightarrow\mathscr{E}[SM^{T[L/X]}$ $((\Lambda X.e_{M})$ $\lbrace L\rbrace)]$ \\

\vspace{5pt}

% sm - universal - ms
$\mathscr{E}[SM^{\forall X.T}$ $(^{\forall X.T}MS$ $v_{S})]_{S}\rightarrow\mathscr{E}[v_{S}]$
\end{tabular}
\caption{Scheme-ML operational semantics}
\label{isos}
\end{figure}

\clearpage

\begin{figure}[p]
\centering
\begin{tabular}{rcl}

\tyunbrand{\cslump} & $=$ & \tylump \\
\tyunbrand{\csnum} & $=$ & \tynum \\
\tyunbrand{\csvarh} & $=$ & \tyvarh \\
\tyunbrand{\csvarm} & $=$ & \tyvarm \\
\tyunbrand{\cslist{\varcsh}} & $=$ & \cslist{\tyunbrand{\varcsh}} \\
\tyunbrand{\cslist{\varcsm}} & $=$ & \cslist{\tyunbrand{\varcsm}} \\
\tyunbrand{\csfun{\varcsh}{\varcsh}} & $=$ & \csfun{\tyunbrand{\varcsh}}{\tyunbrand{\varcsh}} \\
\tyunbrand{\csfun{\varcsm}{\varcsm}} & $=$ & \csfun{\tyunbrand{\varcsm}}{\tyunbrand{\varcsm}} \\
\tyunbrand{\csfor{\csvarh}{\varcsh}} & $=$ & \csfor{\csvarh}{\tyunbrand{\varcsh}} \\
\tyunbrand{\csfor{\csvarm}{\varcsm}} & $=$ & \csfor{\csvarm}{\tyunbrand{\varcsm}} \\
\tyunbrand{\csbrand{\vartyh}{\csvarh}} & $=$ & \vartyh \\
\tyunbrand{\csbrand{\vartym}{\csvarm}} & $=$ & \vartym \\

\end{tabular}
\caption{Unbrand function}
\label{unbrand}
\end{figure}

\clearpage

\begin{figure}[p]
\caption{The lump equality relation.}
\centering
\begin{tabular}{c}

$x \eq x$ \\
$x \eq y \Rightarrow y \eq x$ \\
$x \eq y$ and $y \eq z \Rightarrow x \eq z$ \\
$\vartyh \eq \tylump$ \\
$\vartym \eq \tylump$ \\
$\vartyh = \vartym \Rightarrow \vartyh \symlumpeq \vartym$ \\

\end{tabular}
\label{figequality}
\end{figure}

\clearpage
\section{Proof of Type Soundness}

The proof of correctness is similar to that of Kinghorn \cite{kinghorn07}, mutatis mutandis.

\begin{lemma}{Inversion of the Typing Relation}

\label{leminv}
The syntactic forms of well-typed expressions determine the types of their subexpressions.
\begin{proof}
Immediate from the typing rules.
\end{proof}
\end{lemma}

\begin{lemma}{Uniqueness of Types}

\label{lemuni}
If \varexph, \varexpm, and \varexps are well-typed then they have only one type.
\begin{proof}
By structural induction on \varexph, \varexpm, and \varexps and the \proinv.
\end{proof}
\end{lemma}

\begin{lemma}{Canonical Forms}

\label{lemcan}
The syntactic forms of \prouvs for each type.
\begin{proof}
Immediate from the definitions of unforced values and the typing relations.
\end{proof}
\end{lemma}

\begin{theorem}{Haskell Progress}

\label{thmpsh}
If \judeh{}{\first{\varexph}}{\vartyh} then \pshyp{\first{\varexph}}{\second{\varexph}}.
\begin{proof}
By structural induction on \first{\varexph} and theorems \ref{thmpsm} and \ref{thmpss}.
\end{proof}
\end{theorem}

\begin{theorem}{ML Progress}

\label{thmpsm}
If \judem{}{\first{\varexpm}}{\vartym} then \pshyp{\first{\varexpm}}{\second{\varexpm}}.
\begin{proof}
By structural induction on \first{\varexpm} and theorems \ref{thmpsh} and \ref{thmpss}.
\end{proof}
\end{theorem}

\begin{theorem}{Scheme Progress}

\label{thmpss}
If \judes{}{\first{\varexps}}{\tytst} then \pshyp{\first{\varexps}}{\second{\varexps}}.
\begin{proof}
By structural induction on \first{\varexps} and theorems \ref{thmpsh} and \ref{thmpsm}.
\end{proof}
\end{theorem}

\begin{lemma}{Expression Substitution Preservation}

\label{lemexp}
If \judeh{\envexte{\env}{\first{\varvarh}}{\first{\vartyh}}}{\first{\varexph}}{\second{\vartyh}} and \judeh{\env}{\second{\varexph}}{\first{\vartyh}} then \judeh{\env}{\expsubst{\first{\varexph}}{\second{\varexph}}{\first{\varvarh}}}{\second{\vartyh}}.  If \judem{\envexte{\env}{\first{\varvarm}}{\first{\vartym}}}{\first{\varexpm}}{\second{\vartym}} and \judem{\env}{\second{\varexpm}}{\first{\vartym}} then \judem{\env}{\expsubst{\first{\varexpm}}{\second{\varexpm}}{\first{\varvarm}}}{\second{\vartym}}.  If \judes{\envexte{\env}{\first{\varvars}}{\tytst}}{\first{\varexps}}{\tytst} and \judes{\env}{\second{\varexps}}{\tytst} then \judes{\env}{\expsubst{\first{\varexps}}{\second{\varexps}}{\first{\varvars}}}{\tytst}.
\begin{proof}
By structural induction.
\end{proof}
\end{lemma}

\begin{lemma}{Type Substitution Preservation}

\label{lemtyp}
If \judeh{\envextt{\env}{\first{\tyvarh}}}{\first{\varexph}}{\first{\vartyh}} and \judth{\env}{\second{\vartyh}} then \judeh{\env}{\expsubst{\first{\varexph}}{\second{\vartyh}}{\first{\tyvarh}}}{\tysubst{\first{\vartyh}}{\second{\vartyh}}{\first{\tyvarh}}}.  If \judem{\envextt{\env}{\first{\tyvarm}}}{\first{\varexpm}}{\first{\vartym}} and \judtm{\env}{\second{\vartym}} then \judem{\env}{\expsubst{\first{\varexpm}}{\second{\vartym}}{\first{\tyvarm}}}{\tysubst{\first{\vartym}}{\second{\vartym}}{\first{\tyvarm}}}.
\begin{proof}
By structural induction.
\end{proof}
\end{lemma}

\begin{lemma}{Evaluation Context Preservation}

\label{lemeva}
If \judeh{}{\first{\varexph}}{\first{\vartyh}}, \judeh{}{\second{\varexph}}{\first{\vartyh}}, and \judeh{}{\redconh{\first{\varexph}}}{\second{\vartyh}} then \judeh{}{\redconh{\second{\varexph}}}{\second{\vartyh}}.
If \judem{}{\first{\varexpm}}{\first{\vartym}}, \judem{}{\second{\varexpm}}{\first{\vartym}}, and \judem{}{\redconm{\first{\varexpm}}}{\second{\vartym}} then \judem{}{\redconm{\second{\varexpm}}}{\second{\vartym}}.
If \judes{}{\first{\varexps}}{\tytst}, \judes{}{\second{\varexps}}{\tytst}, and \judes{}{\redcons{\first{\varexps}}}{\tytst} then \judes{}{\redcons{\second{\varexps}}}{\tytst}.
\begin{proof}
By structural induction.
\end{proof}
\end{lemma}

\begin{theorem}{Haskell Preservation}

\label{thmpnh}
If \judeh{\env}{\first{\varexph}}{\first{\vartyh}} and \redruleh{\first{\varexph}}{\second{\varexph}} then \judeh{\env}{\second{\varexph}}{\first{\vartyh}}.
\begin{proof}
By cases on the reduction \redruleh{\first{\varexph}}{\second{\varexph}}, lemma \ref{lemeva}, and theorems \ref{thmpnm} and \ref{thmpns}.
\end{proof}
\end{theorem}

\begin{theorem}{ML Preservation}

\label{thmpnm}
If \judem{\env}{\first{\varexpm}}{\first{\vartym}} and \first{\varexpm} \red \second{\varexpm} then \judem{\env}{\second{\varexpm}}{\first{\vartym}}.
\begin{proof}
By cases on the reduction \redruleh{\first{\varexpm}}{\second{\varexpm}}, lemma \ref{lemeva}, and theorems \ref{thmpnh} and \ref{thmpns}.
\end{proof}
\end{theorem}

\begin{theorem}{Scheme Preservation}

\label{thmpns}
If \judes{\env}{\first{\varexps}}{\tytst} and \first{\varexps} \red \second{\varexps} then \judes{\env}{\second{\varexps}}{\tytst}.
\begin{proof}
By cases on the reduction \redrules{\first{\varexps}}{\second{\varexps}}, lemma \ref{lemeva}, and theorems \ref{thmpnh} and \ref{thmpnm}.
\end{proof}
\end{theorem}

\clearpage

\bibliography{bibliography}
\bibliographystyle{plain}

\end{document}