\begin{hcf}
\label{hcf}
\begin{enumerate}
\item If $e_{H}$ is a normal form of type $N$ then $e_{H}=\overline{n}$.
\item If $e_{H}$ is a normal form of type $T_{1}\rightarrow T_{2}$ then $e_{H}=\lambda x:T_{1}.e_{H}$.
\item If $e_{H}$ is a normal form of type $\forall X.T$ then $e_{H}=\Lambda X.e_{H}$ or $e_{H}=\;^{\forall X.T}HS\;v_{S}$.
\item If $e_{H}$ is a normal form of type $[T]$ then $e_{H}=\mathtt{cons}\;e_{H}^{1}\;e_{H}^{2}$.
\item If $e_{H}$ is a normal form of type $L$ then $e_{H}=\;^{L}HS\;v_{S}$.
\end{enumerate}
\begin{proof}
[???] Straightforward from the definition of Haskell normal forms and the definition of the Haskell typing relation.
%For part (1), according to the typing rules in Figure \ref{mctr}, the only normal form of type $N$ is $\overline{n}$.  For part (2), 
\end{proof}
\end{hcf}