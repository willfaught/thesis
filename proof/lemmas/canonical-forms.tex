The types of Haskell and ML values determine their syntactic forms.

\begin{lemma}{Canonical Forms}

\label{lemcan}

The syntactic forms of \prouvs for each type.

\begin{enumerate}

% Haskell

% L

\item If \judeh{\env}{\varvaluh}{\tylump} then $\varvaluh \in \lbrace \exphm{\tylump}{\vartym}{\varvalfm}, \exphs{\cslump}{\varvalfs} \rbrace$.

% N

\item If \judeh{\env}{\varvaluh}{\tynum} then $\varvaluh = \expnum{\symnum}$.

% {t}

\item If \judeh{\env}{\varvaluh}{\tylist{\vartyh}} then $\varvaluh \in \lbrace \expnils{\vartyh}, \expcons{\first{\varexph}}{\second{\varexph}} \rbrace$.

% t->t

\item If \judeh{\env}{\varvaluh}{\tyfun{\first{\vartyh}}{\second{\vartyh}}} then $\varvaluh = \expfabss{\varvarh}{\first{\vartyh}}{\varexph}$.

% Au.t

\item If \judeh{\env}{\varvaluh}{\tyfor{\tyvarh}{\vartyh}} then $\varvaluh = \exptabs{\tyvarh}{\varexph}$.

% ML

% L

\item If \judem{\env}{\varvalfm}{\tylump} then $\varvalfm \in \lbrace \expmh{\tylump}{\vartym}{\varvalfm}, \expms{\cslump}{\varvalfs} \rbrace$.

% N

\item If \judem{\env}{\varvalfm}{\tynum} then $\varvalfm = \expnum{\symnum}$.

% {t}

\item If \judem{\env}{\varvalfm}{\tylist{\vartym}} then $\varvalfm \in \lbrace \expnils{\vartym}, \expcons{\first{\varvalum}}{\second{\varvalum}} \rbrace$.

% t->t

\item If \judem{\env}{\varvalfm}{\tyfun{\first{\vartym}}{\second{\vartym}}} then $\varvalfm = \expfabss{\varvarh}{\first{\vartym}}{\varexph}$.

% Au.t

\item If \judem{\env}{\varvalfm}{\tyfor{\tyvarm}{\vartym}} then $\varvalfm = \exptabs{\tyvarm}{\varexph}$.

\end{enumerate}

\begin{proof}

Immediate from the definitions of unforced values and the typing relations.

\end{proof}

\end{lemma}
