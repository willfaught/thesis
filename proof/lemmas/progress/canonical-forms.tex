\begin{lemma}
\label{cf}
\onehalfspacing
The possible syntactic forms of values of various types.
\begin{enumerate}
\item If $v_{A}:N$ then $v_{A}=\overline{n}$ where $A\in\lbrace H,M\rbrace$.
\item If $v_{A}:T_{1}\rightarrow T_{2}$ then $v_{A}=\lambda x:T_{1}.e_{A}$ where $A\in\lbrace H,M\rbrace$.
\item If $v_{H}:\forall X.T$ then $v_{H}=\Lambda X.e_{H}$ or $v_{H}=\,^{\forall X.T}HS\;v_{S}$.
\item If $v_{M}:\forall X.T$ then $v_{M}=\Lambda X.e_{M}$ or $v_{M}=\,^{\forall X.T}MS\;v_{S}$.
\item If $v_{H}:[T]$ then $v_{H}=\mathtt{cons}\;e_{H}^{1}\;e_{H}^{2}$ or $v_{H}=\mathtt{nil}^{T}$.
\item If $v_{M}:[T]$ then $v_{M}=\mathtt{cons}\;v_{M}^{1}\;v_{M}^{2}$ or $v_{M}=\mathtt{nil}^{T}$ or $v_{M}=\,^{[T]}MH^{[T]}\;v_{H}$.
\item If $v_{H}:L$ then $v_{H}=\,^{L}HS\;v_{S}$.
\item If $v_{M}:L$ then $v_{M}=\,^{L}MS\;v_{S}$.
\end{enumerate}
\begin{proof}
Immediate from the definitions of values and the typing relations.
\end{proof}
\end{lemma}