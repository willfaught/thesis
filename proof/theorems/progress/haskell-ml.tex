\newcommand{\pscases}[4]{If #1 is a \profv then #2 and #3 determine the reduction of #4.}

\newcommand{\pserr}[2]{If #1 \red \emph{\experr{\varstr}} then #2 \red \emph{\experr{\varstr}}.}
\newcommand{\pserrand}[3]{If #1 \red \emph{\experr{\varstr}} and #2 is a \profv then #3 \red \emph{\experr{\varstr}}.}

\newcommand{\pshyp}[2]{#1 is a \profv or #1 \red #2 or #1 \red \emph{\experr{\varstr}}}
\newcommand{\pshypby}[2]{\pshyp{#1}{#2} by the induction hypothesis.}

\newcommand{\pssub}[4]{If #1 \red #2 then #3 \red #4.}
\newcommand{\pssuband}[5]{If #1 \red #2 and #3 is a \profv then #4 \red #5.}

\newcommand{\psred}[2]{\redrule{#1}{#2}.}
\newcommand{\psrednote}[3]{\redrule{#1}{#2} $(#3)$.}

\newcommand{\psval}[2]{#1 by lemmas \ref{leminv} and \ref{lemuni} and #2 by lemma \ref{lemcan}.}
\newcommand{\psvalh}[3]{\psval{\judeh{}{#1}{#2}}{#3}}
\newcommand{\psvalm}[3]{\psval{\judem{}{#1}{#2}}{#3}}
\newcommand{\psvaleqh}[3]{\psvalh{#1}{#2}{#1 $=$ #3}}
\newcommand{\psvaleqm}[3]{\psvalm{#1}{#2}{#1 $=$ #3}}
\newcommand{\psvalinh}[3]{\psvalh{#1}{#2}{$#1 \in \lbrace #3 \rbrace$}}
\newcommand{\psvalinm}[3]{\psvalm{#1}{#2}{$#1 \in \lbrace #3 \rbrace$}}
\newcommand{\psvalif}[2]{If #1 is a \profv then #2}
\newcommand{\psvalifeqh}[3]{\psvalif{#1}{\psvaleqh{#1}{#2}{#3}}}
\newcommand{\psvalifeqm}[3]{\psvalif{#1}{\psvaleqm{#1}{#2}{#3}}}
\newcommand{\psvalifinh}[3]{\psvalif{#1}{\psvalinh{#1}{#2}{#3}}}
\newcommand{\psvalifinm}[3]{\psvalif{#1}{\psvalinm{#1}{#2}{#3}}}

\begin{theorem}{Haskell Progress}

\label{thmhps}

If \judeh{}{\varexph}{\vartyh} then \pshyp{\first{\varexph}}{\second{\varexph}}.

\begin{proof}

By structural induction on \varexph.

% Haskell

% \x:t.e

\newcommand{\psfabss}{\expfabss{\varvarh}{\vartyh}{\varexph}\xspace}

\begin{case}{\psfabss}

\psfabss is a \profv.

\end{case}

% \\u.e

\newcommand{\pstabs}{\exptabs{\tyvarh}{\varexph}\xspace}

\begin{case}{\pstabs}

\pstabs is a \profv.

\end{case}

% n

\newcommand{\psnum}{\expnum{\symnum}\xspace}

\begin{case}{\psnum}

\psnum is a \profv.

\end{case}

% nil t

\newcommand{\psnils}{\expnils{\vartyh}\xspace}

\begin{case}{\psnils}

\psnils is a \profv.

\end{case}

% cons e e

\newcommand{\psconsh}{\expcons{\first{\varexph}}{\second{\varexph}}\xspace}

\begin{case}{\psconsh}

\psconsh is a \profv.

\end{case}

% x

\newcommand{\psvar}{\varvarh\xspace}

\begin{case}{\psvar}

Cannot occur because \varexph is closed.

\end{case}

% e e

\newcommand{\psfapp}{\expfapp{\first{\varexph}}{\second{\varexph}}\xspace}
\newcommand{\x}{\expfabss{\varvarh}{\first{\vartyh}}{\third{\varexph}}\xspace}

\begin{case}{\psfapp}

\pshypby
{\first{\varexph}}
{\third{\varexph}}
\psvalifeqh
{\first{\varexph}}
{\tyfun{\first{\vartyh}}{\second{\vartyh}}}
{\x}
\psred
{\expfapp{(\x)}{\second{\varexph}}}
{\expsubst{\third{\varexph}}{\second{\varexph}}{\varvarh}}
\pssub
{\first{\varexph}}
{\third{\varexph}}
{\psfapp}
{\expfapp{\third{\varexph}}{\second{\varexph}}}
\pserr
{\first{\varexph}}
{\psfapp}

\end{case}

% fix e

\newcommand{\psfix}{\expfix{\first{\varexph}}\xspace}
\renewcommand{\x}{\expfabss{\varvarh}{\vartyh}{\second{\varexph}}\xspace}
\newcommand{\y}{\expfix{(\x)}}

\begin{case}{\psfix}

\pshypby
{\first{\varexph}}
{\second{\varexph}}
\psvalifeqh
{\first{\varexph}}
{\tyfun{\vartyh}{\vartyh}}
{\x}
\psred
{\y}
{\expsubst{\second{\varexph}}{\y}{\varvarh}}
\pssub
{\first{\varexph}}
{\second{\varexph}}
{\psfix}
{\expfix{\second{\varexph}}}
\pserr
{\first{\varexph}}
{\psfix}

\end{case}

% e<t>

\newcommand{\pstapp}{\exptapp{\first{\varexph}}{\first{\vartyh}}\xspace}
\renewcommand{\x}{\exptabs{\tyvarh}{\second{\varexph}}\xspace}

\begin{case}{\pstapp}

\pshypby
{\first{\varexph}}
{\second{\varexph}}
\psvalifeqh
{\first{\varexph}}
{\tyfor{\tyvarh}{\second{\vartyh}}}
{\x}
\psred
{\exptapp{(\x)}{\first{\vartyh}}}
{\expsubst{\second{\varexph}}{\first{\vartyh}}{\tyvarh}}
\pssub
{\first{\varexph}}
{\second{\varexph}}
{\pstapp}
{\exptapp{\second{\varexph}}{\first{\vartyh}}}
\pserr
{\first{\varexph}}
{\pstapp}

\end{case}

% f e

\newcommand{\psfield}{\expfield{\first{\varexph}}\xspace}
\renewcommand{\x}{\expnils{\vartyh}\xspace}
\renewcommand{\y}{\expcons{\second{\varexph}}{\third{\varexph}}\xspace}

\begin{case}{\psfield}

\pshypby
{\first{\varexph}}
{\second{\varexph}}
\psvalifinh
{\first{\varexph}}
{\tylist{\vartyh}}
{\x, \y}
\psred
{\exphd{(\x)}}
{\expwrongs{\vartyh}{\errempty}}
\psred
{\exptl{(\x)}}
{\expwrongs{\tylist{\vartyh}}{\errempty}}
\psred
{\exphd{(\y)}}
{\second{\varexph}}
\psred
{\exptl{(\y)}}
{\third{\varexph}}
\pssub
{\first{\varexph}}
{\second{\varexph}}
{\psfield}
{\expfield{\second{\varexph}}}
\pserr
{\first{\varexph}}
{\psfield}

\end{case}

% o e e

\newcommand{\psop}{\expop{\first{\varexph}}{\second{\varexph}}\xspace}
\renewcommand{\x}{\first{\expnum{\varnum}}\xspace}
\renewcommand{\y}{\second{\expnum{\varnum}}\xspace}

\begin{case}{\psop}

\pshypby
{\first{\varexph}}
{\third{\varexph}}
\psvalifeqh
{\first{\varexph}}
{\tynum}
{\x}
\pssub
{\first{\varexph}}
{\third{\varexph}}
{\psop}
{\expop{\third{\varexph}}{\second{\varexph}}}
\pserr
{\first{\varexph}}
{\psop}
\pshypby
{\second{\varexph}}
{\third{\varexph}}
\psvalifeqh
{\second{\varexph}}
{\tynum}
{\y}
\pssuband
{\second{\varexph}}
{\third{\varexph}}
{\first{\varexph}}
{\psop}
{\expop{\first{\varexph}}{\third{\varexph}}}
\pserrand
{\second{\varexph}}
{\first{\varexph}}
{\psop}
\psred
{\expadd{\x}{\y}}
{\expnum{\first{\varnum} + \second{\varnum}}}
\psred
{\expsub{\x}{\y}}
{\expnum{\formvar{max}(\first{\varnum} - \second{\varnum}, 0)}}

\end{case}

% null? e

\newcommand{\pspnull}{\exppnull{\first{\varexph}}\xspace}
\renewcommand{\x}{\expnils{\vartyh}\xspace}
\renewcommand{\y}{\expcons{\second{\varexph}}{\third{\varexph}}}

\begin{case}{\pspnull}

\pshypby
{\first{\varexph}}
{\second{\varexph}}
\psvalifinh
{\first{\varexph}}
{\tylist{\vartyh}}
{\x, \y}
\psred
{\exppnull{(\x)}}
{\expnum{0}}
\psred
{\exppnull{(\y)}}
{\expnum{1}}
\pssub
{\first{\varexph}}
{\second{\varexph}}
{\pspnull}
{\exppnull{\second{\varexph}}}
\pserr
{\first{\varexph}}
{\pspnull}

\end{case}

% if0 e e e

\newcommand{\psif}{\expif{\first{\varexph}}{\second{\varexph}}{\third{\varexph}}\xspace}
\renewcommand{\x}{\expnum{\varnum}\xspace}

\begin{case}{\psif}

\pshypby
{\first{\varexph}}
{\fourth{\varexph}}
\psvalifeqh
{\first{\varexph}}
{\tynum}
{\x}
\psred
{\expif{\expnum{0}}{\second{\varexph}}{\third{\varexph}}}
{\second{\varexph}}
\psrednote
{\expif{\x}{\second{\varexph}}{\third{\varexph}}}
{\third{\varexph}}
{n \neq 0}
\pssub
{\first{\varexph}}
{\fourth{\varexph}}
{\psif}
{\expif{\fourth{\varexph}}{\second{\varexph}}{\third{\varexph}}}
\pserr
{\first{\varexph}}
{\psif}

\end{case}

% wrong t string

\newcommand{\pswrongs}{\expwrongs{\vartyh}{\varstr}\xspace}

\begin{case}{\pswrongs}

\psred
{\pswrongs}
{\emph{\experr{\varstr}}}

\end{case}

% hm t t e

\newcommand{\pshm}{\exphm{\first{\vartyh}}{\first{\vartym}}{\first{\varexpm}}}

\begin{case}{\pshm}

\pshypby
{\first{\varexpm}}
{\second{\varexpm}}
\pscases
{\first{\varexpm}}
{\first{\vartyh}}
{\first{\vartym}}
{\pshm}

\begin{subcase}{\first{\vartyh} $=$ \tylump}

\exphm{\tylump}{\first{\vartym}}{\first{\varexpm}} is a \profv.

\end{subcase}

\begin{subcase}{\first{\vartyh} $\neq$ \tylump and \first{\vartym} $=$ \tylump}

\psvalinh
{\first{\varexpm}}
{\tylump}
{\expmh{\tylump}{\second{\vartyh}}{\varexph}, \expms{\cslump}{\varvalfs}}
\psrednote
{\exphm{\first{\vartyh}}{\tylump}{(\expmh{\tylump}{\second{\vartyh}}{\varexph})}}
{\varexph}
{\first{\vartyh} = \second{\vartyh}}
\psrednote
{\exphm{\first{\vartyh}}{\tylump}{(\expmh{\tylump}{\second{\vartyh}}{\varexph})}}
{\varexph}
{\first{\vartyh} \neq \second{\vartyh}}
\psred
{\exphm{\first{\vartyh}}{\tylump}{(\expms{\cslump}{\varvalfs})}}
{\expwrongs{\first{\vartyh}}{\errvalue}}

\end{subcase}

\begin{subcase}{\first{\vartyh} $=$ \tynum and \first{\vartym} $=$ \tynum}

\psvaleqm
{\first{\varexpm}}
{\tynum}
{\expnum{\varnum}}
\psred
{\exphm{\tynum}{\tynum}{\expnum{\varnum}}}
{\expnum{\varnum}}

\end{subcase}

\begin{subcase}{\first{\vartyh} $=$ \tylist{\second{\vartyh}} and \first{\vartym} $=$ \tylist{\second\vartym}}

\renewcommand{\x}{\expnils{\third{\vartym}}\xspace}
\renewcommand{\y}{\expcons{\first{\varvalum}}{\second{\varvalum}}\xspace}

\psvalinm
{\first{\varexpm}}
{\tylist{\third{\vartym}}}
{\x, \y}
\psred
{\exphm{\tylist{\second{\vartyh}}}{\tylist{\second{\vartym}}}{(\x)}}
{\expnils{\second{\vartyh}}}
\psred
{\exphm{\tylist{\second{\vartyh}}}{\tylist{\second{\vartym}}}{(\y)}}
{\expcons{(\exphm{\second{\vartyh}}{\second{\vartym}}{\first{\varvalum}})}{(\exphm{\tylist{\second{\vartyh}}}{\tylist{\second{\vartym}}}{\second{\varvalum}})}}

\end{subcase}

\begin{subcase}{\first{\vartyh} $=$ \tyfun{\second{\vartyh}}{\third{\vartyh}} and \first{\vartym} $=$ \tyfun{\second{\vartym}}{\third{\vartym}}}

%TODO:

$T=T_{1}\rightarrow T_{2}$

$e_{M}^{1}=\lambda x_{1}:T_{1}.e_{M}^{3}$ by canonical forms (Lemma \ref{cf}).  $^{T_{1}\rightarrow T_{2}}HM$ $(\lambda x_{1}:T_{1}.e_{M}^{3})\rightarrow\lambda x_{2}:T_{1}.(^{T_{2}}HM$ $((\lambda x_{1}:T_{1}.e_{M}^{3})$ $(^{T_{1}}MH$ $x_{2})))$.

\end{subcase}

\begin{subcase}

$T=\forall X.T_{1}$

$e_{M}^{1}\in\lbrace\Lambda X.e_{M}^{3},{^{\forall X.T_{1}}M}S$ $v_{S}\rbrace$ by canonical forms (Lemma \ref{cf}).  $^{\forall X.T_{1}}HM$ $(\Lambda X.e_{M}^{3})\rightarrow\Lambda X.(^{T_{1}}HM$ $e_{M}^{3})$.  $^{\forall X.T_{1}}HM$ $(^{\forall X.T_{1}}MS$ $v_{S})\rightarrow{^{\forall X.T_{1}}H}S$ $v_{S}$.

\end{subcase}

\pssub
{\first{\varexpm}}
{\second{\varexpm}}
{\pshm}
{\exphm{\first{\vartyh}}{\first{\vartym}}{\second{\varexpm}}}
\pserr
{\first{\varexpm}}
{\pshm}

\end{case}

% mh t t e

\begin{case}

$e_{M}={^{T}M}H$ $e_{H}$

$^{T}MH$ $e_{H}$ is an unforced value.

\end{case}

% hs k e

\begin{case}

$e_{A}={^{T}A}S$ $e_{S}^{1}$ where $A\in\lbrace H,M\rbrace$

$e_{S}^{1}$ is an unforced value or $e_{S}^{1}\rightarrow e_{S}^{2}$ or $e_{S}^{1}\rightarrow$ \emph{\textbf{Error}: string} by Scheme progress (Theorem \ref{sps}).  If $e_{S}^{1}\rightarrow e_{S}^{2}$ then $^{T}AS$ $e_{S}^{1}\rightarrow{^{T}A}S$ $e_{S}^{2}$.  If $e_{S}^{1}\rightarrow$ \emph{\textbf{Error}: string} then $^{T}AS$ $e_{S}^{1}\rightarrow$ \emph{\textbf{Error}: string}.  If $e_{S}^{1}$ is an unforced value then $T$ determines the reduction of $^{T}AS$ $e_{S}^{1}$:

\begin{subcase}

$T=L$

$^{L}AS$ $e_{S}^{1}$ is an unforced value.

\end{subcase}

\begin{subcase}

$T=N$

$^{N}AS$ $\overline{n}\rightarrow\overline{n}$.  $^{N}AS$ $e_{S}^{1}\rightarrow\mathtt{wrong}^{N}$ \emph{``Not a number"} $(e_{S}^{1}\neq\overline{n})$.

\end{subcase}

\begin{subcase}

$T=[T_{1}]$

$^{[T_{1}]}AS$ $\mathtt{nil}\rightarrow\mathtt{nil}^{T_{1}}$.  $^{[T_{1}]}AS$ $(\mathtt{cons}$ $v_{S}^{1}$ $v_{S}^{2})\rightarrow\mathtt{cons}$ $(^{T_{1}}AS$ $v_{S}^{1})$ $(^{[T_{1}]}AS$ $v_{S}^{2})$.  $^{[T_{1}]}HS$ $(SH^{[T_{1}]}$ $(\mathtt{cons}$ $e_{H}^{1}$ $e_{H}^{2}))\rightarrow\mathtt{cons}$ $e_{H}^{1}$ $e_{H}^{2}$.  $^{[T_{1}]}MS$ $(SH^{[T_{1}]}$ $(\mathtt{cons}$ $e_{H}^{1}$ $e_{H}^{2}))\rightarrow{^{[T_{1}]}M}H^{[T_{1}]}$ $(\mathtt{cons}$ $e_{H}^{1}$ $e_{H}^{2})$.  $^{[T_{1}]}AS$ $e_{S}^{1}\rightarrow\mathtt{wrong}^{[T_{1}[T_{i}/T_{i}^{a}]]}$ \emph{``Not a list"} $(e_{S}^{1}\not\in\lbrace\mathtt{nil},\mathtt{cons}$ $v_{S}^{1}$ $v_{S}^{2},SH^{[T_{1}]}$ $(\mathtt{cons}$ $e_{H}^{1}$ $e_{H}^{2})\rbrace)$.

\end{subcase}

\begin{subcase}

$T=T_{1}^{a}$

$^{T_{1}^{a}}HS$ $(SH^{T_{1}^{a}}$ $e_{H})\rightarrow e_{H}$.  $^{T_{1}^{a}}HS$ $e_{S}^{1}\rightarrow\mathtt{wrong}^{T_{1}}$ \emph{``Parametricity violated"} $(e_{S}^{1}\neq SH^{T_{1}^{a}}$ $e_{H})$.  $^{T_{1}^{a}}MS$ $(SM^{T_{1}^{a}}$ $v_{M})\rightarrow v_{M}$.  $^{T_{1}^{a}}MS$ $e_{S}^{1}\rightarrow\mathtt{wrong}^{T_{1}}$ \emph{``Parametricity violated"} $(e_{S}^{1}\neq SM^{T_{1}^{a}}$ $v_{M})$.

\end{subcase}

\begin{subcase}

$T=T_{1}\rightarrow T_{2}$

$^{T_{1}\rightarrow T_{2}}AS$ $(\lambda x_{1}.e_{S}^{3})\rightarrow\lambda x_{2}:T_{1}[T_{i}/T^{a}_{i}].(^{T_{2}}AS$ $((\lambda x_{1}.e_{S}^{3})$ $(SA^{T_{1}}$ $x_{2})))$.  $^{T_{1}\rightarrow T_{2}}AS$ $e_{S}^{1}\rightarrow\mathtt{wrong}^{(T_{1}\rightarrow T_{2})[T_{i}/T_{i}^{a}]]}$ \emph{``Not a function"} $(e_{S}^{1}\neq\lambda x_{1}.e_{S}^{3})$.

\end{subcase}

\begin{subcase}

$T=\forall X.T_{1}$

$^{\forall X.T_{1}}AS$ $e_{S}^{1}$ is an unforced value.

\end{subcase}

\end{case}

% ML

% cons v v

\newcommand{\psconsm}{\expcons{\first{\varvalum}}{\second{\varvalum}}\xspace}

\begin{case}

\psconsm

\psconsm is a \profv.

\end{case}

% ML e e

\begin{case}

$e_{M}=e_{M}^{1}$ $e_{M}^{2}$

$e_{M}^{1}$ is an unforced value or $e_{M}^{1}\rightarrow e_{M}^{3}$ or $e_{M}^{1}\rightarrow$ \emph{\textbf{Error}: string} by the induction hypothesis.  If $e_{M}^{1}$ is an unforced value then $e_{M}^{1}:T_{1}\rightarrow T_{2}$ by inversion (Lemma \ref{i}) and uniqueness of types (Lemma \ref{uot}) and $e_{M}^{1}=\lambda x:T_{1}.e_{M}^{4}$ by canonical forms (Lemma \ref{cf}).  If $e_{M}^{1}\rightarrow e_{M}^{3}$ then $e_{M}^{1}$ $e_{M}^{2}\rightarrow e_{M}^{3}$ $e_{M}^{2}$.  If $e_{M}^{1}\rightarrow$ \emph{\textbf{Error}: string} then $e_{M}^{1}$ $e_{M}^{2}\rightarrow$ \emph{\textbf{Error}: string}.  $e_{M}^{2}$ is an unforced value or $e_{M}^{2}\rightarrow e_{M}^{5}$ or $e_{M}^{2}\rightarrow$ \emph{\textbf{Error}: string} by the induction hypothesis.  If $e_{M}^{2}\rightarrow e_{M}^{5}$ and $e_{M}^{1}$ is an unforced value then $e_{M}^{1}$ $e_{M}^{2}\rightarrow e_{M}^{1}$ $e_{M}^{5}$.  If $e_{M}^{2}\rightarrow$ \emph{\textbf{Error}: string} and $e_{M}^{1}$ is an unforced value then $e_{M}^{1}$ $e_{M}^{2}\rightarrow$ \emph{\textbf{Error}: string}.  If $e_{M}^{1}$ is an unforced value and $e_{M}^{2}$ is an unforced value then $(\lambda x:T_{1}.e_{M}^{4})$ $e_{M}^{2}\rightarrow e_{M}^{4}[e_{M}^{2}/x]$.

\end{case}

% ML cons e e

\begin{case}

$e_{M}=\mathtt{cons}$ $e_{M}^{1}$ $e_{M}^{2}$

$e_{M}^{1}$ is an unforced value or $e_{M}^{1}\rightarrow e_{M}^{3}$ or $e_{M}^{1}\rightarrow$ \emph{\textbf{Error}: string} by the induction hypothesis.  If $e_{M}^{1}\rightarrow e_{M}^{3}$ then $\mathtt{cons}$ $e_{M}^{1}$ $e_{M}^{2}\rightarrow\mathtt{cons}$ $e_{M}^{3}$ $e_{M}^{2}$.  If $e_{M}^{1}\rightarrow$ \emph{\textbf{Error}: string} then $\mathtt{cons}$ $e_{M}^{1}$ $e_{M}^{2}\rightarrow$ \emph{\textbf{Error}: string}.  $e_{M}^{2}$ is an unforced value or $e_{M}^{2}\rightarrow e_{M}^{4}$ or $e_{M}^{2}\rightarrow$ \emph{\textbf{Error}: string} by the induction hypothesis.  If $e_{M}^{2}\rightarrow e_{M}^{4}$ and $e_{M}^{1}$ is an unforced value then $\mathtt{cons}$ $e_{M}^{1}$ $e_{M}^{2}\rightarrow\mathtt{cons}$ $e_{M}^{1}$ $e_{M}^{4}$.  If $e_{M}^{2}\rightarrow$ \emph{\textbf{Error}: string} and $e_{M}^{1}$ is an unforced value then $\mathtt{cons}$ $e_{M}^{1}$ $e_{M}^{2}\rightarrow$ \emph{\textbf{Error}: string}.  If $e_{M}^{1}$ and $e_{M}^{2}$ are unforced values then $\mathtt{cons}$ $e_{M}^{1}$ $e_{M}^{2}$ is an unforced value.

\end{case}

% ML f e

\begin{case}

$e_{M}=f$ $e_{M}^{1}$

$e_{M}^{1}$ is an unforced value or $e_{M}^{1}\rightarrow e_{M}^{2}$ or $e_{M}^{1}\rightarrow$ \emph{\textbf{Error}: string} by the induction hypothesis.  If $e_{M}^{1}$ is an unforced value then $e_{M}^{1}:[T]$ by inversion (Lemma \ref{i}) and uniqueness of types (Lemma \ref{uot}) and $e_{M}^{1}\in\lbrace\mathtt{nil}^{T},\mathtt{cons}$ $v_{M}^{1}$ $v_{M}^{2},{^{[T]}M}H^{[T]}$ $(\mathtt{cons}$ $e_{H}^{1}$ $e_{H}^{2})\rbrace$ by canonical forms (Lemma \ref{cf}).  $\mathtt{hd}$ $\mathtt{nil}^{T}\rightarrow\mathtt{wrong}^{T}$ \emph{``Empty list"}.  $\mathtt{tl}$ $\mathtt{nil}^{T}\rightarrow\mathtt{wrong}^{[T]}$ \emph{``Empty list"}.  $\mathtt{hd}$ $(\mathtt{cons}$ $v_{M}^{1}$ $v_{M}^{2})\rightarrow v_{M}^{1}$.  $\mathtt{tl}$ $(\mathtt{cons}$ $v_{M}^{1}$ $v_{M}^{2})\rightarrow v_{M}^{2}$.  $\mathtt{hd}$ $({^{[T]}M}H^{[T]}$ $(\mathtt{cons}$ $e_{H}^{1}$ $e_{H}^{2}))\rightarrow{^{T}M}H^{T}$ $e_{H}^{1}$.  $\mathtt{tl}$ $({^{[T]}M}H^{[T]}$ $(\mathtt{cons}$ $e_{H}^{1}$ $e_{H}^{2}))\rightarrow{^{[T]}M}H^{[T]}$ $e_{H}^{2}$.  If $e_{M}^{1}\rightarrow e_{M}^{2}$ then $f$ $e_{M}^{1}\rightarrow f$ $e_{M}^{2}$.  If $e_{M}^{1}\rightarrow$ \emph{\textbf{Error}: string} then $f$ $e_{M}^{1}\rightarrow$ \emph{\textbf{Error}: string}.

\end{case}

% ML null? e

\begin{case}

$e_{M}=\mathtt{null?}$ $e_{M}^{1}$

$e_{M}^{1}$ is an unforced value or $e_{M}^{1}\rightarrow e_{M}^{2}$ or $e_{M}^{1}\rightarrow$ \emph{\textbf{Error}: string} by the induction hypothesis.  If $e_{M}^{1}$ is an unforced value then $e_{M}^{1}:[T]$ by inversion (Lemma \ref{i}) and uniqueness of types (Lemma \ref{uot}) and $e_{M}^{1}\in\lbrace\mathtt{nil}^{T},\mathtt{cons}$ $v_{M}^{1}$ $v_{M}^{2},{^{[T]}M}H^{[T]}$ $(\mathtt{cons}$ $e_{H}^{1}$ $e_{H}^{2})\rbrace$ by canonical forms (Lemma \ref{cf}).  $\mathtt{null?}$ $\mathtt{nil}^{T}\rightarrow\overline{0}$.  If $e_{M}^{1}\in\lbrace\mathtt{cons}$ $v_{M}^{1}$ $v_{M}^{2},{^{[T]}M}H^{[T]}$ $(\mathtt{cons}$ $e_{H}^{1}$ $e_{H}^{2})\rbrace$ then $\mathtt{null?}$ $e_{M}^{1}\rightarrow\overline{1}$.  If $e_{M}^{1}\rightarrow e_{M}^{2}$ then $\mathtt{null?}$ $e_{M}^{1}\rightarrow\mathtt{null?}$ $e_{M}^{2}$.  If $e_{M}^{1}\rightarrow$ \emph{\textbf{Error}: string} then $\mathtt{null?}$ $e_{M}^{1}\rightarrow$ \emph{\textbf{Error}: string}.

\end{case}

\end{proof}

\end{theorem}
