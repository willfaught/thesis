%\newcommand{\}{}
\newcommand{\pshyp}[2]{#1 is a \profv or #1 \red #2 or #1 \red \emph{\experr{\varstr}}\xspace}
\newcommand{\pshypref}[2]{\pshyp{#1}{#2} by the induction hypothesis.\xspace}
\newcommand{\psval}[3]{If #1 is a \profv then \judeh{}{#1}{#2} by lemmas \ref{leminv} and \ref{lemuni} and #1 $=$ #3 by lemma \ref{lemcan}.\xspace}
\newcommand{\pssub}[4]{If #1 \red #2 then #3 \red #4.\xspace}
\newcommand{\psred}[2]{\redrule{#1}{#2}.\xspace}
\newcommand{\pserr}[2]{If #1 \red \emph{\experr{\varstr}} then #2 \red \emph{\experr{\varstr}}.\xspace}

\begin{theorem}{Haskell Progress}

\label{thmhps}

If \judeh{}{\varexph}{\vartyh} then \pshyp{\first{\varexph}}{\second{\varexph}}.

\begin{proof}

By structural induction on \varexph.

% Haskell

% \x:t.e

\newcommand{\psfabss}{\expfabss{\varvarh}{\vartyh}{\varexph}\xspace}

\begin{case}

\psfabss

\psfabss is a \profv.

\end{case}

% \\u.e

\newcommand{\pstabs}{\exptabs{\tyvarh}{\varexph}\xspace}

\begin{case}

\pstabs

\pstabs is a \profv.

\end{case}

% n

\newcommand{\psnum}{\expnum{\symnum}\xspace}

\begin{case}

\psnum

\psnum is a \profv.

\end{case}

% nil t

\newcommand{\psnils}{\expnils{\vartyh}\xspace}

\begin{case}

\psnils

\psnils is a \profv.

\end{case}

% cons e e

\newcommand{\psconsh}{\expcons{\first{\varexph}}{\second{\varexph}}\xspace}

\begin{case}

\psconsh

\psconsh is a \profv.

\end{case}

% x

\newcommand{\psvar}{\varvarh\xspace}

\begin{case}

\psvar

Cannot occur because \varexph is closed.

\end{case}

% e e

\newcommand{\psfapp}{\expfapp{\first{\varexph}}{\second{\varexph}}\xspace}
\newcommand{\x}{\expfabss{\varvarh}{\first{\vartyh}}{\third{\varexph}}\xspace}

\begin{case}

\psfapp

\pshypref
{\first{\varexph}}
{\third{\varexph}}
\psval
{\first{\varexph}}
{\tyfun{\first{\vartyh}}{\second{\vartyh}}}
{\x}
\psred
{\expfapp{(\x)}{\second{\varexph}}}
{\expsubst{\third{\varexph}}{\second{\varexph}}{\varvarh}}
\pssub
{\first{\varexph}}
{\third{\varexph}}
{\psfapp}
{\expfapp{\third{\varexph}}{\second{\varexph}}}
\pserr
{\first{\varexph}}
{\psfapp}

\end{case}

% fix e

\begin{case}

$e_{A}=\mathtt{fix}$ $e_{A}^{1}$ where $A\in\lbrace H,M\rbrace$

$e_{A}^{1}$ is an unforced value or $e_{A}^{1}\rightarrow e_{A}^{2}$ or $e_{A}^{1}\rightarrow$ \emph{\textbf{Error}: string} by the induction hypothesis.  If $e_{A}^{1}$ is an unforced value then $e_{A}^{1}:T\rightarrow T$ by inversion (Lemma \ref{i}) and uniqueness of types (Lemma \ref{uot}) and $e_{A}^{1}=\lambda x:T.e_{A}^{3}$ by canonical forms (Lemma \ref{cf}).  $\mathtt{fix}$ $(\lambda x:T.e_{A}^{3})\rightarrow e_{A}^{3}[\mathtt{fix}$ $(\lambda x:T.e_{A}^{3})/x]$.  If $e_{A}^{1}\rightarrow e_{A}^{2}$ then $\mathtt{fix}$ $e_{A}^{1}\rightarrow\mathtt{fix}$ $e_{A}^{2}$.  If $e_{A}^{1}\rightarrow$ \emph{\textbf{Error}: string} then $\mathtt{fix}$ $e_{A}^{1}\rightarrow$ \emph{\textbf{Error}: string}.

\end{case}

% e<t>

\begin{case}

$e_{A}=e_{A}^{1}$ $\lbrace T_{1}\rbrace$ where $A\in\lbrace H,M\rbrace$

$e_{A}^{1}$ is an unforced value or $e_{A}^{1}\rightarrow e_{A}^{2}$ or $e_{A}^{1}\rightarrow$ \emph{\textbf{Error}: string} by the induction hypothesis.  If $e_{A}^{1}$ is an unforced value then $e_{A}^{1}:\forall X.T_{2}$ by inversion (Lemma \ref{i}) and uniqueness of types (Lemma \ref{uot}) and $e_{A}^{1}\in\lbrace\Lambda X.e_{A}^{3},{^{\forall X.T_{2}}A}S$ $v_{S}\rbrace$ by canonical forms (Lemma \ref{cf}).  $(\Lambda X.e_{A}^{3})$ $\lbrace T_{1}\rbrace\rightarrow e_{A}^{3}[T_{1}/X]$.  $(^{\forall X.T_{2}}AS$ $v_{S})$ $\lbrace T_{1}\rbrace\rightarrow{^{T_{2}[T_{1}^{a}/X]}A}S$ $v_{S}$.  If $e_{A}^{1}\rightarrow e_{A}^{2}$ then $e_{A}^{1}$ $\lbrace T_{1}\rbrace\rightarrow e_{A}^{2}$ $\lbrace T_{1}\rbrace$.  If $e_{A}^{1}\rightarrow$ \emph{\textbf{Error}: string} then $e_{A}^{1}$ $\lbrace T_{1}\rbrace\rightarrow$ \emph{\textbf{Error}: string}.

\end{case}

% ML cons e e

\begin{case}

$e_{M}=\mathtt{cons}$ $e_{M}^{1}$ $e_{M}^{2}$

$e_{M}^{1}$ is an unforced value or $e_{M}^{1}\rightarrow e_{M}^{3}$ or $e_{M}^{1}\rightarrow$ \emph{\textbf{Error}: string} by the induction hypothesis.  If $e_{M}^{1}\rightarrow e_{M}^{3}$ then $\mathtt{cons}$ $e_{M}^{1}$ $e_{M}^{2}\rightarrow\mathtt{cons}$ $e_{M}^{3}$ $e_{M}^{2}$.  If $e_{M}^{1}\rightarrow$ \emph{\textbf{Error}: string} then $\mathtt{cons}$ $e_{M}^{1}$ $e_{M}^{2}\rightarrow$ \emph{\textbf{Error}: string}.  $e_{M}^{2}$ is an unforced value or $e_{M}^{2}\rightarrow e_{M}^{4}$ or $e_{M}^{2}\rightarrow$ \emph{\textbf{Error}: string} by the induction hypothesis.  If $e_{M}^{2}\rightarrow e_{M}^{4}$ and $e_{M}^{1}$ is an unforced value then $\mathtt{cons}$ $e_{M}^{1}$ $e_{M}^{2}\rightarrow\mathtt{cons}$ $e_{M}^{1}$ $e_{M}^{4}$.  If $e_{M}^{2}\rightarrow$ \emph{\textbf{Error}: string} and $e_{M}^{1}$ is an unforced value then $\mathtt{cons}$ $e_{M}^{1}$ $e_{M}^{2}\rightarrow$ \emph{\textbf{Error}: string}.  If $e_{M}^{1}$ and $e_{M}^{2}$ are unforced values then $\mathtt{cons}$ $e_{M}^{1}$ $e_{M}^{2}$ is an unforced value.

\end{case}

% Haskell f e

\begin{case}

$e_{H}=f$ $e_{H}^{1}$

$e_{H}^{1}$ is an unforced value or $e_{H}^{1}\rightarrow e_{H}^{2}$ or $e_{H}^{1}\rightarrow$ \emph{\textbf{Error}: string} by the induction hypothesis.  If $e_{H}^{1}$ is an unforced value then $e_{H}^{1}:[T]$ by inversion (Lemma \ref{i}) and uniqueness of types (Lemma \ref{uot}) and $e_{H}^{1}\in\lbrace\mathtt{nil}^{T},\mathtt{cons}$ $e_{H}^{3}$ $e_{H}^{4}\rbrace$ by canonical forms (Lemma \ref{cf}).  $\mathtt{hd}$ $\mathtt{nil}^{T}\rightarrow\mathtt{wrong}^{T}$ \emph{``Empty list"}.  $\mathtt{tl}$ $\mathtt{nil}^{T}\rightarrow\mathtt{wrong}^{[T]}$ \emph{``Empty list"}.  $\mathtt{hd}$ $(\mathtt{cons}$ $e_{H}^{3}$ $e_{H}^{4})\rightarrow e_{H}^{3}$.  $\mathtt{tl}$ $(\mathtt{cons}$ $e_{H}^{3}$ $e_{H}^{4})\rightarrow e_{H}^{4}$.  If $e_{H}^{1}\rightarrow e_{H}^{2}$ then $f$ $e_{H}^{1}\rightarrow f$ $e_{H}^{2}$.  If $e_{H}^{1}\rightarrow$ \emph{\textbf{Error}: string} then $f$ $e_{H}^{1}\rightarrow$ \emph{\textbf{Error}: string}.

\end{case}

% ML f e

\begin{case}

$e_{M}=f$ $e_{M}^{1}$

$e_{M}^{1}$ is an unforced value or $e_{M}^{1}\rightarrow e_{M}^{2}$ or $e_{M}^{1}\rightarrow$ \emph{\textbf{Error}: string} by the induction hypothesis.  If $e_{M}^{1}$ is an unforced value then $e_{M}^{1}:[T]$ by inversion (Lemma \ref{i}) and uniqueness of types (Lemma \ref{uot}) and $e_{M}^{1}\in\lbrace\mathtt{nil}^{T},\mathtt{cons}$ $v_{M}^{1}$ $v_{M}^{2},{^{[T]}M}H^{[T]}$ $(\mathtt{cons}$ $e_{H}^{1}$ $e_{H}^{2})\rbrace$ by canonical forms (Lemma \ref{cf}).  $\mathtt{hd}$ $\mathtt{nil}^{T}\rightarrow\mathtt{wrong}^{T}$ \emph{``Empty list"}.  $\mathtt{tl}$ $\mathtt{nil}^{T}\rightarrow\mathtt{wrong}^{[T]}$ \emph{``Empty list"}.  $\mathtt{hd}$ $(\mathtt{cons}$ $v_{M}^{1}$ $v_{M}^{2})\rightarrow v_{M}^{1}$.  $\mathtt{tl}$ $(\mathtt{cons}$ $v_{M}^{1}$ $v_{M}^{2})\rightarrow v_{M}^{2}$.  $\mathtt{hd}$ $({^{[T]}M}H^{[T]}$ $(\mathtt{cons}$ $e_{H}^{1}$ $e_{H}^{2}))\rightarrow{^{T}M}H^{T}$ $e_{H}^{1}$.  $\mathtt{tl}$ $({^{[T]}M}H^{[T]}$ $(\mathtt{cons}$ $e_{H}^{1}$ $e_{H}^{2}))\rightarrow{^{[T]}M}H^{[T]}$ $e_{H}^{2}$.  If $e_{M}^{1}\rightarrow e_{M}^{2}$ then $f$ $e_{M}^{1}\rightarrow f$ $e_{M}^{2}$.  If $e_{M}^{1}\rightarrow$ \emph{\textbf{Error}: string} then $f$ $e_{M}^{1}\rightarrow$ \emph{\textbf{Error}: string}.

\end{case}

% o e e

\begin{case}

$e_{A}=o$ $e_{A}^{1}$ $e_{A}^{2}$ where $A\in\lbrace H,M\rbrace$

$e_{A}^{1}$ is an unforced value or $e_{A}^{1}\rightarrow e_{A}^{3}$ or $e_{A}^{1}\rightarrow$ \emph{\textbf{Error}: string} by the induction hypothesis.  If $e_{A}^{1}\rightarrow e_{A}^{3}$ then $o$ $e_{A}^{1}$ $e_{A}^{2}\rightarrow o$ $e_{A}^{3}$ $e_{A}^{2}$.  If $e_{A}^{1}\rightarrow$ \emph{\textbf{Error}: string} then $o$ $e_{A}^{1}$ $e_{A}^{2}\rightarrow$ \emph{\textbf{Error}: string}.  $e_{A}^{2}$ is an unforced value or $e_{A}^{2}\rightarrow e_{A}^{4}$ or $e_{A}^{2}\rightarrow$ \emph{\textbf{Error}: string} by the induction hypothesis.  If $e_{A}^{2}\rightarrow e_{A}^{4}$ and $e_{A}^{1}$ is an unforced value then $o$ $e_{A}^{1}$ $e_{A}^{2}\rightarrow o$ $e_{A}^{1}$ $e_{A}^{4}$.  If $e_{A}^{2}\rightarrow$ \emph{\textbf{Error}: string} and $e_{A}^{1}$ is an unforced value then $o$ $e_{A}^{1}$ $e_{A}^{2}\rightarrow$ \emph{\textbf{Error}: string}.  $e_{A}^{1}$ and $e_{A}^{2}$ are unforced values otherwise.  $e_{A}^{1}:N$ and $e_{A}^{2}:N$ by inversion (Lemma \ref{i}) and uniqueness of types (Lemma \ref{uot}) and $e_{A}^{1}=\overline{n_{1}}$ and $e_{A}^{2}=\overline{n_{2}}$ by canonical forms (Lemma \ref{cf}).  $+$ $\overline{n_{1}}$ $\overline{n_{2}}\rightarrow\overline{n_{1}+n_{2}}$.  $-$ $\overline{n_{1}}$ $\overline{n_{2}}\rightarrow\overline{max(n_{1}-n_{2},0)}$.

\end{case}

% Haskell null? e

\begin{case}

$e_{H}=\mathtt{null?}$ $e_{H}^{1}$

$e_{H}^{1}$ is an unforced value or $e_{H}^{1}\rightarrow e_{H}^{2}$ or $e_{H}^{1}\rightarrow$ \emph{\textbf{Error}: string} by the induction hypothesis.  If $e_{H}^{1}$ is an unforced value then $e_{H}^{1}:[T]$ by inversion (Lemma \ref{i}) and uniqueness of types (Lemma \ref{uot}) and $e_{H}^{1}\in\lbrace\mathtt{nil}^{T},\mathtt{cons}$ $e_{H}^{1}$ $e_{H}^{2}\rbrace$ by canonical forms (Lemma \ref{cf}).  $\mathtt{null?}$ $\mathtt{nil}^{T}\rightarrow\overline{0}$.  $\mathtt{null?}$ $(\mathtt{cons}$ $e_{H}^{1}$ $e_{H}^{2})\rightarrow\overline{1}$.  If $e_{H}^{1}\rightarrow e_{H}^{2}$ then $\mathtt{null?}$ $e_{H}^{1}\rightarrow\mathtt{null?}$ $e_{H}^{2}$.  If $e_{H}^{1}\rightarrow$ \emph{\textbf{Error}: string} then $\mathtt{null?}$ $e_{H}^{1}\rightarrow$ \emph{\textbf{Error}: string}.

\end{case}

% ML null? e

\begin{case}

$e_{M}=\mathtt{null?}$ $e_{M}^{1}$

$e_{M}^{1}$ is an unforced value or $e_{M}^{1}\rightarrow e_{M}^{2}$ or $e_{M}^{1}\rightarrow$ \emph{\textbf{Error}: string} by the induction hypothesis.  If $e_{M}^{1}$ is an unforced value then $e_{M}^{1}:[T]$ by inversion (Lemma \ref{i}) and uniqueness of types (Lemma \ref{uot}) and $e_{M}^{1}\in\lbrace\mathtt{nil}^{T},\mathtt{cons}$ $v_{M}^{1}$ $v_{M}^{2},{^{[T]}M}H^{[T]}$ $(\mathtt{cons}$ $e_{H}^{1}$ $e_{H}^{2})\rbrace$ by canonical forms (Lemma \ref{cf}).  $\mathtt{null?}$ $\mathtt{nil}^{T}\rightarrow\overline{0}$.  If $e_{M}^{1}\in\lbrace\mathtt{cons}$ $v_{M}^{1}$ $v_{M}^{2},{^{[T]}M}H^{[T]}$ $(\mathtt{cons}$ $e_{H}^{1}$ $e_{H}^{2})\rbrace$ then $\mathtt{null?}$ $e_{M}^{1}\rightarrow\overline{1}$.  If $e_{M}^{1}\rightarrow e_{M}^{2}$ then $\mathtt{null?}$ $e_{M}^{1}\rightarrow\mathtt{null?}$ $e_{M}^{2}$.  If $e_{M}^{1}\rightarrow$ \emph{\textbf{Error}: string} then $\mathtt{null?}$ $e_{M}^{1}\rightarrow$ \emph{\textbf{Error}: string}.

\end{case}

% if0 e e e

\begin{case}

$e_{A}=\mathtt{if0}$ $e_{A}^{1}$ $e_{A}^{2}$ $e_{A}^{3}$ where $A\in\lbrace H,M\rbrace$

$e_{A}^{1}$ is an unforced value or $e_{A}^{1}\rightarrow e_{A}^{4}$ or $e_{A}^{1}\rightarrow$ \emph{\textbf{Error}: string} by the induction hypothesis.  If $e_{A}^{1}$ is an unforced value then $e_{A}^{1}:N$ by inversion (Lemma \ref{i}) and uniqueness of types (Lemma \ref{uot}) and $e_{A}^{1}=\overline{n}$ by canonical forms (Lemma \ref{cf}).  $\mathtt{if0}$ $\overline{0}$ $e_{A}^{2}$ $e_{A}^{3}\rightarrow e_{A}^{2}$.  $\mathtt{if0}$ $\overline{n}$ $e_{A}^{2}$ $e_{A}^{3}\rightarrow e_{A}^{3}$ $(n\neq 0)$.  If $e_{A}^{1}\rightarrow e_{A}^{4}$ then $\mathtt{if0}$ $e_{A}^{1}$ $e_{A}^{2}$ $e_{A}^{3}\rightarrow \mathtt{if0}$ $e_{A}^{4}$ $e_{A}^{2}$ $e_{A}^{3}$.  If $e_{A}^{1}\rightarrow$ \emph{\textbf{Error}: string} then $\mathtt{if0}$ $e_{A}^{1}$ $e_{A}^{2}$ $e_{A}^{3}\rightarrow$ \emph{\textbf{Error}: string}.

\end{case}

% wrong t string

\begin{case}

$e_{A}=\mathtt{wrong}^{T}$ \emph{string} where $A\in\lbrace H,M\rbrace$

$\mathtt{wrong}^{T}$ $\mathrm{string}\rightarrow$ \emph{\textbf{Error}: string}.

\end{case}

% hm t t e

\begin{case}

$e_{H}={^{T}H}M$ $e_{M}^{1}$

$e_{M}^{1}$ is an unforced value or $e_{M}^{1}\rightarrow e_{M}^{2}$ or $e_{M}^{1}\rightarrow$ \emph{\textbf{Error}: string} by the induction hypothesis.  If $e_{M}^{1}\rightarrow e_{M}^{2}$ then $^{T}HM$ $e_{M}^{1}\rightarrow{^{T}H}M$ $e_{M}^{2}$.  If $e_{M}^{1}\rightarrow$ \emph{\textbf{Error}: string} then $^{T}HM$ $e_{M}^{1}\rightarrow$ \emph{\textbf{Error}: string}.  If $e_{M}^{1}$ is an unforced value then $T$ determines the reduction of $^{T}HM$ $e_{M}^{1}$:

\begin{subcase}

$T=L$

$e_{M}^{1}={^{L}M}S$ $v_{S}$ by canonical forms (Lemma \ref{cf}).  $^{L}HM$ $(^{L}MS$ $v_{S})\rightarrow{^{L}H}S$ $v_{S}$.

\end{subcase}

\begin{subcase}

$T=N$

$e_{M}^{1}=\overline{n}$ by canonical forms (Lemma \ref{cf}).  $^{N}HM$ $\overline{n}\rightarrow\overline{n}$.

\end{subcase}

\begin{subcase}

$T=[T_{1}]$

$e_{M}^{1}\in\lbrace\mathtt{nil}^{T_{1}},\mathtt{cons}$ $v_{M}^{1}$ $v_{M}^{2},{^{[T_{1}]}M}H$ $(\mathtt{cons}$ $e_{H}^{1}$ $e_{H}^{2})\rbrace$ by canonical forms (Lemma \ref{cf}).  $^{[T_{1}]}HM$ $\mathtt{nil}^{T}\rightarrow\mathtt{nil}^{T}$.  $^{[T_{1}]}HM$ $(\mathtt{cons}$ $v_{M}^{1}$ $v_{M}^{2})\rightarrow\mathtt{cons}$ $(^{T_{1}}HM$ $v_{M}^{1})$ $(^{[T_{1}]}HM$ $v_{M}^{2})$.  $^{[T_{1}]}HM$ $(^{[T_{1}]}MH$ $(\mathtt{cons}$ $e_{H}^{1}$ $e_{H}^{2}))\rightarrow\mathtt{cons}$ $e_{H}^{1}$ $e_{H}^{2}$.

\end{subcase}

\begin{subcase}

$T=T_{1}^{a}$

Cannot occur because $T_{1}^{a}$ occurs only in $^{T_{1}^{a}}HS$ $e_{S}$.

\end{subcase}

\begin{subcase}

$T=T_{1}\rightarrow T_{2}$

$e_{M}^{1}=\lambda x_{1}:T_{1}.e_{M}^{3}$ by canonical forms (Lemma \ref{cf}).  $^{T_{1}\rightarrow T_{2}}HM$ $(\lambda x_{1}:T_{1}.e_{M}^{3})\rightarrow\lambda x_{2}:T_{1}.(^{T_{2}}HM$ $((\lambda x_{1}:T_{1}.e_{M}^{3})$ $(^{T_{1}}MH$ $x_{2})))$.

\end{subcase}

\begin{subcase}

$T=\forall X.T_{1}$

$e_{M}^{1}\in\lbrace\Lambda X.e_{M}^{3},{^{\forall X.T_{1}}M}S$ $v_{S}\rbrace$ by canonical forms (Lemma \ref{cf}).  $^{\forall X.T_{1}}HM$ $(\Lambda X.e_{M}^{3})\rightarrow\Lambda X.(^{T_{1}}HM$ $e_{M}^{3})$.  $^{\forall X.T_{1}}HM$ $(^{\forall X.T_{1}}MS$ $v_{S})\rightarrow{^{\forall X.T_{1}}H}S$ $v_{S}$.

\end{subcase}

\end{case}

% mh t t e

\begin{case}

$e_{M}={^{T}M}H$ $e_{H}$

$^{T}MH$ $e_{H}$ is an unforced value.

\end{case}

% hs k e

\begin{case}

$e_{A}={^{T}A}S$ $e_{S}^{1}$ where $A\in\lbrace H,M\rbrace$

$e_{S}^{1}$ is an unforced value or $e_{S}^{1}\rightarrow e_{S}^{2}$ or $e_{S}^{1}\rightarrow$ \emph{\textbf{Error}: string} by Scheme progress (Theorem \ref{sps}).  If $e_{S}^{1}\rightarrow e_{S}^{2}$ then $^{T}AS$ $e_{S}^{1}\rightarrow{^{T}A}S$ $e_{S}^{2}$.  If $e_{S}^{1}\rightarrow$ \emph{\textbf{Error}: string} then $^{T}AS$ $e_{S}^{1}\rightarrow$ \emph{\textbf{Error}: string}.  If $e_{S}^{1}$ is an unforced value then $T$ determines the reduction of $^{T}AS$ $e_{S}^{1}$:

\begin{subcase}

$T=L$

$^{L}AS$ $e_{S}^{1}$ is an unforced value.

\end{subcase}

\begin{subcase}

$T=N$

$^{N}AS$ $\overline{n}\rightarrow\overline{n}$.  $^{N}AS$ $e_{S}^{1}\rightarrow\mathtt{wrong}^{N}$ \emph{``Not a number"} $(e_{S}^{1}\neq\overline{n})$.

\end{subcase}

\begin{subcase}

$T=[T_{1}]$

$^{[T_{1}]}AS$ $\mathtt{nil}\rightarrow\mathtt{nil}^{T_{1}}$.  $^{[T_{1}]}AS$ $(\mathtt{cons}$ $v_{S}^{1}$ $v_{S}^{2})\rightarrow\mathtt{cons}$ $(^{T_{1}}AS$ $v_{S}^{1})$ $(^{[T_{1}]}AS$ $v_{S}^{2})$.  $^{[T_{1}]}HS$ $(SH^{[T_{1}]}$ $(\mathtt{cons}$ $e_{H}^{1}$ $e_{H}^{2}))\rightarrow\mathtt{cons}$ $e_{H}^{1}$ $e_{H}^{2}$.  $^{[T_{1}]}MS$ $(SH^{[T_{1}]}$ $(\mathtt{cons}$ $e_{H}^{1}$ $e_{H}^{2}))\rightarrow{^{[T_{1}]}M}H^{[T_{1}]}$ $(\mathtt{cons}$ $e_{H}^{1}$ $e_{H}^{2})$.  $^{[T_{1}]}AS$ $e_{S}^{1}\rightarrow\mathtt{wrong}^{[T_{1}[T_{i}/T_{i}^{a}]]}$ \emph{``Not a list"} $(e_{S}^{1}\not\in\lbrace\mathtt{nil},\mathtt{cons}$ $v_{S}^{1}$ $v_{S}^{2},SH^{[T_{1}]}$ $(\mathtt{cons}$ $e_{H}^{1}$ $e_{H}^{2})\rbrace)$.

\end{subcase}

\begin{subcase}

$T=T_{1}^{a}$

$^{T_{1}^{a}}HS$ $(SH^{T_{1}^{a}}$ $e_{H})\rightarrow e_{H}$.  $^{T_{1}^{a}}HS$ $e_{S}^{1}\rightarrow\mathtt{wrong}^{T_{1}}$ \emph{``Parametricity violated"} $(e_{S}^{1}\neq SH^{T_{1}^{a}}$ $e_{H})$.  $^{T_{1}^{a}}MS$ $(SM^{T_{1}^{a}}$ $v_{M})\rightarrow v_{M}$.  $^{T_{1}^{a}}MS$ $e_{S}^{1}\rightarrow\mathtt{wrong}^{T_{1}}$ \emph{``Parametricity violated"} $(e_{S}^{1}\neq SM^{T_{1}^{a}}$ $v_{M})$.

\end{subcase}

\begin{subcase}

$T=T_{1}\rightarrow T_{2}$

$^{T_{1}\rightarrow T_{2}}AS$ $(\lambda x_{1}.e_{S}^{3})\rightarrow\lambda x_{2}:T_{1}[T_{i}/T^{a}_{i}].(^{T_{2}}AS$ $((\lambda x_{1}.e_{S}^{3})$ $(SA^{T_{1}}$ $x_{2})))$.  $^{T_{1}\rightarrow T_{2}}AS$ $e_{S}^{1}\rightarrow\mathtt{wrong}^{(T_{1}\rightarrow T_{2})[T_{i}/T_{i}^{a}]]}$ \emph{``Not a function"} $(e_{S}^{1}\neq\lambda x_{1}.e_{S}^{3})$.

\end{subcase}

\begin{subcase}

$T=\forall X.T_{1}$

$^{\forall X.T_{1}}AS$ $e_{S}^{1}$ is an unforced value.

\end{subcase}

\end{case}

% ML

% cons v v

\newcommand{\psconsm}{\expcons{\first{\varvalum}}{\second{\varvalum}}\xspace}

\begin{case}

\psconsm

\psconsm is a \profv.

\end{case}

% ML e e

\begin{case}

$e_{M}=e_{M}^{1}$ $e_{M}^{2}$

$e_{M}^{1}$ is an unforced value or $e_{M}^{1}\rightarrow e_{M}^{3}$ or $e_{M}^{1}\rightarrow$ \emph{\textbf{Error}: string} by the induction hypothesis.  If $e_{M}^{1}$ is an unforced value then $e_{M}^{1}:T_{1}\rightarrow T_{2}$ by inversion (Lemma \ref{i}) and uniqueness of types (Lemma \ref{uot}) and $e_{M}^{1}=\lambda x:T_{1}.e_{M}^{4}$ by canonical forms (Lemma \ref{cf}).  If $e_{M}^{1}\rightarrow e_{M}^{3}$ then $e_{M}^{1}$ $e_{M}^{2}\rightarrow e_{M}^{3}$ $e_{M}^{2}$.  If $e_{M}^{1}\rightarrow$ \emph{\textbf{Error}: string} then $e_{M}^{1}$ $e_{M}^{2}\rightarrow$ \emph{\textbf{Error}: string}.  $e_{M}^{2}$ is an unforced value or $e_{M}^{2}\rightarrow e_{M}^{5}$ or $e_{M}^{2}\rightarrow$ \emph{\textbf{Error}: string} by the induction hypothesis.  If $e_{M}^{2}\rightarrow e_{M}^{5}$ and $e_{M}^{1}$ is an unforced value then $e_{M}^{1}$ $e_{M}^{2}\rightarrow e_{M}^{1}$ $e_{M}^{5}$.  If $e_{M}^{2}\rightarrow$ \emph{\textbf{Error}: string} and $e_{M}^{1}$ is an unforced value then $e_{M}^{1}$ $e_{M}^{2}\rightarrow$ \emph{\textbf{Error}: string}.  If $e_{M}^{1}$ is an unforced value and $e_{M}^{2}$ is an unforced value then $(\lambda x:T_{1}.e_{M}^{4})$ $e_{M}^{2}\rightarrow e_{M}^{4}[e_{M}^{2}/x]$.

\end{case}

\end{proof}

\end{theorem}
