\section{Introduction}

The complexities of software interoperation in part engender the proverbial reinvention of the wheel. Programmers forgo existing solutions to problems in other languages where interoperation proves too cumbersome; they reimplement software components, rather than reuse them. Disparate programming language features complicate the conversion of values exchanged between components of different languages. Resolving language incompatibilities transparently at boundaries between component languages facilitates reuse by unburdening programmers. To address part of this problem, this paper presents a model of computation that resolves incompatible evaluation strategies.

The systems of interoperation presented by Matthews and Findler \cite{matthews07} do not resolve incompatible evaluation strategies. The evaluation of expressions of the inner language within guards follows the evaluation strategy of the outer language, so where the outer and inner languages have incompatible evaluation strategies, interoperation may change the meaning of values converted between languages.

Since Matthews and Findler used call-by-value evaluation strategies for their ML and Scheme models, it is illustrative to introduce a third language that uses an incompatible evaluation strategy: call-by-name. Thus their model is extended with a third language, identical to the ML model except it uses a call-by-name evaluation strategy, and named after Haskell, to which it is more similar than ML.

Call-by-name and call-by-value evaluation strategies use orders of evaluation that take opposite approaches. Call-by-name evaluates expressions needed only by primitive operations, whereas call-by-value evaluates all expressions. As such, call-by-name evaluates a proper subset of the expressions that call-by-value does. In other words, the set of call-by-name strictness points is a proper subset of that of call-by-value. The exclusive disjunction between these two sets is the set of incompatible strictness points that may change the meaning of expressions that are converted from call-by-name to call-by-value. In call-by-name, expressions at these points may be assumed to never be evaluated, or assumed to be evaluated only a finite number of times. Call-by-value may violate these assumptions.

Interoperation requires preserving these strictness points for each evaluation strategy, even after a call-by-name value is converted to a call-by-value value. Otherwise, expressions may be evaluated where they were not before and cause errors or diverge. For an expression imported from one language to another, if its conversion requires it to be evaluated, then the evaluation must follow the order of evaluation defined by the inner language's evaluation strategy. This means deferring the evaluation of converted call-by-name expressions in these incompatible points using a dual notion of values and evaluation contexts in call-by-value languages that handles both call-by-value values and converted call-by-name expressions, called \emph{forced} and \emph{unforced} values and evaluation contexts.