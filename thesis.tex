\documentclass[12pt]{ucthesis}
\newif \ifpdf
\ifx \pdfoutput \undefined
  \pdffalse
\else
  \pdfoutput=1
  \pdftrue
\fi
\usepackage{url}
\ifpdf
  \usepackage[pdftex]{graphicx}
  \usepackage[pdftex,
    plainpages=false,
    breaklinks=true,
    colorlinks=true,
    urlcolor=blue,
    citecolor=blue,
    linkcolor=blue,
    bookmarks=true,
    bookmarksopen=true,
    bookmarksopenlevel=3,
    pdfstartview=FitV,
    pdfauthor={William Faught},
    pdftitle={Interoperation Between Haskell, ML, and Scheme},
    pdfkeywords={thesis, masters, cal poly}]{hyperref}
  \pdfcompresslevel=1
\else
  \usepackage{graphicx}
\fi
\usepackage{amsmath}
\usepackage{amssymb}
\usepackage{amsthm}
\usepackage[letterpaper]{geometry}
\usepackage{mathrsfs}
\usepackage{setspace}
\usepackage[overload]{textcase}
\newtheorem{theorem}{Theorem}
\newtheorem{lemma}{Lemma}
\newtheorem{case}{Case}[theorem]
\newtheorem{subcase}{Case}[case]
\bibliographystyle{abbrv}
\setlength{\parindent}{0.25in}
\setlength{\parskip}{6pt}
\geometry{verbose,nohead,tmargin=1.25in,bmargin=1in,lmargin=1.5in,rmargin=1.3in}
\setcounter{tocdepth}{2}
\newcommand{\captionfonts}{\small\bf\ssp}
\makeatletter
\long\def\@makecaption#1#2{%
  \vskip\abovecaptionskip
  \sbox\@tempboxa{{\captionfonts #1: #2}}
  \ifdim \wd \@tempboxa > \hsize
    {\captionfonts #1: #2\par}
  \else
    \hbox to \hsize{\hfil\box\@tempboxa\hfil}
  \fi
  \vskip\belowcaptionskip}
%\let\@currsize\normalsize
\makeatother
\begin{document}
\title{Interoperation Between Haskell, ML, and Scheme}
\author{William Faught}
\degreemonth{June}
\degreeyear{2008}
\degree{Master of Science}
\defensemonth{August}
\defenseyear{2008}
\numberofmembers{3}
\chair{Dr. John Clements}
\othermemberA{Dr. Gene Fisher}
\othermemberB{Dr. Aaron Keen}
\field{Computer Science}
\campus{San Luis Obispo}
\copyrightyears{seven}
\maketitle
\begin{frontmatter}
\copyrightpage
\approvalpage
\begin{abstract}
Programming language interoperation is the cooperation of software components written in different languages.  Components cooperate by exchanging data across a boundary between their languages that converts it from one form to another.  The boundary resolves conflicts between languages and must not violate the type system of either language.  This paper explores the resolution of conflicting type systems, conflicting preservation of parametricity, and conflicting evaluation strategies for a system of interoperation for Haskell, ML, and Scheme by defining a model of computation and interoperation, providing a proof of its type soundness, and describing an implementation of it.
\end{abstract}
\begin{acknowledgements}
\indent\indent I want to thank my father and mother, Jerry and Jo Ann, for their encouragement, advice, and support, without which this would not have been possible.

I want to thank my adviser, John Clements, for helping me along the way.  I very much appreciate the time he set aside for me and his advice.
\end{acknowledgements}
\tableofcontents
\listoffigures
\end{frontmatter}
\pagestyle{plain}
\renewcommand{\baselinestretch}{1.66}
\chapter{Introduction}

Software components comprise various programming languages, interfaces, and execution platforms.  Components of homogeneous configurations of these and other properties cooperate more easily than components of heterogeneous configurations interoperate.  This relative ease encourages components to be redefined in common configurations, rather than reusing existing solutions to problems from less common configurations.  Components of different languages interoperate by exchanging data across boundaries between their languages.  Language boundaries convert data from one form to another and ensure that interoperation does not violate any language property.  Language incompatibilities complicate interoperation and arise where equivalent forms of data do not exist or where interoperation would violate a language property.  This paper explores and resolves three such incompatibilities with a model of computation, proves its type soundness, and describes an implementation of it.

\section{Evaluation Strategies}

The first incompatibility is evaluation strategies.  Evaluation strategies determine the order in which languages evaluate expressions.  Eager evaluation reduces expressions regardless of necessity, and lazy evaluation reduces expressions only where necessary.  Lazy languages --- languages that use lazy evaluation --- can construct infinite streams as lists because they do not evaluate list elements when lists are constructed, but eager languages cannot because they do.  Not all lazy lists --- lists in lazy languages --- can be converted to eager lists.  Instead, elements of lazy lists can be converted when accessed by eager languages if they are not lazy lists too.

\section{Type Systems}

The second incompatibility is type systems.  Static type systems calculate and validate the types of expressions before run time and guarantee that well-typed programs will not encounter type errors during run time.  Run-time type calculations or validations are unnecessary.  Dynamic type systems do not calculate or validate the types of expressions.  Programs can use predicates to calculate the kinds of values during run time.

In the model, components exchange a common set of values.

Statically-typed languages --- languages that use static type systems --- must verify that the actual type of data received from a server matches their expected type.  Mismatched types may cause type errors during run time, which statically-typed languages cannot detect because they do not calculate or validate types during run time.  If data was sent from statically-typed languages, their actual type can be calculated and validated before run time.  If data was sent from dynamically-typed languages, their actual type cannot be calculated until run time, except functions.  Since dynamically-typed functions can produce results of various types, and determining function behavior is undecidable \cite{blume04}, equivalent types for these functions cannot be reliably calculated.  Since the types of dynamically-typed functions cannot be reliably calculated at boundaries, those functions are wrapped in contracts \cite{findler02} that calculate and validate the types of their arguments and results during run time.

\section{Parametricity}

The third incompatibility is parametricity.  Parametricity ensures that parametric polymorphic functions from statically-typed languages behave the same regardless of the types and values of their arguments, and that those functions with variable result types produce as their results an argument with the same variable type.  Dynamically-typed functions can use type predicates and conditions to determine their behavior by the types and values of their arguments.  If dynamically-typed functions are used as parametric polymorphic functions in languages that support parametricity, they can break parametricity.  Arguments for these functions must be wrapped such that type predicates and conditions cannot examine them.

The languages in the model must be able to express programs in which the aforementioned three incompatibilities arise.  Haskell, ML, and Scheme each possess a unique combination of properties that together are sufficient for this purpose.  Haskell and ML use static type systems and support parametricity.  ML and Scheme use eager evaluation.  Haskell uses lazy evaluation.  Scheme uses a dynamic type system.  Any more languages would uselessly complicate the model.

The rest of the paper is organized as follows: Chapter 2 defines the model of computation.  Chapter 3 proves the type soundness of the model.  Chapter 4 describes an implementation of the model.  Chapter 5 discusses related work.  Chapter 6 discusses future work.  Chapter 7 discusses the conclusions.
\chapter{Model of Computation}

The model of computation is based on that of Matthews and Findler \ref{matthews07}. Their model consists of two simple models, one representing ML and the other Scheme.

The ML model is a simply-typed lambda calculus extended with parametric polymorphism called System F. The substitution semantics by which type abstractions are applied means the ML model has parametricity, which is a property that ensures that programs behave the same regardless of the types applied to by type abstractions.  The ML model introduces new expressions, type abstractions and type applications, to express parametric polymorphism, and new types, type variables and forall types, for them. The ML model uses an eager evaluation strategy.

The Scheme model is an extended untyped lambda calculus using an eager evaluation strategy. Value predicates enable ad-hoc polymorphism.

To this mix we introduce a Haskell model identical to the ML model, except it uses a lazy evaluation strategy.

The Haskell, ML, and Scheme models are defined in figures \ref{hg}, \ref{mg}, and \ref{sg}.



\clearpage

\begin{figure}[p]
\centering

\begin{tabular}{rcl}

\varexph & $=$ & \varvarh $|$ \varvaluh $|$ \expfapp{\varexph}{\varexph} $|$ \exptapp{\varexph}{\vartyh} $|$ \expfix{\varexph} $|$ \expop{\varexph}{\varexph} $|$ \expif{\varexph}{\varexph}{\varexph} $|$ \expfield{\varexph} $|$ \expnull{\varexph} \\

&& \expwrongs{\vartyh}{\formvar{string}} $|$ \exphm{\vartyh}{\varexpm} $|$ \exphs{\vartyh}{\varexps} \\

\varvaluh & $=$ & \expfabss{\varvarh}{\vartyh}{\varexph} $|$ \exptabs{\tyvarh}{\varexph} $|$ \expnum{\varnum} $|$ \expnils{\vartyh} $|$ \expcons{\varexph}{\varexph} $|$ \exphs{\tylump}{\varvalus} $|$ \exphs{(\tyfor{\tyvarh}{\vartyh})}{\varvalus} \\

\vartyh & $=$ & \tylump $|$ \tynum $|$ \tyvarh $|$ \tylist{\tyvarh} $|$ \tylabel{\vartyh}{\tyvarh} $|$ \tyfun{\vartyh}{\vartyh} $|$ \tyfor{\tyvarh}{\vartyh} \\

\formvar{\symop} & $=$ & \formsym{\symadd} $|$ \formsym{\symsub} \\

\formvar{\symfield} & $=$ & \formsym{\symhd} $|$ \formsym{\symtl} \\

\varconfh & $=$ & \symholeh $|$ \expfapp{\varconfh}{\varexph} $|$ \exptapp{\varconfh}{\vartyh} $|$ \expfix{\varconfh} $|$ \expop{\varconfh}{\varexph} $|$ \expop{\varvaluh}{\varconfh} $|$ \expif{\varconfh}{\varexph}{\varexph} \\

&& \expfield{\varconfh} $|$ \expnull{\varconfh} $|$ \exphm{\vartyh}{\varconfm} $|$ \exphs{\vartyh}{\varconfs}

\end{tabular}
\caption{Haskell grammar and evaluation contexts}
\label{hg}
\end{figure}


\clearpage

\begin{figure}[p]
\[
% L
\frac
{}
{\judth{}{\tylump}}
\quad
% N
\frac
{}
{\judth{}{\tynum}}
\quad
% u
\frac
{}
{\judth{\envextt{\env}{\tyvarh}}{\tyvarh}}
\]
\[
% {t}
\frac
{\judth{\env}{\vartyh}}
{\judth{\env}{\tylist{\vartyh}}}
\quad
% t->t
\frac
{\judth{\env}{\first{\vartyh}} \quad \judth{\env}{\second{\vartyh}}}
{\judth{\env}{\tyfun{\first{\vartyh}}{\second{\vartyh}}}}
\quad
% Au.t
\frac
{\judth{\env, \tyvarh}{\vartyh}}
{\judth{\env}{\tyfor{\tyvarh}{\vartyh}}}
\]
\bigskip
\[
% \x:t.e
\frac
{\judth{\env}{\first{\vartyh}} \quad \judeh{\envexte{\env}{\varvarh}{\first{\vartyh}}}{\varexph}{\second{\vartyh}}}
{\judeh{\env}{(\expfabss{\varvarh}{\first{\vartyh}}{\varexph})}{\tyfun{\first{\vartyh}}{\second{\vartyh}}}}
\quad
% \\u.e
\frac
{\judeh{\envextt{\env}{\tyvarh}}{\varexph}{\vartyh}}
{\judeh{\env}{\exptabs{\tyvarh}{\varexph}}{\tyfor{\tyvarh}{\vartyh}}}
\quad
% n
\frac
{}
{\judeh{}{\expnum{n}}{\tynum}}
\]
\[
% nil t
\frac
{\judeh{\env}{\vartyh}}
{\judeh{\env}{\expnils{\vartyh}}{\tylist{\vartyh}}}
\quad
% cons e e
\frac
{\judeh{\env}{\first{\varexph}}{\vartyh} \quad \judeh{\env}{\second{\varexph}}{\tylist{\vartyh}}}
{\judeh{\env}{\expcons{\first{\varexph}}{\second{\varexph}}}{\tylist{\vartyh}}}
\quad
% x
\frac
{}
{\judeh{\envexte{\env}{\varvarh}{\vartyh}}{\varvarh}{\vartyh}}
\]
\[
% e e
\frac
{\judeh{\env}{\first{\varexph}}{\tyfun{\first{\vartyh}}{\second{\vartyh}}} \quad \judeh{\env}{\second{\varexph}}{\first{\vartyh}}}
{\env\symjudh\expfapp{\first{\varexph}}{\second{\varexph}}:\second{\vartyh}}
\quad
% fix e
\frac
{\judeh{\env}{\varexph}{\tyfun{\vartyh}{\vartyh}}}
{\judeh{\env}{\expfix{\varexph}}{\vartyh}}
\]
\[
% e<t>
\frac
{\judth{\env}{\first{\vartyh}} \quad \judeh{\env}{\varexph}{\tyfor{\tyvarh}{\second{\vartyh}}}}
{\judeh{\env}{\exptapp{\varexph}{\first{\vartyh}}}{\tysubst{\second{\vartyh}}{\first{\vartyh}}{\tyvarh}}}
\quad
% hd e
\frac
{\judeh{\env}{\varexph}{\tylist{\vartyh}}}
{\judeh{\env}{\exphd{\varexph}}{\vartyh}}
\quad
% tl e
\frac
{\judeh{\env}{\varexph}{\tylist{\vartyh}}}
{\judeh{\env}{\exptl{\varexph}}{\tylist{\vartyh}}}
\]
\[
% o e e
\frac
{\judeh{\env}{\first{\varexph}}{\tynum} \quad \judeh{\env}{\second{\varexph}}{\tynum}}
{\judeh{\env}{\expop{\first{\varexph}}{\second{\varexph}}}{\tynum}}
\quad
% null? e
\frac
{\judeh{\env}{\varexph}{\tylist{\vartyh}}}
{\judeh{\env}{\exppnull{\varexph}}{\tynum}}
\quad
% hs k e
\frac
{\judth{\env}{\tyunbrand{\varcsh}} \quad \judes{\env}{\varexps}{\tytst}}
{\judeh{\env}{\exphs{\varcsh}{\varexps}}{\tyunbrand{\varcsh}}}
\]
\[
% if0 e e e
\frac
{\judeh{\env}{\first{\varexph}}{\tynum} \quad \judeh{\env}{\second{\varexph}}{\vartyh} \quad \judeh{\env}{\third{\varexph}}{\vartyh}}
{\judeh{\env}{\expif{\first{\varexph}}{\second{\varexph}}{\third{\varexph}}}{\vartyh}}
\quad
% wrong t s
\frac
{\judth{\env}{\vartyh}}
{\judeh{\env}{\expwrongs{\vartyh}{\formvar{string}}}{\vartyh}}
\]
\[
% hm k k e
\frac
{\judth{\env}{\vartyh} \quad \judtm{\env}{\first{\vartym}} \quad \judem{\env}{\varexpm}{\second{\vartym}} \quad \vartyh \eq \vartym \quad \first{\vartym} = \second{\vartym}}
{\judeh{\env}{\exphm{\vartyh}{\first{\vartym}}{\varexpm}}{\vartyh}}
\]
\caption{Haskell typing rules}
\label{htr}
\end{figure}


\clearpage

\begin{figure}[p]
\centering
\begin{tabular}{l}

% Function application

\redruleh
{\expfapp{(\expfabss{\varvarh}{\vartyh}{\first{\varexph}})}{\second{\varexph}}}
{\expsubst{\first{\varexph}}{\second{\varexph}}{\varvarh}} \\

% Type application

\redruleh
{\exptapp{(\exptabs{\tyvarh}{\varexph})}{\vartyh}}
{\expsubst{\varexph}{\vartyh}{\tyvarh}} \\

% Fix

\redruleh
{\expfix{(\expfabss{\varvarh}{\vartyh}{\varexph})}}
{\expsubst{\varexph}{\expfix{(\expfabss{\varvarh}{\vartyh}{\varexph})}}{\varvarh}} \\

% Add

\redruleh
{\expadd{\first{\expnum{\varnum}}}{\second{\expnum{\varnum}}}}
{\expnum{\first{\varnum} + \second{\varnum}}} \\

% Subtract

\redruleh
{\expsub{\first{\expnum{\varnum}}}{\second{\expnum{\varnum}}}}
{\expnum{\formvar{max}(\first{\varnum} - \second{\varnum}, 0)}} \\

% If0 true

\redruleh
{\expif{\expnum{0}}{\first{\varexph}}{\second{\varexph}}}
{\first{\varexph}} \\

% If0 false

\redruleh
{\expif{\expnum{\varnum}}{\first{\varexph}}{\second{\varexph}}}
{\second{\varexph}}
$(\varnum \neq 0)$ \\

% Head nil

\redruleh
{\exphd{(\expnils{\vartyh})}}
{\expwrongs{\vartyh}{\str{Empty\;list}}} \\

% Tail nil

\redruleh
{\exptl{(\expnils{\vartyh})}}
{\expwrongs{\tylist{\vartyh}}{\str{Empty\;list}}} \\

% Head cons

\redruleh
{\exphd{(\expcons{\first{\varexph}}{\second{\varexph}})}}
{\first{\varexph}} \\

% Tail cons

\redruleh
{\exptl{(\expcons{\first{\varexph}}{\second{\varexph}})}}
{\second{\varexph}} \\

% Null nil

\redruleh
{\expnull{(\expnils{\vartyh})}}
{\expnum{0}} \\

% Null cons

\redruleh
{\expnull{(\expcons{\first{\varexph}}{\second{\varexph}})}}
{\expnum{1}} \\

% Wrong

\redrule
{\redenvh{\expwrongs{\vartyh}{\formvar{string}}}}
{\experr{\varstr}}

\end{tabular}
\caption{Haskell operational semantics}
\label{hos}
\end{figure}


\clearpage

\begin{figure}[p]
\centering
\begin{tabular}{l}

% hm L (ms L v)

\redruleh
{\exphm{\tylump}{(\expms{\tylump}{\varvalfs})}}
{\exphs{\tylump}{\varvalfs}} \\

% hm N n

\redruleh
{\exphm{\tynum}{\expnum{\varnum}}}
{\expnum{\varnum}} \\

% hm [t] (nil t)

\redruleh
{\exphm{\tylist{\varcsh}}{(\expnils{\vartym})}}
{\expnils{\tyunbrand{\varcsh}}} \\

% hm [t] (cons v v)

\redruleh
{\exphm{\tylist{\varcsh}}{(\expcons{\first{\varvalum}}{\second{\varvalum}})}}
{\expcons{(\exphm{\tyunbrand{\varcsh}}{\first{\varvalum}})}{(\exphm{\tylist{\tyunbrand{\varcsh}}}{\second{\varvalum}})}} \\

% hm (t->t) (\x:t.e)

\redrule
{\redconh{\exphm{(\tyfun{\first{\varcsh}}{\second{\varcsh}})}{(\expfabss{\varvarm}{\vartym}{\varexpm})}}}
{} \\

\redsp \redcon{\expfabss{\varvarh}{\tyunbrand{\first{\varcsh}}}{\exphm{\second{\varcsh}}{\expfapp{((\expfabss{\varvarm}{\vartym}{\varexpm})}{(\expmh{\vartym}{\varvarh})})}}} \\

% hm (Ax.t) (\\x.e)

\redruleh
{\exphm{(\csfor{\csvarh}{\varcsh})}{(\exptabs{\tyvarm}{\varexpm})}}
{\exptabs{\tyvarh}{\exphm{\varcsh}{(\exptapp{(\exptabs{\tyvarm}{\varexpm})}{\tyconv{\tyvarh}})}}} \\

\end{tabular}
\caption{Haskell-ML operational semantics}
\label{hmos}
\end{figure}

\clearpage

\begin{figure}[p]
\onehalfspacing
\centering
\begin{tabular}{l}

% hs N n

\redruleh
{\exphs{\csnum}{\expnum{\varnum}}}
{{\expnum{\varnum}}} \\

% hs N v

\redruleh
{\exphs{\csnum}{\varvalfs}}
{\expwrongs{\tynum}{\errnum}}
$(\varvalfs \neq \expnum{\varnum})$ \\

% hs {k} nil

\redruleh
{\exphs{\cslist{\varcsh}}{\expnild}}
{\expnils{\tyunbrand{\varcsh}}} \\

% hs {k} (cons v v)

\redruleh
{\exphs{\cslist{\varcsh}}{(\expcons{\first{\varvalus}}{\second{\varvalus}})}}
{\expcons{(\exphs{\varcsh}{\first{\varvalus}})}{(\exphs{\cslist{\varcsh}}{\second{\varvalus}})}} \\

% hs {k} v

\redruleh
{\exphs{\cslist{\varcsh}}{\varvalfs}}
{\expwrongs{\tyunbrand{\cslist{\varcsh}}}{\errlist}} \\

\redsp $(\varvalfs \neq \expnild$ and $\varvalfs \neq \expcons{\first{\varvalus}}{\second{\varvalus}})$ \\

% hs (b.t) (sh (b.t) e)

\redruleh
{\exphs{(\csbrand{\varbrand}{\vartyh})}{(\expsh{(\csbrand{\varbrand}{\vartyh})}{\varexph})}}
{\varexph} \\

% hs (b.t) v

\redruleh
{\exphs{(\csbrand{\varbrand}{\vartyh})}{\varvalfs}}
{\expwrongs{\vartyh}{\errbrand}}
$(\varvalfs \neq \expsh{(\csbrand{\varbrand}{\vartyh})}{\varexph})$ \\

% hs (k->k) (\x.e)

\redruleh
{\exphs{(\csfun{\first{\varcsh}}{\second{\varcsh}})}{(\expfabsd{\varvars}{\varexps})}}
{\expfabss{\varvarh}{\tyunbrand{\first{\varcsh}}}{\exphs{\second{\varcsh}}{(\expfapp{(\expfabsd{\varvars}{\varexps})}{(\expsh{\first{\varcsh}}{\varvarh})})}}} \\

% hs (k->k) v

\redruleh
{\exphs{(\csfun{\first{\varcsh}}{\second{\varcsh}})}{\varvalfs}}
{\expwrongs{\tyunbrand{\csfun{\first{\varcsh}}{\second{\varcsh}}}}{\errfun}} \\

\redsp $(\varvalfs \neq \expfabsd{\varvars}{\varexps})$ \\

% hs (Au.k) w

\redruleh
{\exphs{(\csfor{\csvarh}{\varcsh})}{\varvalfs}}
{\exptabs{\tyvarh}{\exphs{\varcsh}{\varvalfs}}} \\

\end{tabular}
\caption{Haskell-Scheme operational semantics}
\label{fighsos}
\end{figure}

\clearpage

\begin{figure}[p]
\centering
\begin{tabular}{rcl}

$e^m$ & $=$ & $x^m$ $|$ $v^m$ $|$ $e^m$ $e^m$ $|$ $e^m$ $\lbrace t\rbrace$ $|$ $\mathtt{fix}$ $e^m$ $|$ $o$ $e^m$ $e^m$ $|$ $\mathtt{if0}$ $e^m$ $e^m$ $e^m$ \\

&& $\mathtt{cons}$ $e^m$ $e^m$ $|$ $f$ $e^m$ $|$ $\mathtt{null?}$ $e^m$ $|$ $\mathtt{wrong}$ $t$ string $|$ $\mathtt{ms}$ $t$ $e^s$ \\

$v^m$ & $=$ & $w^m$ $|$ $\mathtt{mh}$ $t$ $e^h$ \\

$w^m$ & $=$ & $\lambda x^m:t.e^m$ $|$ $\Lambda x^m.e^m$ $|$ $\overline{n}$ $|$ $\mathtt{nil}$ $t$ $|$ $\mathtt{cons}$ $v^m$ $v^m$ $|$ $\mathtt{ms}$ $\mathtt{L}$ $v^s$ $|$ $\mathtt{ms}$ $(\forall x^m.t)$ $v^s$ \\

$t$ & $=$ & $\mathtt{L}$ $|$ $\mathtt{N}$ $|$ $x^m$ $|$ $[t]$ $|$ $t.x^m$ $|$ $t\rightarrow t$ $|$ $\forall x^m.t$ \\

$o$ & $=$ & $+$ $|$ $-$ \\

$f$ & $=$ & $\mathtt{hd}$ $|$ $\mathtt{tl}$ \\

$E^m$ & $=$ & $[\,]^m$ $|$ $F^m$ $e^m$ $|$ $w^m$ $E^m$ $|$ $F^m$ $\lbrace t\rbrace$ $|$ $\mathtt{fix}$ $F^m$ $|$ $o$ $F^m$ $e^m$ $|$ $o$ $w^m$ $F^m$ \\

&& $\mathtt{if0}$ $F^m$ $e^m$ $e^m$ $|$ $\mathtt{cons}$ $E^m$ $e^m$ $|$ $\mathtt{cons}$ $v^m$ $E^m$ $|$ $f$ $F^m$ $|$ $\mathtt{null?}$ $F^m$ \\

&& $\mathtt{ms}$ $t$ $E^s$ \\

$F^m$ & $=$ & $E^m$ $|$ $\mathtt{mh}$ $t$ $E^h$

\end{tabular}
\caption{ML grammar and evaluation contexts}
\label{mg}
\end{figure}

\clearpage

\begin{figure}[p]
\[
% Lump
\frac
{}
{\judtm{}{\tylump}}
\quad
% Number
\frac
{}
{\judtm{}{\tynum}}
\quad
% Variable
\frac
{}
{\judtm{\envextt{\tyvarm}}{\tyvarm}}
\]
\[
% List
\frac
{\judtm{\env}{\vartym}}
{\judtm{\env}{\tylist{\vartym}}}
\quad
% Label
\frac
{\judtm{\env}{\vartym}}
{\judtm{\env}{\tylabel{\vartym}{\tyvarm}}}
\quad
% Function
\frac
{\judtm{\env}{\first{\vartym}} \quad \judtm{\env}{\second{\vartym}}}
{\judtm{\env}{\tyfun{\first{\vartym}}{\second{\vartym}}}}
\quad
% Forall
\frac
{\judtm{\env, \tyvarm}{\vartym}}
{\judtm{\env}{\tyfor{\tyvarm}{\vartym}}}
\]
\bigskip
\[
% Function abstraction
\frac
{\judtm{}{\first{\vartym}} \quad \judem{\envexte{\varvarm}{\first{\vartym}}}{\varexpm}{\second{\vartym}}}
{\judem{\env}{(\expfabss{\varvarm}{\first{\vartym}}{\varexpm})}{\tyfor{\first{\vartym}}{\second{\vartym}}}}
\quad
% Type abstraction
\frac
{\judem{\envextt{\tyvarm}}{\varexpm}{\vartym}}
{\judem{\env}{\exptabs{\tyvarm}{\varexpm}}{\tyfor{\tyvarm}{\vartym}}}
\quad
% Number
\frac
{}
{\judem{}{\expnum{n}}{\tynum}}
\]
\[
% Nil
\frac
{\judem{\env}{\vartym}}
{\judem{\env}{\expnils{\vartym}}{\tylist{\vartym}}}
\quad
% Cons
\frac
{\judem{\env}{\first{\varexpm}}{\vartym} \quad \judem{\env}{\second{\varexpm}}{\tylist{\vartym}}}
{\judem{\env}{\expcons{\first{\varexpm}}{\second{\varexpm}}}{\tylist{\vartym}}}
\quad
% Variable
\frac
{}
{\judem{\envexte{\varvarm}{\vartym}}{\varvarm}{\vartym}}
\]
\[
% Function application
\frac
{\judem{\env}{\first{\varexpm}}{\tyfun{\first{\vartym}}{\second{\vartym}}} \quad \judem{\env}{\second{\varexpm}}{\first{\vartym}}}
{\env\symjudh\expfapp{\first{\varexpm}}{\second{\varexpm}}:\second{\vartym}}
\quad
% Fix
\frac
{\judem{\env}{\varexpm}{\tyfun{\vartym}{\vartym}}}
{\judem{\env}{\expfix{\varexpm}}{\vartym}}
\]
\[
% Type application
\frac
{\judtm{\env}{\first{\vartym}} \quad \judem{\env}{\varexpm}{\tyfor{\tyvarm}{\second{\vartym}}}}
{\judem{\env}{\exptapp{\varexpm}{\first{\vartym}}}{\tysubst{\second{\vartym}}{\first{\vartym}}{\tyvarm}}}
\quad
% Head
\frac
{\judem{\env}{\varexpm}{\tylist{\vartym}}}
{\judem{\env}{\exphd{\varexpm}}{\vartym}}
\quad
% Tail
\frac
{\judem{\env}{\varexpm}{\tylist{\vartym}}}
{\judem{\env}{\exptl{\varexpm}}{\tylist{\vartym}}}
\]
\[
% Arithmetic
\frac
{\judem{\env}{\first{\varexpm}}{\tynum} \quad \judem{\env}{\second{\varexpm}}{\tynum}}
{\judem{\env}{\expop{\first{\varexpm}}{\second{\varexpm}}}{\tynum}}
\quad
% Null
\frac
{\judem{\env}{\varexpm}{\tylist{\vartym}}}
{\judem{\env}{\exppnull{\varexpm}}{\tynum}}
\]
\[
% If0
\frac
{\judem{\env}{\first{\varexpm}}{\tynum} \quad \judem{\env}{\second{\varexpm}}{\vartym} \quad \judem{\env}{\third{\varexpm}}{\vartym}}
{\judem{\env}{\expif{\first{\varexpm}}{\second{\varexpm}}{\third{\varexpm}}}{\vartym}}
\quad
% Wrong
\frac
{\judtm{\env}{\vartym}}
{\judem{\env}{\expwrongs{\vartym}{\formvar{string}}}{\vartym}}
\]
\[
% ML
\frac
{\judtm{\env}{\vartym} \quad \judeh{\env}{\varexph}{\vartyh} \quad \vartym=\vartyh}
{\judem{\env}{\expmh{\vartym}{\varexph}}{\vartym}}
\quad
% Scheme
\frac
{\judtm{\env}{\vartym} \quad \judes{\env}{\varexps}{\tytst}}
{\judem{\env}{\expms{\vartym}{\varexps}}{\tyunlabm{\vartym}}}
\]
\caption{ML typing rules}
\label{mtr}
\end{figure}

\clearpage

\begin{figure}[p]
\centering
\begin{tabular}{l}

$\mathscr{E}[(\lambda x^m:t.e^m)$ $v^m]^m\rightarrow\mathscr{E}[e^m[v^m/x^m]]$ \\

$\mathscr{E}[(\Lambda x^m.e^m)$ $\lbrace t\rbrace]]^m\rightarrow\mathscr{E}[e^m[t/x^m]]$ \\

$\mathscr{E}[\mathtt{fix}$ $(\lambda x^m:t.e^m)]^m\rightarrow\mathscr{E}[e^m[(\mathtt{fix}$ $(\lambda x^m:t.e^m))/x^m]]$ \\

$\mathscr{E}[+$ $\overline{n_{1}}$ $\overline{n_{2}}]^m\rightarrow\mathscr{E}[\overline{n_{1}+n_{2}}]$ \\

$\mathscr{E}[-$ $\overline{n_{1}}$ $\overline{n_{2}}]^m\rightarrow\mathscr{E}[\overline{max^m(n_{1}-n_{2},0)}]$ \\

$\mathscr{E}[\mathtt{if0}$ $\overline{0}$ $e^m_1$ $e^m_2]^m\rightarrow\mathscr{E}[e^m_1]$ \\

$\mathscr{E}[\mathtt{if0}$ $\overline{n}$ $e^m_1$ $e^m_2]^m\rightarrow\mathscr{E}[e^m_2]$ $(n\neq0)$ \\

$\mathscr{E}[\mathtt{hd}$ $(\mathtt{nil}$ $t)]^m\rightarrow\mathscr{E}[\mathtt{wrong}$ $t$ ``Empty list"$]$ \\

$\mathscr{E}[\mathtt{tl}$ $(\mathtt{nil}$ $t)]^m\rightarrow\mathscr{E}[\mathtt{wrong}$ $[t]$ ``Empty list"$]$ \\

$\mathscr{E}[\mathtt{hd}$ $(\mathtt{cons}$ $v^m_1$ $v^m_2)]^m\rightarrow\mathscr{E}[v^m_1]$ \\

$\mathscr{E}[\mathtt{tl}$ $(\mathtt{cons}$ $v^m_1$ $v^m_2)]^m\rightarrow\mathscr{E}[v^m_2]$ \\

$\mathscr{E}[\mathtt{hd}$ $(\mathtt{mh}$ $[t]$ $(\mathtt{cons}$ $e^h_1$ $e^h_2))]^m\rightarrow\mathscr{E}[\mathtt{mh}$ $t$ $e^h_1]$ \\

$\mathscr{E}[\mathtt{tl}$ $(\mathtt{mh}$ $[t]$ $(\mathtt{cons}$ $e^h_1$ $e^h_2))]^m\rightarrow\mathscr{E}[\mathtt{mh}$ $[t]$ $e^h_2]$ \\

$\mathscr{E}[\mathtt{null?}$ $(\mathtt{nil}$ $t)]^m\rightarrow\mathscr{E}[\overline{0}]$ \\

$\mathscr{E}[\mathtt{null?}$ $(\mathtt{cons}$ $v^m_1$ $v^m_2)]^m\rightarrow\mathscr{E}[\overline{1}]$ \\

$\mathscr{E}[\mathtt{null?}$ $(\mathtt{mh}$ $[t]$ $(\mathtt{cons}$ $e^h_1$ $e^h_2))]^m\rightarrow\mathscr{E}[\overline{1}]$ \\

$\mathscr{E}[\mathtt{wrong}$ $t$ string$]^m\rightarrow$ \textbf{Error}: string

\end{tabular}
\caption{ML operational semantics}
\label{mos}
\end{figure}

\clearpage

\begin{figure}[p]
\centering
\begin{tabular}{l}

$\mathscr{E}[\mathtt{mh}$ $\mathtt{L}$ $(\mathtt{hs}$ $\mathtt{L}$ $v^s)]^m\rightarrow\mathscr{E}[\mathtt{ms}$ $\mathtt{L}$ $v^s]$ \\

$\mathscr{E}[\mathtt{mh}$ $\mathtt{N}$ $\overline{n}]^m\rightarrow\mathscr{E}[\overline{n}]$ \\

$\mathscr{E}[\mathtt{mh}$ $[t]$ $(\mathtt{nil}$ $t)]^m\rightarrow\mathscr{E}[\mathtt{nil}$ $t]$ \\

$\mathscr{E}[\mathtt{mh}$ $[t]$ $(\mathtt{cons}$ $e^h_1$ $e^h_2)]^m\rightarrow\mathscr{E}[\mathtt{cons}$ $(\mathtt{mh}$ $t$ $e^h_1)$ $(\mathtt{mh}$ $[t]$ $e^h_2)]$ \\

$\mathscr{E}[\mathtt{mh}$ $(t_1\rightarrow t_2)$ $(\lambda x^h:t_1.e^h)]^m\rightarrow\mathscr{E}[\lambda x^m:t_1.\mathtt{mh}$ $t_2$ $((\lambda x^h:t_1.e^h)$ $(\mathtt{hm}$ $t_1$ $x^m))]$ \\

$\mathscr{E}[\mathtt{mh}$ $(\forall x^m.t)$ $(\Lambda x^m.e^h)]^m\rightarrow\mathscr{E}[\Lambda x^m.\mathtt{mh}$ $t$ $e^h]$ \\

$\mathscr{E}[\mathtt{mh}$ $(\forall x^m.t)$ $(\mathtt{hs}$ $(\forall x^h.t)$ $v^s)]^m\rightarrow\mathscr{E}[\mathtt{ms}$ $(\forall x^m.t)$ $v^s]$ \\

\end{tabular}
\caption{ML-Haskell operational semantics}
\label{mhos}
\end{figure}

\clearpage

\begin{figure}[p]
\centering
\begin{tabular}{l}

% ms N n

\redrulem
{\expms{\tynum}{\expnum{\varnum}}}
{{\expnum{\varnum}}} \\

% ms N v

\redrulem
{\expms{\tynum}{\varvalus}}
{\expwrongs{\tynum}{\str{Not \; a \; number}}}
$(\varvalus \neq \expnum{\varnum})$ \\

% ms [t] nil

\redrulem
{\expms{\tylist{\vartym}}{\expnild}}
{\expnils{\tyunlabm{\vartym}}} \\

% ms [t] (cons v v)

\redrulem
{\expms{\tylist{\vartym}}{(\expcons{\first{\varvalus}}{\second{\varvalus}})}}
{\expcons{(\expms{\vartym}{\first{\varvalus}})}{(\expms{\tylist{\vartym}}{\second{\varvalus}})}} \\

% ms [t] v

\redrulem
{\expms{\tylist{\vartym}}{\first{\varvalus}}}
{\expwrongs{\tyunlabm{\vartym}}{\str{Not \; a \; list}}} \\

\redsp $(\first{\varvalus} \neq \expnild$ and $\first{\varvalus} \neq \expcons{\second{\varvalus}}{\third{\varvalus}})$ \\

% ms (t.x) (sm (t.x) v)

\redrulem
{\expms{(\tylabel{\vartym}{\tyvarm})}{(\expsm{(\tylabel{\vartyh}{\tyvarh})}{\varvalum})}}
{\varvalum} \\

% ms (t.x) v

\redrulem
{\expms{(\tylabel{\vartym}{\tyvarm})}{\varvalus}}
{\expwrongs{\vartym}{\str{Parametricity \; violated}}}
$(\varvalus \neq \expsm{(\tylabel{\vartym}{\tyvarm})}{\varexpm})$ \\

% ms (t->t) (\x.e)

\redrulem
{\expms{(\tyfun{\first{\vartym}}{\second{\vartym}})}{(\expfabsd{\varvars}{\varexps})}}
{\expfabss{\varvarm}{\tyunlabm{\first{\vartym}}}{\expms{\second{\vartym}}{(\expfapp{(\expfabsd{\varvars}{\varexps})}{(\expsm{\first{\vartym}}{\varvarm})})}}} \\

% ms (t->t) v

\redrulem
{\expms{(\tyfun{\first{\vartym}}{\second{\vartym}})}{\varvalus}}
{\expwrongs{\tyunlabm{(\tyfun{\first{\vartym}}{\second{\vartym}})}}{\str{Not \; a \; function}}} \\

\redsp $(\varvalus \neq \expfabsd{\varvars}{\varexps})$ \\

% (ms (Ax.t) v) {t}

\redrulem
{\exptapp{(\expms{(\tyfor{\tyvarm}{\first{\vartym}})}{\varvalus})}{\second{\vartym}}}
{\expms{\tysubst{\first{\vartym}}{\tylabel{\second{\vartym}}{\tyvarm}}{\tyvarm}}{\varvalus}} \\

\end{tabular}
\caption{ML-Scheme operational semantics}
\label{msos}
\end{figure}

\clearpage

\begin{figure}[p]
\centering
\begin{tabular}{lcl}
\vspace{5pt}

$e_{S}$ & $=$ & $v_{S}$ $\vert$ $x$ $\vert$ $e_{S}$ $e_{S}$ $\vert$ $o$ $e_{S}$ $e_{S}$ $\vert$ $p$ $e_{S}$ $\vert$ $\mathtt{if0}$ $e_{S}$ $e_{S}$ $e_{S}$ $\vert$ $\mathtt{cons}$ $e_{S}$ $e_{S}$ $\vert$ $f$ $e_{S}$ \\

\vspace{5pt}

&& $\vert$ $\mathtt{wrong}$ string $\vert$ $SH^{T}$ $e_{H}$ $\vert$ $SM^{T}$ $e_{M}$ \\

\vspace{5pt}

$v_{S}$ & $=$ & $\lambda x.e_{S}$ $\vert$ $\overline{n}$ $\vert$ $\mathtt{nil}$ $\vert$ $\mathtt{cons}$ $v_{S}$ $v_{S}$ $\vert$ $SH^{[T]}$ $(\mathtt{cons}$ $e_{H}$ $e_{H})$ $\vert$ $SH^{T^{a}}$ $v_{H}$ \\

\vspace{5pt}

&& $\vert$ $SM^{T^{a}}$ $v_{M}$ \\

\vspace{5pt}

$o$ & $=$ & $+$ $\vert$ $-$ \\

\vspace{5pt}

$p$ & $=$ & $\mathtt{fun?}$ $\vert$ $\mathtt{list?}$ $\vert$ $\mathtt{null?}$ $\vert$ $\mathtt{num?}$ \\

\vspace{5pt}

$f$ & $=$ & $\mathtt{hd}$ $\vert$ $\mathtt{tl}$ \\

\vspace{5pt}

$E_{S}$ & $=$ & $[\,]_{S}$ $\vert$ $E_{S}$ $e_{S}$ $\vert$ $v_{S}$ $E_{S}$ $\vert$ $o$ $E_{S}$ $e_{S}$ $\vert$ $o$ $\overline{n}$ $E_{S}$ $\vert$ $p$ $E_{S}$ $\vert$ $\mathtt{if0}$ $E_{S}$ $e_{S}$ $e_{S}$ \\

\vspace{5pt}

&& $\vert$ $\mathtt{cons}$ $E_{S}$ $e_{S}$ $\vert$ $\mathtt{cons}$ $v_{S}$ $E_{S}$ $\vert$ $f$ $E_{S}$ $\vert$ $SH^{T}$ $E_{H}$ $\vert$ $SM^{T}$ $E_{M}$
\end{tabular}
\caption{Scheme grammar and evaluation contexts}
\label{csg}
\end{figure}

\clearpage

\begin{figure}[p]
\[
\frac{}{\vdash^s\mathtt{TST}}
\]
\bigskip
\[
\frac{\Gamma,x^s:\mathtt{TST}\vdash^se^s:\mathtt{TST}}{\Gamma\vdash^s\lambda x^s.e^s:\mathtt{TST}}
\quad
\frac{}{\vdash^s\overline{n}:\mathtt{TST}}
\quad
\frac{}{\vdash^s\mathtt{nil}:\mathtt{TST}}
\]
\[
\frac{\Gamma\vdash^se^s_1:\mathtt{TST}\quad\Gamma\vdash^se^s_2:\mathtt{TST}}{\Gamma\vdash^s\mathtt{cons}\;e^s_1\;e^s_2:\mathtt{TST}}
\quad
\frac{}{\Gamma,x^s:\mathtt{TST}\vdash^sx^s:\mathtt{TST}}
\]
\[
\frac{\Gamma\vdash^se^s_1:\mathtt{TST}\quad\Gamma\vdash^se^s_2:\mathtt{TST}}{\Gamma\vdash^se^s_1\;e^s_2:\mathtt{TST}}
\quad
\frac{\Gamma\vdash^se^s:\mathtt{TST}}{\Gamma\vdash^sf\;e^s:\mathtt{TST}}
\]
\[
\frac{\Gamma\vdash^se^s_1:\mathtt{TST}\quad\Gamma\vdash^se^s_2:\mathtt{TST}}{\Gamma\vdash^so\;e^s_1\;e^s_2:\mathtt{TST}}
\quad
\frac{\Gamma\vdash^se^s:\mathtt{TST}}{\Gamma\vdash^sp\;e^s:\mathtt{TST}}
\]
\[
\frac{\Gamma\vdash^se^s_1:\mathtt{TST}\quad\Gamma\vdash^se^s_2:\mathtt{TST}\quad\Gamma\vdash^se^s_3:\mathtt{TST}}{\Gamma\vdash^s\mathtt{if0}\;e^s_1\;e^s_2\;e^s_3:\mathtt{TST}}
\quad
\frac{}{\vdash^s\mathtt{wrong}\;\mathrm{string}:\mathtt{TST}}
\]
\[
\frac{\Gamma\vdash^ht\quad\Gamma\vdash^he^h:t[t_i/t_i.x^h]}{\Gamma\vdash^s\mathtt{sh}\;t\;e^h:\mathtt{TST}}
\quad
\frac{\Gamma\vdash^mt\quad\Gamma\vdash^me^m:t[t_i/t_i.x^m]}{\Gamma\vdash^s\mathtt{sm}\;t\;e^m:\mathtt{TST}}
\]
\caption{Scheme typing rules}
\label{str}
\end{figure}

\clearpage

\begin{figure}[p]
\centering
\begin{tabular}{l}

% (\x.e) v

\redrules
{\expfapp{(\expfabsd{\varvars}{\varexps})}{\varvalus}}
{\expsubst{\varexps}{\varvalus}{\varvars}} \\

% w v

\redrules
{\expfapp{\varvalfs}{\varvalus}}
{\expwrongd{\str{Not \; a \; function}}}
$(\varvalfs \neq \expfabsd{\varvars}{\varexps})$ \\

% + n n

\redrules
{\expadd{\first{\expnum{\varnum}}}{\second{\expnum{\varnum}}}}
{\expnum{\first{\varnum} + \second{\varnum}}} \\

% - n n

\redrules
{\expsub{\first{\expnum{\varnum}}}{\second{\expnum{\varnum}}}}
{\expnum{\formvar{max}(\first{\varnum} - \second{\varnum}, 0)}} \\

% o w w

\redrules
{\expop{\first{\varvalfs}}{\second{\varvalfs}}}
{\expwrongd{\str{Not \; a \; number}}}
$(\first{\varvalfs} \neq \expnum{\varnum}$ or $\second{\varvalfs} \neq \expnum{\varnum})$ \\

% if0 0 e e

\redrules
{\expif{\expnum{0}}{\first{\varexps}}{\second{\varexps}}}
{\first{\varexps}} \\

% if0 n e e

\redrules
{\expif{\expnum{\varnum}}{\first{\varexps}}{\second{\varexps}}}
{\second{\varexps}}
$(\varnum \neq 0)$ \\

% if0 w e e

\redrules
{\expif{\varvalfs}{\first{\varexps}}{\second{\varexps}}}
{\expwrongd{\str{Not \; a \; number}}}
$(\varvalfs \neq \expnum{\varnum})$ \\

% f nil

\redrules
{\expfield{\expnild}}
{\expwrongd{\str{Empty \; list}}} \\

% hd (cons v v)

\redrules
{\exphd{(\expcons{\first{\varvalus}}{\second{\varvalus}})}}
{\first{\varvalus}} \\

% tl (cons v v)

\redrules
{\exptl{(\expcons{\first{\varvalus}}{\second{\varvalus}})}}
{\second{\varvalus}} \\

% f w

\redrules
{\expfield{\varvalfs}}
{\expwrongd{\str{Not \; a \; list}}}
$(\varvalfs \neq \expnild$ and $\varvalfs \neq \expcons{\first{\varvalus}}{\second{\varvalus}})$ \\

% fun? (\x.e)

\redrules
{\exppfun{(\expfabsd{\varvars}{\varexps})}}
{\expnum{0}} \\

% fun? w

\redrules
{\exppfun{\varvalfs}}
{\expnum{1}}
$(\varvalfs \neq \expfabsd{\varvars}{\varexps})$ \\

% list? nil

\redrules
{\expplist{\expnild}}
{\expnum{0}} \\

% list? (cons v v)

\redrules
{\expplist{(\expcons{\first{\varvalus}}{\second{\varvalus}})}}
{\expnum{0}} \\

% list? w

\redrules
{\expplist{\varvalfs}}
{\expnum{1}}
$(\varvalfs \neq \expnild$ and $\varvalfs \neq \expcons{\first{\varvalus}}{\second{\varvalus}})$ \\

% null? nil

\redrules
{\exppnull{\expnild}}
{\expnum{0}} \\

% null? w

\redrules
{\exppnull{\varvalfs}}
{\expnum{1}}
$(\varvalfs \neq \expnild)$ \\

% num? n

\redrules
{\exppnum{\expnum{\varnum}}}
{\expnum{0}} \\

% num? w

\redrules
{\exppnum{\varvalfs}}
{\expnum{1}}
$(\varvalfs \neq \expnum{\varnum})$ \\

% wrong string

\redrule
{\redcons{\expwrongd{\formvar{string}}}}
{\experr{\varstr}}

\end{tabular}
\caption{Scheme operational semantics}
\label{sos}
\end{figure}

\clearpage

\begin{figure}[p]
\centering
\begin{tabular}{l}
\vspace{5pt}

% sh - lump
$\mathscr{E}[SH^{L}$ $(^{L}HS$ $v_{S})]_{F}\rightarrow\mathscr{E}[v_{S}]$ \\

\vspace{5pt}

% sh - number
$\mathscr{E}[SH^{N}$ $\overline{n}]_{F}\rightarrow\mathscr{E}[\overline{n}]$ \\

\vspace{5pt}

% sh - list - nil
$\mathscr{E}[SH^{[T]}$ $\mathtt{nil}^{T[T_{i}/T_{i}^{a}]}]_{F}\rightarrow\mathscr{E}[\mathtt{nil}]$ \\

\vspace{5pt}

% sh - function
$\mathscr{E}[SH^{T_{1}\rightarrow T_{2}}$ $(\lambda x_{1}:T_{1}[T_{i}/T_{i}^{a}].e_{H})]_{F}\rightarrow$ \\

\vspace{5pt}

$\quad\mathscr{E}[\lambda x_{2}.(SH^{T_{2}}$ $((\lambda x_{1}:T_{1}[T_{i}/T_{i}^{a}].e_{H})$ $(^{T_{1}}HS$ $x_{2})))]$ \\

\vspace{5pt}

% sh - universal
$\mathscr{E}[SH^{\forall X.T}$ $(\Lambda X.e_{H})]_{F}\rightarrow\mathscr{E}[SH^{T[L/X]}$ $((\Lambda X.e_{H})$ $\lbrace L\rbrace)]$ \\

\vspace{5pt}

% sh - universal - hs
$\mathscr{E}[SH^{\forall X.T}$ $(^{\forall X.T}HS$ $v_{S})]_{F}\rightarrow\mathscr{E}[v_{S}]$ \\
\end{tabular}
\caption{Scheme-Haskell operational semantics}
\label{isos}
\end{figure}

\clearpage

\begin{figure}[p]
\centering
\begin{tabular}{l}
\vspace{5pt}

% sm - lump
$\mathscr{E}[SM^{L}$ $(^{L}MS$ $v_{S})]_{S}\rightarrow\mathscr{E}[v_{S}]$ \\

\vspace{5pt}

% sm - number
$\mathscr{E}[SM^{N}$ $\overline{n}]_{S}\rightarrow\mathscr{E}[\overline{n}]$ \\

\vspace{5pt}

% sm - list - nil
$\mathscr{E}[SM^{[T]}$ $\mathtt{nil}^{T[T_{i}/T_{i}^{a}]}]_{S}\rightarrow\mathscr{E}[\mathtt{nil}]$ \\

\vspace{5pt}

% sm - list - cons
$\mathscr{E}[SM^{[T]}$ $(\mathtt{cons}$ $v_{M}^{1}$ $v_{M}^{2})]_{S}\rightarrow\mathscr{E}[\mathtt{cons}$ $(SM^{T}$ $v_{M}^{1})$ $(SM^{[T]}$ $v_{M}^{2})]$ \\

\vspace{5pt}

% sm - list - mh cons
$\mathscr{E}[SM^{[T]}$ $(^{[T]}MH^{[T]}$ $(\mathtt{cons}$ $e_{H}^{1}$ $e_{H}^{2}))]_{S}\rightarrow\mathscr{E}[SH^{[T]}$ $(\mathtt{cons}$ $e_{H}^{1}$ $e_{H}^{2})]$ \\

\vspace{5pt}

% sm - function
$\mathscr{E}[SM^{T_{1}\rightarrow T_{2}}$ $(\lambda x_{1}:T_{1}[T_{i}/T_{i}^{a}].e_{M})]_{S}\rightarrow$ \\

\vspace{5pt}

$\quad\mathscr{E}[\lambda x_{2}.(SM^{T_{2}}$ $((\lambda x_{1}:T_{1}[T_{i}/T_{i}^{a}].e_{M})$ $(^{T_{1}}MS$ $x_{2})))]$ \\

\vspace{5pt}

% sm - universal
$\mathscr{E}[SM^{\forall X.T}$ $(\Lambda X.e_{M})]_{S}\rightarrow\mathscr{E}[SM^{T[L/X]}$ $((\Lambda X.e_{M})$ $\lbrace L\rbrace)]$ \\

\vspace{5pt}

% sm - universal - ms
$\mathscr{E}[SM^{\forall X.T}$ $(^{\forall X.T}MS$ $v_{S})]_{S}\rightarrow\mathscr{E}[v_{S}]$
\end{tabular}
\caption{Scheme-ML operational semantics}
\label{isos}
\end{figure}

\clearpage

\begin{figure}[p]
\centering
\begin{tabular}{rcl}

\tyunbrand{\cslump} & $=$ & \tylump \\
\tyunbrand{\csnum} & $=$ & \tynum \\
\tyunbrand{\csvarh} & $=$ & \tyvarh \\
\tyunbrand{\csvarm} & $=$ & \tyvarm \\
\tyunbrand{\cslist{\varcsh}} & $=$ & \cslist{\tyunbrand{\varcsh}} \\
\tyunbrand{\cslist{\varcsm}} & $=$ & \cslist{\tyunbrand{\varcsm}} \\
\tyunbrand{\csfun{\varcsh}{\varcsh}} & $=$ & \csfun{\tyunbrand{\varcsh}}{\tyunbrand{\varcsh}} \\
\tyunbrand{\csfun{\varcsm}{\varcsm}} & $=$ & \csfun{\tyunbrand{\varcsm}}{\tyunbrand{\varcsm}} \\
\tyunbrand{\csfor{\csvarh}{\varcsh}} & $=$ & \csfor{\csvarh}{\tyunbrand{\varcsh}} \\
\tyunbrand{\csfor{\csvarm}{\varcsm}} & $=$ & \csfor{\csvarm}{\tyunbrand{\varcsm}} \\
\tyunbrand{\csbrand{\vartyh}{\csvarh}} & $=$ & \vartyh \\
\tyunbrand{\csbrand{\vartym}{\csvarm}} & $=$ & \vartym \\

\end{tabular}
\caption{Unbrand function}
\label{unbrand}
\end{figure}

\begin{figure}[p]
\caption{The lump equality relation.}
\centering
\begin{tabular}{c}

$x \eq x$ \\
$x \eq y \Rightarrow y \eq x$ \\
$x \eq y$ and $y \eq z \Rightarrow x \eq z$ \\
$\vartyh \eq \tylump$ \\
$\vartym \eq \tylump$ \\
$\vartyh = \vartym \Rightarrow \vartyh \symlumpeq \vartym$ \\

\end{tabular}
\label{figequality}
\end{figure}
\chapter{Proof of Type Soundness}

The soundness of the type system must be proven for the model to be useful.  Soundness is ensured if progress of terms and preservation of types are ensured.  Progress ensures that a well-typed, closed term either is a value, reduces to another term, or reduces to an error.  Preservation ensures that if a well-typed term reduces to another term, the other term is well-typed and has the same type.  Proving progress and preservation proves soundness.

Progress will be proven by structural induction on a term of each syntactic form.  In each case, the term will be proven to be either a value, reducible to another term, or reducible to an error.  If a subterm is reducible to another term, the reduction of the term substitutes the new subterm for the old subterm.  If a subterm is reducible to an error, the term reduces to the error.  If no subterms are reducible but the term is reducible, the syntactic forms of its subterms must be determined to perform the reduction.  Since the term is well-typed, its subterms are well-typed, and thus the typing relations can be inverted to calculate its type from the types of its subterms.

\begin{lemma}

\label{leminv}

The syntactic forms of well-typed expressions determine the types of their subexpressions.

\begin{enumerate}

% Haskell

% \x:t.e

\item If \judeh{\env}{\expfabss{\varvarh}{\first{\vartyh}}{\varexph}}{\second{\vartyh}} then $\second{\vartyh} = \tyfun{\first{\vartyh}}{\third{\vartyh}}$, \judth{\env}{\first{\vartyh}}, and \judeh{\envexte{\varvarh}{\first{\vartyh}}}{\varexph}{\third{\vartyh}}.

% \\u.e

\item If \judeh{\env}{\exptabs{\tyvarh}{\varexph}}{\first{\vartyh}} then $\first{\vartyh} = \tyfor{\tyvarh}{\second{\vartyh}}$ and \judeh{\envextt{\tyvarh}}{\varexph}{\second{\vartyh}}.

% n

\item If \judeh{}{\expnum{\symnum}}{\vartyh} then $\vartyh = \tynum$.

% nil t

\item If \judeh{\env}{\expnils{\first{\vartyh}}}{\second{\vartyh}} then $\second{\vartyh} = \tylist{\first{\vartyh}}$ and \judth{\env}{\first{\vartyh}}.

% cons e e

\item If \judeh{\env}{\expcons{\first{\varexph}}{\second{\varexph}}}{\first{\vartyh}} then $\first{\vartyh} = \tylist{\second{\vartyh}}$, \judeh{\env}{\first{\varexph}}{\second{\vartyh}}, and \judeh{\env}{\second{\varexph}}{\tylist{\second{\vartyh}}}.

% x

\item \judeh{\envexte{\varvarh}{\vartyh}}{\varvarh}{\vartyh}.

% e e

\item If \judeh{\env}{\expfapp{\first{\varexph}}{\second{\varexph}}}{\first{\vartyh}} then \judeh{\env}{\first{\varexph}}{\tyfun{\second{\vartyh}}{\first{\vartyh}}} and \judeh{\env}{\second{\varexph}}{\second{\vartyh}}.

% fix e

\item If \judeh{\env}{\expfix{\varexph}}{\vartyh} then \judeh{\env}{\varexph}{\tyfun{\vartyh}{\vartyh}}.

% e<t>

\item If \judeh{\env}{\exptapp{\varexph}{\first{\vartyh}}}{\second{\vartyh}} then $\second{\vartyh} = \tysubst{\third{\vartyh}}{\first{\vartyh}}{\tyvarh}$, \judth{\env}{\vartyh}, and \judeh{\env}{\varexph}{\tyfor{\tyvarh}{\third{\vartyh}}}.

% hd e

\item If \judeh{\env}{\exphd{\varexph}}{\vartyh} then \judeh{\env}{\varexph}{\tylist{\vartyh}}.

% tl e

\item If \judeh{\env}{\exptl{\varexph}}{\first{\vartyh}} then $\first{\vartyh} = \tylist{\second{\vartyh}}$ and \judeh{\env}{\varexph}{\tylist{\second{\vartyh}}}.

% o e e

\item If $\Gamma\vdash_{A}o$ $e_{A}^{1}$ $e_{A}^{2}:T$ then $T=N$, $\Gamma\vdash_{A}e_{A}^{1}:N$, and $\Gamma\vdash_{A}e_{A}^{2}:N$ where $A\in\lbrace H,M\rbrace$.

\item If \judeh{\env}{\expop{\first{\varexph}}{\second{\varexph}}}{ % TODO

% null? e

\item If $\Gamma\vdash_{A}\mathtt{null?}$ $e_{A}:T$ then $T=N$ and $\Gamma\vdash_{A}e_{A}:[T_{1}]$ where $A\in\lbrace H,M\rbrace$.

% if0 e e e

\item If $\Gamma\vdash_{A}\mathtt{if0}$ $e_{A}^{1}$ $e_{A}^{2}$ $e_{A}^{3}:T$ then $T=T_{1}$, $\Gamma\vdash_{A}e_{A}^{1}:N$, $\Gamma\vdash_{A}e_{A}^{2}:T_{1}$, and $\Gamma\vdash_{A}e_{A}^{3}:T_{1}$ where $A\in\lbrace H,M\rbrace$.

% wrong t string

\item If $\Gamma\vdash_{A}\mathtt{wrong}^{T_{1}}$ $\mathrm{string}:T$ then $T=T_{1}$ where $A\in\lbrace H,M\rbrace$.

\item If $\Gamma\vdash_{A}{^{T_{1}}A}B^{T_{1}}$ $e_{B}:T$ then $T=T_{1}$, $\Gamma\vdash_{A}T_{1}$, $\Gamma\vdash_{B}T_{1}$, and $\Gamma\vdash_{B}e_{B}:T_{1}$ where $(A,B)\in\lbrace(H,M),(M,H)\rbrace$.

\item If $\Gamma\vdash_{A}{^{T_{1}}A}S$ $e_{S}:T$ then $T=T_{1}[T_{i}/T_{i}^{a}]$, $\Gamma\vdash_{A}T_{1}$, and $\Gamma\vdash_{S}e_{S}:TST$ where $A\in\lbrace H,M\rbrace$.

% ML

% Scheme

\item If $\Gamma\vdash_{S}\lambda x.e_{S}:TST$ then $\Gamma,x:TST\vdash_{S}e_{S}:TST$.

\item $\vdash_{S}\overline{n}:TST$.

\item $\vdash_{S}\mathtt{nil}:TST$.

\item If $\Gamma\vdash_{S}\mathtt{cons}$ $e_{S}^{1}$ $e_{S}^{2}:TST$ then $\Gamma\vdash_{S}e_{S}^{1}:TST$ and $\Gamma\vdash_{S}e_{S}^{2}:TST$.

\item If $\Gamma\vdash_{S}x:TST$ then $x:TST\in\Gamma$.

\item If $\Gamma\vdash_{S}e_{S}^{1}$ $e_{S}^{2}:TST$ then $\Gamma\vdash_{S}e_{S}^{1}:TST$ and $\Gamma\vdash_{S}e_{S}^{2}:TST$.

\item If $\Gamma\vdash_{S}f$ $e_{S}:TST$ then $\Gamma\vdash_{S}e_{S}:TST$.

\item If $\Gamma\vdash_{S}o$ $e_{S}^{1}$ $e_{S}^{2}:TST$ then $\Gamma\vdash_{S}e_{S}^{1}:TST$ and $\Gamma\vdash_{S}e_{S}^{2}:TST$.

\item If $\Gamma\vdash_{S}p$ $e_{S}:TST$ then $\Gamma\vdash_{S}e_{S}:TST$.

\item If $\Gamma\vdash_{S}\mathtt{if0}$ $e_{S}^{1}$ $e_{S}^{2}$ $e_{S}^{3}:TST$ then $\Gamma\vdash_{S}e_{S}^{1}:TST$, $\Gamma\vdash_{S}e_{S}^{2}:TST$, and $\Gamma\vdash_{S}e_{S}^{3}:TST$.

\item $\vdash_{S}\mathtt{wrong}$ $\mathrm{string}:TST$.

\item $\Gamma\vdash_{S}SA^{T_{1}}$ $e_{A}:TST$, $\Gamma\vdash_{A}T_{1}$, and $\Gamma\vdash_{A}e_{A}:T_{1}[T_{i}/T_{i}^{a}]$ where $A\in\lbrace H,M\rbrace$.

\end{enumerate}

\begin{proof}

Immediate from the typing rules.

\end{proof}

\end{lemma}


The type system ensures that a well-typed term has one unique type.

\begin{lemma}
\label{uot}
\onehalfspacing
$e_{A}$ has at most one type $T$ for a given context $\Gamma$ where $A\in\lbrace H,M\rbrace$.
\begin{proof}
By structural induction on $e_{A}$ using inversion (Lemma \ref{i}).
\end{proof}
\end{lemma}

Once the type of an irreducible subterm is determined, its syntactic form can be determined.

\begin{lemma}
\label{cf}
%\onehalfspacing
The possible syntactic forms of values of various types.
\begin{enumerate}
\item If $v_{A}:N$ then $v_{A}=\overline{n}$ where $A\in\lbrace H,M\rbrace$.
\item If $v_{A}:T_{1}\rightarrow T_{2}$ then $v_{A}=\lambda x:T_{1}.e_{A}$ where $A\in\lbrace H,M\rbrace$.
\item If $v_{A}:\forall X.T$ then $v_{A}\in\lbrace\Lambda X.e_{A},{^{\forall X.T}A}S$ $v_{S}\rbrace$ where $A\in\lbrace H,M\rbrace$.
\item If $v_{H}:[T]$ then $v_{H}\in\lbrace\mathtt{cons}$ $e_{H}^{1}$ $e_{H}^{2},\mathtt{nil}^{T}\rbrace$.
\item If $v_{M}:[T]$ then $v_{M}\in\lbrace\mathtt{cons}$ $v_{M}^{1}$ $v_{M}^{2},\mathtt{nil}^{T},{^{[T]}M}H^{[T]}$ $(\mathtt{cons}$ $e_{H}^{1}$ $e_{H}^{2})\rbrace$.
\item If $v_{A}:L$ then $v_{A}={^{L}A}S$ $v_{S}$ where $A\in\lbrace H,M\rbrace$.
\end{enumerate}
\begin{proof}
Immediate from the definitions of values and the typing relations.
\end{proof}
\end{lemma}

If the term is irreducible, it is a value.

\begin{theorem}
\label{ps}
%\onehalfspacing
If $\vdash_{A}e_{A}:T$ then $e_{A}$ is a value or $e_{A}\rightarrow e_{A}'$ or $e_{A}\rightarrow$ \emph{\textbf{Error}:\;string} where $A\in\lbrace H,M,S\rbrace$.
\begin{proof}
By structural induction on $e_{A}$.
\begin{case}

$e_{A}=\overline{n}$ where $A\in\lbrace H,M\rbrace$

$\overline{n}$ is an unforced value.

\end{case}
\input{proof/cases/progress/cons-h.tex}
\input{proof/cases/progress/cons-v-ms.tex}
\begin{case}
$e_{A}=\mathtt{nil}^{T}$ where $A\in\lbrace H,M\rbrace$

$\mathtt{nil}^{T}$ is a value.
\end{case}
\begin{case}
$^{[T]}B\;\mathtt{nil}\rightarrow\mathtt{nil}^{T}$ where $B\in\lbrace HS,MS\rbrace$

$\vdash_{HM}\,^{[T]}B\;\mathtt{nil}:[T]$ by premise and inversion (Lemma \ref{i}) and uniqueness of types (Lemma \ref{uot}).  $\vdash_{HM}\mathtt{nil}^{T}:[T]$ by inversion (Lemma \ref{i}) and uniqueness of types (Lemma \ref{uot}).
\end{case}
\input{proof/cases/progress/term-abstraction-hm.tex}
\input{proof/cases/progress/term-abstraction-s.tex}
\begin{case}

$e_{A}=\Lambda X.e_{A}^{1}$ where $A\in\lbrace H,M\rbrace$

$\Lambda X.e_{A}^{1}$ is a value.

\end{case}
\begin{case}

$e_{S}=x$

Cannot occur because $e_{S}$ is closed.

\end{case}
\input{proof/cases/progress/term-application-h.tex}
\input{proof/cases/progress/term-application-m.tex}
\input{proof/cases/progress/term-application-s.tex}
\begin{case}
$e_{A}=e_{A}^{1}\;\lbrace T_{1}\rbrace$ where $A\in\lbrace H,M\rbrace$

$e_{A}^{1}$ is a value or $e_{A}^{1}\rightarrow e_{A}^{2}$ or $e_{A}^{1}\rightarrow$ \emph{\textbf{Error}:\;string} by the induction hypothesis.  If $e_{A}^{1}$ is a value then $e_{A}^{1}:\forall X.T_{2}$ by inversion (Lemma \ref{i}) and uniqueness of types (Lemma \ref{uot}) and $e_{A}^{1}=\Lambda X.e_{A}^{3}$ or $e_{A}^{1}=\,^{\forall X.T_{2}}AS\;v_{S}$ by canonical forms (Lemma \ref{cf}).
\begin{subcase}
$e_{A}^{1}=\Lambda X.e_{A}^{3}$

$(\Lambda X.e_{A}^{3})\;\lbrace T_{1}\rbrace\rightarrow e_{A}^{3}[T_{1}/X]$.
\end{subcase}
\begin{subcase}
$e_{A}^{1}={^{\forall X.T_{2}}A}S\;v_{S}$

$(^{\forall X.T_{2}}AS\;v_{S})\;\lbrace T_{1}\rbrace\rightarrow{^{T_{2}[T_{1}^{a}/X]}A}S\;v_{S}$.
\end{subcase}
If $e_{A}^{1}\rightarrow e_{A}^{2}$ then $e_{A}^{1}\;\lbrace T_{1}\rbrace\rightarrow e_{A}^{2}\;\lbrace T_{1}\rbrace$.  If $e_{A}^{1}\rightarrow$ \emph{\textbf{Error}:\;string} then $e_{A}^{1}\;\lbrace T_{1}\rbrace\rightarrow$ \emph{\textbf{Error}:\;string}.
\end{case}
\begin{case}
$e_{A}=o\;e_{A}^{1}\;e_{A}^{2}$ where $A\in\lbrace H,M\rbrace$

$e_{A}^{1}$ is a value or $e_{A}^{1}\rightarrow e_{A}^{3}$ or $e_{A}^{1}\rightarrow$ \emph{\textbf{Error}:\;string} by the induction hypothesis.  If $e_{A}^{1}$ is a value then $e_{A}^{1}:N$ by inversion (Lemma \ref{i}) and uniqueness of types (Lemma \ref{uot}) and $e_{A}^{1}=\overline{n_{1}}$ by canonical forms (Lemma \ref{cf}).  If $e_{A}^{1}\rightarrow e_{A}^{3}$ then $o\;e_{A}^{1}\;e_{A}^{2}\rightarrow o\;e_{A}^{3}\;e_{A}^{2}$.  If $e_{A}^{1}\rightarrow$ \emph{\textbf{Error}:\;string} then $o\;e_{A}^{1}\;e_{A}^{2}\rightarrow$ \emph{\textbf{Error}:\;string}.  $e_{A}^{2}$ is a value or $e_{A}^{2}\rightarrow e_{A}^{4}$ or $e_{A}^{2}\rightarrow$ \emph{\textbf{Error}:\;string} by the induction hypothesis.  If $e_{A}^{2}$ is a value then $e_{A}^{2}:N$ by inversion (Lemma \ref{i}) and uniqueness of types (Lemma \ref{uot}) and $e_{A}^{2}=\overline{n_{2}}$ by canonical forms (Lemma \ref{cf}).  If $e_{A}^{2}\rightarrow e_{A}^{4}$ and $e_{A}^{1}$ is a value then $o\;e_{A}^{1}\;e_{A}^{2}\rightarrow o\;e_{A}^{1}\;e_{A}^{4}$.  If $e_{A}^{2}\rightarrow$ \emph{\textbf{Error}:\;string} and $e_{A}^{1}$ is a value then $o\;e_{A}^{1}\;e_{A}^{2}\rightarrow$ \emph{\textbf{Error}:\;string}.  If $e_{A}^{1}=\overline{n_{1}}$ and $e_{A}^{2}=\overline{n_{2}}$ then $o\;e_{A}^{1}\;e_{A}^{2}\rightarrow\overline{n_{1}+n_{2}}$ if $o=+$ or $o\;e_{A}^{1}\;e_{A}^{2}\rightarrow\overline{max(n_{1}-n_{2},0)}$ if $o=-$.
\end{case}
\input{proof/cases/progress/arithmetic-s.tex}
\input{proof/cases/progress/if0-hm.tex}
\input{proof/cases/progress/if0-s.tex}
\input{proof/cases/progress/cons-e-ms.tex}
\begin{case}
$e_{H}=f$ $e_{H}^{1}$

$e_{H}^{1}$ is a value or $e_{H}^{1}\rightarrow e_{H}^{2}$ or $e_{H}^{1}\rightarrow$ \emph{\textbf{Error}:\;string} by the induction hypothesis.  If $e_{H}^{1}$ is a value then $e_{H}^{1}:[T]$ by inversion (Lemma \ref{i}) and uniqueness of types (Lemma \ref{uot}) and $e_{H}^{1}\in\lbrace\mathtt{cons}$ $e_{H}^{3}$ $e_{H}^{4},\mathtt{nil}^{T}\rbrace$ by canonical forms (Lemma \ref{cf}).  If $e_{H}^{1}=\mathtt{cons}$ $e_{H}^{3}$ $e_{H}^{4}$ then $\mathtt{hd}$ $(\mathtt{cons}$ $e_{H}^{3}$ $e_{H}^{4})\rightarrow e_{H}^{3}$ and $\mathtt{tl}$ $(\mathtt{cons}$ $e_{H}^{3}$ $e_{H}^{4})\rightarrow e_{H}^{4}$.  If $e_{H}^{1}=\mathtt{nil}^{T}$ then $\mathtt{hd}$ $\mathtt{nil}^{T}\rightarrow{^{T}H}S$ $(\mathtt{wrong}$ \emph{``Empty list"}$)$ and $\mathtt{tl}$ $\mathtt{nil}^{T}\rightarrow\mathtt{nil}^{T}$.  If $e_{H}^{1}\rightarrow e_{H}^{2}$ then $f$ $e_{H}^{1}\rightarrow f$ $e_{H}^{2}$.  If $e_{H}^{1}\rightarrow$ \emph{\textbf{Error}:\;string} then $f$ $e_{H}^{1}\rightarrow$ \emph{\textbf{Error}:\;string}.
\end{case}
\begin{case}
$e_{M}=f\;e_{M}^{1}$

$e_{M}^{1}$ is a value or $e_{M}^{1}\rightarrow e_{M}^{2}$ or $e_{M}^{1}\rightarrow$ \emph{\textbf{Error}:\;string} by the induction hypothesis.  If $e_{M}^{1}$ is a value then $e_{M}^{1}:[T]$ by inversion (Lemma \ref{i}) and uniqueness of types (Lemma \ref{uot}) and $e_{M}^{1}\in\lbrace\mathtt{cons}\;e_{M}^{3}\;e_{M}^{4},\mathtt{nil}^{T},{^{[T]}M}H^{[T]}\;v_{H}\rbrace$ by canonical forms (Lemma \ref{cf}).  If $e_{M}^{1}=\mathtt{cons}\;e_{M}^{3}\;e_{M}^{4}$ then $\mathtt{hd}\;(\mathtt{cons}\;e_{M}^{3}\;e_{M}^{4})\rightarrow e_{M}^{3}$ or $\mathtt{tl}\;(\mathtt{cons}\;e_{M}^{3}\;e_{M}^{4})\rightarrow e_{M}^{4}$.  If $e_{M}^{1}=\mathtt{nil}^{T}$ then $\mathtt{hd}\;\mathtt{nil}^{T}\rightarrow{^{T}M}S\;(\mathtt{wrong}\;\mathrm{``Empty\;list"})$ and $\mathtt{tl}\;\mathtt{nil}^{T}\rightarrow\mathtt{nil}^{T}$.  If $e_{M}^{1}={^{[T]}M}H^{[T]}\;v_{H}$ then $v_{H}:[T]$ by inversion and uniqueness of types and $v_{H}\in\lbrace\mathtt{cons}\;e_{H}^{1}\;e_{H}^{2},\mathtt{nil}^{T}\rbrace$ by canonical forms.  If $v_{H}=\mathtt{cons}\;e_{H}^{1}\;e_{H}^{2}$ then $\mathtt{hd}\;(^{[T]}MH^{[T]}\;(\mathtt{cons}\;e_{H}^{1}\;e_{H}^{2}))\rightarrow{^{T}M}H^{T}\;e_{H}^{1}$ and $\mathtt{tl}\;(^{[T]}MH^{[T]}\;(\mathtt{cons}\;e_{H}^{1}\;e_{H}^{2}))\rightarrow\,^{[T]}MH^{[T]}\;e_{H}^{2}$.  If $v_{H}=\mathtt{nil}^{T}$ then $\mathtt{hd}\;\mathtt{nil}^{T}\rightarrow{^{T}M}S\;(\mathtt{wrong}\;\mathrm{``Empty\;list"})$ and $\mathtt{tl}\;\mathtt{nil}^{T}\rightarrow\mathtt{nil}^{T}$.  If $e_{M}^{1}\rightarrow e_{M}^{2}$ then $f\;e_{M}^{1}\rightarrow f\;e_{M}^{2}$.  If $e_{M}^{1}\rightarrow e_{M}^{2}$ then $f\;e_{M}^{1}\rightarrow f\;e_{M}^{2}$.  If $e_{M}^{1}\rightarrow$ \emph{\textbf{Error}:\;string} then $f\;e_{M}^{1}\rightarrow$ \emph{\textbf{Error}:\;string}.
\end{case}
\begin{case}
$e_{S}=f\;e_{S}^{1}$

$e_{S}^{1}$ is a value or $e_{S}^{1}\rightarrow e_{S}^{2}$ or $e_{S}^{1}\rightarrow$ \emph{\textbf{Error}:\;string} by the induction hypothesis.  If $e_{S}^{1}\rightarrow e_{S}^{2}$ then $f\;e_{S}^{1}\rightarrow f\;e_{S}^{2}$.  If $e_{S}^{1}\rightarrow$ \emph{\textbf{Error}:\;string} then $f\;e_{S}^{1}\rightarrow$ \emph{\textbf{Error}:\;string}.  $e_{S}^{1}$ is a value otherwise.  If $e_{S}^{1}=\mathtt{cons}\;e_{S}^{3}\;e_{S}^{4}$ then $\mathtt{hd}\;(\mathtt{cons}\;e_{S}^{3}\;e_{S}^{4})\rightarrow e_{S}^{3}$ and $\mathtt{tl}\;(\mathtt{cons}\;e_{S}^{3}\;e_{S}^{4})\rightarrow e_{S}^{4}$.  If $e_{S}^{1}=\mathtt{nil}$ then $\mathtt{hd}\;\mathtt{nil}\rightarrow\mathtt{wrong}\;\mathrm{``Empty\;list"}$ and $\mathtt{tl}\;\mathtt{nil}\rightarrow\mathtt{nil}$.  If $e_{S}^{1}=SH^{[T]}\;(\mathtt{cons}\;e_{H}^{1}\;e_{H}^{2})$ then $\mathtt{hd}\;(SH^{[T]}\;(\mathtt{cons}\;e_{H}^{1}\;e_{H}^{2}))\rightarrow SH^{T}\;e_{H}^{1}$ and $\mathtt{tl}\;(SH^{[T]}\;(\mathtt{cons}\;e_{H}^{1}\;e_{H}^{2}))\rightarrow SH^{[T]}\;e_{H}^{2}$.  $f\;e_{S}^{1}\rightarrow\mathtt{wrong}\;\mathrm{``Not\;a\;list"}$ otherwise.
\end{case}
\begin{case}
$e_{A}=\mathtt{ifnil}\;e_{A}^{1}\;e_{A}^{2}\;e_{A}^{3}$ where $A\in\lbrace H,M\rbrace$

$e_{A}^{1}$ is a value or $e_{A}^{1}\rightarrow e_{A}^{4}$ or $e_{A}^{1}\rightarrow$ \emph{\textbf{Error}:\;string} by the induction hypothesis.  If $e_{A}^{1}$ is a value then $e_{A}^{1}:[T]$ by inversion (Lemma \ref{i}) and uniqueness of types (Lemma \ref{uot}).  If $A=H$ then $e_{A}^{1}\in\lbrace\mathtt{cons}$ $e_{H}^{1}$ $e_{H}^{2},\mathtt{nil}^{T}\rbrace$ by canonical forms (Lemma \ref{cf}).  If $e_{A}^{1}=\mathtt{cons}$ $e_{H}^{1}$ $e_{H}^{2}$ then $\mathtt{ifnil}\;(\mathtt{cons}$ $e_{H}^{1}$ $e_{H}^{2})\;e_{A}^{2}\;e_{A}^{3}\rightarrow e_{A}^{3}$.  If $e_{A}^{1}=\mathtt{nil}^{T}$ then $\mathtt{ifnil}\;\mathtt{nil}^{T}\;e_{A}^{2}\;e_{A}^{3}\rightarrow e_{A}^{2}$.  If $A=M$ then $e_{A}^{1}\in\lbrace\mathtt{cons}$ $v_{M}^{1}$ $v_{M}^{2},\mathtt{nil}^{T},{^{[T]}M}H^{[T]}$ $(\mathtt{cons}$ $e_{H}^{1}$ $e_{H}^{2})\rbrace$ by canonical forms.  If $e_{A}^{1}=\mathtt{cons}$ $v_{M}^{1}$ $v_{M}^{2}$ then $\mathtt{ifnil}\;(\mathtt{cons}$ $v_{M}^{1}$ $v_{M}^{2})\;e_{A}^{2}\;e_{A}^{3}\rightarrow e_{A}^{3}$.  If $e_{A}^{1}=\mathtt{nil}^{T}$ then $\mathtt{ifnil}\;\mathtt{nil}^{T}\;e_{A}^{2}\;e_{A}^{3}\rightarrow e_{A}^{2}$.  If $e_{A}^{1}={^{[T]}M}H^{[T]}$ $(\mathtt{cons}$ $e_{H}^{1}$ $e_{H}^{2})$ then $\mathtt{ifnil}\;({^{[T]}M}H^{[T]}$ $(\mathtt{cons}$ $e_{H}^{1}$ $e_{H}^{2}))\;e_{A}^{2}\;e_{A}^{3}\rightarrow e_{A}^{3}$.  If $e_{A}^{1}\rightarrow e_{A}^{4}$ then $\mathtt{ifnil}\;e_{A}^{1}\;e_{A}^{2}\;e_{A}^{3}\rightarrow \mathtt{if0}\;e_{A}^{4}\;e_{A}^{2}\;e_{A}^{3}$.  If $e_{A}^{1}\rightarrow$ \emph{\textbf{Error}:\;string} then $\mathtt{ifnil}\;e_{A}^{1}\;e_{A}^{2}\;e_{A}^{3}\rightarrow$ \emph{\textbf{Error}:\;string}.
\end{case}
\begin{case}
$e_{S}=\mathtt{ifnil}\;e_{S}^{1}\;e_{S}^{2}\;e_{S}^{3}$

$e_{S}^{1}$ is a value or $e_{S}^{1}\rightarrow e_{S}^{4}$ or $e_{S}^{1}\rightarrow$ \emph{\textbf{Error}:\;string} by the induction hypothesis.  If $e_{S}^{1}$ is a value then $\mathtt{ifnil}\;e_{S}^{1}\;e_{S}^{2}\;e_{S}^{3}\rightarrow e_{S}^{2}$ if $e_{S}^{1}=\mathtt{nil}$ or $\mathtt{ifnil}\;e_{S}^{1}\;e_{S}^{2}\;e_{S}^{3}\rightarrow e_{S}^{3}$ otherwise.  If $e_{S}^{1}\rightarrow e_{S}^{4}$ then $\mathtt{ifnil}\;e_{S}^{1}\;e_{S}^{2}\;e_{S}^{3}\rightarrow \mathtt{ifnil}\;e_{S}^{4}\;e_{S}^{2}\;e_{S}^{3}$.  If $e_{S}^{1}\rightarrow$ \emph{\textbf{Error}:\;string} then $\mathtt{ifnil}\;e_{S}^{1}\;e_{S}^{2}\;e_{S}^{3}\rightarrow$ \emph{\textbf{Error}:\;string}.
\end{case}
\begin{case}
$\mathtt{fix}\;(\lambda x:T_{1}.e_{A})\rightarrow e_{A}[(\mathtt{fix}\;(\lambda x:T_{1}.e_{A}))/x]$ where $A\in\lbrace H,M\rbrace$

$\Gamma\vdash_{A}\mathtt{fix}\;(\lambda x:T_{1}.e_{A}):T$ by premise and uniqueness of types (Lemma \ref{uot}).  $T=T_{1}$, $\Gamma,x:T_{1}\vdash_{A}e_{A}:T_{1}$, and $\Gamma,x:T_{1}\vdash_{A}x:T_{1}$ by inversion (Lemma \ref{i}) and uniqueness of types.  $\Gamma\vdash_{A}e_{A}[(\mathtt{fix}\;(\lambda x:T_{1}.e_{A}))/x]:T_{1}$ by term substitution (Lemma \ref{tms}).  $\Gamma\vdash_{A}e_{A}[(\mathtt{fix}\;(\lambda x:T_{1}.e_{A}))/x]:T$ because $T_{1}=T$.
\end{case}
\begin{case}
$e_{S}=p\;e_{S}^{1}$

$e_{S}^{1}$ is a value or $e_{S}^{1}\rightarrow e_{S}^{2}$ or $e_{S}^{1}\rightarrow$ \emph{\textbf{Error}:\;string} by the induction hypothesis.  If $e_{S}^{1}\rightarrow e_{S}^{2}$ then $p\;e_{S}^{1}\rightarrow p\;e_{S}^{2}$.  If $e_{S}^{1}\rightarrow$ \emph{\textbf{Error}:\;string} then $p\;e_{S}^{1}\rightarrow$ \emph{\textbf{Error}:\;string}.  $e_{S}^{1}$ is a value otherwise.  If $p=\mathtt{nat?}$ then $p\;e_{S}^{1}\rightarrow\overline{0}$ if $e_{S}^{1}=\overline{n}$ and $p\;e_{S}^{1}\rightarrow\overline{1}$ otherwise.  If $p=\mathtt{list?}$ then $p\;e_{S}^{1}\rightarrow\overline{0}$ if $e_{S}^{1}\in\lbrace\mathtt{cons}\;e_{S}^{3}\;e_{S}^{4},\mathtt{nil}\rbrace$ and $p\;e_{S}^{1}\rightarrow\overline{1}$ otherwise.  If $p=\mathtt{proc?}$ then $p\;e_{S}^{1}\rightarrow\overline{0}$ if $e_{S}^{1}=\lambda x.e_{S}^{5}$ and $p\;e_{S}^{1}\rightarrow\overline{1}$ otherwise.
\end{case}
\input{proof/cases/progress/wrong-hm.tex}
\input{proof/cases/progress/wrong-s.tex}
\begin{case}
\label{ab}
$e_{A}={^{T}A}B^{T}$ $e_{B}^{1}$ where $(A,B,C)\in\lbrace(H,M,v),(M,H,e)\rbrace$

$e_{B}^{1}$ is a value or $e_{B}^{1}\rightarrow e_{B}^{2}$ or $e_{B}^{1}\rightarrow$ \emph{\textbf{Error}:\;string} by the induction hypothesis.  If $e_{B}^{1}$ is a value then $T$ determines its reduction.
\begin{subcase}
$T=N$

$e_{B}^{1}=\overline{n}$ by canonical forms (Lemma \ref{cf}).  $^{N}AB^{N}$ $\overline{n}\rightarrow\overline{n}$.
\end{subcase}
\begin{subcase}
$T=T_{1}\rightarrow T_{2}$

$e_{B}^{1}=\lambda x_{1}:T_{1}.e_{B}^{2}$ by canonical forms (Lemma \ref{cf}).  $^{T_{1}\rightarrow T_{2}}AB^{T_{1}\rightarrow T_{2}}$ $(\lambda x_{1}:T_{1}.e_{B}^{2})\rightarrow\lambda x_{2}:T_{1}[T_{i}/T^{a}_{i}].(^{T_{2}}AB^{T_{2}}$ $((\lambda x_{1}:T_{1}.e_{B}^{2})$ $(^{T_{1}}BA^{T_{1}}$ $x_{2})))$.
\end{subcase}
\begin{subcase}
$T=\forall X.T_{1}$

$e_{B}^{1}\in\lbrace\Lambda X.e_{B}^{2},{^{\forall X.T_{1}}B}S$ $v_{S}\rbrace$ by canonical forms (Lemma \ref{cf}).  If $e_{B}^{1}=\Lambda X.e_{B}^{2}$ then $^{\forall X.T_{1}}AB^{\forall X.T_{1}}$ $(\Lambda X.e_{B}^{2})\rightarrow\Lambda X.(^{T_{1}}AB^{T_{1}}$ $e_{B}^{2})$.  If $e_{B}^{1}={^{\forall X.T_{1}}B}S$ $v_{S}$ then $^{\forall X.T_{1}}AB^{\forall X.T_{1}}$ $(^{\forall X.T_{1}}BS$ $v_{S})\rightarrow{^{\forall X.T_{1}}A}S$ $v_{S}$.
\end{subcase}
\begin{subcase}
$T=[T_{1}]$

If $(A,B)=(H,M)$ then $e_{B}^{1}\in\lbrace\mathtt{cons}$ $v_{M}^{1}$ $v_{M}^{2},\mathtt{nil}^{T_{1}},{^{[T_{1}]}M}H^{[T_{1}]}$ $(\mathtt{cons}$ $e_{H}^{1}$ $e_{H}^{2})\rbrace$ by canonical forms (Lemma \ref{cf}).  If $e_{B}^{1}=\mathtt{cons}$ $v_{M}^{1}$ $v_{M}^{2}$ then $^{[T_{1}]}HM^{[T_{1}]}$ $(\mathtt{cons}$ $v_{M}^{1}$ $v_{M}^{2})\rightarrow\mathtt{cons}$ $(^{T_{1}}HM^{T_{1}}$ $v_{M}^{1})$ $(^{[T_{1}]}HM^{[T_{1}]}$ $v_{M}^{2})]$.  If $e_{B}^{1}=\mathtt{nil}^{T_{1}}$ then $^{[T_{1}]}HM^{[T_{1}]}$ $\mathtt{nil}^{T}\rightarrow\mathtt{nil}^{T}$.  If $e_{B}^{1}={^{[T_{1}]}M}H^{[T_{1}]}$ $(\mathtt{cons}$ $e_{H}^{1}$ $e_{H}^{2})$ then $^{[T_{1}]}HM^{[T_{1}]}$ $(^{[T_{1}]}MH^{[T_{1}]}$ $(\mathtt{cons}$ $e_{H}^{1}$ $e_{H}^{2}))\rightarrow\mathtt{cons}$ $e_{H}^{1}$ $e_{H}^{2}$.

If $(A,B)=(M,H)$ then $e_{B}^{1}\in\lbrace\mathtt{cons}$ $e_{H}^{3}$ $e_{H}^{4},\mathtt{nil}^{T_{1}}\rbrace$ by canonical forms.  If $e_{B}^{1}=\mathtt{cons}$ $e_{H}^{3}$ $e_{H}^{4}$ then $^{[T_{1}]}MH^{[T_{1}]}$ $(\mathtt{cons}$ $e_{H}^{3}$ $e_{H}^{4})$ is a value.  If $e_{B}^{1}=\mathtt{nil}^{T_{1}}$ then $^{[T_{1}]}MH^{[T_{1}]}$ $\mathtt{nil}^{T_{1}}\rightarrow\mathtt{nil}^{T_{1}}$.
\end{subcase}
\begin{subcase}
$T=L$

$e_{B}^{1}={^{L}B}S$ $v_{S}$ by canonical forms (Lemma \ref{cf}).  $^{L}AB^{L}$ $(^{L}BS$ $v_{S})\rightarrow{^{L}A}S$ $v_{S}$.
\end{subcase}
\begin{subcase}
$T=T_{1}^{a}$

Cannot occur because $T_{1}^{a}$ occurs only in $^{T_{1}^{a}}AS$ $e_{S}$.
\end{subcase}
If $e_{B}^{1}\rightarrow e_{B}^{2}$ then $^{T}AB^{T}$ $e_{B}^{1}\rightarrow{^{T}A}B$ $e_{B}^{2}$.  If $e_{B}^{1}\rightarrow$ \emph{\textbf{Error}:\;string} then $^{T}AB^{T}$ $e_{B}^{1}\rightarrow$ \emph{\textbf{Error}:\;string}.
\end{case}
\input{proof/cases/progress/as.tex}
\begin{case}
$e_{S}=SA^{T}\;e_{A}^{1}$ where $A\in\lbrace H,M\rbrace$

$e_{A}^{1}$ is a value or $e_{A}^{1}\rightarrow e_{A}^{2}$ or $e_{A}^{1}\rightarrow$ \emph{\textbf{Error}:\;string} by the induction hypothesis.  If $e_{A}^{1}$ is a value then $T$ determines its reduction.
\begin{subcase}
$T=N$

$e_{A}^{1}=\overline{n}$ by canonical forms (Lemma \ref{cf}).  $SA^{N}\;\overline{n}\rightarrow\overline{n}$.
\end{subcase}
\begin{subcase}
$T=T_{1}\rightarrow T_{2}$

$e_{A}^{1}=\lambda x_{1}:T_{1}[T_{i}/T_{i}^{a}].e_{A}^{3}$ by canonical forms (Lemma \ref{cf}).  $SA^{T_{1}\rightarrow T_{2}}\;(\lambda x_{1}:T_{1}[T_{i}/T_{i}^{a}].e_{A}^{3})\rightarrow\lambda x_{2}.(SA^{T_{2}}\;((\lambda x_{1}:T_{1}[T_{i}/T_{i}^{a}].e_{A}^{3})\;(^{T_{1}}AS\;x_{2})))$.
\end{subcase}
\begin{subcase}
$T=\forall X_{1}.T_{1}$

$e_{A}^{1}\in\lbrace\Lambda X_{1}.e_{A}^{3},{^{\forall X_{1}.T_{1}}A}S\;v_{S}\rbrace$ by canonical forms (Lemma \ref{cf}).  If $e_{A}^{1}=\Lambda X_{1}.e_{A}^{3}$ then $SA^{\forall X_{1}.T_{1}}\;(\Lambda X_{1}.e_{A}^{3})\rightarrow\Lambda X_{2}.(SA^{T_{1}[X_{2}/X_{1}]}\;((\Lambda X_{1}.e_{A}^{3})\;\lbrace X_{2}\rbrace))$.  If $e_{A}^{1}={^{\forall X_{1}.T_{1}}A}S\;v_{S}$ then it reduces by Case \ref{as}.
\end{subcase}
\begin{subcase}
$T=[T_{1}]$

If $A=H$ then $SA^{[T_{1}]}\;e_{A}$ is a value.  If $A=M$ then $e_{A}^{1}=\mathtt{cons}\;v_{M}^{1}\;v_{M}^{2}$ by canonical forms (Lemma \ref{cf}).  $SA^{[T_{1}]}\;\mathtt{cons}\;v_{M}^{1}\;v_{M}^{2})\rightarrow\mathtt{cons}\;(SA^{T}\;v_{M}^{1})\;(SA^{[T]}\;v_{M}^{2})$.
\end{subcase}
\begin{subcase}
$T=L$

$e_{A}^{1}={^{L}A}S\;v_{S}$ by canonical forms (Lemma \ref{cf}).  $SA^{L}\;(^{L}AS\;v_{S})\rightarrow v_{S}$.
\end{subcase}
\begin{subcase}
$T=T_{1}^{a}$

$SA^{T_{1}^{a}}\;e_{A}^{3}$ is a value.
\end{subcase}
If $e_{A}^{1}\rightarrow e_{A}^{2}$ then $SA^{T}\;e_{A}^{1}\rightarrow SA^{T}\;e_{A}^{2}$.  If $e_{A}^{1}\rightarrow$ \emph{\textbf{Error}:\;string} then $SA^{T}\;e_{A}^{1}\rightarrow$ \emph{\textbf{Error}:\;string}.
\end{case}
\end{proof}
\end{theorem}

Preservation will be proven by cases on the reduction rules.  In each case, the new term will be proven to be well-typed and have the same type as the old term.  Inversion (Lemma \ref{i}) and uniqueness of types (Lemma \ref{uot}) will be used to determine the type of the old term, the types of the subterms of the old term that occur within the new term, and the type of the new term.  Some reduction rules use substitution.

Term substitution substitutes one term for a second term within a third term.  The result of a term substitution has the same type as the third term.

\begin{lemma}
\label{tms}
\onehalfspacing
If $\Gamma,x:T_{1}\vdash_{A}e_{A}:T_{2}$ and $\Gamma\vdash_{A}y:T_{1}$ then $\Gamma\vdash_{A}e_{A}[y/x]:T_{2}$ where $A\in\lbrace H,M\rbrace$.  If $\Gamma,x:TST\vdash_{S}e_{S}:TST$ and $\Gamma\vdash_{S}y:TST$ then $\Gamma\vdash_{S}e_{S}[y/x]:TST$.
\begin{proof}
By structural induction.
\end{proof}
\end{lemma}

Type substitution substitutes a type for a type variable within a term.  The type of the result is the type substituted for the type variable within the term.

\begin{lemma}
\label{tes}
If $\Gamma,X\vdash_{HM}e_{HM}:T_{1}$ and $\vdash_{HM}T_{2}$ then $\Gamma\vdash_{HM}e_{HM}[T_{2}/X]:T_{1}[T_{2}/X]$.
\begin{proof}
By structural induction.
\end{proof}
\end{lemma}
\begin{lemma}
\label{ecp}
\onehalfspacing
If $\Gamma\vdash_{H}e_{H}^{1}:T_{1}$, $\Gamma\vdash_{H}e_{H}^{2}:T_{2}$, and $\mathscr{E}[e_{H}^{1}]:T_{1}$ then $\mathscr{E}[e_{H}^{2}]:T_{2}$.
\begin{proof}
By structural induction because typing rules use types and not forms of sub-terms.
\end{proof}
\end{lemma}
\begin{theorem}
\label{pn}
\onehalfspacing
If $\Gamma\vdash_{HMS}e_{HMS}^{1}:T$ and $e_{HMS}^{1}\rightarrow e_{HMS}^{2}$ then $\Gamma\vdash_{HMS}e_{HMS}^{2}:T$.
\begin{proof}
By cases on the reduction $e_{HMS}^{1}\rightarrow e_{HMS}^{2}$.  Scheme reductions that do not contain Haskell or ML terms are omitted because demonstrating the preservation of $TST$ is straightforward.
\begin{case}
$(\lambda x:T_{1}.e_{HM}^{1})\;e_{HM}^{2}\rightarrow e_{HM}^{1}[e_{HM}^{2}/x]$

$\Gamma\vdash_{HM}(\lambda x:T_{1}.e_{HM}^{1})\;e_{HM}^{2}:T$ by the premise and uniqueness of types (Lemma \ref{uot}).  $\Gamma\vdash_{HM}\lambda x:T_{1}.e_{HM}^{1}:T_{1}\rightarrow T$, $\Gamma,x:T_{1}\vdash_{HM}e_{HM}^{1}:T$, $\Gamma\vdash_{HM}e_{HM}^{2}:T_{1}$, and $\Gamma,x:T_{1}\vdash_{HM}x:T_{1}$ by inversion (Lemma \ref{i}) and uniqueness of types.  $e_{HM}^{1}[e_{HM}^{2}/x]:T$ by term substitution (Lemma \ref{tms}).
\end{case}
\begin{case}
$(\Lambda X.e_{HM})\;\lbrace T_{1}\rbrace\rightarrow e_{HM}[T_{1}/X]$

$\Gamma\vdash_{HM}(\Lambda X.e_{H})\;\lbrace T_{1}\rbrace:T$ by premise and uniqueness of types (Lemma \ref{uot}).  $\Gamma\vdash_{HM}\Lambda X.e_{HM}:\forall X.T_{2}$, $\Gamma,X\vdash_{HM}e_{HM}:T_{2}$, and $T=T_{2}[T_{1}/X]$ by inversion (Lemma \ref{i}) and uniqueness of types.  $\Gamma\vdash_{HM}e_{HM}[T_{1}/X]:T_{2}[T_{1}/X]$ by type substitution (Lemma \ref{tes}).  $\Gamma\vdash_{HM}e_{HM}[T_{1}/X]:T$ because $T=T_{2}[T_{1}/X]$.
\end{case}
\begin{case}
$\mathtt{if0}\;\overline{0}\;e_{HM}^{1}\;e_{HM}^{2}\rightarrow e_{HM}^{1}$

$\Gamma\vdash_{HM}\mathtt{if0}\;\overline{0}\;e_{HM}^{1}\;e_{HM}^{2}:T$ by premise and uniqueness of types (Lemma \ref{uot}).  $\Gamma\vdash_{HM}e_{HM}^{1}:T$ by inversion (Lemma \ref{i}) and uniqueness of types.
\end{case}
\begin{case}
$\mathtt{if0}\;\overline{n}\;e_{HM}^{1}\;e_{HM}^{2}\rightarrow e_{HM}^{2}\;(n\neq0)$

$\Gamma\vdash_{HM}\mathtt{if0}\;\overline{n}\;e_{HM}^{1}\;e_{HM}^{2}:T$ by premise and uniqueness of types (Lemma \ref{uot}).  $\Gamma\vdash_{HM}e_{HM}^{2}:T$ by inversion (Lemma \ref{i}) and uniqueness of types.
\end{case}
\begin{case}
$+\;\overline{n_{1}}\;\overline{n_{2}}\rightarrow\overline{n_{1}+n_{2}}$

$\vdash_{HM}+\;\overline{n_{1}}\;\overline{n_{2}}:N$ by inversion (Lemma \ref{i}) and uniqueness of types (Lemma \ref{uot}).  $\vdash_{HM}\overline{n_{1}+n_{2}}:N$ by inversion and uniqueness of types.
\end{case}
\begin{case}
$-\;\overline{n_{1}}\;\overline{n_{2}}\rightarrow\overline{max(n_{1}-n_{2},0)}$ where $A\in\lbrace H,M\rbrace$

$\vdash_{A}-\;\overline{n_{1}}\;\overline{n_{2}}:N$ by inversion (Lemma \ref{i}) and uniqueness of types (Lemma \ref{uot}).  $\vdash_{A}\overline{max(n_{1}-n_{2},0)}:N$ by inversion and uniqueness of types.
\end{case}
\input{cases/preservation/head-cons.tex}
\begin{case}
$\mathtt{tl}\;(\mathtt{cons}\;e_{HM}^{1}\;e_{HM}^{2})\rightarrow e_{HM}^{2}$

$\Gamma\vdash_{HM}\mathtt{tl}\;(\mathtt{cons}\;e_{HM}^{1}\;e_{HM}^{2}):T$ by premise and uniqueness of types (Lemma \ref{uot}).  $\Gamma\vdash_{HM}e_{HM}^{2}:T$ by inversion and uniqueness of types (Lemma \ref{uot}).
\end{case}
\begin{case}
$\mathtt{hd}\;\mathtt{nil}^{T_{1}}\rightarrow\,^{T}HS\;(\mathtt{wrong}\;\mathrm{``Empty\;list"})$

$\Gamma\vdash_{HM}\mathtt{hd}\;\mathtt{nil}^{T_{1}}:T$ by premise and uniqueness of types (Lemma \ref{uot}).  $\Gamma\vdash_{HM}\,^{T}HS\;(\mathtt{wrong}\;\mathrm{``Empty\;list"}):T$ by inversion (Lemma \ref{i}) and uniqueness of types (Lemma \ref{uot}).
\end{case}
\begin{case}
$\mathtt{tl}\;\mathtt{nil}^{T_{1}}\rightarrow\mathtt{nil}^{T_{1}}$ where $A\in\lbrace H,M\rbrace$

$\Gamma\vdash_{A}\mathtt{tl}\;\mathtt{nil}^{T_{1}}:T$ by premise and uniqueness of types (Lemma \ref{uot}).  $\Gamma\vdash_{A}\mathtt{nil}^{T_{1}}:[T_{1}]$ and $T=[T_{1}]$ by inversion (Lemma \ref{i}) and uniqueness of types.  $\Gamma\vdash_{A}\mathtt{nil}^{T_{1}}:T$ because $[T_{1}]=T$.
\end{case}
\begin{case}
$\mathtt{fix}\;(\lambda x:T_{1}.e_{A})\rightarrow e_{A}[(\mathtt{fix}\;(\lambda x:T_{1}.e_{A}))/x]$ where $A\in\lbrace H,M\rbrace$

$\Gamma\vdash_{A}\mathtt{fix}\;(\lambda x:T_{1}.e_{A}):T$ by premise and uniqueness of types (Lemma \ref{uot}).  $T=T_{1}$, $\Gamma,x:T_{1}\vdash_{A}e_{A}:T_{1}$, and $\Gamma,x:T_{1}\vdash_{A}x:T_{1}$ by inversion (Lemma \ref{i}) and uniqueness of types.  $\Gamma\vdash_{A}e_{A}[(\mathtt{fix}\;(\lambda x:T_{1}.e_{A}))/x]:T_{1}$ by term substitution (Lemma \ref{tms}).  $\Gamma\vdash_{A}e_{A}[(\mathtt{fix}\;(\lambda x:T_{1}.e_{A}))/x]:T$ because $T_{1}=T$.
\end{case}
\begin{case}

$e_{A}=\overline{n}$ where $A\in\lbrace H,M\rbrace$

$\overline{n}$ is an unforced value.

\end{case}
\begin{case}
$^{\forall X_{1}.T}B^{\forall X_{1}.T}\;(\Lambda X_{1}.e_{HM})\rightarrow\Lambda X_{2}.(^{T[X_{2}/X_{1}]}B^{T[X_{2}/X_{1}]}\;((\Lambda X_{1}.e_{HM})\;\lbrace X_{2}\rbrace))$ where $B\in\lbrace HM,MH\rbrace$

$\Gamma\vdash_{HM}\,^{\forall X_{1}.T}B^{\forall X_{1}.T}\;(\Lambda X_{1}.e_{HM}):\forall X_{1}.T$ by premise and inversion (Lemma \ref{i}) and uniqueness of types (Lemma \ref{uot}).  NOT DONE.
\end{case}
\begin{case}
$e_{S}=\lambda x.e_{S}^{1}$

$\lambda x.e_{S}^{1}$ is a forced value.
\end{case}
\begin{case}
$^{[T]}HM^{[T]}\;(\mathtt{cons}\;v_{M}^{1}\;v_{M}^{2})\rightarrow\mathtt{cons}\;(^{T}HM^{T}\;v_{M}^{1})\;(^{[T]}HM^{[T]}\;v_{M}^{2})$

$\Gamma\vdash_{H}^{[T]}HM^{[T]}\;(\mathtt{cons}\;v_{M}^{1}\;v_{M}^{2}):[T]$ by premise and inversion (Lemma \ref{i}) and uniqueness of types (Lemma \ref{uot}).  $\Gamma\vdash_{M}v_{M}^{1}:T$ and $\Gamma\vdash_{M}v_{M}^{1}:[T]$ by inversion (Lemma \ref{i}) and uniqueness of types (Lemma \ref{uot}).  $\Gamma\vdash_{H}\,^{T}HM^{T}\;v_{M}^{1}:T$ and $\Gamma\vdash_{H}\,^{[T]}HM^{[T]}\;v_{M}^{2}:[T]$ by the induction hypothesis and uniqueness of types (Lemma \ref{uot}).  $\Gamma\vdash_{H}\mathtt{cons}\;(^{T}HM^{T}\;v_{M}^{1})\;(^{[T]}HM^{[T]}\;v_{M}^{2}):[T]$.
\end{case}
\input{cases/preservation/list-cons-m.tex}
\input{cases/preservation/list-nil.tex}
%\input{cases/preservation/.tex}
\end{proof}
\end{theorem}
\chapter{Implementation of the Model}

The model of computation defined above was implemented in the DrScheme integrated development environment.  Haskell and ML modules are written in subsets of their real syntaxes and compiled to Scheme modules.  Scheme modules are written with the Scheme language supported by the Scheme compiler of DrScheme.  A module defined in a particular language exports sets of declarations for the other languages and only imports sets of declarations from other modules meant only for its language.  Haskell can import ML and Scheme expressions, ML can import Scheme expressions, and Scheme can import Haskell expressions.  Haskell and ML use a form of type reconstruction called let-polymorphism instead of the explicit type system System F.  The domain of numbers for all languages is integers.  All languages have boolean values and operations, additional arithmetic operations, and composite data.  The majority of the ML implementation was taken from Kinghorn \ref{TODO: cite kinghorn}.

\section{Importation Syntax}

The model expressed language interoperation through nesting languages within each other as expressions.  Real-world code libraries interoperate through interfaces.  In the implementation, modules are the code libraries of interest.  Languages interoperate by importing a declaration exported by a module by specifying its name and type.  The Haskell syntax that imports ML and Scheme expressions is \texttt{:ml type "name"} and \texttt{:scheme type "name"}, respectively, where \texttt{type} and \texttt{name} are the types and names of the expressions, respectively.  The ML syntax that imports Scheme expressions is \texttt{name :G type}.  The Scheme syntax that imports Haskell expressions is \texttt{(:haskell name type)}.  For example, \texttt{(:scheme a $\rightarrow$ a "identity") 0} imports the Scheme identity function to Haskell and applies it to zero.

\section{Importation Type Verification}

The Haskell import expressions, the ML import expression for Scheme, and the Scheme import expression for Haskell verify the expected types match the actual types.  Where expressions from Haskell and ML are imported, their actual types can be checked immediately because Haskell and ML modules export the types of their exported declarations for import verification.  Where Scheme expressions are imported, their actual types can be checked only if they are not functions.  If they are functions, a DrScheme library that implements contracts for higher-order functions is used to delay the verifications of their types.

Contracts are the mechanism that verifies values crossing boundaries match their expected types.  The library also provides a mechanism for assigning blame \ref{TODO: cite contracts} to languages for type errors to indicate which is at fault.  If a Scheme value does not match its expected type, Scheme is at fault.  If an ML value does not match its expected type, ML is at fault.  If Scheme applies an ML function to the wrong value, Scheme is at fault.  ML cannot apply a Scheme function to the wrong type because such a function application expression would be ill-typed.

\section{Limiting Scheme}

The real-world Scheme of DrScheme represents Scheme within the system of interoperation.  The real-world Scheme has language constructs not supported or allowed by Haskell and ML.  Therefore Scheme modules are presumed to use only a subset of the features of Scheme that are compatible with Haskell and ML.  For example, a Scheme module could use side effects to break parametricity by determining the behavior of a function by the state of a global variable.
\chapter{Related Work}

This work extends the work of Kinghorn \cite{kinghorn07} by adding Haskell and lists to his model of computation, his proof of type soundness, and his implementation of the model.  Kinghorn extended the work of Matthews and Findler \cite{matthews07} by adding parametric polymorphism and parametricity to their model of computation, providing a more thorough proof of its type soundness, and implementing it with a fully-featured Scheme and a subset of Objective Caml, a dialect of ML.

Guha et al. \cite{guha07} describe a system of parametric polymorphic contracts for higher-order functions that assign blame for contract violations and protect parametricity.  The system both ensures function arguments match the contract parameters and prevents functions from examining the types and values of their arguments. This work uses two separate mechanisms, boundary expressions and label types, to achieve the same result.

Perhaps the most mainstream systems of interoperation are the Common Object Request Broker Architecture (CORBA), the Component Object Model (COM), and the .NET Framework, yet not one of them supports interoperation between Haskell, ML, and Scheme as this work does.  CORBA, COM, and the .NET Framework support the interoperation of static and dynamic type systems and strict evaluation, but not higher-order functions, parametric polymorphism, parametricity, or lazy evaluation \cite{omg04} \cite{microsoft07} \cite{ecma06}.

Tobin-Hochstadt and Felleisen \cite{tobin-hochstadt06} describe a system of mechanically translating programs written in a dynamically-typed language to an equivalent form in a similar, statically-typed language.  The system has higher-order functions, static and dynamic type systems, and strict evaluation, but not parametric polymorphism, parametricity, or lazy evaluation.  The system enables the interoperation of higher-order functions, dynamic type systems, and strict evaluation, but not parametric polymorphism, parametricity, static type systems, or lazy evaluation.  It uses contracts for higher-order functions to assign blame to languages for type errors, which this model does not do.

Henglein and Rehof \cite{henglein95} describe a system of polymorphic type inference for Scheme that infers types and run-time type operations, thereby giving a high-level translation from Scheme to ML.  ML programs cannot be translated to equivalent Scheme programs.  The system has higher-order functions, parametric polymorphism, parametricity, static and dynamic type systems, and strict evaluation, but not lazy evaluation.  The system enables the interoperation of higher-order functions, dynamic type systems, and strict evaluation, but not parametric polymorphism, parametricity, static type systems, or lazy evaluation.

Benton \cite{benton05} describes a system of embedding dynamically-typed languages within the statically-typed language ML and projecting dynamically-typed values back into ML.  The system has higher-order functions, parametric polymorphism, parametricity, static and dynamic type systems, and strict evaluation, but does not have lazy evaluation.  The system enables the interoperation of higher-order functions, parametric polymorphism, static and dynamic type systems, and strict evaluation, but not parametricity or lazy evaluation.
\chapter{Future Work}

The model of computation is sufficient to express the interoperation of languages with conflicting evaluation strategies and demonstrate the resolution of those conflicts with lists.  Certainly other data types could be added to the model, but they would add nothing new to the method of resolving conflicts between evaluation strategies and would further complicate the model.  The implementation of the model would benefit from additional support of standard language constructs and data types because its ability to create useful programs would much improve.  Compiling language modules to a more efficient implementation would increase the speed and usability of the implementation.  Adding languages with other evaluation strategies, such as normal order and applicative order, would be interesting but would grow the size and complexity of the model and proof at an exponential rate.  Further explorations of conflicting evaluation strategies would best be tackled with pairs of languages to minimize complexity.

!!! TODO: A set of $n$ interoperable languages requires $n\times(n-1)$ boundaries, Three interoperable languages require six boundaries.
\chapter{Conclusions}

This work sought to resolve three types of conflict between languages in a system of interoperation.  It resolves conflicting type systems by converting between equivalent expressions and applying type abstractions to lump types.  It resolves conflicting preservation of parametricity by making Haskell and ML values opaque to Scheme with label types.  It resolves conflicting evaluation strategies by delaying the conversion of lists until their contents are needed.  It enables the interoperation of functions with dynamic checks and contracts for higher-order functions.  It defined a model of computation that can express language interoperation where such incompatibilities arise, provided a proof of type soundness, and described an implementation of the model that supported additional language features.
\clearpage
\bibliography{bibliography}
\bibliographystyle{plain}
\end{document}