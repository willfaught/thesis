\documentclass[12pt]{ucthesis}
\newif \ifpdf
\ifx \pdfoutput \undefined
  \pdffalse
\else
  \pdfoutput=1
  \pdftrue
\fi
\usepackage{url}
\ifpdf
  \usepackage[pdftex]{graphicx}
  \usepackage[pdftex,
    plainpages=false,
    breaklinks=true,
    colorlinks=true,
    urlcolor=blue,
    citecolor=blue,
    linkcolor=blue,
    bookmarks=true,
    bookmarksopen=true,
    bookmarksopenlevel=3,
    pdfstartview=FitV,
    pdfauthor={William Faught},
    pdftitle={Interoperation Between Haskell, ML, and Scheme},
    pdfkeywords={thesis, masters, cal poly}]{hyperref}
  \pdfcompresslevel=1
\else
  \usepackage{graphicx}
\fi

\usepackage{amsmath}
\usepackage{amssymb}
\usepackage{amsthm}
\usepackage[letterpaper]{geometry}
\usepackage{mathrsfs}
\usepackage{setspace}
\usepackage[overload]{textcase}

\newtheorem{theorem}{Theorem}
\newtheorem{lemma}{Lemma}
\newtheorem{case}{Case}[theorem]
\newtheorem{subcase}{Case}[case]

\bibliographystyle{abbrv}

\setlength{\parindent}{0.25in} \setlength{\parskip}{6pt}

\geometry{verbose,nohead,tmargin=1.25in,bmargin=1in,lmargin=1.5in,rmargin=1.3in}

\setcounter{tocdepth}{2}

\newcommand{\captionfonts}{\small\bf\ssp}

\makeatletter
\long\def\@makecaption#1#2{%
  \vskip\abovecaptionskip
  \sbox\@tempboxa{{\captionfonts #1: #2}}
  \ifdim \wd \@tempboxa > \hsize
    {\captionfonts #1: #2\par}
  \else
    \hbox to \hsize{\hfil\box\@tempboxa\hfil}
  \fi
  \vskip\belowcaptionskip}
\makeatother

\begin{document}

\title{Interoperation Between Haskell, ML, and Scheme}
\author{William Faught}
\degreemonth{June}
\degreeyear{2008}
\degree{Master of Science}
\defensemonth{September}
\defenseyear{2008}
\numberofmembers{3}
\chair{Dr. John Clements}
\othermemberA{Dr. Gene Fisher}
\othermemberB{Dr. Aaron Keen}
\field{Computer Science}
\campus{San Luis Obispo}
\copyrightyears{seven}
\maketitle

\begin{frontmatter}

\copyrightpage
\approvalpage
\begin{abstract}
Programming language interoperation is the cooperation of software components written in different languages.  When two components cooperate, they exchange data across a language boundary that converts it from one form to another.  The boundary resolves conflicts between languages and must not violate the type system of either language.  This paper explores the resolution of conflicting type systems, parametricity support, and evaluation strategies for a system of interoperation for Haskell, ML, and Scheme by defining a model of computation and interoperation, providing a proof of its type soundness, and describing an implementation of it.
\end{abstract}
\begin{acknowledgements}
\indent\indent I want to thank my father and mother, Jerry and Jo Ann, for their encouragement, advice, and support, without which this would not have been possible.

I want to thank my adviser, John Clements, for helping me along the way.  I very much appreciate the time he set aside for me and his advice.
\end{acknowledgements}
\addcontentsline{toc}{chapter}{Contents}
\tableofcontents
\listoffigures

\end{frontmatter}

\pagestyle{plain}
\renewcommand{\baselinestretch}{1.66}

\chapter{Introduction}

Software components written in different programming languages can cooperate through interoperation.  Interoperation is a client-server interaction concerning the reuse of server resources by clients.  Servers send data and clients receive data across a boundary between their languages that converts the data into a form understandable to clients.  The data exchange must not violate the type system or any property of either language for interoperation to be useful.  Differences between languages---incompatibilities---complicate interoperation, especially where equivalent forms of data do not exist.  This paper explores and resolves three such incompatibilities with a model of computation, gives a thorough proof of its type soundness, and describes an implementation of it.

The first incompatibility is the type system.  Static type systems calculate and validate the types of expressions before run time.  They guarantee that well-typed programs will not encounter type errors during run time.  Run-time type calculations and validations are unnecessary.  Dynamic type systems calculate and validate the types of expressions during run time to detect type errors.  Programs can explicitly calculate the types of expressions with type predicates and report type errors with an error mechanism.  If they detect a type error, they report it and terminate the computation.

Statically-typed languages---languages that use static type systems---must verify that the actual type of data received from a server matches their expected type.  Mismatched types may cause type errors during run time, which statically-typed languages cannot detect because they do not calculate or validate types during run time.  If data was sent from statically-typed languages, their actual type can be calculated and validated before run time.  If data was sent from dynamically-typed languages, their actual type cannot be calculated until run time, except functions.  Since dynamically-typed functions can produce results of various types, and determining function behavior is undecidable \cite{blume04}, equivalent types for these functions cannot be reliably calculated.  Since the types of dynamically-typed functions cannot be reliably calculated at boundaries, those functions are wrapped in contracts \cite{findler02} that calculate and validate the types of their arguments and results during run time.

The second incompatibility is the support for parametricity.  Parametricity ensures that parametric polymorphic functions from statically-typed languages behave the same regardless of the types and values of their arguments, and that those functions with variable result types produce as their results an argument with the same variable type.  Dynamically-typed functions can use type predicates and conditions to determine their behavior by the types and values of their arguments.  If dynamically-typed functions are used as parametric polymorphic functions in languages that support parametricity, they can break parametricity.  Arguments for these functions must be wrapped such that type predicates and conditions cannot examine them.

The third incompatibility is the evaluation strategy.  Evaluation strategies determine the order in which languages evaluate expressions.  Eager evaluation reduces expressions regardless of necessity, and lazy evaluation reduces expressions only where necessary.  Lazy languages---languages using lazy evaluation---can construct infinite streams as lists because they do not evaluate list elements when lists are constructed, but eager languages cannot because they do.  Therefore there are some lazy lists---lists in lazy languages---for which there are no equivalent eager lists.  Since lazy lists cannot be mechanically converted to eager lists, lazy lists cannot cross to eager languages.  Instead, elements of lazy lists can cross when accessed by eager languages if they are not lazy lists too.

The languages in the model must be able to express programs in which the aforementioned three incompatibilities arise.  Haskell, ML, and Scheme each possess a unique combination of properties that together are sufficient for this purpose.  Haskell and ML use static type systems and support parametricity.  ML and Scheme use eager evaluation.  Haskell uses lazy evaluation.  Scheme uses a dynamic type system.  Any more languages would uselessly complicate the model.

The rest of the paper is organized as follows: Chapter 2 defines the model of computation.  Chapter 3 proves the type soundness of the model.  Chapter 4 describes an implementation of the model.  Chapter 5 discusses related work.  Chapter 6 discusses future work.  Chapter 7 discusses the conclusions.
\newcommand{\haskell}{Haskell model\xspace}
\newcommand{\ml}{ML model\xspace}
\newcommand{\scheme}{Scheme model\xspace}

\newcommand{\haskellml}{Haskell and ML models\xspace}

\newcommand{\articlehaskell}[1]{#1 \haskell}
\newcommand{\articleml}[1]{#1 \ml}
\newcommand{\articlescheme}[1]{#1 \scheme}

\newcommand{\articlehaskellml}[1]{#1 \haskellml}

\newcommand{\thehaskell}{\articlehaskell{the}}
\newcommand{\Thehaskell}{\articlehaskell{The}}
\newcommand{\theml}{\articleml{the}}
\newcommand{\Theml}{\articleml{The}}
\newcommand{\thescheme}{\articlescheme{the}}
\newcommand{\Thescheme}{\articlescheme{The}}

\newcommand{\thehaskellml}{\articlehaskellml{the}}
\newcommand{\Thehaskellml}{\articlehaskellml{The}}

\newcommand{\hastype}[1]{has the type #1}
\newcommand{\havetype}[1]{have the type #1}

\chapter{Model of Computation}

The model of computation is based on that of Matthews and Findler \ref{matthews07}. Their model consists of two simple models, one representing ML and the other Scheme.

The ML model is a simply-typed lambda calculus extended with parametric polymorphism called System F. The substitution semantics by which type abstractions are applied means the ML model has parametricity, which is a property that ensures that programs behave the same regardless of the types applied to by type abstractions.  The ML model introduces new expressions, type abstractions and type applications, to express parametric polymorphism, and new types, type variables and forall types, for them. The ML model uses an eager evaluation strategy.

The Scheme model is an extended untyped lambda calculus using an eager evaluation strategy. Value predicates enable ad-hoc polymorphism. Uses a simple type system to check for free variables.

To this mix we introduce a Haskell model identical to the ML model, except it uses a lazy evaluation strategy.

The definitions of the Haskell, ML, and Scheme models begin in figures \ref{hg}, \ref{mg}, and \ref{sg}.

Expressions are written \varexp, types are written \varty, forced values are written \varvalf, unforced values are written \varvalu, forced evaluation contexts are written \varconf, and unforced evaluation contexts are written \varconu. Symbols that represent grammar non-terminals or relations typically have letter subscripts that specify a model.

\Thehaskellml have static type systems that use typing environments, written \env, and typing relations, written \jud. Typing judgements for expressions are written \jude{\env}{\jud}{\varexp}{\varty}, where \varexp is bound in \env and has the type \varty. Typing judgements for types are written \judt{\env}{\jud}{\varty} and mean \varty is bound in \env. Extended typing environments are written \envexte{\env}{\varvar}{\varty} for variables and \envextt{\env}{\tyvar} for type variables. Typing environments are omitted where empty. \Thescheme uses a simple type system to ensure no free variables. Every well-typed \scheme expression \hastype{\tytst}. Type substitution within types is written \tysubst{\formvar{x}}{\formvar{y}}{\formvar{z}}, where the type \formvar{y} is substituted for free occurrences of the type variable \formvar{z} in the type \formvar{x}.

TODO: opsem

Expression and type substitutions within expressions are written like type substitutions within types.

\section{Natural Numbers}

Natural number expressions are written \expnum{\varnum}, which syntactically denotes the natural number \varnum. Natural numbers \havetype{\tynum} for \thehaskellml. The arithmetic operations \expadd{\varexph}{\varexph} and \expsub{\varexph}{\varexph}, their operands, and \theml counterparts \havetype{\tynum}. In \thescheme, if either operand of an arithmetic expression is not a natural number, the operation reduces to an error, written \expwrongd{\errnum}. Conditional expressions test whether an expression is the natural number \expnum{0}; if it is, it reduces to the first alternative, otherwise it reduces to the second alternative. In \thehaskellml, conditional test expressions \havetype{\tynum} and the alternative expressions have the same type. \Thescheme has predicates that determine whether values are functions, lists, empty lists, or natural numbers, written \exppfun{\varexps}, \expplist{\varexps}, \exppnull{\varexps}, and \exppnum{\varexps}. Predicates reduce to \expnum{0} if true and \expnum{1} if false.

\section{Lists}

TODO

\section{Functions}

TODO

\section{Types}

TODO

\section{Interoperation}

TODO

\clearpage

\begin{figure}[p]
\centering
\begin{tabular}{rcl}
% Expressions
$e_{H}$ & $=$ & $v_{H}\;|\;x\;|\;e_{H}\;e_{H}\;|\;\mathtt{fix}\;e_{H}\;|\;e_{H}\;\lbrace T\rbrace\;|\;f\;e_{H}\;|\;o\;e_{H}\;e_{H}\;|\;\mathtt{null?}\;e_{H}$ \\
&& $|\;\mathtt{if0}\;e_{H}\;e_{H}\;e_{H}\;|\;\mathtt{wrong}^{T}$ string $|\;^{T}HM\;e_{M}\;|\;^{T}HS\;e_{S}$ \\
% Unforced values
$v_{H}$ & $=$ & $\lambda x:T.e_{H}\;|\;\Lambda X.e_{H}\;|\;\overline{n}\;|\;\mathtt{nil}^{T}\;|\;\mathtt{cons}\;e_{H}\;e_{H}\;|\;^{L}HS\;v_{S}\;|\;^{\forall X.T}HS\;v_{S}$ \\
% Types
$T$ & $=$ & $L\;|\;N\;|\;X\;|\;[T]\;|\;T^{a}\;|\;T\rightarrow T\;|\;\forall X.T$ \\
% List operators
$f$ & $=$ & $\mathtt{hd}\;|\;\mathtt{tl}$ \\
% Arithmetic operators
$o$ & $=$ & $+\;|\;-$ \\
% Unforced evaluation contexts
$E_{H}$ & $=$ & $[\,]_{H}\;|\;E_{H}\;e_{H}\;|\;\mathtt{fix}\;E_{H}\;|\;E_{H}\;\lbrace T\rbrace\;|\;f\;E_{H}\;|\;o\;E_{H}\;e_{H}\;|\;o\;v_{H}\;E_{H}$ \\
&& $|\;\mathtt{null?}\;E_{H}\;|\;\mathtt{if0}\;E_{H}\;e_{H}\;e_{H}\;|\;^{T}HM\;E_{M}\;|\;^{T}HS\;E_{S}$
\end{tabular}
\caption{Haskell grammar and evaluation contexts}
\label{hg}
\end{figure}

\clearpage

\begin{figure}[p]
\[
% Lump
\frac
{}
{\judth{}{\tylump}}
\quad
% Number
\frac
{}
{\judth{}{\tynum}}
\quad
% Variable
\frac
{}
{\judth{\envextt{\tyvarh}}{\tyvarh}}
\]
\[
% List
\frac
{\judth{\env}{\vartyh}}
{\judth{\env}{\tylist{\vartyh}}}
\quad
% Label
\frac
{\judth{\env}{\vartyh}}
{\judth{\env}{\tylabel{\vartyh}{\tyvarh}}}
\quad
% Function
\frac
{\judth{\env}{\first{\vartyh}} \quad \judth{\env}{\second{\vartyh}}}
{\judth{\env}{\tyfun{\first{\vartyh}}{\second{\vartyh}}}}
\quad
% Forall
\frac
{\judth{\env, \tyvarh}{\vartyh}}
{\judth{\env}{\tyfor{\tyvarh}{\vartyh}}}
\]
\bigskip
\[
% Function abstraction
\frac
{\judth{}{\first{\vartyh}} \quad \judeh{\envexte{\varvarh}{\first{\vartyh}}}{\varexph}{\second{\vartyh}}}
{\judeh{\env}{(\expfabs{\varvarh}{\first{\vartyh}}{\varexph})}{\tyfor{\first{\vartyh}}{\second{\vartyh}}}}
\quad
% Type abstraction
\frac
{\judeh{\envextt{\tyvarh}}{\varexph}{\vartyh}}
{\judeh{\env}{\exptabs{\tyvarh}{\varexph}}{\tyfor{\tyvarh}{\vartyh}}}
\quad
% Number
\frac
{}
{\judeh{}{\expnum{n}}{\tynum}}
\]
\[
% Nil
\frac
{\judeh{\env}{\vartyh}}
{\judeh{\env}{\expnil{\vartyh}}{\tylist{\vartyh}}}
\quad
% Cons
\frac
{\judeh{\env}{\first{\varexph}}{\vartyh} \quad \judeh{\env}{\second{\varexph}}{\tylist{\vartyh}}}
{\judeh{\env}{\expcons{\first{\varexph}}{\second{\varexph}}}{\tylist{\vartyh}}}
\quad
% Variable
\frac
{}
{\judeh{\envexte{\tyvarh}{\vartyh}}{\tyvarh}{\vartyh}}
\]
\[
% Function application
\frac
{\judeh{\env}{\first{\varexph}}{\tyfun{\first{\vartyh}}{\second{\vartyh}}} \quad \judeh{\env}{\second{\varexph}}{\first{\vartyh}}}
{\env\symjudh\expfapp{\first{\varexph}}{\second{\varexph}}:\vartyh_2}
\quad
% Fix
\frac
{\judeh{\env}{\varexph}{\tyfun{\vartyh}{\vartyh}}}
{\judeh{\env}{\expfix{\varexph}}{\vartyh}}
\]
\[
% Type application
\frac
{\judth{\env}{\first{\vartyh}} \quad \judeh{\env}{\varexph}{\tyfor{\tyvarh}{\second{\vartyh}}}}
{\judeh{\env}{\exptapp{\varexph}{\first{\vartyh}}}{\tysubst{\second{\vartyh}}{\first{\vartyh}}{\tyvarh}}}
\quad
% Head
\frac
{\judeh{\env}{\varexph}{\tylist{\vartyh}}}
{\judeh{\env}{\exphd{\varexph}}{\vartyh}}
\quad
% Tail
\frac
{\judeh{\env}{\varexph}{\tylist{\vartyh}}}
{\judeh{\env}{\exptl{\varexph}}{\tylist{\vartyh}}}
\]
\[
% Arithmetic
\frac
{\judeh{\env}{\first{\varexph}}{\tynum} \quad \judeh{\env}{\second{\varexph}}{\tynum}}
{\judeh{\env}{\expop{\first{\varexph}}{\second{\varexph}}}{\tynum}}
\quad
% Null
\frac
{\judeh{\env}{\varexph}{\tylist{\vartyh}}}
{\judeh{\env}{\expnull{\varexph}}{\tynum}}
\]
\[
% If0
\frac
{\judeh{\env}{\first{\varexph}}{\tynum} \quad \judeh{\env}{\second{\varexph}}{\vartyh} \quad \judeh{\env}{\third{\varexph}}{\vartyh}}
{\judeh{\env}{\expifzero{\first{\varexph}}{\second{\varexph}}{\third{\varexph}}}{\vartyh}}
\quad
% Wrong
\frac
{\judth{\env}{\vartyh}}
{\judeh{\env}{\expwrongs{\vartyh}{\formvar{string}}}{\vartyh}}
\]
\[
% ML
\frac
{\judth{\env}{\vartyh} \quad \judem{\env}{\varexpm}{\vartym} \quad \vartyh=\vartym}
{\judeh{\env}{\exphm{\vartyh}{\varexpm}}{\vartyh}}
\quad
% Scheme
\frac
{\judth{\env}{\vartyh} \quad \judeh{\env}{\varexps}{\tytst}}
{\judeh{\env}{\exphs{\vartyh}{\varexps}}{\tyunlabh{\vartyh}}}
\]
\caption{Haskell typing rules}
\label{htr}
\end{figure}


\clearpage

\begin{figure}[p]
\centering
\begin{tabular}{l}
%\vspace{5pt}

$\mathscr{E}[(\lambda x:T.e_{H_1})$ $e_{H_2}]_{H}\rightarrow\mathscr{E}[e_{H_1}[e_{H_2}/x]]$ \\

%\vspace{5pt}

$\mathscr{E}[\mathtt{fix}$ $(\lambda x:T.e_{H})]_{H}\rightarrow\mathscr{E}[e_{H}[(\mathtt{fix}$ $(\lambda x:T.e_{H}))/x]]$ \\

%\vspace{5pt}

$\mathscr{E}[(\Lambda X.e_{H})$ $\lbrace T\rbrace]_{H}\rightarrow\mathscr{E}[e_{H}[T/X]]$ \\

%\vspace{5pt}

$\mathscr{E}[\mathtt{hd}$ $\mathtt{nil}^{T}]_{H}\rightarrow\mathscr{E}[\mathtt{wrong}^{T}$ ``Empty list"$]$ \\

%\vspace{5pt}

$\mathscr{E}[\mathtt{tl}$ $\mathtt{nil}^{T}]_{H}\rightarrow\mathscr{E}[\mathtt{wrong}^{[T]}$ ``Empty list"$]$ \\

%\vspace{5pt}

$\mathscr{E}[\mathtt{hd}$ $(\mathtt{cons}$ $e_{H}^{1}$ $e_{H}^{2})]_{H}\rightarrow\mathscr{E}[e_{H}^{1}]$ \\

%\vspace{5pt}

$\mathscr{E}[\mathtt{tl}$ $(\mathtt{cons}$ $e_{H}^{1}$ $e_{H}^{2})]_{H}\rightarrow\mathscr{E}[e_{H}^{2}]$ \\

%\vspace{5pt}

$\mathscr{E}[+$ $\overline{n_{1}}$ $\overline{n_{2}}]_{H}\rightarrow\mathscr{E}[\overline{n_{1}+n_{2}}]$ \\

%\vspace{5pt}

$\mathscr{E}[-$ $\overline{n_{1}}$ $\overline{n_{2}}]_{H}\rightarrow\mathscr{E}[\overline{max(n_{1}-n_{2},0)}]$ \\

%\vspace{5pt}

$\mathscr{E}[\mathtt{null?}$ $\mathtt{nil}^{T}]_{H}\rightarrow\mathscr{E}[\overline{0}]$ \\

%\vspace{5pt}

$\mathscr{E}[\mathtt{null?}$ $(\mathtt{cons}$ $e_{H}^{1}$ $e_{H}^{2})]_{H}\rightarrow\mathscr{E}[\overline{1}]$ \\

%\vspace{5pt}

$\mathscr{E}[\mathtt{if0}$ $\overline{0}$ $e_{H}^{1}$ $e_{H}^{2}]_{H}\rightarrow\mathscr{E}[e_{H}^{1}]$ \\

%\vspace{5pt}

$\mathscr{E}[\mathtt{if0}$ $\overline{n}$ $e_{H}^{1}$ $e_{H}^{2}]_{H}\rightarrow\mathscr{E}[e_{H}^{2}]$ $(n\neq0)$ \\

%\vspace{5pt}

$\mathscr{E}[\mathtt{wrong}^{T}$ string$]_{H}\rightarrow$ \textbf{Error}: string
\end{tabular}
\caption{Haskell operational semantics}
\label{hos}
\end{figure}

\clearpage

\begin{figure}[p]
\centering
\begin{tabular}{l}

% hm L (ms L v)

\redruleh
{\exphm{\tylump}{(\expms{\tylump}{\varvalfs})}}
{\exphs{\tylump}{\varvalfs}} \\

% hm N n

\redruleh
{\exphm{\tynum}{\expnum{\varnum}}}
{\expnum{\varnum}} \\

% hm [t] (nil t)

\redruleh
{\exphm{\tylist{\varcsh}}{(\expnils{\vartym})}}
{\expnils{\tyunbrand{\varcsh}}}
$(\tyunbrand{\varcsh} = \vartym)$ \\

% hm [t] (cons v v)

\redruleh
{\exphm{\tylist{\varcsh}}{(\expcons{\first{\varvalum}}{\second{\varvalum}})}}
{\expcons{(\exphm{\tyunbrand{\varcsh}}{\first{\varvalum}})}{(\exphm{\tylist{\tyunbrand{\varcsh}}}{\second{\varvalum}})}} \\

% hm (t->t) (\x:t.e)

\redrule
{\redconh{\exphm{(\tyfun{\first{\varcsh}}{\second{\varcsh}})}{(\expfabss{\varvarm}{\vartym}{\varexpm})}}}
{} \\

\redsp \redcon{\expfabss{\varvarh}{\tyunbrand{\first{\varcsh}}}{\exphm{\second{\varcsh}}{\expfapp{((\expfabss{\varvarm}{\vartym}{\varexpm})}{(\expmh{\vartym}{\varvarh})})}}} \\

% hm (Ax.t) (\\x.e)

\redruleh
{\exphm{(\csfor{\csvarh}{\varcsh})}{(\exptabs{\tyvarm}{\varexpm})}}
{\exptabs{\tyvarh}{\exphm{\varcsh}{(\exptapp{(\exptabs{\tyvarm}{\varexpm})}{\tyconv{\tyvarh}})}}} \\

\end{tabular}
\caption{Haskell-ML operational semantics}
\label{hmos}
\end{figure}

\clearpage

\begin{figure}[p]
\centering
\begin{tabular}{l}

% hs N n

\redruleh
{\exphs{\csnum}{\expnum{\varnum}}}
{{\expnum{\varnum}}} \\

% hs N v

\redruleh
{\exphs{\csnum}{\varvalfs}}
{\expwrongs{\tynum}{\errnum}}
$(\varvalfs \neq \expnum{\varnum})$ \\

% hs {k} nil

\redruleh
{\exphs{\cslist{\varcsh}}{\expnild}}
{\expnils{\tyunbrand{\varcsh}}} \\

% hs {k} (cons v v)

\redruleh
{\exphs{\cslist{\varcsh}}{(\expcons{\first{\varvalus}}{\second{\varvalus}})}}
{\expcons{(\exphs{\varcsh}{\first{\varvalus}})}{(\exphs{\cslist{\varcsh}}{\second{\varvalus}})}} \\

% hs {k} v

\redruleh
{\exphs{\cslist{\varcsh}}{\varvalfs}}
{\expwrongs{\tyunbrand{\cslist{\varcsh}}}{\errlist}} \\

\redsp $(\varvalfs \neq \expnild$ and $\varvalfs \neq \expcons{\first{\varvalus}}{\second{\varvalus}})$ \\

% hs (b.t) (sh (b.t) e)

\redruleh
{\exphs{(\csbrand{\varbrand}{\vartyh})}{(\expsh{(\csbrand{\varbrand}{\vartyh})}{\varexph})}}
{\varexph} \\

% hs (b.t) v

\redruleh
{\exphs{(\csbrand{\varbrand}{\vartyh})}{\varvalfs}}
{\expwrongs{\vartyh}{\errbrand}}
$(\varvalfs \neq \expsh{(\csbrand{\varbrand}{\vartyh})}{\varexph})$ \\

% hs (k->k) (\x.e)

\redruleh
{\exphs{(\csfun{\first{\varcsh}}{\second{\varcsh}})}{(\expfabsd{\varvars}{\varexps})}}
{\expfabss{\varvarh}{\tyunbrand{\first{\varcsh}}}{\exphs{\second{\varcsh}}{(\expfapp{(\expfabsd{\varvars}{\varexps})}{(\expsh{\first{\varcsh}}{\varvarh})})}}} \\

% hs (k->k) v

\redruleh
{\exphs{(\csfun{\first{\varcsh}}{\second{\varcsh}})}{\varvalfs}}
{\expwrongs{\tyunbrand{\csfun{\first{\varcsh}}{\second{\varcsh}}}}{\errfun}} \\

\redsp $(\varvalfs \neq \expfabsd{\varvars}{\varexps})$ \\

% hs (Au.k) w

\redruleh
{\exphs{(\csfor{\csvarh}{\varcsh})}{\varvalfs}}
{\exptabs{\tyvarh}{\exphs{\varcsh}{\varvalfs}}} \\

\end{tabular}
\caption{Haskell-Scheme operational semantics}
\label{fighsos}
\end{figure}

\clearpage

\begin{figure}[p]
\centering
\begin{tabular}{rcl}

\varexpm & $=$ & \varvarm $|$ \varvalum $|$ \expfapp{\varexpm}{\varexpm} $|$ \exptapp{\varexpm}{\vartym} $|$ \expfix{\varexpm} $|$ \expop{\varexpm}{\varexpm} $|$ \expif{\varexpm}{\varexpm}{\varexpm} \\

&& \expcons{\varexpm}{\varexpm} $|$ \expfield{\varexpm} $|$ \exppnull{\varexpm} $|$ \expwrongs{\vartym}{\formvar{string}} $|$ \expms{\vartym}{\varexps} \\

\varvalum & $=$ & \varvalfm $|$ \expmh{\vartym}{\vartyh}{\varexph} \\

\varvalfm & $=$ & \expfabss{\varvarm}{\vartym}{\varexpm} $|$ \exptabs{\tyvarm}{\varexpm} $|$ \expnum{\varnum} $|$ \expnils{\vartym} $|$ \expcons{\varvalum}{\varvalum} $|$ \expmh{\tylump}{\vartyh}{\varvaluh} \\

&& \expms{\tylump}{\varvalfs} \\

\vartym & $=$ & \tylump $|$ \tynum $|$ \tyvarm $|$ \tylist{\vartym} $|$ \tylabel{\vartym}{\tyvarm} $|$ \tyfun{\vartym}{\vartym} $|$ \tyfor{\tyvarm}{\vartym} \\

\formvar{\symop} & $=$ & \formsym{\symadd} $|$ \formsym{\symsub} \\

\formvar{\symfield} & $=$ & \formsym{\symhd} $|$ \formsym{\symtl} \\

\varconfm & $=$ & \varconum $|$ \expmh{\vartym}{\vartyh}{\varconfh} \\

\varconum & $=$ & \symholem $|$ \expfapp{\varconfm}{\varexpm} $|$ \expfapp{\varvalfm}{\varconum} $|$ \exptapp{\varconfm}{\vartym} $|$ \expfix{\varconfm} $|$ \expop{\varconfm}{\varexpm} $|$ \expop{\varvalfm}{\varconfm} \\

&& \expif{\varconfm}{\varexpm}{\varexpm} $|$ \expcons{\varconum}{\varexpm} $|$ \expcons{\varvalum}{\varconum} $|$ \expfield{\varconfm} $|$ \exppnull{\varconfm} \\

&& \expms{\vartym}{\varconfs}

\end{tabular}
\caption{ML grammar and evaluation contexts}
\label{mg}
\end{figure}

\clearpage

\begin{figure}[p]
\[
% L
\frac
{}
{\judtm{}{\tylump}}
\quad
% N
\frac
{}
{\judtm{}{\tynum}}
\quad
% x
\frac
{}
{\judtm{\envextt{\env}{\tyvarm}}{\tyvarm}}
\]
\[
% {t}
\frac
{\judtm{\env}{\vartym}}
{\judtm{\env}{\tylist{\vartym}}}
\quad
% t->t
\frac
{\judtm{\env}{\first{\vartym}} \quad \judtm{\env}{\second{\vartym}}}
{\judtm{\env}{\tyfun{\first{\vartym}}{\second{\vartym}}}}
\quad
% Au.t
\frac
{\judtm{\env, \tyvarm}{\vartym}}
{\judtm{\env}{\tyfor{\tyvarm}{\vartym}}}
\]
\bigskip
\[
% \x:t.e
\frac
{\judtm{\env}{\first{\vartym}} \quad \judem{\envexte{\env}{\varvarm}{\first{\vartym}}}{\varexpm}{\second{\vartym}}}
{\judem{\env}{(\expfabss{\varvarm}{\first{\vartym}}{\varexpm})}{\tyfun{\first{\vartym}}{\second{\vartym}}}}
\quad
% \\u.e
\frac
{\judem{\envextt{\env}{\tyvarm}}{\varexpm}{\vartym}}
{\judem{\env}{\exptabs{\tyvarm}{\varexpm}}{\tyfor{\tyvarm}{\vartym}}}
\quad
% n
\frac
{}
{\judem{}{\expnum{n}}{\tynum}}
\]
\[
% nil t
\frac
{\judtm{\env}{\vartym}}
{\judem{\env}{\expnils{\vartym}}{\tylist{\vartym}}}
\quad
% cons e e
\frac
{\judem{\env}{\first{\varexpm}}{\vartym} \quad \judem{\env}{\second{\varexpm}}{\tylist{\vartym}}}
{\judem{\env}{\expcons{\first{\varexpm}}{\second{\varexpm}}}{\tylist{\vartym}}}
\quad
% x
\frac
{}
{\judem{\envexte{\env}{\varvarm}{\vartym}}{\varvarm}{\vartym}}
\]
\[
% e e
\frac
{\judem{\env}{\first{\varexpm}}{\tyfun{\first{\vartym}}{\second{\vartym}}} \quad \judem{\env}{\second{\varexpm}}{\first{\vartym}}}
{\env\symjudh\expfapp{\first{\varexpm}}{\second{\varexpm}}:\second{\vartym}}
\quad
% fix e
\frac
{\judem{\env}{\varexpm}{\tyfun{\vartym}{\vartym}}}
{\judem{\env}{\expfix{\varexpm}}{\vartym}}
\]
\[
% e<t>
\frac
{\judtm{\env}{\first{\vartym}} \quad \judem{\env}{\varexpm}{\tyfor{\tyvarm}{\second{\vartym}}}}
{\judem{\env}{\exptapp{\varexpm}{\first{\vartym}}}{\tysubst{\second{\vartym}}{\first{\vartym}}{\tyvarm}}}
\quad
% hd e
\frac
{\judem{\env}{\varexpm}{\tylist{\vartym}}}
{\judem{\env}{\exphd{\varexpm}}{\vartym}}
\quad
% tl e
\frac
{\judem{\env}{\varexpm}{\tylist{\vartym}}}
{\judem{\env}{\exptl{\varexpm}}{\tylist{\vartym}}}
\]
\[
% o e e
\frac
{\judem{\env}{\first{\varexpm}}{\tynum} \quad \judem{\env}{\second{\varexpm}}{\tynum}}
{\judem{\env}{\expop{\first{\varexpm}}{\second{\varexpm}}}{\tynum}}
\quad
% null? e
\frac
{\judem{\env}{\varexpm}{\tylist{\vartym}}}
{\judem{\env}{\exppnull{\varexpm}}{\tynum}}
\quad
% ms t e
\frac
{\judtm{\env}{\tyunbrand{\varcsm}} \quad \judes{\env}{\varexps}{\tytst}}
{\judem{\env}{\expms{\varcsm}{\varexps}}{\tyunbrand{\varcsm}}}
\]
\[
% if0 e e e
\frac
{\judem{\env}{\first{\varexpm}}{\tynum} \quad \judem{\env}{\second{\varexpm}}{\vartym} \quad \judem{\env}{\third{\varexpm}}{\vartym}}
{\judem{\env}{\expif{\first{\varexpm}}{\second{\varexpm}}{\third{\varexpm}}}{\vartym}}
\quad
% wrong t string
\frac
{\judtm{\env}{\vartym}}
{\judem{\env}{\expwrongs{\vartym}{\formvar{string}}}{\vartym}}
\]
\[
% mh k k e
\frac
{\judtm{\env}{\vartym} \quad \judth{\env}{\first{\vartyh}} \quad \judeh{\env}{\varexph}{\second{\vartyh}} \quad \vartym \eq \vartyh \quad \first{\vartyh} = \second{\vartyh}}
{\judem{\env}{\expmh{\vartym}{\vartyh}{\first{\varexph}}}{\vartym}}
\]
\caption{ML typing rules}
\label{mtr}
\end{figure}


\clearpage

\begin{figure}[p]
\caption{ML operational semantics.}
\centering
\begin{tabular}{l}

% Function application

\redrulem
{\expfapp{(\expfabss{\varvarm}{\vartym}{\varexpm})}{\varvalum}}
{\expsubst{\varexpm}{\varvalum}{\varvarm}} \\

% (\\u.e)<t>

\redrulem
{\exptapp{(\exptabs{\tyvarm}{\varexpm})}{\vartym}}
{\expsubst{\varexpm}{\csbrand{\varbrand}{\vartym}}{\tyvarm}} \\

% Fix

\redrulem
{\expfix{(\expfabss{\varvarm}{\vartym}{\varexpm})}}
{\expsubst{\varexpm}{\expfix{(\expfabss{\varvarm}{\vartym}{\varexpm})}}{\varvarm}} \\

% Add

\redrulem
{\expadd{\first{\expnum{\varnum}}}{\second{\expnum{\varnum}}}}
{\expnum{\first{\varnum} + \second{\varnum}}} \\

% Subtract

\redrulem
{\expsub{\first{\expnum{\varnum}}}{\second{\expnum{\varnum}}}}
{\expnum{\formvar{max}(\first{\varnum} - \second{\varnum}, 0)}} \\

% If0 true

\redrulem
{\expif{\expnum{0}}{\first{\varexpm}}{\second{\varexpm}}}
{\first{\varexpm}} \\

% If0 false

\redrulem
{\expif{\expnum{\varnum}}{\first{\varexpm}}{\second{\varexpm}}}
{\second{\varexpm}}
$(\varnum \neq 0)$ \\

% Head nil

\redrulem
{\exphd{(\expnils{\vartym})}}
{\expwrongs{\vartym}{\str{Empty \; list}}} \\

% Tail nil

\redrulem
{\exptl{(\expnils{\vartym})}}
{\expwrongs{\tylist{\vartym}}{\str{Empty \; list}}} \\

% Head cons

\redrulem
{\exphd{(\expcons{\first{\varvalum}}{\second{\varvalum}})}}
{\first{\varvalum}} \\

% Tail cons

\redrulem
{\exptl{(\expcons{\first{\varvalum}}{\second{\varvalum}})}}
{\second{\varvalum}} \\

% Null nil

\redrulem
{\exppnull{(\expnils{\vartym})}}
{\expnum{0}} \\

% Null cons

\redrulem
{\exppnull{(\expcons{\first{\varvalum}}{\second{\varvalum}})}}
{\expnum{1}} \\

% wrong t string

\redrule
{\redconh{\expwrongs{\vartym}{\formvar{string}}}}
{\experr{\varstr}}

\end{tabular}
\label{figmos}
\end{figure}

\clearpage

\begin{figure}[p]
\onehalfspacing
\centering
\begin{tabular}{l}

% mh t L (hm L t e)

\redrulem
{\expmh{\first{\vartym}}{\tylump}{(\exphm{\tylump}{\second{\vartym}}{\varvalfm})}}
{\varvalfm}
$(\first{\vartym} = \second{\vartym}$ and $\first{\vartym} \neq \tylump)$ \\

% mh t L (hm L t e)

\redrulem
{\expmh{\first{\vartym}}{\tylump}{(\exphm{\tylump}{\second{\vartym}}{\varvalfm})}}
{\expwrongs{\vartym}{\errtype}}
$(\first{\vartym} \neq \second{\vartym}$ and $\first{\vartym} \neq \tylump)$ \\

% mh t L (hs L v)

\redruleh
{\expmh{\vartym}{\tylump}{(\exphs{\cslump}{\varvalfs})}}
{\expwrongs{\vartym}{\errvalue}}
$(\vartym \neq \tylump)$ \\

% mh N N n

\redrulem
{\expmh{\tynum}{\tynum}{\expnum{\varnum}}}
{\expnum{\varnum}} \\

% mh {t} {t} (nil t)

\redrulem
{\expmh{\tylist{\vartym}}{\tylist{\first{\vartyh}}}{(\expnils{\second{\vartyh}})}}
{\expnils{\vartym}} \\

% mh {t} {t} (cons v v)

\redrulem
{\expmh{\tylist{\vartym}}{\tylist{\vartyh}}{(\expcons{\first{\varexph}}{\second{\varexph}})}}
{\expcons{(\expmh{\vartym}{\vartyh}{\first{\varexph}})}{(\expmh{\tylist{\vartym}}{\tylist{\vartyh}}{\second{\varexph}})}} \\

% mh (t->t) (t->t) (\x:t.e)

\redrule
{\redconm{\expmh{(\tyfun{\first{\vartym}}{\second{\vartym}})}{(\tyfun{\first{\vartyh}}{\second{\vartyh}})}{(\expfabss{\varvarh}{\third{\vartyh}}{\varexph})}}}
{} \\

\redsp \redcon{\expfabss{\varvarm}{\first{\vartym}}{\expmh{\second{\vartym}}{\second{\vartyh}}{\expfapp{((\expfabss{\varvarh}{\third{\vartyh}}{\varexph})}{(\exphm{\first{\vartyh}}{\first{\vartym}}{\varvarm})})}}} \\

% mh (Au.t) (Au.t) (\\x.e)

\redrulem
{\expmh{(\tyfor{\tyvarm}{\vartym})}{(\tyfor{\first{\tyvarh}}{\vartyh})}{(\exptabs{\second{\tyvarh}}{\varexph})}}
{\exptabs{\tyvarm}{\expmh{\vartym}{\tysubst{\vartyh}{\tylump}{\tyvarh}}{\expsubst{\varexph}{\tylump}{\second{\tyvarh}}}}} \\

\end{tabular}
\caption{ML-Haskell operational semantics}
\label{mhos}
\end{figure}


\clearpage

\begin{figure}[p]
\centering
\begin{tabular}{l}

% ms N n

\redrulem
{\expms{\csnum}{\expnum{\varnum}}}
{{\expnum{\varnum}}} \\

% ms N w

\redrulem
{\expms{\csnum}{\varvalfs}}
{\expwrongs{\tynum}{\str{Not \; a \; number}}}
$(\varvalfs \neq \expnum{\varnum})$ \\

% ms {k} nil

\redrulem
{\expms{\cslist{\varcsm}}{\expnild}}
{\expnils{\tyunbrand{\varcsm}}} \\

% ms {k} (cons v v)

\redrulem
{\expms{\cslist{\varcsm}}{(\expcons{\first{\varvalus}}{\second{\varvalus}})}}
{\expcons{(\expms{\varcsm}{\first{\varvalus}})}{(\expms{\cslist{\varcsm}}{\second{\varvalus}})}} \\

% ms {k} w

\redrulem
{\expms{\cslist{\varcsm}}{\varvalfs}}
{\expwrongs{\tyunbrand{\cslist{\varcsm}}}{\str{Not \; a \; list}}} \\

\redsp $(\varvalfs \neq \expnild$ and $\varvalfs \neq \expcons{\first{\varvalus}}{\second{\varvalus}})$ \\

% ms (b.t) (sm (b.t) v)

\redrulem
{\expms{(\csbrand{\varbrand}{\vartym})}{(\expsm{(\csbrand{\varbrand}{\vartym})}{\varvalum})}}
{\varvalum} \\

% ms (b.t) v

\redrulem
{\expms{(\csbrand{\varbrand}{\vartym})}{\varvalfs}}
{\expwrongs{\tyunbrand{\csbrand{\varbrand}{\vartym}}}{\str{Brand \; mismatch}}} \\

\redsp $(\varvalfs \neq \expsm{(\csbrand{\varbrand}{\vartym})}{\varexpm})$ \\

% ms (k->k) (\x.e)

\redrule
{\redconm{\expms{(\csfun{\first{\varcsm}}{\second{\varcsm}})}{(\expfabsd{\varvars}{\varexps})}}}
{} \\

\redsp \redcon{\expfabss{\varvarm}{\tyunbrand{\first{\varcsm}}}{\expms{\second{\varcsm}}{(\expfapp{(\expfabsd{\varvars}{\varexps})}{(\expsm{\first{\varcsm}}{\varvarm})})}}} \\

% ms (k->k) v

\redrulem
{\expms{(\tyfun{\first{\varcsm}}{\second{\varcsm}})}{\varvalfs}}
{\expwrongs{\tyunbrand{\tyfun{\first{\varcsm}}{\second{\varcsm}}}}{\str{Not \; a \; function}}} \\

\redsp $(\varvalfs \neq \expfabsd{\varvars}{\varexps})$ \\

% ms (Au.k) w

\redrulem
{\expms{(\csfor{\csvarm}{\varcsm})}{\varvalfs}}
{\exptabs{\tyvarm}{\expms{\varcsm}{\varvalfs}}} \\

\end{tabular}
\caption{ML-Scheme operational semantics}
\label{msos}
\end{figure}

\clearpage

\begin{figure}[p]
\centering
\begin{tabular}{lcl}

$e^s$ & $=$ & $v^s$ $\vert$ $x^s$ $\vert$ $e^s$ $e^s$ $\vert$ $\mathtt{cons}$ $e^s$ $e^s$ $\vert$ $f$ $e^s$ $\vert$ $o$ $e^s$ $e^s$ $\vert$ $p$ $e^s$ $\vert$ $\mathtt{if0}$ $e^s$ $e^s$ $e^s$ \\

&& $\vert$ $\mathtt{wrong}$ string $\vert$ $\mathtt{sm}$ $t$ $e^m$ \\

$v^s$ & $=$ & $w^s$ $\vert$ $\mathtt{sh}$ $t$ $e^h$ \\

$w^s$ & $=$ & $\lambda x^s.e^s$ $\vert$ $\overline{n}$ $\vert$ $\mathtt{nil}$ $\vert$ $\mathtt{cons}$ $v^s$ $v^s$ $\vert$ $\mathtt{sm}$ $(t.x^m)$ $v^m$ \\

$f$ & $=$ & $\mathtt{hd}$ $\vert$ $\mathtt{tl}$ \\

$o$ & $=$ & $+$ $\vert$ $-$ \\

$p$ & $=$ & $\mathtt{fun?}$ $\vert$ $\mathtt{list?}$ $\vert$ $\mathtt{null?}$ $\vert$ $\mathtt{num?}$ \\

$E^s$ & $=$ & $[\,]^s$ $\vert$ $F^s$ $e^s$ $\vert$ $v^s$ $E^s$ $\vert$ $\mathtt{cons}$ $E^s$ $e^s$ $\vert$ $\mathtt{cons}$ $v^s$ $E^s$ $\vert$ $f$ $F^s$ $\vert$ $o$ $F^s$ $e^s$ \\

&& $\vert$ $o$ $v^s$ $F^s$ $\vert$ $p$ $F^s$ $\vert$ $\mathtt{if0}$ $F^s$ $e^s$ $e^s$ $\vert$ $\mathtt{sm}$ $t$ $E^m$ \\

$F^s$ & $=$ & $E^s$ $\vert$ $\mathtt{sh}$ $t$ $E^h$

\end{tabular}
\caption{Scheme grammar and evaluation contexts}
\label{sg}
\end{figure}

\clearpage

\begin{figure}[p]
\caption{The Scheme typing rules.}
\[
% TST
\frac
{}
{\judts{}{\tytst}}
\]
\bigskip
\[
% \x.e
\frac
{\judes{\envexte{\env}{\varvars}{\tytst}}{\varexps}{\tytst}}
{\judes{\env}{\expfabsd{\varvars}{\varexps}}{\tytst}}
\quad
% n
\frac
{}
{\judes{}{\expnum{n}}{\tytst}}
\quad
% nil
\frac
{}
{\judes{}{\expnild}{\tytst}}
\]
\[
% cons e e
\frac
{\judes{\env}{\first{\varexps}}{\tytst} \quad \judes{\env}{\second{\varexps}}{\tytst}}
{\judes{\env}{\expcons{\first{\varexps}}{\second{\varexps}}}{\tytst}}
\quad
% x
\frac
{}
{\judes{\envexte{\env}{\varvars}{\tytst}}{\varvars}{\tytst}}
\]
\[
% e e
\frac
{\judes{\env}{\first{\varexps}}{\tytst} \quad \judes{\env}{\second{\varexps}}{\tytst}}
{\env\symjudh\expfapp{\first{\varexps}}{\second{\varexps}}:\tytst}
\quad
% a e e
\frac
{\judes{\env}{\first{\varexps}}{\tytst} \quad \judes{\env}{\second{\varexps}}{\tytst}}
{\judes{\env}{\expop{\first{\varexps}}{\second{\varexps}}}{\tytst}}
\]
\[
% if0 e e e
\frac
{\judes{\env}{\first{\varexps}}{\tytst} \quad \judes{\env}{\second{\varexps}}{\tytst} \quad \judes{\env}{\third{\varexps}}{\tytst}}
{\judes{\env}{\expif{\first{\varexps}}{\second{\varexps}}{\third{\varexps}}}{\tytst}}
\quad
% p e
\frac
{\judes{\env}{\varexps}{\tytst}}
{\judes{\env}{\exppred{\varexps}}{\tytst}}
\]
\[
% c e
\frac
{\judes{\env}{\varexps}{\tytst}}
{\judes{\env}{\expfield{\varexps}}{\tytst}}
\quad
% wrong string
\frac
{}
{\judes{}{\expwrongd{\formvar{string}}}{\tytst}}
\]
\[
% sh k e
\frac
{\judth{\env}{\tyunbrand{\varcsh}} \quad \judeh{\env}{\varexph}{\vartyh} \quad \tyunbrand{\varcsh} = \vartyh}
{\judes{\env}{\expsh{\varcsh}{\varexph}}{\tytst}}
\]
\[
% sm k e
\frac
{\judtm{\env}{\tyunbrand{\varcsm}} \quad \judem{\env}{\varexpm}{\vartym} \quad \tyunbrand{\varcsm} = \vartym}
{\judes{\env}{\expsm{\varcsm}{\varexpm}}{\tytst}}
\]
\label{figstr}
\end{figure}

\clearpage

\begin{figure}[p]
\centering
\begin{tabular}{l}
\vspace{5pt}

$\mathscr{E}[(\lambda x.e_{S})$ $v_{S}]_{S}\rightarrow\mathscr{E}[e_{S}[v_{S}/x]]$ \\

\vspace{5pt}

$\mathscr{E}[v_{S}^{1}$ $v_{S}^{2}]_{S}\rightarrow\mathscr{E}[\mathtt{wrong}$ ``Not a function"$]$ $(v_{S}^{1}\neq\lambda x.e_{S})$ \\

\vspace{5pt}

$\mathscr{E}[f$ $\mathtt{nil}]_{S}\rightarrow\mathscr{E}[\mathtt{wrong}$ ``Empty list"$]$ \\

\vspace{5pt}

$\mathscr{E}[\mathtt{hd}$ $(\mathtt{cons}$ $v_{S}^{1}$ $v_{S}^{2})]_{S}\rightarrow\mathscr{E}[v_{S}^{1}]$ \\

\vspace{5pt}

$\mathscr{E}[\mathtt{tl}$ $(\mathtt{cons}$ $v_{S}^{1}$ $v_{S}^{2})]_{S}\rightarrow\mathscr{E}[v_{S}^{2}]$ \\

\vspace{5pt}

$\mathscr{E}[\mathtt{hd}$ $(SH^{[T]}$ $(\mathtt{cons}$ $e_{H}^{1}$ $e_{H}^{2}))]_{S}\rightarrow\mathscr{E}[SH^{T}$ $e_{H}^{1}]$ \\

\vspace{5pt}

$\mathscr{E}[\mathtt{tl}$ $(SH^{[T]}$ $(\mathtt{cons}$ $e_{H}^{1}$ $e_{H}^{2}))]_{S}\rightarrow\mathscr{E}[SH^{[T]}$ $e_{H}^{2}]$ \\

\vspace{5pt}

$\mathscr{E}[f$ $v_{S}^{1}]_{S}\rightarrow\mathscr{E}[\mathtt{wrong}$ ``Not a list"$]$ \\

\vspace{5pt}

$\quad(v_{S}^{1}\not\in\lbrace\mathtt{nil},\mathtt{cons}$ $v_{S}^{2}$ $v_{S}^{3},SH^{[T]}$ $(\mathtt{cons}$ $e_{H}^{1}$ $e_{H}^{2})\rbrace)$ \\

\vspace{5pt}

$\mathscr{E}[+$ $\overline{n_{1}}$ $\overline{n_{2}}]_{S}\rightarrow\mathscr{E}[\overline{n_{1}+n_{2}}]$ \\

\vspace{5pt}

$\mathscr{E}[-$ $\overline{n_{1}}$ $\overline{n_{2}}]_{S}\rightarrow\mathscr{E}[\overline{max(n_{1}-n_{2},0)}]$ \\

\vspace{5pt}

$\mathscr{E}[o$ $v_{S}^{1}$ $v_{S}^{2}]_{S}\rightarrow\mathscr{E}[\mathtt{wrong}$ ``Not a number"$]$ $(v_{S}^{1}\neq\overline{n}$ or $v_{S}^{2}\neq\overline{n})$ \\

\vspace{5pt}

$\mathscr{E}[\mathtt{fun?}$ $(\lambda x.e_{S})]_{S}\rightarrow\mathscr{E}[\overline{0}]$ \\

\vspace{5pt}

$\mathscr{E}[\mathtt{fun?}$ $v_{S}]_{S}\rightarrow\mathscr{E}[\overline{1}]$ ($v_{S}\neq\lambda x.e_{S}$) \\

\vspace{5pt}

$\mathscr{E}[\mathtt{list?}$ $\mathtt{nil}]_{S}\rightarrow\mathscr{E}[\overline{0}]$ \\

\vspace{5pt}

$\mathscr{E}[\mathtt{list?}$ $(\mathtt{cons}$ $v_{S}^{1}$ $v_{S}^{2})]_{S}\rightarrow\mathscr{E}[\overline{0}]$ \\

\vspace{5pt}

$\mathscr{E}[\mathtt{list?}$ $(SH^{[T]}$ $(\mathtt{cons}$ $e_{H}^{1}$ $e_{H}^{2}))]_{S}\rightarrow\mathscr{E}[\overline{0}]$ \\

\vspace{5pt}

$\mathscr{E}[\mathtt{list?}$ $v_{S}^{1}]_{S}\rightarrow\mathscr{E}[\overline{1}]$ $(v_{S}^{1}\not\in\lbrace\mathtt{nil},\mathtt{cons}$ $v_{S}^{1}$ $v_{S}^{2},SH^{[T]}$ $(\mathtt{cons}$ $e_{H}^{1}$ $e_{H}^{2})\rbrace)$ \\

\vspace{5pt}

$\mathscr{E}[\mathtt{null?}$ $\mathtt{nil}]_{S}\rightarrow\mathscr{E}[\overline{0}]$ \\

\vspace{5pt}

$\mathscr{E}[\mathtt{null?}$ $(\mathtt{cons}$ $v_{S}^{1}$ $v_{S}^{2})]_{S}\rightarrow\mathscr{E}[\overline{1}]$ \\

\vspace{5pt}

$\mathscr{E}[\mathtt{null?}$ $(SH^{[T]}$ $(\mathtt{cons}$ $e_{H}^{1}$ $e_{H}^{2}))]_{S}\rightarrow\mathscr{E}[\overline{1}]$ \\

\vspace{5pt}

$\mathscr{E}[\mathtt{null?}$ $v_{S}^{1}]_{S}\rightarrow\mathscr{E}[\mathtt{wrong}$ ``Not a list"$]$ \\

\vspace{5pt}

$\quad(v_{S}^{1}\not\in\lbrace\mathtt{nil},\mathtt{cons}$ $v_{S}^{2}$ $v_{S}^{3},SH^{[T]}$ $(\mathtt{cons}$ $e_{H}^{1}$ $e_{H}^{2})\rbrace)$ \\

\vspace{5pt}

$\mathscr{E}[\mathtt{num?}$ $\overline{n}]_{S}\rightarrow\mathscr{E}[\overline{0}]$ \\

\vspace{5pt}

$\mathscr{E}[\mathtt{num?}$ $v_{S}]_{S}\rightarrow\mathscr{E}[\overline{1}]$ $(v_{S}\neq\overline{n})$ \\

\vspace{5pt}

$\mathscr{E}[\mathtt{if0}$ $\overline{0}$ $e_{S}^{1}$ $e_{S}^{2}]_{S}\rightarrow\mathscr{E}[e_{S}^{1}]$ \\

\vspace{5pt}

$\mathscr{E}[\mathtt{if0}$ $\overline{n}$ $e_{S}^{1}$ $e_{S}^{2}]_{S}\rightarrow\mathscr{E}[e_{S}^{2}]$ $(n\neq 0)$ \\

\vspace{5pt}

$\mathscr{E}[\mathtt{if0}$ $v_{S}$ $e_{S}^{1}$ $e_{S}^{2}]_{S}\rightarrow\mathscr{E}[\mathtt{wrong}$ ``Not a number"$]$ $(v_{S}\neq\overline{n})$ \\

\vspace{5pt}

$\mathscr{E}[\mathtt{wrong}$ $\mathrm{string}]_{S}\rightarrow$ \textbf{Error}: string
\end{tabular}
\caption{Scheme operational semantics}
\label{sos}
\end{figure}

\clearpage

\begin{figure}[p]
\centering
\begin{tabular}{l}
% sh - lump
$\mathscr{E}[SH^{L}$ $(^{L}HS$ $v_{S})]_{S}\rightarrow\mathscr{E}[v_{S}]$ \\
% sh - number
$\mathscr{E}[SH^{N}$ $\overline{n}]_{S}\rightarrow\mathscr{E}[\overline{n}]$ \\
% sh - list - nil
$\mathscr{E}[SH^{[T]}$ $\mathtt{nil}^{T[T_{i}/T_{i}^{a}]}]_{S}\rightarrow\mathscr{E}[\mathtt{nil}]$ \\
% sh - function
$\mathscr{E}[SH^{T_{1}\rightarrow T_{2}}$ $(\lambda x_{1}:T_{1}[T_{i}/T_{i}^{a}].e_{H})]_{S}\rightarrow$ \\
$\quad\mathscr{E}[\lambda x_{2}.(SH^{T_{2}}$ $((\lambda x_{1}:T_{1}[T_{i}/T_{i}^{a}].e_{H})$ $(^{T_{1}}HS$ $x_{2})))]$ \\
% sh - universal
$\mathscr{E}[SH^{\forall X.T}$ $(\Lambda X.e_{H})]_{S}\rightarrow\mathscr{E}[SH^{T[L/X]}$ $((\Lambda X.e_{H})$ $\lbrace L\rbrace)]$ \\
% sh - universal - hs
$\mathscr{E}[SH^{\forall X.T}$ $(^{\forall X.T}HS$ $v_{S})]_{S}\rightarrow\mathscr{E}[v_{S}]$ \\

$\mathscr{E}[\mathtt{hd}$ $(\mathtt{sh}^{[t]}$ $(\mathtt{cons}$ $e^h_1$ $e^h_2))]^s\rightarrow\mathscr{E}[\mathtt{sh}^{t}$ $e^h_1]$ \\

$\mathscr{E}[\mathtt{tl}$ $(\mathtt{sh}^{[t]}$ $(\mathtt{cons}$ $e^h_1$ $e^h_2))]^s\rightarrow\mathscr{E}[\mathtt{sh}^{[t]}$ $e^h_2]$ \\

\end{tabular}
\caption{Scheme-Haskell operational semantics}
\label{isos}
\end{figure}

\clearpage

\begin{figure}[p]
\centering
\begin{tabular}{l}

$\mathscr{E}[\mathtt{sm}$ $t$ $(\mathtt{mh}$ $t$ $e^h)]^s\rightarrow\mathscr{E}[\mathtt{sh}$ $t$ $e^h]$ \\

$\mathscr{E}[\mathtt{sm}$ $\mathtt{L}$ $(\mathtt{ms}$ $\mathtt{L}$ $v^s)]^s\rightarrow\mathscr{E}[v^s]$ \\

$\mathscr{E}[\mathtt{sm}$ $\mathtt{N}$ $\overline{n}]^s\rightarrow\mathscr{E}[\overline{n}]$ \\

$\mathscr{E}[\mathtt{sm}$ $[t]$ $(\mathtt{nil}$ $t[t_i/t_i.x^m])]^s\rightarrow\mathscr{E}[\mathtt{nil}]$ \\

$\mathscr{E}[\mathtt{sm}$ $[t]$ $(\mathtt{cons}$ $v^m_1$ $v^m_2)]^s\rightarrow\mathscr{E}[\mathtt{cons}$ $(\mathtt{sm}$ $t$ $v^m_1)$ $(\mathtt{sm}$ $[t]$ $v^m_2)]$ \\

$\mathscr{E}[\mathtt{sm}$ $(t_1\rightarrow t_2)$ $(\lambda x^m:t_1[t_i/t_i.x^m].e^m)]^s\rightarrow$ \\

$\quad\quad\mathscr{E}[\lambda x^s.\mathtt{sm}$ $t_2$ $((\lambda x^m:t_1[t_i/t_i.x^m].e^m)$ $(\mathtt{ms}$ $t_1$ $x^s))]$ \\

$\mathscr{E}[\mathtt{sm}$ $(\forall x^m.t)$ $(\Lambda x^m.e^m)]^s\rightarrow\mathscr{E}[\mathtt{sm}$ $t[\mathtt{L}/x^m]$ $e^m[\mathtt{L}/x^m]]$ \\

$\mathscr{E}[\mathtt{sm}$ $(\forall x^m.t)$ $(\mathtt{ms}$ $(\forall x^m.t)$ $v^s)]^s\rightarrow\mathscr{E}[v^s]$

\end{tabular}
\caption{Scheme-ML operational semantics}
\label{smos}
\end{figure}

\clearpage

\begin{figure}[p]
\centering
\begin{tabular}{rcl}

\tyunbrand{\cslump} & $=$ & \tylump \\
\tyunbrand{\csnum} & $=$ & \tynum \\
\tyunbrand{\csvarh} & $=$ & \tyvarh \\
\tyunbrand{\csvarm} & $=$ & \tyvarm \\
\tyunbrand{\cslist{\varcsh}} & $=$ & \cslist{\tyunbrand{\varcsh}} \\
\tyunbrand{\cslist{\varcsm}} & $=$ & \cslist{\tyunbrand{\varcsm}} \\
\tyunbrand{\csfun{\varcsh}{\varcsh}} & $=$ & \csfun{\tyunbrand{\varcsh}}{\tyunbrand{\varcsh}} \\
\tyunbrand{\csfun{\varcsm}{\varcsm}} & $=$ & \csfun{\tyunbrand{\varcsm}}{\tyunbrand{\varcsm}} \\
\tyunbrand{\csfor{\csvarh}{\varcsh}} & $=$ & \csfor{\csvarh}{\tyunbrand{\varcsh}} \\
\tyunbrand{\csfor{\csvarm}{\varcsm}} & $=$ & \csfor{\csvarm}{\tyunbrand{\varcsm}} \\
\tyunbrand{\csbrand{\varbrand}{\vartyh}} & $=$ & \vartyh \\
\tyunbrand{\csbrand{\varbrand}{\vartym}} & $=$ & \vartym \\

\end{tabular}
\caption{Unbrand function}
\label{unbrand}
\end{figure}

\begin{figure}[p]
\onehalfspacing
\centering

$x \eq x$

$x \eq y \Rightarrow y \eq x$

$x \eq y$ and $y \eq z \Rightarrow x \eq z$

$\vartyh \eq \tylump$

$\vartym \eq \tylump$

$\vartyh = \vartym \Rightarrow \vartyh \symlumpeq \vartym$

\caption{Lump equality relation}
\label{figequality}
\end{figure}
\chapter{Proof of Type Soundness}

The type soundness of the model of computation must be proven for it to be useful.  Type soundness is ensured if progress of expressions and preservation of types are ensured.  Progress ensures that a well-typed, closed expression is either a value, reducible to another expression, or reducible to an error.  Preservation ensures that if a well-typed expression reduces to another expression, the other expression is well-typed and has the same type.  Proving progress and preservation proves type soundness.  The proof extends the proof by Kinghorn \cite{kinghorn07}, which was based on proofs by Pierce \cite{pierce02} and Matthews and Findler \cite{matthews07}.

\section{Progress}

Progress will be proven by structural induction on a well-typed, closed expression of each syntactic form.  In each case, the expression will be proven to be either a value, reducible to another expression, or reducible to an error.  Reductions of subexpressions are reductions of the top expression.  If a subexpression reduces to an error, the top expression reduces to the error.  In some cases, the syntactic forms of subexpressions must be determined to reduce the top expression.  If unique types for those subexpressions can be determined, they can be used to determine their syntactic forms.

Inverting the typing relations enables the syntactic forms of well-typed expressions to determine the types of their subexpressions:

\begin{lemma}
\label{i}
\onehalfspacing
The type of a term of each syntactic form can be calculated from the types of its subterms.
\begin{enumerate}
\item If $\Gamma\vdash_{A}x:T$ then $x:T_{1}\in\Gamma$ and $T=T_{1}$ where $A\in\lbrace H,M\rbrace$.
\item If $\Gamma\vdash_{S}x:TST$ then $x:TST\in\Gamma$.
\item If $\vdash_{A}\overline{n}:T$ then $T=N$ where $A\in\lbrace H,M\rbrace$.
\item $\vdash_{S}\overline{n}:TST$.
\item If $\Gamma\vdash_{A}\lambda x:T_{1}.e_{A}:T$ then $\Gamma\vdash_{A}T_{1}$, $\Gamma,x:T_{1}\vdash_{A}e_{A}:T_{2}$, and $T=T_{1}\rightarrow T_{2}$ where $A\in\lbrace H,M\rbrace$.
\item If $\Gamma\vdash_{S}\lambda x.e_{S}:TST$ then $\Gamma,x:TST\vdash_{S}e_{S}:TST$.
\item If $\Gamma\vdash_{A}\Lambda X.e_{A}:T$ then $\Gamma,X\vdash_{A}e_{A}:T_{1}$ and $T=\forall X.T_{1}$ where $A\in\lbrace H,M\rbrace$.
\item If $\Gamma\vdash_{A}\mathtt{cons}\;e_{A}^{1}\;e_{A}^{2}:T$ then $\Gamma\vdash_{A}e_{A}^{1}:T_{1}$, $\Gamma\vdash_{A}e_{A}^{2}:[T_{1}]$, and $T=[T_{1}]$ where $A\in\lbrace H,M\rbrace$.
\item If $\Gamma\vdash_{S}\mathtt{cons}\;e_{S}^{1}\;e_{S}^{2}:TST$ then $\Gamma\vdash_{S}e_{S}^{1}:TST$ and $\Gamma\vdash_{S}e_{S}^{2}:TST$.
\item If $\Gamma\vdash_{A}\mathtt{nil}^{T_{1}}:T$ then $\Gamma\vdash_{A}T_{1}$ and $T=[T_{1}]$ where $A\in\lbrace H,M\rbrace$.
\item $\vdash_{S}\mathtt{nil}:TST$.
\item If $\Gamma\vdash_{A}e_{A}^{1}\;e_{A}^{2}:T$ then $\Gamma\vdash_{A}e_{A}^{1}:T_{1}\rightarrow T_{2}$, $\Gamma\vdash_{A}e_{A}^{2}:T_{1}$, and $T=T_{2}$ where $A\in\lbrace H,M\rbrace$.
\item If $\Gamma\vdash_{S}e_{S}^{1}\;e_{S}^{2}:TST$ then $\Gamma\vdash_{S}e_{S}^{1}:TST$ and $\Gamma\vdash_{S}e_{S}^{2}:TST$.
\item If $\Gamma\vdash_{A}e_{A}\;\lbrace T_{1}\rbrace:T$ then $\Gamma\vdash_{A}T_{1}$, $\Gamma\vdash_{A}e_{A}:\forall X.T_{2}$, and $T=T_{2}[T_{1}/X]$ where $A\in\lbrace H,M\rbrace$.
\item If $\Gamma\vdash_{A}\mathtt{if0}\;e_{A}^{1}\;e_{A}^{2}\;e_{A}^{3}:T$ then $\Gamma\vdash_{A}e_{A}^{1}:N$, $\Gamma\vdash_{A}e_{A}^{2}:T_{1}$, $\Gamma\vdash_{A}e_{A}^{3}:T_{1}$, and $T=T_{1}$ where $A\in\lbrace H,M\rbrace$.
\item If $\Gamma\vdash_{S}\mathtt{if0}\;e_{S}^{1}\;e_{S}^{2}\;e_{S}^{3}:TST$ then $\Gamma\vdash_{S}e_{S}^{1}:TST$, $\Gamma\vdash_{S}e_{S}^{2}:TST$, and $\Gamma\vdash_{S}e_{S}^{3}:TST$.
\item If $\Gamma\vdash_{A}o\;e_{A}^{1}\;e_{A}^{2}:T$ then $\Gamma\vdash_{A}e_{A}^{1}:N$, $\Gamma\vdash_{A}e_{A}^{2}:N$, and $T=N$ where $A\in\lbrace H,M\rbrace$.
\item If $\Gamma\vdash_{S}o\;e_{S}^{1}\;e_{S}^{2}:TST$ then $\Gamma\vdash_{S}e_{S}^{1}:TST$ and $\Gamma\vdash_{S}e_{S}^{2}:TST$.
\item If $\Gamma\vdash_{A}\mathtt{hd}\;e_{A}:T$ then $\Gamma\vdash_{A}e_{A}:[T_{1}]$ and $T=T_{1}$ where $A\in\lbrace H,M\rbrace$.
\item If $\Gamma\vdash_{A}\mathtt{tl}\;e_{A}:T$ then $\Gamma\vdash_{A}e_{A}:[T_{1}]$ and $T=[T_{1}]$ where $A\in\lbrace H,M\rbrace$.
\item If $\Gamma\vdash_{S}f\;e_{S}:TST$ then $\Gamma\vdash_{S}e_{S}:TST$.
\item If $\Gamma\vdash_{A}\mathtt{fix}\;e_{A}:T$ then $\Gamma\vdash_{A}e_{A}:T_{1}\rightarrow T_{1}$ and $T=T_{1}$ where $A\in\lbrace H,M\rbrace$.
\item If $\Gamma\vdash_{S}p\;e_{S}:TST$ then $\Gamma\vdash_{S}e_{S}:TST$.
\item $\vdash_{S}\mathtt{wrong}\;\mathrm{string}:TST$.
\item If $\Gamma\vdash_{A}{^{T_{1}}A}B^{T_{1}}\;e_{B}:T$ then $\Gamma\vdash_{A}T_{1}$, $\Gamma\vdash_{B}T_{1}$, $\Gamma\vdash_{B}e_{B}:T_{1}$, and $T=T_{1}$ where $(A,B)\in\lbrace(H,M),(M,H)\rbrace$.
\item If $\Gamma\vdash_{A}{^{T_{1}}A}S\;e_{S}:T$ then $\Gamma\vdash_{A}T_{1}$, $\Gamma\vdash_{S}e_{S}:TST$, and $T=T_{1}[T_{i}/T_{i}^{a}]$ where $A\in\lbrace H,M\rbrace$.
\item If $\Gamma\vdash_{S}SA^{T_{1}}\;e_{A}:T$ then $\Gamma\vdash_{A}T_{1}$, $\Gamma\vdash_{A}e_{A}:T_{1}[T_{i}/T_{i}^{a}]$, and $T=TST$ where $A\in\lbrace H,M\rbrace$.
\end{enumerate}
\begin{proof}
Immediate from the definitions of the typing relations.
\end{proof}
\end{lemma}

Well-typed Haskell and ML expressions have unique types:

\begin{lemma}
\label{uot}
%\onehalfspacing
$e_{A}$ has at most one type $T$ for a given context $\Gamma$ where $A\in\lbrace H,M\rbrace$.
\begin{proof}
By structural induction on $e_{A}$ using inversion (Lemma \ref{i}).
\end{proof}
\end{lemma}

The types of Haskell and ML values determine their syntactic forms:

\begin{lemma}
\label{cf}
%\onehalfspacing
The possible syntactic forms of values of various types.
\begin{enumerate}
\item If $v_{A}:N$ then $v_{A}=\overline{n}$ where $A\in\lbrace H,M\rbrace$.
\item If $v_{A}:T_{1}\rightarrow T_{2}$ then $v_{A}=\lambda x:T_{1}.e_{A}$ where $A\in\lbrace H,M\rbrace$.
\item If $v_{A}:\forall X.T$ then $v_{A}\in\lbrace\Lambda X.e_{A},{^{\forall X.T}A}S$ $v_{S}\rbrace$ where $A\in\lbrace H,M\rbrace$.
\item If $v_{H}:[T]$ then $v_{H}\in\lbrace\mathtt{cons}$ $e_{H}^{1}$ $e_{H}^{2},\mathtt{nil}^{T}\rbrace$.
\item If $v_{M}:[T]$ then $v_{M}\in\lbrace\mathtt{cons}$ $v_{M}^{1}$ $v_{M}^{2},\mathtt{nil}^{T},{^{[T]}M}H^{[T]}$ $(\mathtt{cons}$ $e_{H}$ $e_{H})\rbrace$.
\item If $v_{A}:L$ then $v_{A}={^{L}A}S$ $v_{S}$ where $A\in\lbrace H,M\rbrace$.
\end{enumerate}
\begin{proof}
Immediate from the definitions of values and the typing relations.
\end{proof}
\end{lemma}

\begin{theorem}
\label{ps}
If $\vdash_{A}e_{A}:T$ then $e_{A}$ is a value or $e_{A}\rightarrow e_{A}'$ or $e_{A}\rightarrow$ \emph{\textbf{Error}:\;string} where $A\in\lbrace H,M,S\rbrace$.
\begin{proof}
By structural induction on $e_{A}$.
\begin{case}
$e_{A}=\lambda x:T.e_{A}^{1}$ where $A\in\lbrace H,M\rbrace$

$\lambda x:T.e_{A}^{1}$ is a value.
\end{case}
\begin{case}
$e_{S}=\lambda x.e_{S}^{1}$

$\lambda x.e_{S}^{1}$ is a value.
\end{case}
\begin{case}
$e_{A}=\Lambda X.e_{A}^{1}$ where $A\in\lbrace H,M\rbrace$

$\Lambda X.e_{A}^{1}$ is a value.
\end{case}
\begin{case}
$e_{A}=\overline{n}$ where $A\in\lbrace H,M,S\rbrace$

$\overline{n}$ is a value.
\end{case}
\begin{case}
$e_{HM}=\mathtt{nil}^{T}$

$\mathtt{nil}^{T}$ is a value.
\end{case}
\begin{case}
$e_{S}=\mathtt{nil}$

$\mathtt{nil}$ is a value.
\end{case}
\begin{case}
$e_{H}=\mathtt{cons}\;e_{H}^{1}\;e_{H}^{2}$

$\mathtt{cons}\;e_{H}^{1}\;e_{H}^{2}$ is a value.
\end{case}
\begin{case}
$e_{A}=\mathtt{cons}\;v_{A}^{1}\;v_{A}^{2}$ where $A\in\lbrace M,S\rbrace$

$\mathtt{cons}\;v_{A}^{1}\;v_{A}^{2}$ is a value.
\end{case}
\begin{case}
$e_{A}=x$ where $A\in\lbrace H,M,S\rbrace$

Cannot occur because $e_{A}$ is closed.
\end{case}
\begin{case}
$e_{H}=e_{H}^{1}\;e_{H}^{2}$

$e_{H}^{1}$ is a value or $e_{H}^{1}\rightarrow e_{H}^{3}$ or $e_{H}^{1}\rightarrow$ \emph{\textbf{Error}:\;string} by the induction hypothesis.  If $e_{H}^{1}$ is a value then $e_{H}^{1}:T_{1}\rightarrow T_{2}$ by inversion (Lemma \ref{i}) and uniqueness of types (Lemma \ref{uot}) and $e_{H}^{1}=\lambda x:T_{1}.e_{H}^{4}$ by canonical forms (Lemma \ref{cf}).  $(\lambda x:T_{1}.e_{H}^{4})\;e_{H}^{2}\rightarrow e_{H}^{4}[e_{H}^{2}/x]$.  If $e_{H}^{1}\rightarrow e_{H}^{3}$ then $e_{H}^{1}\;e_{H}^{2}\rightarrow e_{H}^{3}\;e_{H}^{2}$.  If $e_{H}^{1}\rightarrow$ \emph{\textbf{Error}:\;string} then $e_{H}^{1}\;e_{H}^{2}\rightarrow$ \emph{\textbf{Error}:\;string}.
\end{case}
\begin{case}
$e_{M}=e_{M}^{1}\;e_{M}^{2}$

$e_{M}^{1}$ is a value or $e_{M}^{1}\rightarrow e_{M}^{3}$ or $e_{M}^{1}\rightarrow$ \emph{\textbf{Error}:\;string} by the induction hypothesis.  If $e_{M}^{1}$ is a value then $e_{M}^{1}:T_{1}\rightarrow T_{2}$ by inversion (Lemma \ref{i}) and uniqueness of types (Lemma \ref{uot}) and $e_{M}^{1}=\lambda x:T_{1}.e_{M}^{4}$ by canonical forms (Lemma \ref{cf}).  If $e_{M}^{1}\rightarrow e_{M}^{3}$ then $e_{M}^{1}\;e_{M}^{2}\rightarrow e_{M}^{3}\;e_{M}^{2}$.  If $e_{M}^{1}\rightarrow$ \emph{\textbf{Error}:\;string} then $e_{M}^{1}\;e_{M}^{2}\rightarrow$ \emph{\textbf{Error}:\;string}.  $e_{M}^{2}$ is a value or $e_{M}^{2}\rightarrow e_{M}^{5}$ or $e_{M}^{2}\rightarrow$ \emph{\textbf{Error}:\;string} by the induction hypothesis.  If $e_{M}^{2}\rightarrow e_{M}^{5}$ and $e_{M}^{1}$ is a value then $e_{M}^{1}\;e_{M}^{2}\rightarrow e_{M}^{1}\;e_{M}^{5}$.  If $e_{M}^{2}\rightarrow$ \emph{\textbf{Error}:\;string} and $e_{M}^{1}$ is a value then $e_{M}^{1}\;e_{M}^{2}\rightarrow$ \emph{\textbf{Error}:\;string}.  If $e_{M}^{1}$ and $e_{M}^{2}$ are values then $(\lambda x:T_{1}.e_{M}^{4})\;e_{M}^{2}\rightarrow e_{M}^{4}[e_{M}^{2}/x]$.
\end{case}
\begin{case}
$e_{S}=e_{S}^{1}$ $e_{S}^{2}$

$e_{S}^{1}$ is a value or $e_{S}^{1}\rightarrow e_{S}^{3}$ or $e_{S}^{1}\rightarrow$ \emph{\textbf{Error}:\;string} by the induction hypothesis.  If $e_{S}^{1}\rightarrow e_{S}^{3}$ then $e_{S}^{1}$ $e_{S}^{2}\rightarrow e_{S}^{3}$ $e_{S}^{2}$.  If $e_{S}^{1}\rightarrow$ \emph{\textbf{Error}:\;string} then $e_{S}^{1}$ $e_{S}^{2}\rightarrow$ \emph{\textbf{Error}:\;string}.  $e_{S}^{2}$ is a value or $e_{S}^{2}\rightarrow e_{S}^{4}$ or $e_{S}^{2}\rightarrow$ \emph{\textbf{Error}:$ $string} by the induction hypothesis.  If $e_{S}^{2}\rightarrow e_{S}^{4}$ and $e_{S}^{1}$ is a value then $e_{S}^{1}$ $e_{S}^{2}\rightarrow e_{S}^{1}$ $e_{S}^{4}$.  If $e_{S}^{2}\rightarrow$ \emph{\textbf{Error}:\;string} and $e_{S}^{1}$ is a value then $e_{S}^{1}$ $e_{S}^{2}\rightarrow$ \emph{\textbf{Error}:\;string}.  If $e_{S}^{1}$ and $e_{S}^{2}$ are values then $(\lambda x.e_{S}^{5})$ $e_{S}^{2}\rightarrow e_{S}^{5}[e_{S}^{2}/x]$ if $e_{S}^{1}=\lambda x.e_{S}^{5}$ or $e_{S}^{1}$ $e_{S}^{2}\rightarrow\mathtt{wrong}$ \emph{``Not a function"} otherwise.
\end{case}
\begin{case}
$e_{A}=e_{A}^{1}\;\lbrace T_{1}\rbrace$ where $A\in\lbrace H,M\rbrace$

$e_{A}^{1}$ is a value or $e_{A}^{1}\rightarrow e_{A}^{2}$ or $e_{A}^{1}\rightarrow$ \emph{\textbf{Error}:\;string} by the induction hypothesis.  If $e_{A}^{1}$ is a value then $e_{A}^{1}:\forall X.T_{2}$ by inversion (Lemma \ref{i}) and uniqueness of types (Lemma \ref{uot}) and $e_{A}^{1}=\Lambda X.e_{A}^{3}$ or $e_{A}^{1}=\,^{\forall X.T_{2}}AS\;v_{S}$ by canonical forms (Lemma \ref{cf}).
\begin{subcase}
$e_{A}^{1}=\Lambda X.e_{A}^{3}$

$(\Lambda X.e_{A}^{3})\;\lbrace T_{1}\rbrace\rightarrow e_{A}^{3}[T_{1}/X]$.
\end{subcase}
\begin{subcase}
$e_{A}^{1}={^{\forall X.T_{2}}A}S\;v_{S}$

$(^{\forall X.T_{2}}AS\;v_{S})\;\lbrace T_{1}\rbrace\rightarrow{^{T_{2}[T_{1}^{a}/X]}A}S\;v_{S}$.
\end{subcase}
If $e_{A}^{1}\rightarrow e_{A}^{2}$ then $e_{A}^{1}\;\lbrace T_{1}\rbrace\rightarrow e_{A}^{2}\;\lbrace T_{1}\rbrace$.  If $e_{A}^{1}\rightarrow$ \emph{\textbf{Error}:\;string} then $e_{A}^{1}\;\lbrace T_{1}\rbrace\rightarrow$ \emph{\textbf{Error}:\;string}.
\end{case}
\begin{case}
$e_{A}=o\;e_{A}^{1}\;e_{A}^{2}$ where $A\in\lbrace H,M\rbrace$

$e_{A}^{1}$ is a value or $e_{A}^{1}\rightarrow e_{A}^{3}$ or $e_{A}^{1}\rightarrow$ \emph{\textbf{Error}:\;string} by the induction hypothesis.  If $e_{A}^{1}$ is a value then $e_{A}^{1}:N$ by inversion (Lemma \ref{i}) and uniqueness of types (Lemma \ref{uot}) and $e_{A}^{1}=\overline{n_{1}}$ by canonical forms (Lemma \ref{cf}).  If $e_{A}^{1}\rightarrow e_{A}^{3}$ then $o\;e_{A}^{1}\;e_{A}^{2}\rightarrow o\;e_{A}^{3}\;e_{A}^{2}$.  If $e_{A}^{1}\rightarrow$ \emph{\textbf{Error}:\;string} then $o\;e_{A}^{1}\;e_{A}^{2}\rightarrow$ \emph{\textbf{Error}:\;string}.  $e_{A}^{2}$ is a value or $e_{A}^{2}\rightarrow e_{A}^{4}$ or $e_{A}^{2}\rightarrow$ \emph{\textbf{Error}:\;string} by the induction hypothesis.  If $e_{A}^{2}$ is a value then $e_{A}^{2}:N$ by inversion and uniqueness of types and $e_{A}^{2}=\overline{n_{2}}$ by canonical forms.  If $e_{A}^{2}\rightarrow e_{A}^{4}$ and $e_{A}^{1}$ is a value then $o\;e_{A}^{1}\;e_{A}^{2}\rightarrow o\;e_{A}^{1}\;e_{A}^{4}$.  If $e_{A}^{2}\rightarrow$ \emph{\textbf{Error}:\;string} and $e_{A}^{1}$ is a value then $o\;e_{A}^{1}\;e_{A}^{2}\rightarrow$ \emph{\textbf{Error}:\;string}.  If $e_{A}^{1}$ and $e_{A}^{2}$ are values then $+\;\overline{n_{1}}\;\overline{n_{2}}\rightarrow\overline{n_{1}+n_{2}}$ and $-\;\overline{n_{1}}\;\overline{n_{2}}\rightarrow\overline{max(n_{1}-n_{2},0)}$.
\end{case}
\begin{case}
$e_{S}=o\;e_{S}^{1}\;e_{S}^{2}$

$e_{S}^{1}$ is a value or $e_{S}^{1}\rightarrow e_{S}^{3}$ or $e_{S}^{1}\rightarrow$ \emph{\textbf{Error}:\;string} by the induction hypothesis.  If $e_{S}^{1}\rightarrow e_{S}^{3}$ then $o\;e_{S}^{1}\;e_{S}^{2}\rightarrow o\;e_{S}^{3}\;e_{S}^{2}$.  If $e_{S}^{1}\rightarrow$ \emph{\textbf{Error}:\;string} then $o\;e_{S}^{1}\;e_{S}^{2}\rightarrow$ \emph{\textbf{Error}:\;string}.  $e_{S}^{2}$ is a value or $e_{S}^{2}\rightarrow e_{S}^{4}$ or $e_{S}^{2}\rightarrow$ \emph{\textbf{Error}:\;string} by the induction hypothesis.  If $e_{S}^{2}\rightarrow e_{S}^{4}$ and $e_{S}^{1}$ is a value then $o\;e_{S}^{1}\;e_{S}^{2}\rightarrow o\;e_{S}^{1}\;e_{S}^{4}$.  If $e_{S}^{2}\rightarrow$ \emph{\textbf{Error}:\;string} and $e_{S}^{1}$ is a value then $o\;e_{S}^{1}\;e_{S}^{2}\rightarrow$ \emph{\textbf{Error}:\;string}.  If $e_{S}^{1}$ and $e_{S}^{2}$ are values then $o\;e_{S}^{1}\;e_{S}^{2}\rightarrow\overline{n_{1}+n_{2}}$ if $o=+$, $e_{S}^{1}=\overline{n_{1}}$, and $e_{S}^{2}=\overline{n_{2}}$ and $o\;e_{S}^{1}\;e_{S}^{2}\rightarrow\overline{max(n_{1}-n_{2},0)}$ if $o=-$, $e_{S}^{1}=\overline{n_{1}}$, and $e_{S}^{2}=\overline{n_{2}}$ and $o\;e_{S}^{1}\;e_{S}^{2}\rightarrow\mathtt{wrong}\;\mathrm{``Not\;a\;number"}$ otherwise.
\end{case}
\begin{case}
$e_{A}=\mathtt{null?}$ $e_{A}^{1}$ where $A\in\lbrace H,M\rbrace$

$e_{A}^{1}$ is a value or $e_{A}^{1}\rightarrow e_{A}^{2}$ or $e_{A}^{1}\rightarrow$ \emph{\textbf{Error}: string} by the induction hypothesis.  If $e_{A}^{1}$ is a value then $e_{A}^{1}:[T]$ by inversion (Lemma \ref{i}) and uniqueness of types (Lemma \ref{uot}).  If $A=H$ then $e_{A}^{1}\in\lbrace\mathtt{nil}^{T},\mathtt{cons}$ $e_{H}^{1}$ $e_{H}^{2}\rbrace$ by canonical forms (Lemma \ref{cf}).  If $e_{A}^{1}=\mathtt{nil}^{T}$ then $\mathtt{null?}$ $e_{A}^{1}\rightarrow\overline{0}$.  If $e_{A}^{1}=\mathtt{cons}$ $e_{H}^{1}$ $e_{H}^{2}$ then $\mathtt{null?}$ $e_{A}^{1}\rightarrow\overline{1}$.  If $A=M$ then $e_{A}^{1}\in\lbrace\mathtt{nil}^{T},\mathtt{cons}$ $v_{M}^{1}$ $v_{M}^{2},{^{[T]}M}H^{[T]}$ $(\mathtt{cons}$ $e_{H}^{1}$ $e_{H}^{2})\rbrace$ by canonical forms.  If $e_{A}^{1}=\mathtt{nil}^{T}$ then $\mathtt{null?}$ $e_{A}^{1}\rightarrow\overline{0}$.  If $e_{A}^{1}\in\lbrace\mathtt{cons}$ $v_{M}^{1}$ $v_{M}^{2},{^{[T]}M}H^{[T]}$ $(\mathtt{cons}$ $e_{H}^{1}$ $e_{H}^{2})\rbrace$ then $\mathtt{null?}$ $e_{A}^{1}\rightarrow\overline{1}$.  If $e_{A}^{1}\rightarrow e_{A}^{2}$ then $\mathtt{null?}$ $e_{A}^{1}\rightarrow\mathtt{null?}$ $e_{A}^{2}$.  If $e_{A}^{1}\rightarrow$ \emph{\textbf{Error}: string} then $\mathtt{null?}$ $e_{A}^{1}\rightarrow$ \emph{\textbf{Error}: string}.
\end{case}
\begin{case}
$e_{S}=p\;e_{S}^{1}$

$e_{S}^{1}$ is a value or $e_{S}^{1}\rightarrow e_{S}^{2}$ or $e_{S}^{1}\rightarrow$ \emph{\textbf{Error}:\;string} by the induction hypothesis.  If $e_{S}^{1}\rightarrow e_{S}^{2}$ then $p\;e_{S}^{1}\rightarrow p\;e_{S}^{2}$.  If $e_{S}^{1}\rightarrow$ \emph{\textbf{Error}:\;string} then $p\;e_{S}^{1}\rightarrow$ \emph{\textbf{Error}:\;string}.  $e_{S}^{1}$ is a value otherwise.  If $p=\mathtt{nat?}$ then $p\;e_{S}^{1}\rightarrow\overline{0}$ if $e_{S}^{1}=\overline{n}$ and $p\;e_{S}^{1}\rightarrow\overline{1}$ otherwise.  If $p=\mathtt{list?}$ then $p\;e_{S}^{1}\rightarrow\overline{0}$ if $e_{S}^{1}\in\lbrace\mathtt{cons}\;e_{S}^{3}\;e_{S}^{4},\mathtt{nil}\rbrace$ and $p\;e_{S}^{1}\rightarrow\overline{1}$ otherwise.  If $p=\mathtt{proc?}$ then $p\;e_{S}^{1}\rightarrow\overline{0}$ if $e_{S}^{1}=\lambda x.e_{S}^{5}$ and $p\;e_{S}^{1}\rightarrow\overline{1}$ otherwise.
\end{case}
\begin{case}
$e_{A}=\mathtt{if0}\;e_{A}^{1}\;e_{A}^{2}\;e_{A}^{3}$ where $A\in\lbrace H,M\rbrace$

$e_{A}^{1}$ is a value or $e_{A}^{1}\rightarrow e_{A}^{4}$ or $e_{A}^{1}\rightarrow$ \emph{\textbf{Error}:\;string} by the induction hypothesis.  If $e_{A}^{1}$ is a value then $e_{A}^{1}:N$ by inversion (Lemma \ref{i}) and uniqueness of types (Lemma \ref{uot}) and $e_{A}^{1}=\overline{n}$ by canonical forms (Lemma \ref{cf}).  $\mathtt{if0}\;\overline{n}\;e_{A}^{2}\;e_{A}^{3}\rightarrow e_{A}^{2}$ if $\overline{n}=\overline{0}$ or $\mathtt{if0}\;\overline{n}\;e_{A}^{2}\;e_{A}^{3}\rightarrow e_{A}^{3}$ otherwise.  If $e_{A}^{1}\rightarrow e_{A}^{4}$ then $\mathtt{if0}\;e_{A}^{1}\;e_{A}^{2}\;e_{A}^{3}\rightarrow \mathtt{if0}\;e_{A}^{4}\;e_{A}^{2}\;e_{A}^{3}$.  If $e_{A}^{1}\rightarrow$ \emph{\textbf{Error}:\;string} then $\mathtt{if0}\;e_{A}^{1}\;e_{A}^{2}\;e_{A}^{3}\rightarrow$ \emph{\textbf{Error}:\;string}.
\end{case}
\begin{case}
$e_{S}=\mathtt{if0}\;e_{S}^{1}\;e_{S}^{2}\;e_{S}^{3}$

$e_{S}^{1}$ is a value or $e_{S}^{1}\rightarrow e_{S}^{4}$ or $e_{S}^{1}\rightarrow$ \emph{\textbf{Error}:\;string} by the induction hypothesis.  If $e_{S}^{1}$ is a value then $\mathtt{if0}\;e_{S}^{1}\;e_{S}^{2}\;e_{S}^{3}\rightarrow e_{S}^{2}$ if $e_{S}^{1}=\overline{0}$ or $\mathtt{if0}\;e_{S}^{1}\;e_{S}^{2}\;e_{S}^{3}\rightarrow e_{S}^{3}$ otherwise.  If $e_{S}^{1}\rightarrow e_{S}^{4}$ then $\mathtt{if0}\;e_{S}^{1}\;e_{S}^{2}\;e_{S}^{3}\rightarrow \mathtt{if0}\;e_{S}^{4}\;e_{S}^{2}\;e_{S}^{3}$.  If $e_{S}^{1}\rightarrow$ \emph{\textbf{Error}:\;string} then $\mathtt{if0}\;e_{S}^{1}\;e_{S}^{2}\;e_{S}^{3}\rightarrow$ \emph{\textbf{Error}:\;string}.
\end{case}
\begin{case}
$e_{A}=\mathtt{cons}\;e_{A}^{1}\;e_{A}^{2}$ where $A\in\lbrace M,S\rbrace$

$e_{A}^{1}$ is a value or $e_{A}^{1}\rightarrow e_{A}^{3}$ or $e_{A}^{1}\rightarrow$ \emph{\textbf{Error}:\;string} by the induction hypothesis.  If $e_{A}^{1}\rightarrow e_{A}^{3}$ then $\mathtt{cons}\;e_{A}^{1}\;e_{A}^{2}\rightarrow\mathtt{cons}\;e_{A}^{3}\;e_{A}^{2}$.  If $e_{A}^{1}\rightarrow$ \emph{\textbf{Error}:\;string} then $\mathtt{cons}\;e_{A}^{1}\;e_{A}^{2}\rightarrow$ \emph{\textbf{Error}:\;string}.  $e_{A}^{2}$ is a value or $e_{A}^{2}\rightarrow e_{A}^{4}$ or $e_{A}^{1}\rightarrow$ \emph{\textbf{Error}:\;string} by the induction hypothesis.  If $e_{A}^{2}\rightarrow e_{A}^{4}$ and $e_{M}^{1}$ is a value then $\mathtt{cons}\;e_{A}^{1}\;e_{A}^{2}\rightarrow\mathtt{cons}\;e_{A}^{1}\;e_{A}^{4}$.  If $e_{A}^{2}\rightarrow$ \emph{\textbf{Error}:\;string} then $\mathtt{cons}\;e_{A}^{1}\;e_{A}^{2}\rightarrow$ \emph{\textbf{Error}:\;string}.  If $e_{A}^{1}$ and $e_{A}^{2}$ are values then $\mathtt{cons}\;e_{A}^{1}\;e_{A}^{2}$ is a value.
\end{case}
\begin{case}
$e_{H}=f$ $e_{H}^{1}$

$e_{H}^{1}$ is a value or $e_{H}^{1}\rightarrow e_{H}^{2}$ or $e_{H}^{1}\rightarrow$ \emph{\textbf{Error}:\;string} by the induction hypothesis.  If $e_{H}^{1}$ is a value then $e_{H}^{1}:[T]$ by inversion (Lemma \ref{i}) and uniqueness of types (Lemma \ref{uot}) and $e_{H}^{1}\in\lbrace\mathtt{cons}$ $e_{H}^{3}$ $e_{H}^{4},\mathtt{nil}^{T}\rbrace$ by canonical forms (Lemma \ref{cf}).  If $e_{H}^{1}=\mathtt{cons}$ $e_{H}^{3}$ $e_{H}^{4}$ then $\mathtt{hd}$ $(\mathtt{cons}$ $e_{H}^{3}$ $e_{H}^{4})\rightarrow e_{H}^{3}$ and $\mathtt{tl}$ $(\mathtt{cons}$ $e_{H}^{3}$ $e_{H}^{4})\rightarrow e_{H}^{4}$.  If $e_{H}^{1}=\mathtt{nil}^{T}$ then $\mathtt{hd}$ $\mathtt{nil}^{T}\rightarrow{^{T}H}S$ $(\mathtt{wrong}$ \emph{``Empty list"}$)$ and $\mathtt{tl}$ $\mathtt{nil}^{T}\rightarrow\mathtt{nil}^{T}$.  If $e_{H}^{1}\rightarrow e_{H}^{2}$ then $f$ $e_{H}^{1}\rightarrow f$ $e_{H}^{2}$.  If $e_{H}^{1}\rightarrow$ \emph{\textbf{Error}:\;string} then $f$ $e_{H}^{1}\rightarrow$ \emph{\textbf{Error}:\;string}.
\end{case}
\begin{case}
$e_{M}=f\;e_{M}^{1}$

$e_{M}^{1}$ is a value or $e_{M}^{1}\rightarrow e_{M}^{2}$ or $e_{M}^{1}\rightarrow$ \emph{\textbf{Error}:\;string} by the induction hypothesis.  If $e_{M}^{1}$ is a value then $e_{M}^{1}:[T]$ by inversion (Lemma \ref{i}) and uniqueness of types (Lemma \ref{uot}) and $e_{M}^{1}\in\lbrace\mathtt{cons}\;e_{M}^{3}\;e_{M}^{4},\mathtt{nil}^{T},{^{[T]}M}H^{[T]}\;v_{H}\rbrace$ by canonical forms (Lemma \ref{cf}).  If $e_{M}^{1}=\mathtt{cons}\;e_{M}^{3}\;e_{M}^{4}$ then $\mathtt{hd}\;(\mathtt{cons}\;e_{M}^{3}\;e_{M}^{4})\rightarrow e_{M}^{3}$ or $\mathtt{tl}\;(\mathtt{cons}\;e_{M}^{3}\;e_{M}^{4})\rightarrow e_{M}^{4}$.  If $e_{M}^{1}=\mathtt{nil}^{T}$ then $\mathtt{hd}\;\mathtt{nil}^{T}\rightarrow{^{T}M}S\;(\mathtt{wrong}\;\mathrm{``Empty\;list"})$ and $\mathtt{tl}\;\mathtt{nil}^{T}\rightarrow\mathtt{nil}^{T}$.  If $e_{M}^{1}={^{[T]}M}H^{[T]}\;v_{H}$ then $v_{H}:[T]$ by inversion and uniqueness of types and $v_{H}\in\lbrace\mathtt{cons}\;e_{H}^{1}\;e_{H}^{2},\mathtt{nil}^{T}\rbrace$ by canonical forms.  If $v_{H}=\mathtt{cons}\;e_{H}^{1}\;e_{H}^{2}$ then $\mathtt{hd}\;(^{[T]}MH^{[T]}\;(\mathtt{cons}\;e_{H}^{1}\;e_{H}^{2}))\rightarrow{^{T}M}H^{T}\;e_{H}^{1}$ and $\mathtt{tl}\;(^{[T]}MH^{[T]}\;(\mathtt{cons}\;e_{H}^{1}\;e_{H}^{2}))\rightarrow{^{[T]}M}H^{[T]}\;e_{H}^{2}$.  If $v_{H}=\mathtt{nil}^{T}$ then $\mathtt{hd}\;\mathtt{nil}^{T}\rightarrow{^{T}M}S\;(\mathtt{wrong}\;\mathrm{``Empty\;list"})$ and $\mathtt{tl}\;\mathtt{nil}^{T}\rightarrow\mathtt{nil}^{T}$.  If $e_{M}^{1}\rightarrow e_{M}^{2}$ then $f\;e_{M}^{1}\rightarrow f\;e_{M}^{2}$.  If $e_{M}^{1}\rightarrow e_{M}^{2}$ then $f\;e_{M}^{1}\rightarrow f\;e_{M}^{2}$.  If $e_{M}^{1}\rightarrow$ \emph{\textbf{Error}:\;string} then $f\;e_{M}^{1}\rightarrow$ \emph{\textbf{Error}:\;string}.
\end{case}
\begin{case}
$e_{S}=f$ $e_{S}^{1}$

$e_{S}^{1}$ is a value or $e_{S}^{1}\rightarrow e_{S}^{2}$ or $e_{S}^{1}\rightarrow$ \emph{\textbf{Error}:\;string} by the induction hypothesis.  If $e_{S}^{1}\rightarrow e_{S}^{2}$ then $f$ $e_{S}^{1}\rightarrow f$ $e_{S}^{2}$.  If $e_{S}^{1}\rightarrow$ \emph{\textbf{Error}:\;string} then $f$ $e_{S}^{1}\rightarrow$ \emph{\textbf{Error}:\;string}.  $e_{S}^{1}$ is a value otherwise.  If $e_{S}^{1}=\mathtt{cons}$ $e_{S}^{3}$ $e_{S}^{4}$ then $f$ $(\mathtt{cons}$ $e_{S}^{3}$ $e_{S}^{4})\rightarrow e_{S}^{3}$ if $f=\mathtt{hd}$ and $f$ $(\mathtt{cons}$ $e_{S}^{3}$ $e_{S}^{4})\rightarrow e_{S}^{4}$ if $f=\mathtt{tl}$.  If $e_{S}^{1}=\mathtt{nil}$ then $f$ $\mathtt{nil}\rightarrow\mathtt{wrong}$ \emph{``Empty list"}.  If $e_{S}^{1}=SH^{[T]}$ $(\mathtt{cons}$ $e_{H}^{1}$ $e_{H}^{2})$ then $f$ $(SH^{[T]}$ $(\mathtt{cons}$ $e_{H}^{1}$ $e_{H}^{2}))\rightarrow SH^{T}$ $e_{H}^{1}$ if $f=\mathtt{hd}$ and $f$ $(SH^{[T]}$ $(\mathtt{cons}$ $e_{H}^{1}$ $e_{H}^{2}))\rightarrow SH^{[T]}$ $e_{H}^{2}$ if $f=\mathtt{tl}$.  $f$ $e_{S}^{1}\rightarrow\mathtt{wrong}$ \emph{``Not a list"} otherwise.
\end{case}
\begin{case}
$e_{A}=\mathtt{fix}\;e_{A}^{1}$ where $A\in\lbrace H,M\rbrace$

$e_{A}^{1}$ is a value or $e_{A}^{1}\rightarrow e_{A}^{2}$ or $e_{A}^{1}\rightarrow$ \emph{\textbf{Error}:\;string} by the induction hypothesis.  If $e_{A}^{1}$ is a value then $e_{A}^{1}:T_{1}\rightarrow T_{2}$ by inversion (Lemma \ref{i}) and uniqueness of types (Lemma \ref{uot}) and $e_{A}^{1}=\lambda x:T_{1}.e_{A}^{3}$ by canonical forms (Lemma \ref{cf}).  $\mathtt{fix}\;e_{A}^{1}\rightarrow e_{A}^{3}[\mathtt{fix}\;(\lambda x:T_{1}.e_{A}^{3})/x]$.  If $e_{A}^{1}\rightarrow e_{A}^{2}$ then $\mathtt{fix}\;e_{A}^{1}\rightarrow\mathtt{fix}\;e_{A}^{2}$.  If $e_{A}^{1}\rightarrow$ \emph{\textbf{Error}:\;string} then $\mathtt{fix}\;e_{A}^{1}\rightarrow$ \emph{\textbf{Error}:\;string}.
\end{case}
\begin{case}
$e_{A}=\mathtt{wrong}^{T}$ $\mathrm{string}$ where $A\in\lbrace H,M\rbrace$

$\mathtt{wrong}^{T}$ $\mathrm{string}\rightarrow$ \emph{\textbf{Error}:\;string}.
\end{case}
\begin{case}
$e_{S}=\mathtt{wrong}$ $\mathrm{string}$

$\mathtt{wrong}$ $\mathrm{string}\rightarrow$ \emph{\textbf{Error}:\;string}.
\end{case}
\begin{case}
\label{ab}
$e_{A}={^{T}A}B^{T}$ $e_{B}^{1}$ where $(A,B,C)\in\lbrace(H,M,v),(M,H,e)\rbrace$

$e_{B}^{1}$ is a value or $e_{B}^{1}\rightarrow e_{B}^{2}$ or $e_{B}^{1}\rightarrow$ \emph{\textbf{Error}:\;string} by the induction hypothesis.  If $e_{B}^{1}$ is a value then $T$ determines its reduction.
\begin{subcase}
$T=N$

$e_{B}^{1}=\overline{n}$ by canonical forms (Lemma \ref{cf}).  $^{N}AB^{N}$ $\overline{n}\rightarrow\overline{n}$.
\end{subcase}
\begin{subcase}
$T=T_{1}\rightarrow T_{2}$

$e_{B}^{1}=\lambda x_{1}:T_{1}.e_{B}^{2}$ by canonical forms (Lemma \ref{cf}).  $^{T_{1}\rightarrow T_{2}}AB^{T_{1}\rightarrow T_{2}}$ $(\lambda x_{1}:T_{1}.e_{B}^{2})\rightarrow\lambda x_{2}:T_{1}[T_{i}/T^{a}_{i}].(^{T_{2}}AB^{T_{2}}$ $((\lambda x_{1}:T_{1}.e_{B}^{2})$ $(^{T_{1}}BA^{T_{1}}$ $x_{2})))$.
\end{subcase}
\begin{subcase}
$T=\forall X.T_{1}$

$e_{B}^{1}\in\lbrace\Lambda X.e_{B}^{2},{^{\forall X.T_{1}}B}S$ $v_{S}\rbrace$ by canonical forms (Lemma \ref{cf}).  If $e_{B}^{1}=\Lambda X.e_{B}^{2}$ then $^{\forall X.T_{1}}AB^{\forall X.T_{1}}$ $(\Lambda X.e_{B}^{2})\rightarrow\Lambda X.(^{T_{1}}AB^{T_{1}}$ $e_{B}^{2})$.  If $e_{B}^{1}={^{\forall X.T_{1}}B}S$ $v_{S}$ then $^{\forall X.T_{1}}AB^{\forall X.T_{1}}$ $(^{\forall X.T_{1}}BS$ $v_{S})\rightarrow{^{\forall X.T_{1}}A}S$ $v_{S}$.
\end{subcase}
\begin{subcase}
$T=[T_{1}]$

If $(A,B)=(H,M)$ then $e_{B}^{1}\in\lbrace\mathtt{cons}$ $v_{M}^{1}$ $v_{M}^{2},\mathtt{nil}^{T_{1}},{^{[T_{1}]}M}H^{[T_{1}]}$ $(\mathtt{cons}$ $e_{H}^{1}$ $e_{H}^{2})\rbrace$ by canonical forms (Lemma \ref{cf}).  If $e_{B}^{1}=\mathtt{cons}$ $v_{M}^{1}$ $v_{M}^{2}$ then $^{[T_{1}]}HM^{[T_{1}]}$ $(\mathtt{cons}$ $v_{M}^{1}$ $v_{M}^{2})\rightarrow\mathtt{cons}$ $(^{T_{1}}HM^{T_{1}}$ $v_{M}^{1})$ $(^{[T_{1}]}HM^{[T_{1}]}$ $v_{M}^{2})]$.  If $e_{B}^{1}=\mathtt{nil}^{T_{1}}$ then $^{[T_{1}]}HM^{[T_{1}]}$ $\mathtt{nil}^{T}\rightarrow\mathtt{nil}^{T}$.  If $e_{B}^{1}={^{[T_{1}]}M}H^{[T_{1}]}$ $(\mathtt{cons}$ $e_{H}^{1}$ $e_{H}^{2})$ then $^{[T_{1}]}HM^{[T_{1}]}$ $(^{[T_{1}]}MH^{[T_{1}]}$ $(\mathtt{cons}$ $e_{H}^{1}$ $e_{H}^{2}))\rightarrow\mathtt{cons}$ $e_{H}^{1}$ $e_{H}^{2}$.

If $(A,B)=(M,H)$ then $e_{B}^{1}\in\lbrace\mathtt{cons}$ $e_{H}^{3}$ $e_{H}^{4},\mathtt{nil}^{T_{1}}\rbrace$ by canonical forms.  If $e_{B}^{1}=\mathtt{cons}$ $e_{H}^{3}$ $e_{H}^{4}$ then $^{[T_{1}]}MH^{[T_{1}]}$ $(\mathtt{cons}$ $e_{H}^{3}$ $e_{H}^{4})$ is a value.  If $e_{B}^{1}=\mathtt{nil}^{T_{1}}$ then $^{[T_{1}]}MH^{[T_{1}]}$ $\mathtt{nil}^{T_{1}}\rightarrow\mathtt{nil}^{T_{1}}$.
\end{subcase}
\begin{subcase}
$T=L$

$e_{B}^{1}={^{L}B}S$ $v_{S}$ by canonical forms (Lemma \ref{cf}).  $^{L}AB^{L}$ $(^{L}BS$ $v_{S})\rightarrow{^{L}A}S$ $v_{S}$.
\end{subcase}
\begin{subcase}
$T=T_{1}^{a}$

Cannot occur because $T_{1}^{a}$ occurs only in $^{T_{1}^{a}}AS$ $e_{S}$.
\end{subcase}
If $e_{B}^{1}\rightarrow e_{B}^{2}$ then $^{T}AB^{T}$ $e_{B}^{1}\rightarrow{^{T}A}B$ $e_{B}^{2}$.  If $e_{B}^{1}\rightarrow$ \emph{\textbf{Error}:\;string} then $^{T}AB^{T}$ $e_{B}^{1}\rightarrow$ \emph{\textbf{Error}:\;string}.
\end{case}
\begin{case}
$e_{A}={^{T}A}S\;e_{S}^{1}$ where $A\in\lbrace H,M\rbrace$

$e_{S}^{1}$ is a value or $e_{S}^{1}\rightarrow e_{S}^{2}$ or $e_{S}^{1}\rightarrow$ \emph{\textbf{Error}:\;string} by the induction hypothesis.  If $e_{S}^{1}$ is a value then $T$ determines its reduction.
\begin{subcase}
$T=N$

If $e_{S}^{1}=\overline{n}$ then $^{N}AS\;\overline{n}\rightarrow\overline{n}$.  $^{N}AS\;e_{S}^{1}\rightarrow{^{N}A}S\;(\mathtt{wrong}\;\mathrm{``Not\;a\;number"})$ otherwise.
\end{subcase}
\begin{subcase}
$T=T_{1}\rightarrow T_{2}$

If $e_{S}^{1}=\lambda x_{1}.e_{S}^{2}$ then $^{T_{1}\rightarrow T_{2}}AS\;(\lambda x_{1}.e_{S}^{2})\rightarrow\lambda x_{2}:T_{1}[T_{i}/T^{a}_{i}].(^{T_{2}}AS\;((\lambda x_{1}.e_{S}^{2})\;(SA^{T_{1}}\;x_{2})))$.  $^{T_{1}\rightarrow T_{2}}AS\;e_{S}^{1}\rightarrow{^{T_{1}\rightarrow T_{2}}A}S\;(\mathtt{wrong}\;\mathrm{``Not\;a\;procedure"})$ otherwise.
\end{subcase}
\begin{subcase}
$T=\forall X.T_{1}$

$^{\forall X.T_{1}}AS\;e_{S}^{1}$ is a value.
\end{subcase}
\begin{subcase}
$T=[T_{1}]$

If $e_{S}^{1}=\mathtt{cons}\;v_{S}^{2}\;v_{S}^{3}$ then $^{[T_{1}]}AS\;(\mathtt{cons}\;v_{S}^{2}\;v_{S}^{3})\rightarrow\mathtt{cons}\;(^{T_{1}}AS\;v_{S}^{2})\;(^{[T_{1}]}AS\;v_{S}^{3})$.  $^{[T_{1}]}AS\;e_{S}^{1}\rightarrow{^{[T_{1}]}A}S\;(\mathtt{wrong}\;\mathrm{``Not\;a\;list"})$ otherwise.
\end{subcase}
\begin{subcase}
$T=L$

$^{L}AS\;e_{S}^{1}$ is a value.
\end{subcase}
\begin{subcase}
$T=T_{1}^{a}$

If $e_{S}^{1}=SA^{T_{1}^{a}}\;e_{A}^{1}$ then $^{T_{1}^{a}}AS\;(SA^{T_{1}^{a}}\;e_{A}^{1})\rightarrow e_{A}^{1}$.  $^{T_{1}^{a}}AS\;e_{S}^{1}\rightarrow{^{T_{1}^{a}}A}S\;(\mathtt{wrong}\;\mathrm{``Parametricity\;violated"})$ otherwise.
\end{subcase}
If $e_{S}^{1}\rightarrow e_{S}^{2}$ then $^{T}AS\;e_{S}^{1}\rightarrow\,^{T}AS\;e_{S}^{2}$.  If $e_{S}^{1}\rightarrow$ \emph{\textbf{Error}:\;string} then $^{T}AS\;e_{S}^{1}\rightarrow$ \emph{\textbf{Error}:\;string}.
\end{case}
\begin{case}
$e_{S}=SA^{T}$ $e_{A}^{1}$ where $(A,B)\in\lbrace(H,e),(M,v)\rbrace$

$e_{A}^{1}$ is a value or $e_{A}^{1}\rightarrow e_{A}^{2}$ or $e_{A}^{1}\rightarrow$ \emph{\textbf{Error}:\;string} by the induction hypothesis.  If $e_{A}^{1}$ is a value then $T$ determines its reduction.
\begin{subcase}
$T=N$

$e_{A}^{1}=\overline{n}$ by canonical forms (Lemma \ref{cf}).  $SA^{N}$ $\overline{n}\rightarrow\overline{n}$.
\end{subcase}
\begin{subcase}
$T=T_{1}\rightarrow T_{2}$

$e_{A}^{1}=\lambda x_{1}:T_{1}[T_{i}/T_{i}^{a}].e_{A}^{3}$ by canonical forms (Lemma \ref{cf}).  $SA^{T_{1}\rightarrow T_{2}}$ $(\lambda x_{1}:T_{1}[T_{i}/T_{i}^{a}].e_{A}^{3})\rightarrow\lambda x_{2}.(SA^{T_{2}}$ $((\lambda x_{1}:T_{1}[T_{i}/T_{i}^{a}].e_{A}^{3})$ $(^{T_{1}}AS$ $x_{2})))$.
\end{subcase}
\begin{subcase}
$T=\forall X_{1}.T_{1}$

$e_{A}^{1}\in\lbrace\Lambda X_{1}.e_{A}^{3},{^{\forall X_{1}.T_{1}}A}S$ $v_{S}\rbrace$ by canonical forms (Lemma \ref{cf}).  If $e_{A}^{1}=\Lambda X_{1}.e_{A}^{3}$ then $SA^{\forall X_{1}.T_{1}}$ $(\Lambda X_{1}.e_{A}^{3})\rightarrow\Lambda X_{2}.(SA^{T_{1}[X_{2}/X_{1}]}$ $((\Lambda X_{1}.e_{A}^{3})$ $\lbrace X_{2}\rbrace))$.  If $e_{A}^{1}={^{\forall X_{1}.T_{1}}A}S$ $v_{S}$ then it reduces by Case \ref{as}.
\end{subcase}
\begin{subcase}
$T=[T_{1}]$

If $A=H$ then $e_{H}^{1}\in\lbrace\mathtt{cons}$ $e_{H}^{1}$ $e_{H}^{2},\mathtt{nil}^{T_{1}}\rbrace$ by canonical forms (Lemma \ref{cf}).  If $e_{H}^{1}=\mathtt{cons}$ $e_{H}^{1}$ $e_{H}^{2}$ then $SH^{[T_{1}]}$ $(\mathtt{cons}$ $e_{H}^{1}$ $e_{H}^{2})$ is a value.  If $e_{H}^{1}=\mathtt{nil}^{T_{1}}$ then $SH^{T_{1}}$ $\mathtt{nil}^{T_{1}}\rightarrow\mathtt{nil}$.

If $A=M$ then $e_{M}^{1}\in\lbrace\mathtt{cons}$ $v_{M}^{1}$ $v_{M}^{2},\mathtt{nil}^{T_{1}},{^{[T_{1}]}M}H^{[T_{1}]}$ $(\mathtt{cons}$ $e_{H}^{1}$ $e_{H}^{2})\rbrace$ by canonical forms (Lemma \ref{cf}).  If $e_{M}^{1}=\mathtt{cons}$ $v_{M}^{1}$ $v_{M}^{2}$ then $SM^{[T_{1}]}$ $(\mathtt{cons}$ $v_{M}^{1}$ $v_{M}^{2})\rightarrow\mathtt{cons}$ $(SM^{T_{1}}$ $v_{M}^{1})$ $(SM^{[T_{1}]}$ $v_{M}^{2})$.  If $e_{M}^{1}=\mathtt{nil}^{T_{1}}$ then $SM^{T_{1}}$ $\mathtt{nil}^{T_{1}}\rightarrow\mathtt{nil}$.  If $e_{M}^{1}={^{[T_{1}]}M}H^{[T_{1}]}$ $(\mathtt{cons}$ $e_{H}^{1}$ $e_{H}^{2})$ then $SM^{[T_{1}]}$ $({^{[T_{1}]}M}H^{[T_{1}]}$ $(\mathtt{cons}$ $e_{H}^{1}$ $e_{H}^{2}))\rightarrow SH^{[T_{1}]}$ $(\mathtt{cons}$ $e_{H}^{1}$ $e_{H}^{2})$.
\end{subcase}
\begin{subcase}
$T=L$

$e_{A}^{1}={^{L}A}S$ $v_{S}$ by canonical forms (Lemma \ref{cf}).  $SA^{L}$ $(^{L}AS$ $v_{S})\rightarrow v_{S}$.
\end{subcase}
\begin{subcase}
$T=T_{1}^{a}$

$SA^{T_{1}^{a}}$ $B_{A}^{3}$ is a value.
\end{subcase}
If $e_{A}^{1}\rightarrow e_{A}^{2}$ then $SA^{T}$ $e_{A}^{1}\rightarrow SA^{T}$ $e_{A}^{2}$.  If $e_{A}^{1}\rightarrow$ \emph{\textbf{Error}:\;string} then $SA^{T}$ $e_{A}^{1}\rightarrow$ \emph{\textbf{Error}:\;string}.
\end{case}
\end{proof}
\end{theorem}

\begin{hps}
\label{hps}
If $\vdash_{H}e_{H}:T$ then either $e_{H}$ is a value, $e_{H}\rightarrow e_{H}'$ for some $e_{H}'$, or $e_{H}\rightarrow$ \emph{\textbf{Error}:\;string} for some \emph{string}.
Inversion of the Haskell typing relation.
\begin{hi}
\label{hi}
\begin{enumerate}
\item If $\vdash_{H}\overline{n}:T$ then $T=N$.
\item If $\Gamma\vdash_{H}x:T$ then $x:T\in\Gamma$.
\item If $\Gamma\vdash_{H}\lambda x:T_{1}.e_{H}:T$ then $\Gamma\vdash_{H}T_{1}$, $\Gamma,x:T_{1}\vdash_{H}e_{H}:T_{2}$, and $T=T_{1}\rightarrow T_{2}$.
\item If $\Gamma\vdash_{H}\Lambda X.e_{H}:T$ then $\Gamma,X\vdash_{H}e_{H}:T_{1}$ and $T=\forall X.T_{1}$.
\item If $\Gamma\vdash_{H}\mathtt{cons}\;e_{H}^{1}\;e_{H}^{2}:T$ then $\Gamma\vdash_{H}e_{H}^{1}:T_{1}$, $\Gamma\vdash_{H}e_{H}^{2}:[T_{1}]$, and $T=[T_{1}]$.
\item If $\Gamma\vdash_{H}\mathtt{nil}:T$ then $T=\forall X.[X]$.
\item If $\Gamma\vdash_{H}e_{H}^{1}\;e_{H}^{2}:T$ then $\Gamma\vdash_{H}e_{H}^{1}:T_{1}\rightarrow T_{2}$, $\Gamma\vdash_{H}e_{H}^{2}:T_{1}$, and $T=T_{2}$.
\item If $\Gamma\vdash_{H}e_{H}\;\lbrace T_{1}\rbrace:T$ then $\Gamma\vdash_{H}e_{H}:\forall X.T_{2}$, $\Gamma\vdash_{H}T_{1}$, and $T=T_{2}[T_{1}/X]$.
\item If $\Gamma\vdash_{H}\mathtt{if0}\;e_{H}^{1}\;e_{H}^{2}\;e_{H}^{3}:T$ then $\Gamma\vdash_{H}e_{H}^{1}:N$, $\Gamma\vdash_{H}e_{H}^{2}:T_{1}$, $\Gamma\vdash_{H}e_{H}^{3}:T_{1}$, and $T=T_{1}$.
\item If $\Gamma\vdash_{H}o\;e_{H}^{1}\;e_{H}^{2}:T$ then $\Gamma\vdash_{H}e_{H}^{1}:N$, $\Gamma\vdash_{H}e_{H}^{2}:N$, and $T=N$.
\item If $\Gamma\vdash_{H}\mathtt{hd}\;e_{H}:T$ then $\Gamma\vdash_{H}e_{H}:[T_{1}]$ and $T=T_{1}$.
\item If $\Gamma\vdash_{H}\mathtt{tl}\;e_{H}:T$ then $\Gamma\vdash_{H}e_{H}:[T_{1}]$ and $T=[T_{1}]$.
\item If $\Gamma\vdash_{H}\mathtt{fix}\;e_{H}:T$ then $\Gamma\vdash_{H}e_{H}:(T_{1}\rightarrow T_{1})\rightarrow T_{1}$ and $T=T_{1}$.
\item If $\Gamma\vdash_{H}\;^{T_{1}}HM^{T_{2}}\;e_{M}:T$ then $\Gamma\vdash_{H}T_{1}$, $\Gamma\vdash_{M}T_{2}$, $T_{1}=T_{2}$, $\Gamma\vdash_{M}e_{M}:T_{2}$, and $T=T_{1}$.
\item If $\Gamma\vdash_{H}\;^{T_{1}}HS\;e_{S}:T$ then $\Gamma\vdash_{H}T_{1}$, $\Gamma\vdash_{S}e_{S}:TST$, and $T=T_{1}$.
\end{enumerate}
\begin{proof}
Immediate from the definition of the Haskell typing relation.
\end{proof}
\end{hi}
Each term $e_{H}$ has at most one type.
\begin{huot}
\label{huot}
If $\Gamma\vdash_{H}e_{H}:T_{1}$ and $\Gamma\vdash_{H}e_{H}:T_{2}$ then $T_{1}=T_{2}$.
\begin{proof}
Straightforward structural induction on $e_{H}$.
\end{proof}
\end{huot}
\begin{hcf}
\label{hcf}
Canonical Forms
\begin{enumerate}
\item If $e_{H}$ is a normal form of type $N$ then $e_{H}=\overline{n}$.
\item If $e_{H}$ is a normal form of type $T_{1}\rightarrow T_{2}$ then $e_{H}=\lambda x:T_{1}.e_{H}$.
\item If $e_{H}$ is a normal form of type $\forall X.T$ then $e_{H}=\Lambda X.e_{H}$ or $e_{H}=\,^{\forall X.T}HS\;v_{S}$.
\item If $e_{H}$ is a normal form of type $[T]$ then $e_{H}=\mathtt{cons}\;e_{H}^{1}\;e_{H}^{2}$.
\item If $e_{H}$ is a normal form of type $L$ then $e_{H}=\,^{L}HS\;v_{S}$.
\end{enumerate}
\begin{proof}
Immediate from the definition of Haskell values and the Haskell typing relation.
\end{proof}
\end{hcf}
\begin{proof}
Straightforward structural induction on $e_{H}$.
\begin{hps-case-1}
$e_{H}=x$

Cannot occur because $e_{H}$ is closed.
\end{hps-case-1}
\begin{hps-case-2}
$e_{H}=\lambda x:T_{1}.e_{H}^{1}$

$\lambda x:T_{1}.e_{H}^{1}$ is a value.
\end{hps-case-2}
\begin{hps-case-3}
$e_{H}=\Lambda X.e_{H}^{1}$

$\Lambda X.e_{H}^{1}$ is a value.
\end{hps-case-3}
\begin{hps-case-4}
$e_{H}=\mathtt{cons}\;e_{H}^{1}\;e_{H}^{2}$

$\mathtt{cons}\;e_{H}^{1}\;e_{H}^{2}$ is a value.
\end{hps-case-4}
\begin{hps-case-5}
$e_{H}=\mathtt{nil}$

$\mathtt{nil}$ is a value.
\end{hps-case-5}
\begin{hps-case-6}
$e_{H}=\overline{n}$

$\overline{n}$ is a value.
\end{hps-case-6}
\begin{hps-case-7}
$e_{H}=e_{H}^{1}\;e_{H}^{2}$

By the induction hypothesis, either $e_{H}^{1}$ is a value, $e_{H}^{1}\rightarrow e_{H}^{3}$ for some $e_{H}^{3}$, or $e_{H}^{1}\rightarrow$ \emph{\textbf{Error}:\;string} for some \emph{string}.  If $e_{H}^{1}$ is a value then $e_{H}^{1}:T_{1}\rightarrow T_{2}$ for some $T_{1}$ and $T_{2}$ by inversion (Lemma \ref{hi}) and uniqueness of types (Lemma \ref{huot}) and $e_{H}^{1}=\lambda x:T_{1}.e_{H}^{3}$ for some $x$ and $e_{H}^{3}$ by canonical forms (Lemma \ref{hcf}).  Therefore $(\lambda x:T.e_{H}^{3})\;e_{H}^{2}\rightarrow e_{H}^{3}[e_{H}^{2}/x]$.  If $e_{H}^{1}\rightarrow e_{H}^{3}$ for some $e_{H}^{3}$ then $e_{H}^{1}\;e_{H}^{2}\rightarrow e_{H}^{3}\;e_{H}^{2}$.  If $e_{H}^{1}\rightarrow$ \emph{\textbf{Error}:\;string} for some \emph{string} then $e_{H}^{1}\;e_{H}^{2}\rightarrow$ \emph{\textbf{Error}:\;string}.
\end{hps-case-7}
\begin{hps-case-8}
$e_{H}=e_{H}^{1}\;\lbrace T_{1}\rbrace$

By the induction hypothesis, either $e_{H}^{1}\rightarrow e_{H}^{2}$ for some $e_{H}^{2}$, $e_{H}^{1}\rightarrow$ \emph{\textbf{Error}:\;string} for some \emph{string}, or $e_{H}^{1}$ is a value.  If $e_{H}^{1}\rightarrow e_{H}^{2}$ for some $e_{H}^{2}$ then $e_{H}^{1}\;\lbrace T_{1}\rbrace\rightarrow e_{H}^{2}\;\lbrace T_{1}\rbrace$.  If $e_{H}^{1}\rightarrow$ \emph{\textbf{Error}:\;string} for some \emph{string} then $e_{H}^{1}\;\lbrace T_{1}\rbrace\rightarrow$ \emph{\textbf{Error}:\;string}.  If $e_{H}^{1}$ is a value then $e_{H}^{1}:\forall X.T_{2}$ for some $X$ and $T_{2}$ by inversion (Lemma \ref{hi}) and uniqueness of types (Lemma \ref{huot}) and either $e_{H}^{1}=\Lambda X.e_{H}^{2}$ for some $e_{H}^{2}$ or $e_{H}^{1}=\,^{\forall X.T_{2}}HS\;v_{S}$ for some $v_{S}$ by canonical forms (Lemma \ref{hcf}).
\end{hps-case-8}
\begin{hps-case-8-1}
$e_{H}^{1}=\Lambda X.e_{H}^{2}$

$(\Lambda X.e_{H}^{2})\;\lbrace T_{1}\rbrace\rightarrow e_{H}^{2}[T_{1}/X]$.
\end{hps-case-8-1}
\begin{hps-case-8-2}
$e_{H}^{1}=\,^{\forall X.T_{2}}HS\;v_{S}$

$(^{\forall X.T_{2}}HS\;v_{S})\;\lbrace T_{1}\rbrace\rightarrow\,^{T_{2}[T_{1}^{a}/X]}HS\;v_{S}$.
\end{hps-case-8-2}
\begin{hps-case-9}
$e_{H}=\mathtt{if0}\;e_{H}^{1}\;e_{H}^{2}\;e_{H}^{3}$

By the induction hypothesis, either $e_{H}^{1}$ is a value, $e_{H}^{1}\rightarrow e_{H}^{4}$ for some $e_{H}^{4}$, or $e_{H}^{1}\rightarrow$ \emph{\textbf{Error}:\;string} for some \emph{string}.  If $e_{H}^{1}$ is a value then $e_{H}^{1}:N$ by inversion (Lemma \ref{hi}) and uniqueness of types (Lemma \ref{huot}) and $e_{H}^{1}=\overline{n}$ for some $\overline{n}$ by canonical forms (Lemma \ref{hcf}).  Therefore $\mathtt{if0}\;e_{H}^{1}\;e_{H}^{2}\;e_{H}^{3}\rightarrow e_{H}^{2}$ if $e_{H}^{1}=\overline{0}$ or $\mathtt{if0}\;e_{H}^{1}\;e_{H}^{2}\;e_{H}^{3}\rightarrow e_{H}^{3}$ if $e_{H}^{1}\neq\overline{0}$.  If $e_{H}^{1}\rightarrow e_{H}^{4}$ for some $e_{H}^{4}$ then $\mathtt{if0}\;e_{H}^{1}\;e_{H}^{2}\;e_{H}^{3}\rightarrow \mathtt{if0}\;e_{H}^{4}\;e_{H}^{2}\;e_{H}^{3}$.  If $e_{H}^{1}\rightarrow$ \emph{\textbf{Error}:\;string} for some \emph{string} then $\mathtt{if0}\;e_{H}^{1}\;e_{H}^{2}\;e_{H}^{3}\rightarrow$ \emph{\textbf{Error}:\;string}.
\end{hps-case-9}
\begin{hps-case-10}
$e_{H}=o\;e_{H}^{1}\;e_{H}^{2}$

By the induction hypothesis, either $e_{H}^{1}$ is a value, $e_{H}^{1}\rightarrow e_{H}^{3}$ for some $e_{H}^{3}$, or $e_{H}^{1}\rightarrow$ \emph{\textbf{Error}:\;string} for some \emph{string}.  If $e_{H}^{1}$ is a value then $e_{H}^{1}:N$ by inversion (Lemma \ref{hi}) and uniqueness of types (Lemma \ref{huot}) and $e_{H}^{1}=\overline{n_{1}}$ for some $\overline{n_{1}}$ by canonical forms (Lemma \ref{hcf}).  If $e_{H}^{1}\rightarrow e_{H}^{3}$ for some $e_{H}^{3}$ then $o\;e_{H}^{1}\;e_{H}^{2}\rightarrow o\;e_{H}^{3}\;e_{H}^{2}$.  If $e_{H}^{1}\rightarrow$ \emph{\textbf{Error}:\;string} for some \emph{string} then $o\;e_{H}^{1}\;e_{H}^{2}\rightarrow$ \emph{\textbf{Error}:\;string}.

By the induction hypothesis, either $e_{H}^{2}$ is a value, $e_{H}^{2}\rightarrow e_{H}^{3}$ for some $e_{H}^{3}$, or $e_{H}^{2}\rightarrow$ \emph{\textbf{Error}:\;string} for some \emph{string}.  If $e_{H}^{2}$ is a value then $e_{H}^{2}:N$ by inversion (Lemma \ref{hi}) and uniqueness of types (Lemma \ref{huot}) and $e_{H}^{2}=\overline{n_{2}}$ for some $\overline{n_{2}}$ by canonical forms (Lemma \ref{hcf}).  If $e_{H}^{2}\rightarrow e_{H}^{3}$ for some $e_{H}^{3}$ then $o\;e_{H}^{1}\;e_{H}^{2}\rightarrow o\;e_{H}^{1}\;e_{H}^{3}$.  If $e_{H}^{1}\rightarrow$ \emph{\textbf{Error}:\;string} for some \emph{string} then $o\;e_{H}^{1}\;e_{H}^{2}\rightarrow$ \emph{\textbf{Error}:\;string}.

If $e_{H}^{1}=\overline{n_{1}}$ and $e_{H}^{2}=\overline{n_{2}}$ then $o\;e_{H}^{1}\;e_{H}^{2}\rightarrow \overline{n_{1}+n_{2}}$ if $o=+$ or $o\;e_{H}^{1}\;e_{H}^{2}\rightarrow \overline{max(n_{1}-n_{2},0)}$ if $o=-$.
\end{hps-case-10}
\begin{hps-case-11}
$e_{H}=f\;e_{H}^{1}$

By the induction hypothesis, either $e_{H}^{1}$ is a value, $e_{H}^{1}\rightarrow e_{H}^{2}$ for some $e_{H}^{2}$, or $e_{H}^{1}\rightarrow$ \emph{\textbf{Error}:\;string} for some \emph{string}.  If $e_{H}^{1}$ is a value then $e_{H}^{1}:[T]$ for some $T$ by inversion (Lemma \ref{hi}) and uniqueness of types (Lemma \ref{huot}) and $e_{H}^{1}=\mathtt{cons}\;e_{H}^{2}\;e_{H}^{3}$ for some $e_{H}^{2}$ and $e_{H}^{3}$ by canonical forms (Lemma \ref{hcf}).  Therefore $f\;(\mathtt{cons}\;e_{H}^{2}\;e_{H}^{3})\rightarrow e_{H}^{2}$ if $f=\mathtt{hd}$ or $f\;(\mathtt{cons}\;e_{H}^{2}\;e_{H}^{3})\rightarrow e_{H}^{3}$ if $f=\mathtt{tl}$.  If $e_{H}^{1}\rightarrow e_{H}^{2}$ for some $e_{H}^{2}$ then $f\;e_{H}^{1}\rightarrow f\;e_{H}^{2}$.  If $e_{H}^{1}\rightarrow$ \emph{\textbf{Error}:\;string} for some \emph{string} then $f\;e_{H}^{1}\rightarrow$ \emph{\textbf{Error}:\;string}.
\end{hps-case-11}
\begin{hps-case-12}
$e_{H}=\mathtt{fix}\;e_{H}^{1}$

By the induction hypothesis, either $e_{H}^{1}$ is a value, $e_{H}^{1}\rightarrow e_{H}^{2}$ for some $e_{H}^{2}$, or $e_{H}^{1}\rightarrow$ \emph{\textbf{Error}:\;string} for some \emph{string}.  If $e_{H}^{1}$ is a value then $e_{H}^{1}:T_{1}\rightarrow T_{2}$ for some $T_{1}$ and $T_{2}$ by inversion (Lemma \ref{hi}) and uniqueness of types (Lemma \ref{huot}) and $e_{H}^{1}=\lambda x:T_{1}.e_{H}^{2}$ for some $x$ and $e_{H}^{2}$ by canonical forms (Lemma \ref{hcf}).  Therefore $\mathtt{fix}\;(\lambda x:T_{1}.e_{H}^{2})\rightarrow e_{H}^{2}[\mathtt{fix}\;(\lambda x:T_{1}.e_{H}^{2})/x]$.  If $e_{H}^{1}\rightarrow e_{H}^{2}$ for some $e_{H}^{2}$ then $\mathtt{fix}\;e_{H}^{1}\rightarrow\mathtt{fix}\;e_{H}^{2}$.  If $e_{H}^{1}\rightarrow$ \emph{\textbf{Error}:\;string} for some \emph{string} then $\mathtt{fix}\;e_{H}^{1}\rightarrow$ \emph{\textbf{Error}:\;string}.
\end{hps-case-12}
\begin{hps-case-13}
$e_{H}=\,^{T}HM^{T}\;e_{M}^{1}$

By the ML progress theorem (Theorem \ref{mps}), either $e_{M}^{1}\rightarrow e_{M}^{2}$ for some $e_{M}^{2}$, $e_{M}^{1}\rightarrow$ \emph{\textbf{Error}:\;string} for some \emph{string}, or $e_{M}^{1}$ is a value.  If $e_{M}^{1}\rightarrow e_{M}^{2}$ for some $e_{M}^{2}$ then $^{T}HS\;e_{M}^{1}\rightarrow\,^{T}HS\;e_{M}^{2}$.  If $e_{M}^{1}\rightarrow$ \emph{\textbf{Error}:\;string} for some \emph{string} then $^{T}HS\;e_{M}^{1}\rightarrow$ \emph{\textbf{Error}:\;string}.  If $e_{M}^{1}$ is a value then $T$ determines its reduction:
\begin{hps-case-13-1}
$T=N$

$e_{M}^{1}=\overline{n}$ for some $\overline{n}$ by canonical forms (Lemma \ref{mcf}).  $^{N}HM^{N}\;\overline{n}\rightarrow\overline{n}$.
\end{hps-case-13-1}
\begin{hps-case-13-2}
$T=T_{1}\rightarrow T_{2}$

$e_{M}^{1}=\lambda x_{1}:T_{1}.e_{M}^{2}$ for some $x_{1}$ and $e_{M}^{2}$ by canonical forms (Lemma \ref{mcf}).  $^{T_{1}\rightarrow T_{2}}HM^{T_{1}\rightarrow T_{2}}\;(\lambda x_{1}:T_{1}.e_{M}^{2})\rightarrow\lambda x_{2}:T_{1}[T_{i}/T^{a}_{i}].(^{T_{2}}HM^{T_{2}}\;((\lambda x_{1}:T_{1}.e_{M}^{2})\;(^{T_{1}}MH^{T_{1}}\;x_{2})))$.
\end{hps-case-13-2}
\begin{hps-case-13-3}
$T=\forall X_{1}.T_{1}$

$e_{M}^{1}=\Lambda X_{1}.e_{M}^{2}$ for some $e_{M}^{2}$ by canonical forms (Lemma \ref{mcf}).  $^{\forall X_{1}.T_{1}}HM^{\forall X_{1}.T_{1}}\;(\Lambda X_{1}.e_{M}^{2})\rightarrow\Lambda X_{2}.(^{T_{1}[X_{2}/X_{1}]}HM^{T_{1}[X_{2}/X_{1}]}\;((\Lambda X_{1}.e_{M}^{2})\;\lbrace X_{2}\rbrace))$.
\end{hps-case-13-3}
\begin{hps-case-13-4}
$T=[T_{1}]$

$e_{M}^{1}=\mathtt{cons}\;v_{M}^{1}\;v_{M}^{2}$ for some $v_{M}^{1}$ and $v_{M}^{1}$ by canonical forms (Lemma \ref{mcf}).  $^{[T_{1}]}HM^{[T_{1}]}\;(\mathtt{cons}\;v_{M}^{1}\;v_{M}^{2})\rightarrow\mathtt{cons}\;(^{T_{1}}HM^{T_{1}}\;v_{M}^{1})\;(^{[T_{1}]}HM^{[T_{1}]}\;v_{M}^{2})$.
\end{hps-case-13-4}
\begin{hps-case-13-5}
$T=L$

Cannot occur because $L$ occurs only in $^{L}HS\;e_{S}$.
\end{hps-case-13-5}
\begin{hps-case-13-6}
$T=T_{1}^{a}$

Cannot occur because $T_{1}^{a}$ occurs only in $^{T_{1}^{a}}HS\;e_{S}$.
\end{hps-case-13-6}
\end{hps-case-13}
\begin{hps-case-14}
$e_{H}=\,^{T}HS\;e_{S}^{1}$

By the Scheme progress theorem (Theorem \ref{sps}), either $e_{S}^{1}\rightarrow e_{S}^{2}$ for some $e_{S}^{2}$, $e_{S}^{1}\rightarrow$ \emph{\textbf{Error}:\;string} for some \emph{string}, or $e_{S}^{1}$ is a value.  If $e_{S}^{1}\rightarrow e_{S}^{2}$ for some $e_{S}^{2}$ then $^{T}HS\;e_{S}^{1}\rightarrow\,^{T}HS\;e_{S}^{2}$.  If $e_{S}^{1}\rightarrow$ \emph{\textbf{Error}:\;string} for some \emph{string} then $^{T}HS\;e_{S}^{1}\rightarrow$ \emph{\textbf{Error}:\;string}.  If $e_{S}^{1}$ is a value then $T$ determines its reduction:
\begin{hps-case-14-1}
$T=N$

If $e_{S}^{1}=\overline{n}$ for some $\overline{n}$ then $^{N}HS\;\overline{n}\rightarrow\overline{n}$.  Otherwise, $^{N}HS\;e_{S}^{1}\rightarrow\,^{N}HS\;(\mathtt{wrong}\;\mathrm{``Not\;a\;number"})$.
\end{hps-case-14-1}
\begin{hps-case-14-2}
$T=T_{1}\rightarrow T_{2}$

If $e_{S}^{1}=\lambda x_{1}.e_{S}^{2}$ for some $x_{1}$ and $e_{S}^{2}$ then $^{T_{1}\rightarrow T_{2}}HS\;(\lambda x_{1}.e_{S}^{2})\rightarrow\lambda x_{2}:T_{1}[T_{i}/T^{a}_{i}].(^{T_{2}}HS\;((\lambda x_{1}.e_{S}^{2})\;(SH^{T_{1}}\;x_{2})))$.  Otherwise, $^{T_{1}\rightarrow T_{2}}HS\;e_{S}^{1}\rightarrow\,^{T_{1}\rightarrow T_{2}}HS\;(\mathtt{wrong}\;\mathrm{``Not\;a\;procedure"})$.
\end{hps-case-14-2}
\begin{hps-case-14-3}
$T=\forall X.T_{1}$

$^{\forall X.T_{1}}HS\;e_{S}^{1}$ is a value.
\end{hps-case-14-3}
\begin{hps-case-14-4}
$T=[T_{1}]$

If $e_{S}^{1}=\mathtt{cons}\;v_{S}^{2}\;v_{S}^{3}$ for some $v_{S}^{2}$ and $v_{S}^{3}$ then $^{[T_{1}]}HS\;(\mathtt{cons}\;v_{S}^{2}\;v_{S}^{3})\rightarrow\mathtt{cons}\;(^{T_{1}}HS\;v_{S}^{2})\;(^{[T_{1}]}HS\;v_{S}^{3})$.  Otherwise, $^{[T_{1}]}HS\;e_{S}^{1}\rightarrow\,^{[T_{1}]}HS\;(\mathtt{wrong}\;\mathrm{``Not\;a\;list"})$.
\end{hps-case-14-4}
\begin{hps-case-14-5}
$T=L$

$^{L}HS\;e_{S}^{1}$ is a value.
\end{hps-case-14-5}
\begin{hps-case-14-6}
$T=T_{1}^{a}$

If $e_{S}^{1}=SH^{T_{1}^{a}}\;e_{H}^{1}$ then $^{T_{1}^{a}}HS\;(SH^{T_{1}^{a}}\;e_{H}^{1})\rightarrow e_{H}^{1}$.  Otherwise, $^{T_{1}^{a}}HS\;e_{S}^{1}\rightarrow\,^{T_{1}^{a}}HS\;(\mathtt{wrong}\;\mathrm{``Parametricity\;violated"})$.
\end{hps-case-14-6}
\end{hps-case-14}
\end{proof}
\end{hps}

\section{Preservation}

Preservation will be proven by cases on the rewrite rules.  In each case, the right side is be proven to be well-typed and have the same type as the left side.  Inversion (Lemma \ref{i}) and uniqueness of types (Lemma \ref{uot}) are used to determine the types of the left side and its subexpressions and the type of the right side.  Some rewrite rules use expression and type substitutions.

If $e_{A}^{1}$ is substituted for free occurrences of $x$ within $e_{A}^{2}$, $e_{A}^{1}$ and $x$ have the same type, and the result has the same type as $e_{A}^{2}$, where $A\in\lbrace H,M,S\rbrace$:

\begin{lemma}
\label{tms}
If $\Gamma,x_{1}:T_{1}\vdash_{A}e_{A}:T_{2}$ and $\Gamma\vdash_{A}x_{2}:T_{1}$ then $\Gamma\vdash_{A}e_{A}[x_{2}/x_{1}]:T_{2}$ where $A\in\lbrace H,M\rbrace$.  If $\Gamma,x_{1}:TST\vdash_{S}e_{S}:TST$ and $\Gamma\vdash_{S}x_{2}:TST$ then $\Gamma\vdash_{S}e_{S}[x_{2}/x_{1}]:TST$.
\begin{proof}
By structural induction.
\end{proof}
\end{lemma}

If $T_{1}$ is substituted for free occurrences of $X$ within $e_{A}$ of type $T_{2}$, the type of the result is $T_{1}$ substituted for free occurrences of $X$ within $T_{2}$, where $A\in\lbrace H,M\rbrace$:

\begin{lemma}
\label{tes}
\onehalfspacing
If $\Gamma,X\vdash_{HM}e_{HM}:T_{1}$ and $\vdash_{HM}T_{2}$ then $\Gamma\vdash_{HM}e_{HM}[T_{2}/X]:T_{1}[T_{2}/X]$.
\begin{proof}
By structural induction.
\end{proof}
\end{lemma}

\begin{lemma}
\label{ecp}
%\onehalfspacing
If $\Gamma\vdash_{A}e_{A}^{1}:T_{1}$, $\Gamma\vdash_{A}e_{A}^{2}:T_{2}$, and $\mathscr{E}[e_{A}^{1}]:T_{1}$ then $\mathscr{E}[e_{A}^{2}]:T_{2}$ where $A\in\lbrace H,M,S\rbrace$.
\begin{proof}
By structural induction.
\end{proof}
\end{lemma}

\begin{theorem}
\label{pn}
If $\Gamma\vdash_{A}e_{A}^{1}:T$ and $e_{A}^{1}\rightarrow e_{A}^{2}$ then $\Gamma\vdash_{A}e_{A}^{2}:T$ where $A\in\lbrace H,M\rbrace$.  If $\Gamma\vdash_{S}e_{S}^{1}:TST$ and $e_{S}^{1}\rightarrow e_{S}^{2}$ then $\Gamma\vdash_{S}e_{S}^{2}:TST$.
\begin{proof}
By cases on the reductions $e_{A}^{1}\rightarrow e_{A}^{2}$ and $e_{S}^{1}\rightarrow e_{S}^{2}$.  Straightforward cases of Scheme preservation are elided.
\begin{case}
$(\lambda x:T_{1}.e_{HM}^{1})\;e_{HM}^{2}\rightarrow e_{HM}^{1}[e_{HM}^{2}/x]$

$\Gamma\vdash_{HM}(\lambda x:T_{1}.e_{HM}^{1})\;e_{HM}^{2}:T$ by the premise and uniqueness of types (Lemma \ref{uot}).  $\Gamma\vdash_{HM}\lambda x:T_{1}.e_{HM}^{1}:T_{1}\rightarrow T$, $\Gamma,x:T_{1}\vdash_{HM}e_{HM}^{1}:T$, $\Gamma\vdash_{HM}e_{HM}^{2}:T_{1}$, and $\Gamma,x:T_{1}\vdash_{HM}x:T_{1}$ by inversion (Lemma \ref{i}) and uniqueness of types.  $e_{HM}^{1}[e_{HM}^{2}/x]:T$ by term substitution (Lemma \ref{tms}).
\end{case}
\begin{case}
$\mathscr{E}[(\Lambda X.e_{H})\;\lbrace T_{1}\rbrace]_{H}\rightarrow\mathscr{E}[e_{H}[T_{1}/X]]$

$(\Lambda X.e_{H})\;\lbrace T_{1}\rbrace:T$ by the induction hypothesis and uniqueness of types (Lemma \ref{uot}).

%\textbf{!!! NOT DONE !!!}
%$\Gamma\vdash_{H}e_{H}:\forall X.T_{2}$, $\Gamma\vdash_{H}T_{1}$, and $T=T_{2}[T_{1}/X]$.
\end{case}
\begin{case}
$\mathtt{if0}\;\overline{0}\;e_{HM}^{1}\;e_{HM}^{2}\rightarrow e_{HM}^{1}$

$\Gamma\vdash_{HM}\mathtt{if0}\;\overline{0}\;e_{HM}^{1}\;e_{HM}^{2}:T$ by premise and uniqueness of types (Lemma \ref{uot}).  $\Gamma\vdash_{HM}e_{HM}^{1}:T$ by inversion (Lemma \ref{i}) and uniqueness of types.
\end{case}
\begin{case}
$\mathtt{if0}\;\overline{n}\;e_{A}^{1}\;e_{A}^{2}\rightarrow e_{A}^{2}\;(n\neq0)$ where $A\in\lbrace H,M\rbrace$

$\Gamma\vdash_{A}\mathtt{if0}\;\overline{n}\;e_{A}^{1}\;e_{A}^{2}:T$ by premise and uniqueness of types (Lemma \ref{uot}).  $T=T_{1}$ and $\Gamma\vdash_{A}e_{A}^{2}:T_{1}$ by inversion (Lemma \ref{i}) and uniqueness of types.  $\Gamma\vdash_{A}e_{A}^{2}:T$ because $T_{1}=T$.
\end{case}
\begin{case}
$+\;\overline{n_{1}}\;\overline{n_{2}}\rightarrow\overline{n_{1}+n_{2}}$ where $A\in\lbrace H,M\rbrace$

$\vdash_{A}+\;\overline{n_{1}}\;\overline{n_{2}}:N$ by inversion (Lemma \ref{i}) and uniqueness of types (Lemma \ref{uot}).  $\vdash_{A}\overline{n_{1}+n_{2}}:N$ by inversion and uniqueness of types.
\end{case}
\begin{case}
$-\;\overline{n_{1}}\;\overline{n_{2}}\rightarrow\overline{max(n_{1}-n_{2},0)}$

$\vdash_{HM}-\;\overline{n_{1}}\;\overline{n_{2}}:N$ by inversion (Lemma \ref{i}) and uniqueness of types (Lemma \ref{uot}).  $\vdash_{HM}\overline{max(n_{1}-n_{2},0)}:N$ by inversion and uniqueness of types.
\end{case}
\begin{case}
$\mathtt{hd}$ $(\mathtt{cons}$ $e_{H}^{1}$ $e_{H}^{2})\rightarrow e_{H}^{1}$

$\Gamma\vdash_{H}\mathtt{hd}$ $(\mathtt{cons}$ $e_{H}^{1}$ $e_{H}^{2}):T$ by premise and uniqueness of types (Lemma \ref{uot}).  $\Gamma\vdash_{H}e_{H}^{1}:T_{1}$, $\Gamma\vdash_{H}\mathtt{cons}$ $e_{H}^{1}$ $e_{H}^{2}:[T_{1}]$, and $T=T_{1}$ by inversion (Lemma \ref{i}) and uniqueness of types.  $\Gamma\vdash_{H}e_{H}^{1}:T$ because $T_{1}=T$.
\end{case}
\begin{case}
$\mathtt{tl}\;(\mathtt{cons}\;e_{H}^{1}\;e_{H}^{2})\rightarrow e_{H}^{2}$

$\Gamma\vdash_{H}\mathtt{tl}\;(\mathtt{cons}\;e_{H}^{1}\;e_{H}^{2}):T$ by premise and uniqueness of types (Lemma \ref{uot}).  $\Gamma\vdash_{H}e_{H}^{2}:[T_{1}]$, $\Gamma\vdash_{H}\mathtt{cons}\;e_{H}^{1}\;e_{H}^{2}:[T_{1}]$, and $T=[T_{1}]$ by inversion (Lemma \ref{i}) and uniqueness of types.  $\Gamma\vdash_{H}e_{H}^{2}:T$ because $[T_{1}]=T$.
\end{case}
\begin{case}
$\mathtt{hd}\;(\mathtt{cons}\;v_{M}^{1}\;v_{M}^{2})\rightarrow v_{M}^{1}$

$\Gamma\vdash_{M}\mathtt{hd}\;(\mathtt{cons}\;v_{M}^{1}\;v_{M}^{2}):T$ by premise and uniqueness of types (Lemma \ref{uot}).  $\Gamma\vdash_{M}v_{M}^{1}:T_{1}$, $\Gamma\vdash_{M}\mathtt{cons}\;v_{M}^{1}\;v_{M}^{2}:[T_{1}]$, and $T=T_{1}$ by inversion (Lemma \ref{i}) and uniqueness of types.  $\Gamma\vdash_{M}v_{M}^{1}:T$ because $T_{1}=T$.
\end{case}
\begin{case}
$\mathtt{tl}\;(\mathtt{cons}\;v_{M}^{1}\;v_{M}^{2})\rightarrow v_{M}^{2}$

$\Gamma\vdash_{M}\mathtt{tl}\;(\mathtt{cons}\;v_{M}^{1}\;v_{M}^{2}):T$ by premise and uniqueness of types (Lemma \ref{uot}).  $T=[T_{1}]$, $\Gamma\vdash_{M}\mathtt{cons}\;v_{M}^{1}\;v_{M}^{2}:[T_{1}]$, and $\Gamma\vdash_{M}v_{M}^{2}:[T_{1}]$ by inversion (Lemma \ref{i}) and uniqueness of types.  $\Gamma\vdash_{M}v_{M}^{2}:T$ because $[T_{1}]=T$.
\end{case}
\begin{case}
$\mathtt{hd}\;\mathtt{nil}^{T}\rightarrow\,^{T}B\;(\mathtt{wrong}\;\mathrm{``Empty\;list"})$ where $B\in\lbrace HS,MS\rbrace$

$\Gamma\vdash_{HM}\mathtt{hd}\;\mathtt{nil}^{T}:T$ by premise and uniqueness of types (Lemma \ref{uot}).  $\Gamma\vdash_{S}\mathtt{wrong}\;\mathrm{``Empty\;list"}:TST$ and $\Gamma\vdash_{HM}{^{T}B}\;(\mathtt{wrong}\;\mathrm{``Empty\;list"}):T$ by inversion (Lemma \ref{i}) and uniqueness of types.
\end{case}
\begin{case}
$\mathtt{tl}$ $\mathtt{nil}^{T_{1}}\rightarrow\mathtt{nil}^{T_{1}}$ where $A\in\lbrace H,M\rbrace$

$\Gamma\vdash_{A}\mathtt{tl}$ $\mathtt{nil}^{T_{1}}:T$ by premise and uniqueness of types (Lemma \ref{uot}).  $\Gamma\vdash_{A}\mathtt{nil}^{T_{1}}:[T_{1}]$ and $T=[T_{1}]$ by inversion (Lemma \ref{i}) and uniqueness of types.  $\Gamma\vdash_{A}\mathtt{nil}^{T_{1}}:T$ because $[T_{1}]=T$.
\end{case}
\begin{case}
$\mathtt{fix}\;(\lambda x:T_{1}.e_{A})\rightarrow e_{A}[(\mathtt{fix}\;(\lambda x:T_{1}.e_{A}))/x]$ where $A\in\lbrace H,M\rbrace$

$\Gamma\vdash_{A}\mathtt{fix}\;(\lambda x:T_{1}.e_{A}):T$ by premise and uniqueness of types (Lemma \ref{uot}).  $\Gamma\vdash_{A}\lambda x:T_{1}.e_{A}:T_{1}\rightarrow T_{1}$, $\Gamma,x:T_{1}\vdash_{A}e_{A}:T_{1}$, and $T=T_{1}$ by inversion (Lemma \ref{i}) and uniqueness of types.  $\Gamma\vdash_{A}e_{A}[(\mathtt{fix}\;(\lambda x:(T\rightarrow T).e_{A}))/x]:T_{1}$ by term substitution (Lemma \ref{tms}).  $\Gamma\vdash_{A}e_{A}[(\mathtt{fix}\;(\lambda x:(T\rightarrow T).e_{A}))/x]:T$ because $T_{1}=T$.
\end{case}
\begin{case}
$^{N}AB^{N}$ $\overline{n}\rightarrow\overline{n}$ where $(A,B)\in\lbrace(H,M),(M,H)\rbrace$

$\vdash_{A}{^{N}A}B^{N}$ $\overline{n}:T$ by premise and uniqueness of types (Lemma \ref{uot}).  $\vdash_{A}\overline{n}:N$ and $T=N$ by inversion (Lemma \ref{i}) and uniqueness of types.
\end{case}
\begin{case}
$^{N}AS$ $\overline{n}\rightarrow\overline{n}$ where $A\in\lbrace H,M\rbrace$

$\vdash_{A}{^{N}A}S$ $\overline{n}:T$ by premise and uniqueness of types (Lemma \ref{uot}).  $\vdash_{A}\overline{n}:N$ and $T=N$ by inversion (Lemma \ref{i}) and uniqueness of types.
\end{case}
\begin{case}
$^{N}AS\;v_{S}\rightarrow{^{N}A}S\;(\mathtt{wrong}\;\mathrm{``Not\;a\;number"})\;(v_{S}\neq\overline{n})$ where $A\in\lbrace H,M\rbrace$

$\Gamma\vdash_{A}{^{N}AS}\;v_{S}:T$ by premise and uniqueness of types (Lemma \ref{uot}).  $T=N$ by inversion (Lemma \ref{i}) and uniqueness of types.  $\vdash_{S}\mathtt{wrong}\;\mathrm{``Not\;a\;number"}:TST$ by inversion.  $\vdash_{A}{^{N}A}S\;(\mathtt{wrong}\;\mathrm{``Not\;a\;number"}):N$ by inversion and uniqueness of types.  $\vdash_{A}{^{N}A}S\;(\mathtt{wrong}\;\mathrm{``Not\;a\;number"}):T$ because $N=T$.
\end{case}
\begin{case}
$^{T_{1}\rightarrow T_{2}}AB^{T_{1}\rightarrow T_{2}}\;(\lambda x_{1}:T_{1}.e_{B})\rightarrow\lambda x_{2}:T_{1}.(^{T_{2}}AB^{T_{2}}\;((\lambda x_{1}:T_{1}.e_{B})\;(^{T_{1}}BA^{T_{1}}\;x_{2})))$ where $(A,B)\in\lbrace(H,M),(M,H)\rbrace$

$\Gamma\vdash_{A}{^{T_{1}\rightarrow T_{2}}}AB^{T_{1}\rightarrow T_{2}}\;(\lambda x_{1}:T_{1}.e_{B}):T$ by premise and uniqueness of types (Lemma \ref{uot}).  $\Gamma\vdash_{B}\lambda x_{1}:T_{1}.e_{B}:T_{1}\rightarrow T_{2}$, $T=T_{1}\rightarrow T_{2}$, $\Gamma,x_{2}:T_{1}\vdash_{A}x_{2}:T_{1}$, $\Gamma,x_{2}:T_{1}\vdash_{B}{^{T_{1}}B}A^{T_{1}}\;x_{2}:T_{1}$, $\Gamma,x_{2}:T_{1}\vdash_{B}(\lambda x_{1}:T_{1}.e_{B})\;(^{T_{1}}BA^{T_{1}}\;x_{2}):T_{2}$, $\Gamma,x_{2}:T_{1}\vdash_{A}{^{T_{2}}A}B^{T_{2}}\;((\lambda x_{1}:T_{1}.e_{B})\;(^{T_{1}}BA^{T_{1}}\;x_{2})):T_{2}$, and $\Gamma\vdash_{A}\lambda x_{2}:T_{1}.(^{T_{2}}AB^{T_{2}}\;((\lambda x_{1}:T_{1}.e_{B})\;(^{T_{1}}BA^{T_{1}}\;x_{2}))):T_{1}\rightarrow T_{2}$ by inversion (Lemma \ref{i}) and uniqueness of types.  $\Gamma\vdash_{A}\lambda x_{2}:T_{1}.(^{T_{2}}AB^{T_{2}}\;((\lambda x_{1}:T_{1}.e_{B})\;(^{T_{1}}BA^{T_{1}}\;x_{2}))):T$ because $T_{1}\rightarrow T_{2}=T$.
\end{case}
\begin{case}
$^{T_{1}\rightarrow T_{2}}AS$ $(\lambda x_{1}.e_{S})\rightarrow\lambda x_{2}:T_{1}[T_{i}/T_{i}^{a}].(^{T_{2}}AS$ $((\lambda x_{1}.e_{S})$ $(SA^{T_{1}}$ $x_{2})))$ where $A\in\lbrace H,M\rbrace$

$\Gamma\vdash_{A}{^{T_{1}\rightarrow T_{2}}A}S$ $(\lambda x_{1}.e_{S}):T$ by premise and uniqueness of types (Lemma \ref{uot}).  $\Gamma\vdash_{S}\lambda x_{1}.e_{S}:TST$ by inversion (Lemma \ref{i}).  $T=(T_{1}\rightarrow T_{2})[T_{i}/T_{i}^{a}]$ by inversion and uniqueness of types.  $\Gamma,x_{2}:T_{1}[T_{i}/T_{i}^{a}]\vdash_{A}x_{2}:T_{1}[T_{i}/T_{i}^{a}]$ by inversion and uniqueness of types.  $\Gamma,x_{2}:T_{1}[T_{i}/T_{i}^{a}]\vdash_{S}SA^{T_{1}}$ $x_{2}:TST$ and $\Gamma,x_{2}:T_{1}[T_{i}/T_{i}^{a}]\vdash_{S}(\lambda x_{1}.e_{S})$ $(SA^{T_{1}}$ $x_{2}):TST$ by inversion.  $\Gamma,x_{2}:T_{1}[T_{i}/T_{i}^{a}]\vdash_{A}{^{T_{2}}A}S$ $((\lambda x_{1}.e_{S})$ $(SA^{T_{1}}$ $x_{2})):T_{2}[T_{i}/T_{i}^{a}]$ and $\Gamma\vdash_{A}\lambda x_{2}:T_{1}[T_{i}/T_{i}^{a}].(^{T_{2}}AS$ $((\lambda x_{1}.e_{S})$ $(SA^{T_{1}}$ $x_{2}))):T_{1}[T_{i}/T_{i}^{a}]\rightarrow T_{2}[T_{i}/T_{i}^{a}]$ by inversion and uniqueness of types.  $\Gamma\vdash_{A}\lambda x_{2}:T_{1}[T_{i}/T_{i}^{a}].(^{T_{2}}AS$ $((\lambda x_{1}.e_{S})$ $(SA^{T_{1}}$ $x_{2}))):T$ because $T_{1}[T_{i}/T_{i}^{a}]\rightarrow T_{2}[T_{i}/T_{i}^{a}]=(T_{1}\rightarrow T_{2})[T_{i}/T_{i}^{a}]=T$.
\end{case}
\begin{case}
$^{T_{1}\rightarrow T_{2}}AS$ $v_{S}\rightarrow{^{T_{1}\rightarrow T_{2}}A}S$ $(\mathtt{wrong}$ $\mathrm{``Not}$ $\mathrm{a}$ $\mathrm{function"})$ $(v_{S}\neq\lambda x.e_{S})$ where $A\in\lbrace H,M\rbrace$

$\Gamma\vdash_{A}{^{T_{1}\rightarrow T_{2}}A}S$ $v_{S}:T$ by premise and uniqueness of types (Lemma \ref{uot}).  $T=(T_{1}\rightarrow T_{2})[T_{i}/T_{i}^{a}]$ by inversion (Lemma \ref{i}) and uniqueness of types.  $\vdash_{S}\mathtt{wrong}$ $\mathrm{``Not}$ $\mathrm{a}$ $\mathrm{function"}:TST$ by inversion.  $\Gamma\vdash_{A}{^{T_{1}\rightarrow T_{2}}A}S$ $(\mathtt{wrong}$ $\mathrm{``Not}$ $\mathrm{a}$ $\mathrm{function"}):(T_{1}\rightarrow T_{2})[T_{i}/T_{i}^{a}]$ by inversion and uniqueness of types.  $\Gamma\vdash_{A}{^{T_{1}\rightarrow T_{2}}A}S$ $(\mathtt{wrong}$ $\mathrm{``Not}$ $\mathrm{a}$ $\mathrm{function"}):T$ because $(T_{1}\rightarrow T_{2})[T_{i}/T_{i}^{a}]=T$.
\end{case}
\begin{case}
$SA^{T_{1}\rightarrow T_{2}}\;(\lambda x_{1}:T_{1}[T_{i}/T_{i}^{a}].e_{A})\rightarrow\lambda x_{2}.(SA^{T_{2}}\;((\lambda x_{1}:T_{1}[T_{i}/T_{i}^{a}].e_{A})\;(^{T_{1}}AS\;x_{2})))$ where $A\in\lbrace H,M\rbrace$

$\Gamma\vdash_{S}SA^{T_{1}\rightarrow T_{2}}\;(\lambda x_{1}:T_{1}[T_{i}/T_{i}^{a}].e_{A}):TST$ by premise.  $\Gamma\vdash_{A}\lambda x_{1}:T_{1}[T_{i}/T_{i}^{a}].e_{A}:T_{1}[T_{i}/T_{i}^{a}]\rightarrow T_{2}[T_{i}/T_{i}^{a}]$ by inversion (Lemma \ref{i}) and uniqueness of types (Lemma \ref{uot}).  $\Gamma,x_{2}:TST\vdash_{S}x_{2}:TST$ by inversion.  $\Gamma,x_{2}:TST\vdash_{A}{^{T_{1}}A}S\;x_{2}:T_{1}[T_{i}/T_{i}^{a}]$ and $\Gamma,x_{2}:TST\vdash_{A}(\lambda x_{1}:T_{1}[T_{i}/T_{i}^{a}].e_{A})\;(^{T_{1}}AS\;x_{2}):T_{2}[T_{i}/T_{i}^{a}]$ by inversion and uniqueness of types.  $\Gamma,x_{2}:TST\vdash_{S}SA^{T_{2}}\;((\lambda x_{1}:T_{1}[T_{i}/T_{i}^{a}].e_{A})\;(^{T_{1}}AS\;x_{2})):TST$ and $\Gamma\vdash_{S}\lambda x_{2}.(SA^{T_{2}}\;((\lambda x_{1}:T_{1}[T_{i}/T_{i}^{a}].e_{A})\;(^{T_{1}}AS\;x_{2}))):TST$ by inversion.
\end{case}
\begin{case}
$^{\forall X.T_{1}}AB^{\forall X_{1}.T_{1}}$ $(\Lambda X.e_{B})\rightarrow\Lambda X.(^{T_{1}}AB^{T_{1}}$ $e_{B})$ where $(A,B)\in\lbrace(H,M),(M,H)\rbrace$

$\Gamma\vdash_{A}{^{\forall X.T_{1}}A}B^{\forall X.T_{1}}$ $(\Lambda X_{1}.e_{B}):T$ by premise and uniqueness of types (Lemma \ref{uot}).  $\Gamma,X\vdash_{B}e_{B}:T_{1}$, $\Gamma\vdash_{B}\Lambda X.e_{B}:\forall X.T_{1}$, $T=\forall X.T_{1}$, $\Gamma,X\vdash_{A}{^{T_{1}}A}B^{T_{1}}$ $e_{B}:T_{1}$, and $\Gamma\vdash_{A}\Lambda X.(^{T_{1}}AB^{T_{1}}$ $e_{B}):\forall X.T_{1}$ by inversion (Lemma \ref{i}) and uniqueness of types.  $\Gamma\vdash_{A}\Lambda X.(^{T_{1}}AB^{T_{1}}$ $e_{B}):T$ because $\forall X.T_{1}=T$.
\end{case}
\begin{case}
$^{\forall X.T_{1}}AB^{\forall X.T_{1}}$ $(^{\forall X.T_{1}}BS$ $v_{S})\rightarrow{^{\forall X.T_{1}}A}S$ $v_{S}$ where $(A,B)\in\lbrace(H,M),$ $(M,H)\rbrace$

$\Gamma\vdash_{A}{^{\forall X.T_{1}}A}B^{\forall X.T_{1}}$ $(^{\forall X.T_{1}}BS$ $v_{S}):T$ by premise and uniqueness of types (Lemma \ref{uot}).  $\Gamma\vdash_{S}v_{S}:TST$ by inversion (Lemma \ref{i}).  $\Gamma\vdash_{B}{^{\forall X.T_{1}}B}S$ $v_{S}:\forall X.T_{1}$, $T=\forall X.T_{1}$, and $\Gamma\vdash_{A}{^{\forall X.T_{1}}A}S$ $v_{S}:\forall X.T_{1}$ by inversion and uniqueness of types.  $\Gamma\vdash_{A}{^{\forall X.T_{1}}A}S$ $v_{S}:T$ because $\forall X.T_{1}=T$.
\end{case}
\begin{case}
$(^{\forall X.T_{1}}AS\;v_{S})\;\lbrace T_{2}\rbrace\rightarrow{^{T_{1}[T_{2}^{a}/X]}A}S\;v_{S}$ where $A\in\lbrace H,M\rbrace$

$\Gamma\vdash_{A}(^{\forall X.T_{1}}AS\;v_{S})\;\lbrace T_{2}\rbrace:T$ by premise and uniqueness of types (Lemma \ref{uot}).  $T=T_{1}[T_{2}/X]$ by inversion (Lemma \ref{i}) and uniqueness of types.  $\Gamma\vdash_{S}v_{S}:TST$ by inversion.  $\Gamma\vdash_{A}{^{T_{1}[T_{2}^{a}/X]}A}S\;v_{S}:T_{1}[T_{2}^{a}/X][T_{i}/T_{i}^{a}]$ by inversion and uniqueness of types.  $\Gamma\vdash_{A}{^{T_{1}[T_{2}^{a}/X]}A}S\;v_{S}:T$ because $T_{1}[T_{2}^{a}/X][T_{i}/T_{i}^{a}]=T_{1}[T_{2}/X]=T$.
\end{case}
\begin{case}
$SA^{\forall X.T_{1}}$ $(\Lambda X.e_{A})\rightarrow SA^{T_{1}[L/X]}$ $((\Lambda X.e_{A})$ $\lbrace L\rbrace)$ where $A\in\lbrace H,M\rbrace$

$\Gamma\vdash_{S}SA^{\forall X.T_{1}}$ $(\Lambda X.e_{A}):TST$ by premise.  $\Gamma\vdash_{A}\Lambda X.e_{A}:\forall X.T_{1}$ and $\Gamma\vdash_{A}(\Lambda X.e_{A})$ $\lbrace L\rbrace:T_{1}[L/X]$ by inversion (Lemma \ref{i}) and uniqueness of types (Lemma \ref{uot}).  $\Gamma\vdash_{S}SA^{T_{1}[L/X]}$ $((\Lambda X.e_{A})$ $\lbrace L\rbrace):TST$ by inversion.
\end{case}
\begin{case}
$^{[T_{1}]}HM^{[T_{1}]}\;(\mathtt{cons}\;v_{M}^{1}\;v_{M}^{2})\rightarrow\mathtt{cons}\;(^{T_{1}}HM^{T_{1}}\;v_{M}^{1})\;(^{[T_{1}]}HM^{[T_{1}]}\;v_{M}^{2})$

$^{[T_{1}]}HM^{[T_{1}]}\;(\mathtt{cons}\;v_{M}^{1}\;v_{M}^{2}):T$ by premise and uniqueness of types (Lemma \ref{uot}).  $T=[T_{1}]$, $\Gamma\vdash_{M}v_{M}^{1}:T_{1}$, $\Gamma\vdash_{M}v_{M}^{2}:[T_{1}]$, $\Gamma\vdash_{H}{^{T_{1}}H}M^{T_{1}}\;v_{M}^{1}:T_{1}$, $\Gamma\vdash_{H}{^{[T_{1}]}H}M^{[T_{1}]}\;v_{M}^{2}:[T_{1}]$, and $\Gamma\vdash_{H}\mathtt{cons}\;(^{T_{1}}HM^{T_{1}}\;v_{M}^{1})\;(^{[T_{1}]}HM^{[T_{1}]}\;v_{M}^{2}):[T_{1}]$ by inversion (Lemma \ref{i}) and uniqueness of types.  $\Gamma\vdash_{H}\mathtt{cons}\;(^{T_{1}}HM^{T_{1}}\;v_{M}^{1})\;(^{[T_{1}]}HM^{[T_{1}]}\;v_{M}^{2}):T$ because $[T_{1}]=T$.
\end{case}
\begin{case}
$^{[T_{1}]}HM^{[T_{1}]}\;(^{[T_{1}]}MH^{[T_{1}]}\;(\mathtt{cons}\;e_{H}^{1}\;e_{H}^{2}))\rightarrow\mathtt{cons}\;e_{H}^{1}\;e_{H}^{2}$

$^{[T_{1}]}HM^{[T_{1}]}\;(^{[T_{1}]}MH^{[T_{1}]}\;(\mathtt{cons}\;e_{H}^{1}\;e_{H}^{2})):T$ by premise and uniqueness of types (Lemma \ref{uot}).  $T=[T_{1}]$, $\Gamma\vdash_{H}{^{[T_{1}]}M}H^{[T_{1}]}\;(\mathtt{cons}\;e_{H}^{1}\;e_{H}^{2}):[T_{1}]$, and $\Gamma\vdash_{H}\mathtt{cons}\;e_{H}^{1}\;e_{H}^{2}:[T_{1}]$ by inversion (Lemma \ref{i}) and uniqueness of types.  $\Gamma\vdash_{H}\mathtt{cons}\;e_{H}^{1}\;e_{H}^{2}:T$ because $[T_{1}]=T$.
\end{case}
\begin{case}
$\mathtt{hd}\;(^{[T_{1}]}MH^{[T_{1}]}\;(\mathtt{cons}\;e_{H}^{1}\;e_{H}^{2}))\rightarrow{^{T_{1}}M}H^{T_{1}}\;e_{H}^{1}$

$\Gamma\vdash_{M}\mathtt{hd}\;(^{[T_{1}]}MH^{[T_{1}]}\;(\mathtt{cons}\;e_{H}^{1}\;e_{H}^{2})):T$ by premise and uniqueness of types (Lemma \ref{uot}).  $T=T_{1}$, $\Gamma\vdash_{H}e_{H}^{1}:T_{1}$, and $^{T_{1}}MH^{T_{1}}\;e_{H}^{1}:T_{1}$ by inversion (Lemma \ref{i}) and uniqueness of types (Lemma \ref{uot}).  $^{T_{1}}MH^{T_{1}}\;e_{H}^{1}:T$ because $T_{1}=T$.
\end{case}
\begin{case}
$\mathtt{hd}\;(^{[T_{1}]}MH^{[T_{1}]}\;(\mathtt{cons}\;e_{H}^{1}\;e_{H}^{2}))\rightarrow{^{[T_{1}]}M}H^{[T_{1}]}\;e_{H}^{2}$

$\Gamma\vdash_{M}\mathtt{hd}\;(^{[T_{1}]}MH^{[T_{1}]}\;(\mathtt{cons}\;e_{H}^{1}\;e_{H}^{2})):T$ by premise and uniqueness of types (Lemma \ref{uot}).  $T=T_{1}$, $\Gamma\vdash_{H}e_{H}^{1}:T_{1}$, and $^{[T_{1}]}MH^{[T_{1}]}\;e_{H}^{1}:[T_{1}]$ by inversion (Lemma \ref{i}) and uniqueness of types (Lemma \ref{uot}).  $^{[T_{1}]}MH^{[T_{1}]}\;e_{H}^{2}:T$ because $[T_{1}]=T$.
\end{case}
\begin{case}
$^{[T_{1}]}AS\;(\mathtt{cons}\;v_{S}^{1}\;v_{S}^{2})\rightarrow\mathtt{cons}\;(^{T_{1}}AS\;v_{S}^{1})\;(^{[T_{1}]}AS\;v_{S}^{2})$ where $A\in\lbrace H,M\rbrace$

$\Gamma\vdash_{A}{^{[T_{1}]}A}S\;(\mathtt{cons}\;v_{S}^{1}\;v_{S}^{2}):T$ by premise and uniqueness of types (Lemma \ref{uot}).  $T=[T_{1}]$ by inversion (Lemma \ref{i}) and uniqueness of types.  $\Gamma\vdash_{S}v_{S}^{1}:TST$, and $\Gamma\vdash_{S}v_{S}^{2}:TST$ by inversion.  $\Gamma\vdash_{A}{^{T_{1}}A}S\;v_{S}^{1}:T_{1}$, $\Gamma\vdash_{A}{^{[T_{1}]}A}S\;v_{S}^{2}:[T_{1}]$, and $\Gamma\vdash_{A}\mathtt{cons}\;(^{T_{1}}AS\;v_{S}^{1})\;(^{[T_{1}]}AS\;v_{S}^{2}):[T_{1}]$ by inversion and uniqueness of types.  $\Gamma\vdash_{A}\mathtt{cons}\;(^{T_{1}}AS\;v_{S}^{1})\;(^{[T_{1}]}AS\;v_{S}^{2}):T$ because $[T_{1}]=T$.
\end{case}
\begin{case}
$^{[T_{1}]}HS$ $(SH^{[T_{1}]}$ $(\mathtt{cons}$ $e_{H}^{1}$ $e_{H}^{2}))\rightarrow\mathtt{cons}$ $e_{H}^{1}$ $e_{H}^{2}$

$^{[T_{1}]}HS$ $(SH^{[T_{1}]}$ $(\mathtt{cons}$ $e_{H}^{1}$ $e_{H}^{2})):T$ by premise and uniqueness of types (Lemma \ref{uot}).  $\Gamma\vdash_{H}\mathtt{cons}$ $e_{H}^{1}$ $e_{H}^{2}:[T_{1}]$ by inversion (Lemma \ref{i}) and uniqueness of types.  $\Gamma\vdash_{S}SH^{[T_{1}]}$ $(\mathtt{cons}$ $e_{H}^{1}$ $e_{H}^{2}):TST$ by inversion.  $T=[T_{1}]$ by inversion and uniqueness of types.  $\Gamma\vdash_{H}\mathtt{cons}$ $e_{H}^{1}$ $e_{H}^{2}:T$ because $[T_{1}]=T$.
\end{case}
\begin{case}
$^{[T_{1}]}MS$ $(SH^{[T_{1}]}$ $(\mathtt{cons}$ $e_{H}^{1}$ $e_{H}^{2}))\rightarrow{^{[T_{1}]}M}H^{[T_{1}]}$ $(\mathtt{cons}$ $e_{H}^{1}$ $e_{H}^{2})$

$^{[T_{1}]}MS$ $(SH^{[T_{1}]}$ $(\mathtt{cons}$ $e_{H}^{1}$ $e_{H}^{2})):T$ by premise and uniqueness of types (Lemma \ref{i}).  $\Gamma\vdash_{H}\mathtt{cons}$ $e_{H}^{1}$ $e_{H}^{2}:[T_{1}]$ by inversion (Lemma \ref{i}) and uniqueness of types.  $\Gamma\vdash_{S}SH^{[T_{1}]}$ $(\mathtt{cons}$ $e_{H}^{1}$ $e_{H}^{2}):TST$ by inversion.  $T=[T_{1}]$ and $\Gamma\vdash_{M}{^{[T_{1}]}M}H^{[T_{1}]}$ $(\mathtt{cons}$ $e_{H}^{1}$ $e_{H}^{2}):[T_{1}]$ by inversion and uniqueness of types.  $\Gamma\vdash_{M}{^{[T_{1}]}M}H^{[T_{1}]}$ $(\mathtt{cons}$ $e_{H}^{1}$ $e_{H}^{2}):T$ because $[T_{1}]=T$.
\end{case}
\begin{case}
$\mathtt{hd}\;(SH^{[T_{1}]}\;(\mathtt{cons}\;e_{H}^{1}\;e_{H}^{2}))\rightarrow SH^{T_{1}}\;e_{H}^{1}$

$\Gamma\vdash_{S}\mathtt{hd}\;(SH^{[T_{1}]}\;(\mathtt{cons}\;e_{H}^{1}\;e_{H}^{2})):TST$ by premise.  $\Gamma\vdash_{H}e_{H}^{1}:T_{1}$ by inversion (Lemma \ref{i}) and uniqueness of types (Lemma \ref{uot}).  $\Gamma\vdash_{S}SH^{T_{1}}\;e_{H}^{1}:TST$ by inversion.
\end{case}
\begin{case}
$\mathtt{tl}\;(SH^{[T_{1}]}\;(\mathtt{cons}\;e_{H}^{1}\;e_{H}^{2}))\rightarrow SH^{[T_{1}]}\;e_{H}^{2}$

$\Gamma\vdash_{S}\mathtt{tl}\;(SH^{[T_{1}]}\;(\mathtt{cons}\;e_{H}^{1}\;e_{H}^{2})):TST$ by premise.  $\Gamma\vdash_{H}e_{H}^{2}:[T_{1}]$ by inversion (Lemma \ref{i}) and uniqueness of types (Lemma \ref{uot}).  $\Gamma\vdash_{S}SH^{[T_{1}]}\;e_{H}^{2}:TST$ by inversion.
\end{case}
\begin{case}
$SM^{[T_{1}]}\;(\mathtt{cons}\;v_{M}^{1}\;v_{M}^{2})\rightarrow\mathtt{cons}\;(SM^{T_{1}}\;v_{M}^{1})\;(SM^{[T_{1}]}\;v_{M}^{2})$

$\Gamma\vdash_{S}SM^{[T_{1}]}\;(\mathtt{cons}\;v_{M}^{1}\;v_{M}^{2}):TST$ by premise.  $\Gamma\vdash_{M}v_{M}^{1}:T_{1}$ and $\Gamma\vdash_{M}v_{M}^{2}:[T_{1}]$ by inversion (Lemma \ref{i}) and uniqueness of types (Lemma \ref{uot}).  $\Gamma\vdash_{S}SM^{T_{1}}\;v_{M}^{1}:TST$, $\Gamma\vdash_{S}SM^{[T_{1}]}\;v_{M}^{2}:TST$, and $\Gamma\vdash_{S}\mathtt{cons}\;(SM^{T_{1}}\;v_{M}^{1})\;(SM^{[T_{1}]}\;v_{M}^{2}):TST$ by inversion.
\end{case}
\begin{case}
$SM^{[T_{1}]}\;(^{[T_{1}]}MH^{[T_{1}]}\;(\mathtt{cons}\;e_{H}^{1}\;e_{H}^{2}))\rightarrow SH^{[T_{1}]}\;(\mathtt{cons}\;e_{H}^{1}\;e_{H}^{2})$

$SM^{[T_{1}]}\;(^{[T_{1}]}MH^{[T_{1}]}\;(\mathtt{cons}\;e_{H}^{1}\;e_{H}^{2})):TST$ by premise.  $\Gamma\vdash_{H}\mathtt{cons}\;e_{H}^{1}\;e_{H}^{2}:[T_{1}]$ and $\Gamma\vdash_{M}{^{[T_{1}]}M}H^{[T_{1}]}\;(\mathtt{cons}\;e_{H}^{1}\;e_{H}^{2}):[T_{1}]$ by inversion (Lemma \ref{i}) and uniqueness of types (Lemma \ref{uot}).  $\Gamma\vdash_{S}SH^{[T_{1}]}\;(\mathtt{cons}\;e_{H}^{1}\;e_{H}^{2}):TST$ by inversion.
\end{case}
\begin{case}
$^{[T_{1}]}AB^{[T_{1}]}\;\mathtt{nil}^{T_{1}}\rightarrow\mathtt{nil}^{T_{1}}$ where $(A,B)\in\lbrace(H,M),(M,H)\rbrace$

$\Gamma\vdash_{A}{^{[T_{1}]}A}B^{[T_{1}]}\;\mathtt{nil}^{T_{1}}:T$ by premise and uniqueness of types (Lemma \ref{uot}).  $\Gamma\vdash_{A}\mathtt{nil}^{T_{1}}:[T_{1}]$ and $T=[T_{1}]$ by inversion (Lemma \ref{i}) and uniqueness of types.  $\Gamma\vdash_{A}\mathtt{nil}^{T_{1}}:T$ because $[T_{1}]=T$.
\end{case}
\begin{case}
$^{[T_{1}]}AS$ $\mathtt{nil}\rightarrow\mathtt{nil}^{T_{1}}$ where $A\in\lbrace H,M\rbrace$

$\Gamma\vdash_{A}{^{[T_{1}]}A}S$ $\mathtt{nil}:T$ by premise and uniqueness of types (Lemma \ref{uot}).  $T=[T_{1}]$ and $\Gamma\vdash_{A}\mathtt{nil}^{T_{1}}:[T_{1}]$ by inversion (Lemma \ref{i}) and uniqueness of types.  $\Gamma\vdash_{A}\mathtt{nil}^{T_{1}}:T$ because $[T_{1}]=T$.
\end{case}
\begin{case}
$^{[T_{1}]}AS\;v_{S}^{1}\rightarrow{^{[T_{1}]}A}S\;(\mathtt{wrong}\;\mathrm{``Not\;a\;list"})$ $(v_{S}^{1}\neq\mathtt{cons}\;v_{S}^{2}\;v_{S}^{3}$ and $v_{S}^{1}\neq\mathtt{nil})$ where $A\in\lbrace H,M\rbrace$

$\Gamma\vdash_{A}{^{[T_{1}]}AS}\;v_{S}^{1}:T$ by premise and uniqueness of types (Lemma \ref{uot}).  $T=[T_{1}]$ by inversion (Lemma \ref{i}) and uniqueness of types.  $\vdash_{S}\mathtt{wrong}\;\mathrm{``Not\;a\;list"}:TST$ by inversion.  $\Gamma\vdash_{A}{^{[T_{1}]}A}S\;(\mathtt{wrong}\;\mathrm{``Not\;a\;list"}):[T_{1}]$ by inversion and uniqueness of types.  $\Gamma\vdash_{A}{^{[T_{1}]}A}S\;(\mathtt{wrong}\;\mathrm{``Not\;a\;list"}):T$ because $[T_{1}]=T$.
\end{case}
\begin{case}
$^{L}AB^{L}\;(^{L}BS\;v_{S})\rightarrow{^{L}A}S\;v_{S}$ where $(A,B)\in\lbrace(H,M),(M,H)\rbrace$

$\Gamma\vdash_{A}{^{L}A}B^{L}\;(^{L}BS\;v_{S}):T$ by premise and uniqueness of types (Lemma \ref{uot}).  $T=L$ by inversion (Lemma \ref{i}) and uniqueness of types.  $\Gamma\vdash_{S}v_{S}:TST$ by inversion.  $\Gamma\vdash_{A}{^{L}A}S\;v_{S}:L$ by inversion and uniqueness of types.  $\Gamma\vdash_{A}{^{L}A}S\;v_{S}:T$ because $L=T$.
\end{case}
\begin{case}
$^{T_{1}^{a}}AS$ $(SA^{T_{1}^{a}}$ $B_{A})\rightarrow B_{A}$ where $(A,B)\in\lbrace(H,e),(M,v)\rbrace$

$^{T_{1}^{a}}AS$ $(SA^{T_{1}^{a}}$ $B_{A}):T$ by premise and uniqueness of types (Lemma \ref{uot}).  $\Gamma\vdash_{A}B_{A}:T_{1}^{a}[T_{i}/T_{i}^{a}]$ by inversion (Lemma \ref{i}) and uniqueness of types.  $\Gamma\vdash_{S}SA^{T_{1}^{a}}$ $B_{A}:TST$ by inversion.  $T=T_{1}^{a}[T_{i}/T_{i}^{a}]$ by inversion and uniqueness of types.  $\Gamma\vdash_{A}B_{A}:T$ because $T_{1}^{a}[T_{i}/T_{i}^{a}]=T$.
\end{case}
\begin{case}
$^{T_{1}^{a}}AS\;v_{S}\rightarrow{^{T_{1}^{a}}A}S\;(\mathtt{wrong}\;\mathrm{``Parametricity\;violated"})$ $(v_{S}\neq SA^{T_{1}^{a}}\;B_{A})$ where $(A,B)\in\lbrace(H,e),(M,v)\rbrace$

$\Gamma\vdash_{A}{^{T_{1}^{a}}A}S\;v_{S}:T$ by premise and uniqueness of types (Lemma \ref{uot}).  $T=T_{1}^{a}[T_{i}/T_{i}^{a}]$ by inversion (Lemma \ref{i}) and uniqueness of types.  $\vdash_{S}\mathtt{wrong}\;\mathrm{``Parametricity\;violated"}:TST$ by inversion.  $^{T_{1}^{a}}AS\;(\mathtt{wrong}\;\mathrm{``Parametricity\;violated"}):T_{1}^{a}[T_{i}/T_{i}^{a}]$ by inversion and uniqueness of types.  $^{T_{1}^{a}}AS\;(\mathtt{wrong}\;\mathrm{``Parametricity\;violated"}):T$ because $T_{1}^{a}[T_{i}/T_{i}^{a}]=T$.
\end{case}
\end{proof}
\end{theorem}
\chapter{Implementation of the Model}

The model of computation defined above was implemented in the DrScheme integrated development environment.  Haskell and ML modules are written in subsets of their real syntaxes and compiled to Scheme modules.  Scheme modules are written with the Scheme language supported by the Scheme compiler of DrScheme.  A module defined in a particular language exports sets of declarations for the other languages and only imports sets of declarations from other modules meant only for its language.  Haskell can import ML and Scheme expressions, ML can import Scheme expressions, and Scheme can import Haskell expressions.  Haskell and ML use a form of type reconstruction called let-polymorphism instead of the explicit type system System F.  The domain of numbers for all languages is integers.  All languages have boolean values and operations, additional arithmetic operations, and composite data.  The majority of the ML implementation was taken from Kinghorn \ref{TODO: cite kinghorn}.

\section{Importation Syntax}

The model expressed language interoperation through nesting languages within each other as expressions.  Real-world code libraries interoperate through interfaces.  In the implementation, modules are the code libraries of interest.  Languages interoperate by importing a declaration exported by a module by specifying its name and type.  The Haskell syntax that imports ML and Scheme expressions is \texttt{:ml type "name"} and \texttt{:scheme type "name"}, respectively, where \texttt{type} and \texttt{name} are the types and names of the expressions, respectively.  The ML syntax that imports Scheme expressions is \texttt{name :G type}.  The Scheme syntax that imports Haskell expressions is \texttt{(:haskell name type)}.  For example, \texttt{(:scheme a $\rightarrow$ a "identity") 0} imports the Scheme identity function to Haskell and applies it to zero.

\section{Importation Type Verification}

The Haskell import expressions, the ML import expression for Scheme, and the Scheme import expression for Haskell verify the expected types match the actual types.  Where expressions from Haskell and ML are imported, their actual types can be checked immediately because Haskell and ML modules export the types of their exported declarations for import verification.  Where Scheme expressions are imported, their actual types can be checked only if they are not functions.  If they are functions, a DrScheme library that implements contracts for higher-order functions \cite{findler02}, is used to delay the verifications of their types.

Contracts are the mechanism that verifies values crossing boundaries match their expected types.  The library also provides a mechanism for assigning blame to languages for type errors to indicate which is at fault.  If a Scheme value does not match its expected type, Scheme is at fault.  If an ML value does not match its expected type, ML is at fault.  If Scheme applies an ML function to the wrong value, Scheme is at fault.  ML cannot apply a Scheme function to the wrong type because such a function application expression would be ill-typed.

\section{Limiting Scheme}

The real-world Scheme of DrScheme represents Scheme within the system of interoperation.  The real-world Scheme has language constructs not supported or allowed by Haskell and ML.  Therefore Scheme modules are presumed to use only a subset of the features of Scheme that are compatible with Haskell and ML.  For example, a Scheme module could use side effects to break parametricity by determining the behavior of a function by the state of a global variable.
\chapter{Future Work}

The model of computation is sufficient to enable the interoperation of languages with incompatible evaluation strategies.  Certainly other data types could be added to the model, but they would add nothing new to the method of resolving incompatible evaluation strategies and would further complicate the model.  The implementation of the model would be more useful if additional language constructs and data types were added.  Performance would improve if modules were compiled to bytecodes or machine code.  Adding languages with other evaluation strategies, such as normal order and applicative order, or languages with static type systems that do not support parametricity, would be interesting, but the sizes of the model and proof would grow exponentially.  Further explorations of incompatible evaluation strategies would best be tackled with pairs of languages to minimize complexity.
\chapter{Future Work}

The model of computation is sufficient to enable the interoperation of languages with incompatible evaluation strategies.  Certainly other data types could be added to the model, but they would add nothing new to the method of resolving incompatible evaluation strategies and would further complicate the model.  The implementation of the model would be more useful if additional language constructs and data types were added.  Performance would improve if modules were compiled to bytecodes or machine code.  Adding languages with other evaluation strategies, such as normal order and applicative order, or languages with static type systems that do not support parametricity, would be interesting, but the sizes of the model and proof would grow exponentially.  Further explorations of incompatible evaluation strategies would best be tackled with pairs of languages to minimize complexity.
\chapter{Conclusions}

This work resolved three language incompatibilities in a system of interoperation for three diverse languages.  It resolved incompatible type systems with contracts for higher-order functions and lump types.  It resolved incompatible support for parametricity with label types.  It resolved incompatible evaluation strategies with delayed conversions for list constructions.  It defined a model of computation that can express interoperation where the aforementioned incompatibilities arise and resolve them, provided a proof of type soundness, and described an implementation of the model that supported additional language features.

\clearpage
\bibliography{bibliography}
\bibliographystyle{plain}
\addcontentsline{toc}{chapter}{Bibliography}

\end{document}